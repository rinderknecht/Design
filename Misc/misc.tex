%%%%%%%

Here, we will follow \cite{Fang:2002}, with our own
improvements, as he stroke a good balance between accuracy and
technical difficulty. Let us start by gathering
deduction~\eqref{eq:num_of_bits}, definition~\eqref{eq:F} and
equation~\eqref{eq:ruler_nu} into what may be considered an
alternative definition of~\(\nu_i\):
\begin{equation*}
\abovedisplayskip=2pt
\belowdisplayskip=0pt
\nu_i = i - \sum_{j=1}^{\floor{\lg n}}
\left\lfloor{\frac{i}{2^j}}\right\rfloor.
\end{equation*}
Therefore,
\begin{equation*}
\abovedisplayskip=0pt
\belowdisplayskip=2pt
\OB{\fun{tms}}{n} = \frac{1}{2}n(n-1) -
\sum_{i=1}^{n-1}\sum_{j=1}^{\floor{\lg i}}
\left\lfloor{\frac{i}{2^j}}\right\rfloor
= \frac{1}{2}n(n-1) -
\sum_{j=1}^{\floor{\lg n}}\sum_{i=1}^{n-1}
\left\lfloor{\frac{i}{2^j}}\right\rfloor.
\index{tms@$\OB{\fun{tms}}{n}$}
\end{equation*}
The inversion of summations is valid because \(\floor{i/2^j} = 0\) for
any \(j > \floor{\lg i}\). In order to evaluate the new inner sum, let
us consider the table in \fig~\vref{fig:bits},
which shows the enumeration of the first integers in binary. The
quantity \(\floor{i/2^j}\) is therein interpreted as the \(i\)th row
with the \(j\)th rightmost bits discarded. For instance,
\(\floor{9/2^2} = 2\) because \((1001)_2=9\) is cut to
\((10)_2=2\). The sum \(\sum_{i=1}^{n-1}\floor{i/2^j}\) is the sum of
the binary numbers lying at the left of a vertical line set between
the \(j\)th and \((j+1)\)th bits. (We advise the reader to write a
longer table and take as an example \(n=(110110)_2=54\) and \(j=3\).)
Under close examination, two chunks of the tables appear to make up
the sum we are interested in:
\begin{enumerate}

  \item Starting at row~\(2^j\), the number at the left of the
    vertical line is~\(1\) and it is repeated \(2^j\)~times; further
    down, the number \((10)_2=2\) is repeated \(2^j\)~times etc. until
    \(\floor{n/2^j}-1\) is repeated \(2^j\)~times. This amounts to
    \(\frac{1}{2}\floor{n/2^j}(\floor{n/2^j}-1)2^j\).

  \item Remains a piece which may be incomplete when reaching
    \(n-1\). It can be seen to be \(\floor{n/2^j}(n \bmod{2^j}) =
    \floor{n/2^j}(n-\floor{n/2^j}2^j)\).

\end{enumerate}
Adding these two partial sums, we obtain
\begin{equation*}
\sum_{i=1}^{n-1}\left\lfloor\frac{i}{2^j}\right\rfloor
= -\frac{1}{2}\left\lfloor\frac{n}{2^j}\right\rfloor 2^j
  + \left(n\left\lfloor\frac{n}{2^j}\right\rfloor
  - \frac{1}{2}\left\lfloor\frac{n}{2^j}\right\rfloor^22^j\right).
\end{equation*}
We can now return back at our double summation:
\begin{equation*}
\OB{\fun{tms}}{n} =
    \frac{1}{2}n(n-1)
  + \frac{1}{2}\sum_{j=1}^{\floor{\lg n}}
      \left\lfloor\frac{n}{2^j}\right\rfloor{2^j}
  - \sum_{j=1}^{\floor{\lg n}}
      \left(n\left\lfloor\frac{n}{2^j}\right\rfloor
      - \frac{1}{2}\left\lfloor\frac{n}{2^j}\right\rfloor^22^j\right).
\index{tms@$\OB{\fun{tms}}{n}$}
\end{equation*}
Let us introduce now the \emph{fractional part}\index{fractional
  part@$\{x\}$|see{fractional part}}\index{fractional part} of a
real~\(x\) as \(\{x\} := x - \floor{x}\). In particular, we have \(0
\leqslant \{x\} < 1\). Then
\begin{equation*}
\sum_{j=1}^{\floor{\lg n}}
      \left(n\left\lfloor\frac{n}{2^j}\right\rfloor
      - \frac{1}{2}\left\lfloor\frac{n}{2^j}\right\rfloor^22^j\right)
= \frac{1}{2}n^2 - \frac{n^2}{2^{\floor{\lg n}+1}}
  - \frac{1}{2}\sum_{j=1}^{\floor{\lg n}}\left\lbrace\frac{n}{2^j}\right\rbrace^22^j,
\end{equation*}
because \(\sum_{j=1}^{\floor{\lg n}}{1/2^j} = 1 - 1/2^{\floor{\lg
    n}}\). Similarly, we have
\begin{equation*}
\sum_{j=1}^{\floor{\lg n}}
      \left\lfloor\frac{n}{2^j}\right\rfloor{2^j}
= n\floor{\lg n}
- \sum_{j=1}^{\floor{\lg n}}\left\lbrace\frac{n}{2^j}\right\rbrace 2^j.
\end{equation*}
We can now express \(\OB{\fun{tms}}{n}\) in terms of~\(n\) and isolate
the fractional parts:
\begin{equation*}
2 \cdot \OB{\fun{tms}}{n} =
  n\floor{\lg n} + \frac{n^2}{2^{\floor{\lg n}}}
  - n
  - \sum_{j=1}^{\floor{\lg n}}
    \left(\left\lbrace\frac{n}{2^j}\right\rbrace -
    \left\lbrace\frac{n}{2^j}\right\rbrace^2\right)2^j.
\end{equation*}
We now approximate the fractional parts \emph{together}. Since we have
the tight bounds \(0 \leqslant x - x^2 \leqslant \myfrac{1}/{4}\) if
\(0 \leqslant x \leqslant 1\), and \(\sum_{j=1}^{\floor{\lg n}}2^j =
2^{\floor{\lg n}+1} - 2\),
\begin{equation*}
n\floor{\lg n} + n^22^{-\floor{\lg n}} - n - 2^{\floor{\lg n}-1} +
\tfrac{1}{2} < 2 \cdot \OB{\fun{tms}}{n} \leqslant n\floor{\lg n} +
n^22^{-\floor{\lg n}} - n.
\end{equation*}
Now, let us approximate the bounds by making apparent \(\{\lg
n\}\). Let \(L_n\) be the lower bound and \(U_n\) the upper bound. We
have
\begin{equation*}
L_n = n\lg n + \theta_L(\{\lg n\}) \cdot n,\; \text{with
  \(\theta_L(x) := 2^x - 2^{-x-1} - x - 1\)}.
\end{equation*}
The derivative of \(\theta_L(x)\) tells us that \(\min_{0 \leqslant x
  < 1}\theta_L(x) = \theta_L(0) = -\myfrac{1}/{2}\). Since \(0
\leqslant \{\lg n\} < 1\), we draw \(L_n \geqslant n\lg n +
\theta_L(0) \cdot n = n\lg n - \frac{1}{2}n\). Also,
\begin{equation*}
U_n = n\lg n + \theta_U(\{\lg n\}) \cdot n,\; \text{with
  \(\theta_U(x) := 2^x - x - 1\)}.
\end{equation*}
The derivative of \(\theta_U(x)\) implies that \(\max_{0 \leqslant x <
  1}\theta_U(x) = \theta_U(0) = 0\), so \(U_n \leqslant n\lg n +
\theta_U(0) \cdot n = n\lg n\). Gathering both approximations yields
\begin{equation}
\tfrac{1}{2}n\lg n - \tfrac{1}{4}n + \tfrac{1}{4}
< \OB{\fun{tms}}{n} \leqslant
\tfrac{1}{2}n\lg n.
\label{ineq:Fang}\index{tms@$\OB{\fun{tms}}{n}$}
\end{equation}
The upper bound is reached when \(n=2^p\), as
equation~\eqref{eq:best_power} \vpageref{eq:best_power} shows. This
can also directly be seen in equation~\eqref{eq:OB_tms}
\vpageref{eq:OB_tms} and \fig~\vref{fig:bits}, as \(n-1=2^p-1\)
maximises the number of \(1\)-bits for a fixed total number of
bits. Unfortunately, we do not know~\(n\) when
\(\OB{\fun{tms}}{n}\)\index{tms@$\OB{\fun{tms}}{n}$} is minimum and
the lower bound itself can be improved.

%%%

This is about \(4.2\%\)~better than the linear coefficient
\(-\myfrac{1}/{4}\) in~\eqref{ineq:Fang}

%%%

\begin{equation}
\OW{\fun{bms}}{n} < n\lg n - 1.
\label{ineq:OWbms_up}
\end{equation}
\index{bms@$\OW{\fun{bms}}{n}$|)}
 If we do not accept these sums as closed
forms, we nevertheless can use them to obtain bounds on the maximum
cost.

Let \(U_n := \sum_{j=1}^{n-1}2^{\rho_j}\). We have \(U_1 = 0\) and
reusing definitions~\eqref{eq:ruler}:
\begin{align*}
U_{2p}   &= \sum_{k=0}^{p-1}2^{\rho_{2k+1}} + \sum_{k=1}^{p-1}2^{\rho_{2k}}
         = p + 2 \cdot U_p,\\
U_{2p+1} &= \sum_{k=0}^{p-1}2^{\rho_{2k+1}} + \sum_{k=1}^{p}2^{\rho_{2k}} 
        = p + 2 \cdot U_{p+1}.
\end{align*}
In other words, \(U_n = 2 \cdot U_{\ceiling{n/2}} + \floor{n/2} = 2
\cdot U_{\ceiling{n/2}} + n - \ceiling{n/2}\), since \(\floor{n/2} +
\ceiling{n/2} = n\). Unrolling the recursion reveals the equation
\begin{equation*}
2 \cdot U_n = 2n + \sum_{k=1}^{\floor{\lg
    n}}\left\lceil{\frac{n}{2^k}}\right\rceil 2^k - 2^{\floor{\lg
    n}+1},
\end{equation*}
if we mind that \(2^{\floor{\lg n}}\) is the highest power of~\(2\) in
the binary notation of~\(n\), so \(0 \leqslant n/2^{\floor{\lg n}+1} <
1\), and if we use the following theorem.
\begin{thm}[Ceilings and Fractions]
\label{thm:ceilings}
\textsl{Let \(x\)~be a real number and \(q\)~a natural number. Then
  \(\ceiling{\ceiling{x}/q} = \ceiling{x/q}\).}
\end{thm}
\begin{proof}
  The equality is equivalent to the conjunction of the two
  complementary inequalities \(\ceiling{\ceiling{x}/q} \geqslant
  \ceiling{x/q}\) and \(\ceiling{\ceiling{x}/q} \leqslant
  \ceiling{x/q}\). The former is straightforward: \(\ceiling{x}
  \geqslant x \Rightarrow \ceiling{x}/q \geqslant x/q \Rightarrow
  \ceiling{\ceiling{x}/q} \geqslant \ceiling{x/q}\). Next, because
  both sides of the inequality are integers, \(\ceiling{\ceiling{x}/q}
  \leqslant \ceiling{x/q}\) is equivalent to state that \(p \leqslant
  \ceiling{\ceiling{x}/q} \Rightarrow p \leqslant \ceiling{x/q}\), for
  any integer~\(p\). An obvious lemma is that if \(i\)~is an integer
  and~\(y\) a real number, \(i \leqslant \ceiling{y} \Leftrightarrow i
  \leqslant y\), so the original inequality is equivalent to \(p
  \leqslant \ceiling{x}/q \Rightarrow p \leqslant x/q\), for any
  integer \(p\), which is \(pq \leqslant \ceiling{x} \Rightarrow pq
  \leqslant x\). The lemma yields this implication, thereby achieving
  the proof.
\end{proof}
\noindent Using the standard inequalities \(x \leqslant \ceiling{x} <
x + 1\) yields
\begin{equation*}
n\floor{\lg n} + 2n - 2^{\floor{\lg n}+1}
\leqslant 2 \cdot U_n < n\floor{\lg n} + 2n - 2.
\end{equation*}
Using \(\floor{x} = x - \{x\}\) we derive
\begin{equation*}
n\lg n + 2n - n\cdot\theta_L(\{\lg n\})
\leqslant 2 \cdot U_n < n\lg n + 2n - 2,
\end{equation*}
with \(\theta_L(x) := x + 2^{1 - x}\). Since \(\max_{0 \leqslant x <
  1}\theta_L(x) = \theta_L(0) = 2\), we conclude
\begin{equation}
n\lg n \leqslant 2 \cdot U_n < n\lg n + 2n - 2.
\label{ineq:Un}
\end{equation}
The lower bound is tight if \(n=2^p\).

Let \(T_n := \sum_{j=1}^{n}{2^{\rho_{j}}}\). We thus have \(T_n = U_n
+ 2^{\rho_n}\) and reusing the bounds~\eqref{ineq:Un} yields \(n\lg n
+ 2 \leqslant 2 \cdot T_n < n\lg n + 4n - 2\), since \(1 \leqslant
2^{\rho_n} \leqslant n\). But the upper bound is far too bad, because
\(n=2^{\rho_n}\) when the \emph{lower} bound on~\(T_n\) is
tight. Therefore, we should apply to~\(T_n\) the same technique we
used to bound~\(U_n\). The recurrences on the ruler
function\index{ruler function} \eqref{eq:ruler} help us in finding a
recurrence for~\(T_n\) as follows:
\begin{align*}
T_{2p} &= \sum_{k=0}^{\smash[t]{p-1}}{2^{\rho_{2k+1}}} +
\sum_{k=1}^{p}{2^{\rho_{2k}}} = p + 2 \cdot T_{p},\\
 T_{2p+1}
&= \sum_{j=1}^{\smash[t]{2p+1}}{2^{\rho_{j}}} = 1 + T_{2p} = (p + 1) +
2 \cdot T_{p}.
\end{align*}
Equivalently, \(T_{n} = 2 \cdot T_{\floor{n/2}} + \ceiling{n/2} = 2
\cdot T_{\floor{n/2}} + n - \floor{n/2}\). Therefore, unravelling a
few terms of the recurrence quickly reveals the equation
\begin{equation*}
2 \cdot T_n = 2n + \sum_{j=1}^{\floor{\lg n}}
            {\left\lfloor{\frac{n}{2^j}}\right\rfloor 2^j}.
\end{equation*}
if we mind the following lemma.
\begin{thm}[Floors and Fractions]
\label{thm:floors}
\textsl{Let \(x\)~be a real number and \(q\)~a natural number. Then
  \(\floor{\floor{x}/q} = \floor{x/q}\).}
\end{thm}
\begin{proof}
  The equality is equivalent to the conjunction of the two
  complementary inequalities \(\floor{\floor{x}/q} \leqslant
  \floor{x/q}\) and \(\floor{x/q} \leqslant \floor{\floor{x}/q}\). The
  former is straightforward because it is a consequence of \(\floor{x}
  \leqslant x\). In the latter, because both sides of the inequality
  are integers, \(\floor{x/q} \leqslant \floor{\floor{x}/q}\) is
  equivalent to state that \(p \leqslant \floor{x/q} \Rightarrow p
  \leqslant \floor{\floor{x}/q}\), for any integer~\(p\). An obvious
  lemma is that if \(i\)~is an integer and \(y\)~a real number, \(i
  \leqslant \floor{y} \Leftrightarrow i \leqslant y\), so the original
  inequality is equivalent to \(p \leqslant x/q \Rightarrow p
  \leqslant \floor{x}/q\), which is trivially equivalent to \(pq
  \leqslant x \Rightarrow pq \leqslant \floor{x}\). Since \(pq\)~is an
  integer, this implication is true from the same lemma.
\end{proof}
\noindent By definition, \(\{x\} := x - \floor{x}\), thus
\begin{equation*}
2 \cdot T_n = n\floor{\lg n} + 2n - \sum_{j=1}^{\floor{\lg
    n}}\left\lbrace\frac{n}{2^j}\right\rbrace 2^j.
\end{equation*}
Using \(0 \leqslant \{x\} < 1\), we obtain the bounds
\begin{equation*}
n\floor{\lg n} + 2n - 2^{\floor{\lg n}+1} + 2 < 2 \cdot T_n \leqslant
n\floor{\lg n} + 2n.
\end{equation*}
Furthermore, \(x - 1 < \floor{x} \leqslant x\) and \(\floor{x} = x -
\{x\}\), therefore
\begin{align*}
n(\lg n - \{\lg n\}) + 2n - 2^{\lg n - \{\lg n\} +
  1} + 2 < 2 \cdot T_n &\leqslant n\lg n + 2n,\\
n\lg n + 2n + 2 - n \cdot \theta_L(\{\lg n\}) < 2 \cdot T_n 
&\leqslant n\lg n + 2n,
\end{align*}
with \(\theta_L(x) := x + 2^{1 - x}\). Since \(\max_{0 \leqslant x <
  1}\theta_L(x) = \theta_L(0) = 2\), we conclude:
\begin{equation}
n\lg n + 2 < 2 \cdot T_n \leqslant n\lg n + 2n.
\label{ineq:Tn}
\end{equation}
The upper bound is tight if \(n=2^p\).

We can now proceed to bound
\(\OW{\fun{bms}}{2p}\)\index{bms@$\OW{\fun{bms}}{n}$|(} and
\(\OW{\fun{bms}}{2p+1}\) from equations~\eqref{eq:OWbms_2p}
and~\eqref{eq:OWbms_2p_1}, but we shall only care for an upper
bound. First, we have
\begin{align*}
\OW{\fun{bms}}{2p+1} = 2 \cdot T_p + 2 \cdot \OB{\fun{bms}}{p} +
\nu_p,
\quad
\OW{\fun{bms}}{2p} = 2 \cdot U_p + 2 \cdot \OB{\fun{bms}}{p} + 1.
\end{align*}
Next, we appeal to~\eqref{ineq:Tn}, above,
and~\eqref{ineq:bounds_Btms}, \vpageref{ineq:bounds_Btms}:
\begin{equation*}
\OW{\fun{bms}}{2p+1}
  \leqslant (p\lg p + 2p) + p\lg p + \nu_p
   \leqslant (2p+1)\lg p + 2p + 1.
\end{equation*}
Setting \(n=2p+1\) entails \(\OW{\fun{bms}}{n} \leqslant
n\lg(\tfrac{1}{2}(n-1)) + n = n\lg(n-1)\). On the other hand,
recalling inequalities~\eqref{ineq:Un} and~\eqref{ineq:bounds_Btms}
brings us
\begin{equation*}
\OW{\fun{bms}}{2p} < 1 + (p\lg p + 2p - 2) + (p\lg p)
             = 2p\lg p + 2p - 1,
\end{equation*}
which is \(\OW{\fun{bms}}{n} < n\lg n - 1\), if we set \(n = 2p\).

Retaining the maximum of the upper bounds for
\(\OW{\fun{tms}}{2p}\) and \(\OW{\fun{tms}}{2p+1}\), we can bound
\(\OW{\fun{bms}}{n}\) for any~\(n > 2\):
