As mentioned earlier, the \emph{height} of a tree is the number of
nodes on a maximal path from the root to a leaf, that is, a node
without subtrees; for example, we can follow down and count the
nodes~\((\circ)\) in \fig~\ref{fig:catalan_tree}. A tree reduced to a
single leaf has height~\(0\).

We begin with the key observation that a Catalan tree with \(n\)~edges
and height~\(h\) is in bijection with a Dyck path of length~\(2n\)
\emph{and height \(h-1\)} (see \fig~\vref{fig:catalan_tree} and
\fig~\ref{fig:bijection}). This simple fact allows us to reckon on the
height of the Dyck paths and transfer the result back to Catalan
trees.

Let~\(h_n\) be the average height of Catalan trees with \(n\)~edges
and \(H_{n,h}\) the number of Catalan trees with \(n\)~edges and
height~\(h\). We then have \(h_n = S_n/C_{n}\), where \(S_n := \sum_{h
  \geqslant 1} h \cdot H_{n,h}\). We need to gain a purchase on that
sum. For example, we might define~\(A_{n,h}\) as being the number of
trees with \(n\)~edges and height less than or equal to~\(h\). Of
course, \(A_{n,h} = A_{n,n+1} = C_{n}\), if \(h > n\). Then \(H_{n,h}
= A_{n,h}-A_{n,h-1}\). Formulas can be further simplified by
letting~\(B_{n,h}\) be the number of trees with \(n\)~edges with
height greater than~\(h\):
\begin{equation}
S_n = \sum_{h \geqslant 1}h(A_{n,h}-A_{n,h-1})
    = \sum_{h \geqslant 1}h(B_{n,h-1}-B_{n,h}) = \sum_{h\geqslant 0} B_{n,h}.
\label{eq:Sn}
\end{equation}

With the determination of~\(A_{n,h}\) in mind, let us consider in
\fig~\ref{fig:height} a Dyck path of length~\(2n\) and height~\(h-1\).
\begin{figure}[t]
\centering
\subfloat[Dyck path of length~\(2n\) and
height \(h-1\) \label{fig:height}]{
\includegraphics[scale=0.9,bb=71 565 247 721]{height}}
\quad
\subfloat[Path from \(A\) to \(\Omega\) avoiding \(y=x+s\) and
\(y=x-t\)\label{fig:mohanty}]{
\includegraphics[scale=0.9,bb=71 621 219 721]{mohanty}}
\caption{Paths avoiding diagonal boundaries\label{fig:boundaries}}
\end{figure}
The double lines are boundaries which cannot be attained by the
path. This is in fact a special case of a general monotonic path
between two diagonal boundaries, as shown in \fig~\ref{fig:mohanty},
where \(s\)~denotes the vertical distance from~\(A\), and \(t\)~the
horizontal distance from~\(A\). It is well known that the number of
monotonic paths from~\(A(0,0)\) to~\(\Omega(a,b)\) avoiding the
boundaries is
\begin{equation}
\left\lvert\mathcal{L}(a,b;t,s)\right\rvert = \sum_{k \in \mathbb{Z}}\left[\binom{a+b}{b+k(t+s)} - \binom{a+b}{b+k(t+s)+t}\right].
\label{eq:mohanty}
\end{equation}

The proof by \citet[p.~6]{Mohanty_1979} of this formula is based on
the reflection principle and the principle of inclusion and exclusion,
whereby a superset is taken and a subset is retracted, because both
sets are easier to enumerate than the whole --~we used it implicitly
to find formula~\eqref{eq:Cn}. We quote his proof here verbatim
because it is not often found in print nowadays.
\begin{proof}
  (\textbf{Mohanty}) For brevity, call the boundaries \(x=y+t\) and
  \(x=y-s\), \(\mathcal{L}^{+}\) and \(\mathcal{L}^{-}\),
  respectively. Denote by~\(A_1\) the set of paths that reach
  \(\mathcal{L}^{+}\), by~\(A_2\) the set of paths that reach
  \(\mathcal{L}^{+}\), \(\mathcal{L}^{-}\) in that order, and in
  general by~\(A_i\) the set of paths reaching \(\mathcal{L}^{+}\),
  \(\mathcal{L}^{-}\), \(\mathcal{L}^{+}\), \ldots (\(i\)~times) in
  the specified order. Similarly, let~\(B_i\) be the set of paths
  reaching \(\mathcal{L}^{-}\), \(\mathcal{L}^{+}\),
  \(\mathcal{L}^{-}\), \ldots (\(i\)~times) in the specified order. An
  application of the usual inclusion\--exclusion method yields
  \begin{equation}
    \left\lvert\mathcal{L}(a,b;t,s)\right\rvert = \binom{a+b}{b} +
    \sum_{i \geqslant 1}(-1)^{i}(\lvert{A_i}\rvert +
    \lvert{B_i}\rvert),\label{eq:L}
  \end{equation}
  where~\(\lvert{A_i}\rvert\) and~\(\lvert{B_i}\rvert\) are evaluated
  by using the reflection principle repeatedly. For example,
  consider~\(A_3\). Since every path in~\(A_3\) must
  reach~\(\mathcal{L}^{+}\), \(A_3\) when reflected about
  \(\mathcal{L}^{+}\) becomes the set of paths from \((t,-t)\) to
  \((a,b)\) each of which reaches~\(\mathcal{L}^{+}\) after reaching
  \(\mathcal{L}^{-}\). Another reflection about \(\mathcal{L}^{-}\)
  would make~\(A_3\) equivalent to the set of paths from
  \((-s-t,s+t)\) to \((a,b)\) that reach~\(\mathcal{L}^{+}\), which in
  turn can be written as \(R(a+s+t,b-s-t; 2s+3t)\). [Note:
    \(R(a,b;t)\) is the set of paths from \((0,0)\) to \((a,b)\)
    reflected about~\(\mathcal{L}^{+}\).] Thus, since
  \(\lvert{R(a,b;t)}\rvert = \binom{a+b}{a-t}\),
\begin{equation*}
\left\lvert{A_3}\right\rvert = \binom{a+b}{a-s-2t},
\end{equation*}
and, more generally,
\begin{equation*}
\left\lvert{A_{2j}}\right\rvert = \binom{a+b}{a+j(s+t)}
\quad\text{and}\quad
\left\lvert{A_{2j+1}}\right\rvert = \binom{a+b}{a-j(s+t)-t}.
\end{equation*}
The expressions for \(\lvert{B_{2j}}\rvert\),
\(\lvert{B_{2j+1}}\rvert\), \(j=0, 1, 2, \dots\), with
\(\lvert{A_0}\rvert\), \(\lvert{B_0}\rvert\) being \(\binom{a+b}{b}\),
are obtained by interchanging \(a\)~with~\(b\) and
\(s\)~with~\(t\). Substitution of these values in~\eqref{eq:L}
yields~\eqref{eq:mohanty} after some simplifications.
\end{proof}

Resuming our argument, if we match the figures in
\fig~\ref{fig:boundaries}, we find \(s=h\), \(t=1\), \(a=b=n\),
hence \(a+b=2n\) and \(b+k(s+t)=n+k(h+1)\),
which we plug into formula~\eqref{eq:mohanty} and change \(h\)~into \(h-1\):
\begin{equation*}
A_{n,h-1} = \sum_{k \in \mathbb{Z}}\left[\binom{2n}{n+kh} -
           \binom{2n}{n+1+kh}\right].
\end{equation*}
After splitting the sum into \(k<0\), \(k=0\) and \(k>0\), then
changing the sign of~\(k\) in the first case, using \(\binom{p}{q} =
\binom{p}{p-q}\) in the second and last, then gathering the remaining
sums ranging over \(k \geqslant 1\), we reach
\begin{align*}
A_{n,h-1}
  &= - \sum_{k \geqslant 1}\left[\binom{2n}{n+1-kh} -
    2\binom{2n}{n-kh} + \binom{2n}{n-1-kh}\right]\\
  &\phantom{=}\; + \binom{2n}{n} - \binom{2n}{n-1}.
\end{align*}
Recognising \(C_n\) as in page~\pageref{eq:Ann}, we settle for
\begin{equation*}
C_n - A_{n,h-1}
  = \sum_{k \geqslant 1}\left[\binom{2n}{n+1-kh} -
    2\binom{2n}{n-kh} + \binom{2n}{n-1-kh}\right].
\end{equation*}
Finally, recalling that \(B_{n,h} = A_{n,n+1} - A_{n,h}\) and \(C_n =
A_{n,n+1}\), we arrive at the formula
\begin{equation}
B_{n,h-1} = \sum_{k \geqslant 1}
            \left[\binom{2n}{n+1-kh} - 2\binom{2n}{n-kh}
            + \binom{2n}{n-1-kh}\right].
\label{eq:Bn}
\end{equation}

\citet*{KnuthdeBruijnRice_1972} published a landmark paper where they
obtain the same result using much more involved mathematics. They
start by modelling the problem with a generating function
\citep{Wilf_1990} which satisfies a recurrent equation whose solution
expresses the generating function in terms of continued fractions of
Fibonacci polynomials. Integration over complex numbers is utilised to
obtain formula~\eqref{eq:Bn}. Alternatively, generating functions can
be used on monotonic lattice paths instead of Catalan trees
\citep[page~64]{Kemp_1984} \citep{FlajoletNebelProdinger_2006}.

\citet*{SedgewickFlajolet_1996} \citep{FlajoletSedgewick_2009} use
analytic combinatorics and real analysis to obtain the asymptotic
approximation of~\(B_{n,h}\). They
write~\cite[p.~260]{SedgewickFlajolet_1996}: ``This analysis is the
hardest nut that we are cracking in this book. It combines techniques
for solving linear recurrences and continued fractions, generating
function expansions, especially by the Lagrange inversion theorem, and
binomial approximations and Euler\--Maclaurin summations.' It is not
possible to go into the details here, but we can sketch how that
asymptotic approximation can be carried out.

Equation~\eqref{eq:Sn} entails \(S_{n} = \sum_{h \geqslant 1}
B_{n,h-1}\), therefore
\begin{equation*}
S_{n} = \sum_{k' \geqslant 1}d(k') \cdot
         \left[\binom{2n}{n+1-k'} - 2\binom{2n}{n-k'}
         + \binom{2n}{n-1-k'}\right],
\end{equation*}
where~\(d(k')\) is the number of positive divisors of~\(k'\), but
complex analysis is needed
\citep{KnuthdeBruijnRice_1972,FlajoletGourdonDumas_1995}. Another way
is to express the binomials in terms of \(\binom{2n}{n-kh}\) as
follows:
\begin{align*}
\binom{2n}{n-m+1} &= \frac{(2n)!}{(n-m+1)!\,(n+m-1)!}\\
                  &= \frac{(2n)!\,(n+m)}{(n-m)!\,(n-m+1)(n+m)!}
                   = \frac{n+m}{n-m+1}\binom{2n}{n-m},\\
\binom{2n}{n-m-1} &= \frac{(2n)!}{(n-m-1)!\,(n+m+1)!}\\
                  &= \frac{(2n)!\,(n-m)}{(n-m)!\,(n+m)!\,(n+m+1)}
                   = \frac{n-m}{n+m+1}\binom{2n}{n-m}.
\end{align*}
Therefore,
\begin{equation*}
\binom{2n}{n-m+1} - 2\binom{2n}{n-m} + \binom{2n}{n-m-1}
= 2 \cdot \frac{2m^2-(n+1)}{(n+1)^2-m^2}\binom{2n}{n-m}.
\end{equation*}
Let \(F_n(m) = (2m^2-n)/(n^2-m^2)\). We have
\begin{equation*}
S_{n} = 2 \cdot \sum_{h \geqslant 1}\sum_{k \geqslant 1} F_{n+1}(kh)
\cdot \binom{2n}{n-kh}.
\end{equation*}
From equation~\eqref{eq:Cn} and \(h_n = S_n/C_n\), we deduce \(h_{n} =
(n+1)S_{n}{\binom{2n}{n}}^{-1}\), hence we must approximate
\((n+1)F_{n+1}(m)\) and \(\binom{2n}{n-m}\binom{2n}{n}^{-1}\). On the
one hand, we have
\begin{equation*}
F_{n+1}(m) \sim \frac{2m^2-n}{n^2} \sim \frac{2m^2-n}{n(n+1)},
\end{equation*}
so \((n+1)F_{n+1}(kh) \sim 2k^2h^2\!/n-1\). On the other hand,
\citet*[4.6, 4.8]{SedgewickFlajolet_1996} show
\begin{equation*}
\binom{2n}{n-m}{\binom{2n}{n}}^{-1} \sim e^{-m^2\!/n}.
\end{equation*}
Assuming that the tails (the implicit error terms) of the two previous
approximations decrease exponentially, we have
\begin{equation*}
h_{n} \sim \sum_{h \geqslant 1}\sum_{k \geqslant 1}
(4k^2h^2\!/n - 2)e^{-k^2h^2\!/n}
= \sum_{h \geqslant 1}H(h/\!\sqrt{n}),
\end{equation*}
with \(H(x) := \sum_{k \geqslant 1}(4k^2x^2-2)e^{-k^2x^2}\). Finally,
\citet*[5.9]{SedgewickFlajolet_1996}, as well as
\citet*[9.6]{GrahamKnuthPatashnik_1994}, use real analysis to conclude
\begin{equation*}
h_{n} \sim \sum_{h \geqslant 1}H(h/\!\sqrt{n})
\sim \sqrt{n} \int_0^{\infty}\!\!H(x) dx \sim \sqrt{\pi n}.
\end{equation*}
The end of this derivation, after \(B_{n,h-1}\) has been obtained, is
difficult and not even fully formal because the error terms in the
bivariate asymptotic approximations should be carefully checked, as
the referenced authors do. Unfortunately, this also means that this
part is unlikely to be simplified further.
As mentioned earlier, the \emph{height} of a tree is the number of
nodes on a maximal path from the root to a leaf, that is, a node
without subtrees; for example, we can follow down and count the
nodes~\((\circ)\) in \fig~\ref{fig:catalan_tree}. A tree reduced to a
single leaf has height~\(0\).

We begin with the key observation that a Catalan tree with \(n\)~edges
and height~\(h\) is in bijection with a Dyck path of length~\(2n\)
\emph{and height \(h-1\)} (see \fig~\vref{fig:catalan_tree} and
\fig~\ref{fig:bijection}). This simple fact allows us to reckon on the
height of the Dyck paths and transfer the result back to Catalan
trees.

Let~\(h_n\) be the average height of Catalan trees with \(n\)~edges
and \(H_{n,h}\) the number of Catalan trees with \(n\)~edges and
height~\(h\). We then have \(h_n = S_n/C_{n}\), where \(S_n := \sum_{h
  \geqslant 1} h \cdot H_{n,h}\). We need to gain a purchase on that
sum. For example, we might define~\(A_{n,h}\) as being the number of
trees with \(n\)~edges and height less than or equal to~\(h\). Of
course, \(A_{n,h} = A_{n,n+1} = C_{n}\), if \(h > n\). Then \(H_{n,h}
= A_{n,h}-A_{n,h-1}\). Formulas can be further simplified by
letting~\(B_{n,h}\) be the number of trees with \(n\)~edges with
height greater than~\(h\):
\begin{equation}
S_n = \sum_{h \geqslant 1}h(A_{n,h}-A_{n,h-1})
    = \sum_{h \geqslant 1}h(B_{n,h-1}-B_{n,h}) = \sum_{h\geqslant 0} B_{n,h}.
\label{eq:Sn}
\end{equation}

With the determination of~\(A_{n,h}\) in mind, let us consider in
\fig~\ref{fig:height} a Dyck path of length~\(2n\) and height~\(h-1\).
\begin{figure}[t]
\centering
\subfloat[Dyck path of length~\(2n\) and
height \(h-1\) \label{fig:height}]{
\includegraphics[scale=0.9,bb=71 565 247 721]{height}}
\quad
\subfloat[Path from \(A\) to \(\Omega\) avoiding \(y=x+s\) and
\(y=x-t\)\label{fig:mohanty}]{
\includegraphics[scale=0.9,bb=71 621 219 721]{mohanty}}
\caption{Paths avoiding diagonal boundaries\label{fig:boundaries}}
\end{figure}
The double lines are boundaries which cannot be attained by the
path. This is in fact a special case of a general monotonic path
between two diagonal boundaries, as shown in \fig~\ref{fig:mohanty},
where \(s\)~denotes the vertical distance from~\(A\), and \(t\)~the
horizontal distance from~\(A\). It is well known that the number of
monotonic paths from~\(A(0,0)\) to~\(\Omega(a,b)\) avoiding the
boundaries is
\begin{equation}
\left\lvert\mathcal{L}(a,b;t,s)\right\rvert = \sum_{k \in \mathbb{Z}}\left[\binom{a+b}{b+k(t+s)} - \binom{a+b}{b+k(t+s)+t}\right].
\label{eq:mohanty}
\end{equation}

The proof by \citet[p.~6]{Mohanty_1979} of this formula is based on
the reflection principle and the principle of inclusion and exclusion,
whereby a superset is taken and a subset is retracted, because both
sets are easier to enumerate than the whole --~we used it implicitly
to find formula~\eqref{eq:Cn}. We quote his proof here verbatim
because it is not often found in print nowadays.
\begin{proof}
  (\textbf{Mohanty}) For brevity, call the boundaries \(x=y+t\) and
  \(x=y-s\), \(\mathcal{L}^{+}\) and \(\mathcal{L}^{-}\),
  respectively. Denote by~\(A_1\) the set of paths that reach
  \(\mathcal{L}^{+}\), by~\(A_2\) the set of paths that reach
  \(\mathcal{L}^{+}\), \(\mathcal{L}^{-}\) in that order, and in
  general by~\(A_i\) the set of paths reaching \(\mathcal{L}^{+}\),
  \(\mathcal{L}^{-}\), \(\mathcal{L}^{+}\), \ldots (\(i\)~times) in
  the specified order. Similarly, let~\(B_i\) be the set of paths
  reaching \(\mathcal{L}^{-}\), \(\mathcal{L}^{+}\),
  \(\mathcal{L}^{-}\), \ldots (\(i\)~times) in the specified order. An
  application of the usual inclusion\--exclusion method yields
  \begin{equation}
    \left\lvert\mathcal{L}(a,b;t,s)\right\rvert = \binom{a+b}{b} +
    \sum_{i \geqslant 1}(-1)^{i}(\lvert{A_i}\rvert +
    \lvert{B_i}\rvert),\label{eq:L}
  \end{equation}
  where~\(\lvert{A_i}\rvert\) and~\(\lvert{B_i}\rvert\) are evaluated
  by using the reflection principle repeatedly. For example,
  consider~\(A_3\). Since every path in~\(A_3\) must
  reach~\(\mathcal{L}^{+}\), \(A_3\) when reflected about
  \(\mathcal{L}^{+}\) becomes the set of paths from \((t,-t)\) to
  \((a,b)\) each of which reaches~\(\mathcal{L}^{+}\) after reaching
  \(\mathcal{L}^{-}\). Another reflection about \(\mathcal{L}^{-}\)
  would make~\(A_3\) equivalent to the set of paths from
  \((-s-t,s+t)\) to \((a,b)\) that reach~\(\mathcal{L}^{+}\), which in
  turn can be written as \(R(a+s+t,b-s-t; 2s+3t)\). [Note:
    \(R(a,b;t)\) is the set of paths from \((0,0)\) to \((a,b)\)
    reflected about~\(\mathcal{L}^{+}\).] Thus, since
  \(\lvert{R(a,b;t)}\rvert = \binom{a+b}{a-t}\),
\begin{equation*}
\left\lvert{A_3}\right\rvert = \binom{a+b}{a-s-2t},
\end{equation*}
and, more generally,
\begin{equation*}
\left\lvert{A_{2j}}\right\rvert = \binom{a+b}{a+j(s+t)}
\quad\text{and}\quad
\left\lvert{A_{2j+1}}\right\rvert = \binom{a+b}{a-j(s+t)-t}.
\end{equation*}
The expressions for \(\lvert{B_{2j}}\rvert\),
\(\lvert{B_{2j+1}}\rvert\), \(j=0, 1, 2, \dots\), with
\(\lvert{A_0}\rvert\), \(\lvert{B_0}\rvert\) being \(\binom{a+b}{b}\),
are obtained by interchanging \(a\)~with~\(b\) and
\(s\)~with~\(t\). Substitution of these values in~\eqref{eq:L}
yields~\eqref{eq:mohanty} after some simplifications.
\end{proof}

Resuming our argument, if we match the figures in
\fig~\ref{fig:boundaries}, we find \(s=h\), \(t=1\), \(a=b=n\),
hence \(a+b=2n\) and \(b+k(s+t)=n+k(h+1)\),
which we plug into formula~\eqref{eq:mohanty} and change \(h\)~into \(h-1\):
\begin{equation*}
A_{n,h-1} = \sum_{k \in \mathbb{Z}}\left[\binom{2n}{n+kh} -
           \binom{2n}{n+1+kh}\right].
\end{equation*}
After splitting the sum into \(k<0\), \(k=0\) and \(k>0\), then
changing the sign of~\(k\) in the first case, using \(\binom{p}{q} =
\binom{p}{p-q}\) in the second and last, then gathering the remaining
sums ranging over \(k \geqslant 1\), we reach
\begin{align*}
A_{n,h-1}
  &= - \sum_{k \geqslant 1}\left[\binom{2n}{n+1-kh} -
    2\binom{2n}{n-kh} + \binom{2n}{n-1-kh}\right]\\
  &\phantom{=}\; + \binom{2n}{n} - \binom{2n}{n-1}.
\end{align*}
Recognising \(C_n\) as in page~\pageref{eq:Ann}, we settle for
\begin{equation*}
C_n - A_{n,h-1}
  = \sum_{k \geqslant 1}\left[\binom{2n}{n+1-kh} -
    2\binom{2n}{n-kh} + \binom{2n}{n-1-kh}\right].
\end{equation*}
Finally, recalling that \(B_{n,h} = A_{n,n+1} - A_{n,h}\) and \(C_n =
A_{n,n+1}\), we arrive at the formula
\begin{equation}
B_{n,h-1} = \sum_{k \geqslant 1}
            \left[\binom{2n}{n+1-kh} - 2\binom{2n}{n-kh}
            + \binom{2n}{n-1-kh}\right].
\label{eq:Bn}
\end{equation}

\citet*{KnuthdeBruijnRice_1972} published a landmark paper where they
obtain the same result using much more involved mathematics. They
start by modelling the problem with a generating function
\citep{Wilf_1990} which satisfies a recurrent equation whose solution
expresses the generating function in terms of continued fractions of
Fibonacci polynomials. Integration over complex numbers is utilised to
obtain formula~\eqref{eq:Bn}. Alternatively, generating functions can
be used on monotonic lattice paths instead of Catalan trees
\citep[page~64]{Kemp_1984} \citep{FlajoletNebelProdinger_2006}.

\citet*{SedgewickFlajolet_1996} \citep{FlajoletSedgewick_2009} use
analytic combinatorics and real analysis to obtain the asymptotic
approximation of~\(B_{n,h}\). They
write~\cite[p.~260]{SedgewickFlajolet_1996}: `This analysis is the
hardest nut that we are cracking in this book. It combines techniques
for solving linear recurrences and continued fractions, generating
function expansions, especially by the Lagrange inversion theorem, and
binomial approximations and Euler\--Maclaurin summations.' It is not
possible to go into the details here, but we can sketch how that
asymptotic approximation can be carried out.

Equation~\eqref{eq:Sn} entails \(S_{n} = \sum_{h \geqslant 1}
B_{n,h-1}\), therefore
\begin{equation*}
S_{n} = \sum_{k' \geqslant 1}d(k') \cdot
         \left[\binom{2n}{n+1-k'} - 2\binom{2n}{n-k'}
         + \binom{2n}{n-1-k'}\right],
\end{equation*}
where~\(d(k')\) is the number of positive divisors of~\(k'\), but
complex analysis is needed
\citep{KnuthdeBruijnRice_1972,FlajoletGourdonDumas_1995}. Another way
is to express the binomials in terms of \(\binom{2n}{n-kh}\) as
follows:
\begin{align*}
\binom{2n}{n-m+1} &= \frac{(2n)!}{(n-m+1)!\,(n+m-1)!}\\
                  &= \frac{(2n)!\,(n+m)}{(n-m)!\,(n-m+1)(n+m)!}
                   = \frac{n+m}{n-m+1}\binom{2n}{n-m},\\
\binom{2n}{n-m-1} &= \frac{(2n)!}{(n-m-1)!\,(n+m+1)!}\\
                  &= \frac{(2n)!\,(n-m)}{(n-m)!\,(n+m)!\,(n+m+1)}
                   = \frac{n-m}{n+m+1}\binom{2n}{n-m}.
\end{align*}
Therefore,
\begin{equation*}
\binom{2n}{n-m+1} - 2\binom{2n}{n-m} + \binom{2n}{n-m-1}
= 2 \cdot \frac{2m^2-(n+1)}{(n+1)^2-m^2}\binom{2n}{n-m}.
\end{equation*}
Let \(F_n(m) = (2m^2-n)/(n^2-m^2)\). We have
\begin{equation*}
S_{n} = 2 \cdot \sum_{h \geqslant 1}\sum_{k \geqslant 1} F_{n+1}(kh)
\cdot \binom{2n}{n-kh}.
\end{equation*}
From equation~\eqref{eq:Cn} and \(h_n = S_n/C_n\), we deduce \(h_{n} =
(n+1)S_{n}{\binom{2n}{n}}^{-1}\), hence we must approximate
\((n+1)F_{n+1}(m)\) and \(\binom{2n}{n-m}\binom{2n}{n}^{-1}\). On the
one hand, we have
\begin{equation*}
F_{n+1}(m) \sim \frac{2m^2-n}{n^2} \sim \frac{2m^2-n}{n(n+1)},
\end{equation*}
so \((n+1)F_{n+1}(kh) \sim 2k^2h^2\!/n-1\). On the other hand,
\citet*[4.6, 4.8]{SedgewickFlajolet_1996} show
\begin{equation*}
\binom{2n}{n-m}{\binom{2n}{n}}^{-1} \sim e^{-m^2\!/n}.
\end{equation*}
Assuming that the tails (the implicit error terms) of the two previous
approximations decrease exponentially, we have
\begin{equation*}
h_{n} \sim \sum_{h \geqslant 1}\sum_{k \geqslant 1}
(4k^2h^2\!/n - 2)e^{-k^2h^2\!/n}
= \sum_{h \geqslant 1}H(h/\!\sqrt{n}),
\end{equation*}
with \(H(x) := \sum_{k \geqslant 1}(4k^2x^2-2)e^{-k^2x^2}\). Finally,
\citet*[5.9]{SedgewickFlajolet_1996}, as well as
\citet*[9.6]{GrahamKnuthPatashnik_1994}, use real analysis to conclude
\begin{equation*}
h_{n} \sim \sum_{h \geqslant 1}H(h/\!\sqrt{n})
\sim \sqrt{n} \int_0^{\infty}\!\!H(x) dx \sim \sqrt{\pi n}.
\end{equation*}
The end of this derivation, after \(B_{n,h-1}\) has been obtained, is
difficult and not even fully formal because the error terms in the
bivariate asymptotic approximations should be carefully checked, as
the referenced authors do. Unfortunately, this also means that this
part is unlikely to be simplified further.
