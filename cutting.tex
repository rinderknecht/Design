\section{Cutting}
\label{sec:cutting}
\index{stack!cutting}

\index{cut@\fun{cut/2}|(} Let us consider the problem of cutting a
stack~\(s\) at the~\(k\)th place. Obviously, the result is a pair of
stacks. More precisely, let \(\pair{t}{u}\) be the value of
\(\fun{cut}(s,k)\), such that \(\fun{cat}(t,u) \twoheadrightarrow
s\)\index{cat@\fun{cat/2}} and \(t\)~contains \(k\)~items, that is to
say, \(\fun{len}(t) \twoheadrightarrow k\). In particular, if~\(k =
0\), then \(t = \el\); invalid inputs lead to unspecified
behaviours. For instance, \(\fun{cut}([4,2],0) \twoheadrightarrow
\pair{\el}{[4,2]}\) and \(\fun{cut}([5,3,6,0,2],3) \twoheadrightarrow
\pair{[5,3,6]}{[0,2]}\), but, for the sake of simplicity, nothing is
said about \(\fun{cut}([0],7)\) and \(\fun{cut}([0],-1)\). We derive
two cases: \(k = 0\) or else the stack is not empty. The former is
easy to guess:
\begin{equation*}
\fun{cut}(s,0)           \rightarrow \pair{\el}{s};\qquad
\fun{cut}(\cons{x}{s},k) \rightarrow \fbcode{CCCCCC}\,.
\end{equation*}
A big\hyp{}step design\index{design!big-step $\sim$} uses some
sub\-structural recursive calls to set the structure of the value in
the right\hyp{}hand side. Because \fun{cut/2} takes two arguments, we
expect a lexicographic order\index{induction!lexicographic order}
(definition~\eqref{def:lexico}, page~\pageref{def:lexico}):
\begin{equation*}
\fun{cut}(s_0,k_0) \succ \fun{cut}(s_1,k_1)
:\Leftrightarrow
\text{\(s_0 \succ s_1\) or (\(s_0 = s_1\) and \(k_0 > k_1\))}.
\end{equation*}
Using the proper subterm order\index{induction!proper subterm order}
(section~\ref{par:well-founded}, page~\pageref{par:well-founded}) on
stacks for (\(\succ\)),
\begin{equation*}
\fun{cut}(\cons{x}{s},k) \succ \fun{cut}(s,j);\quad
\fun{cut}(\cons{x}{s},k) \succ \fun{cut}(\cons{x}{s},j),\;
\text{if \(k > j\)}.
\end{equation*}
In the latter case, we can set \(j=k-1\), but the value of
\(\fun{cut}(s,j)\) needs to be projected into \(\pair{t}{u}\) so
\(x\)~is injected in and yields \(\pair{\cons{x}{t}}{u}\). This can be
achieved with an auxiliary
function~\fun{push/2}\index{push@\fun{push/2}}:
\begin{equation*}
\begin{array}{@{}r@{\;}l@{\;}lr@{\;}l@{\;}l@{}}
\fun{cut}(s,0) & \rightarrow & \pair{\el}{s};
& \fun{push}(x,\pair{t}{u}) & \rightarrow & \pair{\cons{x}{t}}{u}.\\
\fun{cut}(\cons{x}{s},k) & \rightarrow
& \fun{push}(x,\fun{cut}(s,k-1)).
\end{array}
\end{equation*}


\paragraph{Inference systems}
\label{par:infsys}
\index{inference system}

When the value of a recursive call needs to be destructured, it is
convenient to use an extension of our language to avoid creating
auxiliary functions like \fun{push/2}:
\begin{mathpar}
\inferrule*{}{\fun{cut}(s,0) \rightarrow \pair{\el}{s}}
\;\TirName{Nil}
\qquad
\inferrule
  {\fun{cut}(s,k-1)         \twoheadrightarrow \pair{t}{u}}
  {\fun{cut}(\cons{x}{s},k) \twoheadrightarrow \pair{\cons{x}{t}}{u}}
\,\TirName{Pref}
\end{mathpar}
The new construct is called an \emph{inference rule}\index{inference
  system!rule} because it means: `For the value of
\(\fun{cut}(\cons{x}{s},k)\) to be \(\pair{\cons{x}{t}}{u}\), we infer
that the value of \(\fun{cut}(s,k-1)\) must be \(\pair{t}{u}\).' This
interpretation corresponds to an upwards reading of the rule
\TirName{Pref} (\emph{prefix}). Just as we compose horizontally
rewrite rules, we compose inference rules vertically, stacking them,
as in
\begin{mathpar}
\inferrule
  {\inferrule{
     \inferrule{\fun{cut}([0,2],0) \rightarrow \pair{\el}{[0,2]}}
               {\fun{cut}([6,0,2],1) \twoheadrightarrow \pair{[6]}{[0,2]}}}
      {\fun{cut}([3,6,0,2],2) \twoheadrightarrow \pair{[3,6]}{[0,2]}}}
  {\fun{cut}([5,3,6,0,2],3) \twoheadrightarrow \pair{[5,3,6]}{[0,2]}}
\end{mathpar}
When determining the cost of \(\fun{cut}(s,k)\), we take into account
the hidden function \fun{push/2}, so \(\Call{\fun{cut}([5,3,6,0,2],3)}
= 7\). In general, \index{cut@$\C{\fun{cut}}{n}$} \(\C{\fun{cut}}{k} =
2k + 1\).

Beyond simplifying programs, what makes this formalism interesting is
that it enables two kinds of interpretation: logical and
computational. The computational reading, called
\emph{inductive}\index{inference system!inductive reading} in some
contexts, has just been illustrated. The logical understanding
considers inference rules as logical implications of the form \(P_1
\wedge P_2 \wedge \ldots \wedge P_n \Rightarrow C\), written
\begin{equation*}
\inferrule
  {P_1 \\ P_2 \\ \ldots \\ P_n}
  {C}
\end{equation*}
The propositions~\(P_i\) are called \emph{premises}\index{inference
  system!rule!premise} and \(C\)~is the \emph{conclusion}. In the case
of \TirName{Pref}, there is only one premise. When premises are
lacking, as in \TirName{Nil}, then \(C\)~is called an \emph{axiom} and
no horizontal line is drawn. The composition of inference rules is a
\emph{derivation}. In the case of \fun{cut/2}, all derivations are
isomorphic to a stack, whose top is the conclusion.

The logical reading of rule \TirName{Pref} is: `If \(\fun{cut}(s,k-1)
\twoheadrightarrow \pair{t}{u}\), then we have
\(\fun{cut}(\cons{x}{s},k) \twoheadrightarrow
\pair{\cons{x}{t}}{u}\).'  This top\hyp{}down reading qualifies as
\emph{deductive}\index{inference system!deductive reading}. The
previous derivation then can be regarded as the proof of the relation
instance \(\fun{cut}([5,3,6,0,2],3) \twoheadrightarrow
\pair{[5,3,6]}{[0,2]}\).


\paragraph{Induction on proofs}
\index{induction!$\sim$ on proofs}

A single formalism with such a dual interpretation is powerful because
a definition by means of inference rules enables the proof of theorems
about a function by \emph{induction on the structure of the proof}. As
we have done previously, structural induction can be applied to stacks
considered as a data type (objects). Since, in the case of
\fun{cut/2}, derivations are stacks in themselves, so can induction be
applied to their structure (as meta\hyp{}objects).

Let us illustrate this elegant inductive technique with a proof of the
\emph{soundness}\index{soundness} of
\fun{cut/2}\index{cut@\fun{cut/2}}.

\mypar{Soundness}
\label{par:cut_sound}

The concept of soundness or \emph{correctness} \citep{McCarthy_1962,
  Floyd_1967, Hoare_1971,
  Dijkstra_1976}\index{correctness|see{soundness}} is a binary
relationship, so we always ought to speak of the soundness of a
program with respect to its
\emph{specification}\index{specification}. A specification is a
logical description of the expected properties of the output of a
program, given some assumptions on its input. In the case of
\(\fun{cut}(s,k)\), we already mentioned what to expect: the value
must be a pair \(\pair{t}{u}\) such that the concatenation of
\(t\)~and~\(u\) is~\(s\) and the length of~\(t\) is~\(k\).

Formally, let \(\pred{CorCut}{s,k}\)\index{CorCut@\predName{CorCut}|(}
be the proposition
\begin{quote}
  \textsl{If \(\fun{cut}(s,k) \twoheadrightarrow \pair{t}{u}\), then
    \(\fun{cat}(t,u) \twoheadrightarrow s\) and \(\fun{len}(t)
    \twoheadrightarrow k\),}
\end{quote}
where the function \fun{len/1}\index{len@\fun{len/1}|(} is defined in
equation~\eqref{eq:len} \vpageref{eq:len}.

Let us suppose the antecedent of the implication to be true, otherwise
the theorem is vacuously true, so there exists a derivation~\(\Delta\)
whose conclusion is \(\fun{cut}(s,k) \twoheadrightarrow
\pair{t}{u}\). This derivation is a (meta) stack whose top is the
conclusion in question, which makes it possible to reckon by induction
on its structure, that is, we assume that \predName{CorCut} holds for
the immediate sub\-derivation of~\(\Delta\) (the induction hypothesis)
and then proceed to prove that \predName{CorCut} holds for the entire
derivation. This is the immediate subterm induction we use when
reasoning on a stack as an object: we assume the theorem to hold
for~\(s\) and then move to prove it for~\(\cons{x}{s}\).

A case by case analysis on the kind of rule that can end~\(\Delta\)
guides the proof. To avoid clashes between variables in the theorem
and in the inference system, we will overline the latter ones, like
\(\overline{s}\), \(\overline{t}\) etc. which may differ from
\(s\)~and~\(t\) in \predName{CorCut}.
\begin{itemize}

\item \emph{Case where \(\Delta\) ends with \TirName{Nil}.} There are
  no premises, as~\TirName{Nil} is an axiom. In this case, we have to
  establish \predName{CorCut} without induction. The matching of
  \(\fun{cut}(s,k) \twoheadrightarrow \pair{t}{u}\) against
  \(\fun{cut}(\overline{s},0) \rightarrow \pair{\el}{\overline{s}}\)
  yields \(\overline{s} = s\), \(0=k\), \(\el=t\) and
  \(\overline{s}=u\). Therefore, \index{cat@\fun{cat/2}}
  \(\fun{cat}(t,u) = \fun{cat}(\el,s) \xrightarrow{\smash{\alpha}}
  s\), which proves half of the conjunction. Moreover
  \(\fun{len}(t) = \fun{len}(\el) \xrightarrow{\smash{a}} 0\). This is
  consistent with \(k=0\), so \(\pred{CorCut}{s,0}\) is
  true.\index{CorCut@\predName{CorCut}|)}

  \item \emph{Case where \(\Delta\) ends with \TirName{Pref}.} The
    shape of~\(\Delta\) is thus as follows:
    \begin{mathpar}
      \inferrule*[right=Pref]
        {\inferrule*
           {\inferrule*[vdots=1.5em]{}{ }}
           {\fun{cut}(\overline{s},\overline{k}-1)
            \twoheadrightarrow \pair{\overline{t}}{\overline{u}}}
        }
        {\fun{cut}(\cons{\overline{x}}{\overline{s}},\overline{k})
         \twoheadrightarrow
         \pair{\cons{\overline{x}}{\overline{t}}}{\overline{u}}}
    \end{mathpar}
    The matching of \(\fun{cut}(s,k) \twoheadrightarrow \pair{t}{u}\)
    against the conclusion yields \(\cons{\overline{x}}{\overline{s}}
    = s\), \(\overline{k} = k\), \(\cons{\overline{x}}{\overline{t}} =
    t\) and \(\overline{u} = u\). The induction hypothesis in this
    case is that the theorem holds for the sub\-derivation,
    therefore\index{cat@\fun{cat/2}|(} \(\fun{cat}(\overline{t},
    \overline{u}) \twoheadrightarrow \overline{s}\) and
    \(\fun{len}(\overline{t}) \twoheadrightarrow \overline{k} -
    1\).\index{len@\fun{len/1}|)} The induction principle requires
    that we establish now
    \(\fun{cat}(\cons{\overline{x}}{\overline{t}}, \overline{u})
    \twoheadrightarrow \cons{\overline{x}}{\overline{s}}\) and
    \(\fun{len}(\cons{\overline{x}}{\overline{t}}) \twoheadrightarrow
    \overline{k}\). From the definition of \fun{cat/2} and part of the
    hypothesis, we easily deduce
    \(\fun{cat}(\cons{\overline{x}}{\overline{t}}, \overline{u})
    \xrightarrow{\smash{\beta}}
    \cons{\overline{x}}{\fun{cat}(\overline{t}, \overline{u})}
    \twoheadrightarrow \cons{\overline{x}}{\overline{s}}\). Now the
    other part: \(\fun{len}(\cons{\overline{x}}{\overline{t}})
    \xrightarrow{\smash{b}} 1 + \fun{len}(\overline{t})
    \twoheadrightarrow 1 + (\overline{k} - 1) =
    \overline{k}\).\index{cat@\fun{cat/2}|)}\hfill\(\Box\)

\end{itemize}

\paragraph{Exercise}

Write an equivalent definition of \fun{cut/2} in tail
form.\index{functional language!tail form}
\index{cut@\fun{cut/2}|)}
