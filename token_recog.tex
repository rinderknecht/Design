\section{Token recognition}

Until now we showed how to specify tokens. Now we show how to
recognise them, \emph{i.e.,} realise lexical analysis. Let us consider the
following token definition:
\begin{align*}
\term{if} & \rightarrow \text{\exc{if}}\\
\term{then} & \rightarrow \text{\exc{then}}\\
\term{else} & \rightarrow \text{\exc{else}}\\
\term{relop} & \rightarrow \text{\exc{<} \disj \exc{<=} \disj \exc{=}
  \disj \exc{<>} \disj \exc{>} \disj \exc{>=}}\\
\term{digit} & \rightarrow \text{\lb\exc{0}\dash\exc{9}\rb{}}\\
\term{letter} & \rightarrow
\text{\lb\exc{A}\dash\exc{Z}\exc{a}\dash\exc{z}\rb}\\
\term{id} & \rightarrow \text{\term{letter} \lparen\term{letter} \disj
  \term{digit}\rparen\kleene}\\
\term{num} & \rightarrow \text{\term{digit}\plus{} \lparen\exc{.}
\term{digit}\plus\rparen\opt{} \lparen\exc{E} \lparen\exc{+} \disj
\exc{-}\rparen\opt{} \term{digit}\plus\rparen\opt}
\end{align*}

\paragraph{Reserved identifiers and white space}

Keywords are commonly considered as \emph{reserved identifiers},
\emph{i.e.,} in this case, a valid identifier cannot be any token
\term{if}, \term{then} or \term{else}. This is usually not specified,
but, instead, programmed. In addition, let us assume that the lexemes
are separated by white spaces, consisting of non-null sequences of
blanks, tabulations and newline characters. The lexer usually strips
out those white spaces by comparing them to the regular definition
\term{white\_space}:
\begin{align*}
\term{delim} & \rightarrow \text{\term{blank} \disj \term{tab} \disj
    \term{newline}}\\
\term{white\_space} & \rightarrow \text{\term{delim}\plus}
\end{align*}
If a match for \term{white\_space} is found, the lexer does
\emph{not} return a token to the parser. Rather, it proceeds to find
a token following the white space and return it to the parser.

\paragraph{Input buffer}

The stream of characters that provides the input to the lexer comes
usually from a file. For efficiency reasons, when this file is opened,
a \emph{buffer} is associated, so the lexer actually reads its
characters from this buffer in memory. A buffer is like a
\emph{queue}, or \emph{FIFO} (\emph{First in, First out}), that is, a
list whose one end is used to put elements in and whose other end is
used to get elements out, one at a time. The only difference is that a
buffer has a \emph{fixed size} (hence a buffer can be full). An empty
buffer of size three is depicted as follows:
\begin{center}
output side
\(\longleftarrow\)
\begin{tabular}{|@{\,}c@{\,}|@{\,}c@{\,}|@{\,}c@{\,}|}
  \hline
  \phantom{=}
& \phantom{=}
& \phantom{=}\\
  \hline
\end{tabular}
\(\longleftarrow\)
input side
\end{center}
If we input characters \exc{A} then \exc{B} in this buffer, we draw
\begin{center}
\begin{tabular}{rcc@{\,}@{\,}c@{\,}@{\,}ccl}
  \cline{3-5}
  lexer
& \(\longleftarrow\)
& \multicolumn{1}{|@{\,}c@{\,}|}{\phantom{=}}
& \multicolumn{1}{@{\,}c@{\,}|}{\exc{A}}
& \multicolumn{1}{@{\,}c@{\,}|}{\exc{B}}
& \(\longleftarrow\)
& file\\
  \cline{3-5}
&
&
& \multicolumn{1}{@{\,}c@{\,}}{\(\upharpoonright\)}
\end{tabular}
\end{center}
The symbol~\(\upharpoonright\) is a pointer to the next character
available for output. Let keep in mind that the blank character will
now be noted `\texttt{\char`\ }', in order to avoid confusion with an
empty cell in a buffer. So, if we input now a blank in our buffer from
the file, we get the full buffer
\begin{center}
\begin{tabular}{rcc@{\,}@{\,}c@{\,}@{\,}ccl}
  \cline{3-5}
  lexer
& \(\longleftarrow\)
& \multicolumn{1}{|@{\,}c@{\,}|}{\exc{A}}
& \multicolumn{1}{@{\,}c@{\,}|}{\exc{B}}
& \multicolumn{1}{@{\,}c@{\,}|}{\texttt{\char`\ }}
& \(\longleftarrow\)
& file\\
  \cline{3-5}
&
& \multicolumn{1}{@{\,}c@{\,}}{\(\upharpoonright\)}
\end{tabular}
\end{center}
and no more inputs are possible until at least one output is done. Let
us be careful: a buffer is full if and only if~\(\upharpoonright\)
points to the leftmost character. For example,
\begin{center}
\begin{tabular}{rcc@{\,}@{\,}c@{\,}@{\,}ccl}
  \cline{3-5}
  lexer
& \(\longleftarrow\)
& \multicolumn{1}{|@{\,}c@{\,}|}{\exc{A}}
& \multicolumn{1}{@{\,}c@{\,}|}{\exc{B}}
& \multicolumn{1}{@{\,}c@{\,}|}{\texttt{\char`\ }}
& \(\longleftarrow\)
& file\\
  \cline{3-5}
&
&
& \multicolumn{1}{@{\,}c@{\,}}{\(\upharpoonright\)}
\end{tabular}
\end{center}
is \emph{not} a full buffer: there is still room for one character. If
we input~\exc{C}, it becomes
\begin{center}
\begin{tabular}{rcc@{\,}@{\,}c@{\,}@{\,}ccl}
  \cline{3-5}
  lexer
& \(\longleftarrow\)
& \multicolumn{1}{|@{\,}c@{\,}|}{\exc{B}}
& \multicolumn{1}{@{\,}c@{\,}|}{\texttt{\char`\ }}
& \multicolumn{1}{@{\,}c@{\,}|}{\exc{C}}

& \(\longleftarrow\)
& file\\
  \cline{3-5}
&
& \multicolumn{1}{@{\,}c@{\,}}{\(\upharpoonright\)}
\end{tabular}
\end{center}
which is now a full buffer. The overflowing character \exc{A} has been
discarded. Now, if we output a character (or, equivalently, the lexer
inputs a character) we get
\begin{center}
\begin{tabular}{rcc@{\,}@{\,}c@{\,}@{\,}ccl}
  \cline{3-5}
  lexer
& \(\longleftarrow\)
& \multicolumn{1}{|@{\,}c@{\,}|}{\exc{B}}
& \multicolumn{1}{@{\,}c@{\,}|}{\texttt{\char`\ }}
& \multicolumn{1}{@{\,}c@{\,}|}{\exc{C}}

& \(\longleftarrow\)
& file\\
  \cline{3-5}
&
&
& \multicolumn{1}{@{\,}c@{\,}}{\(\upharpoonright\)}
\end{tabular}
\end{center}
Let us output another character:
\begin{center}
\begin{tabular}{rcc@{\,}@{\,}c@{\,}@{\,}ccl}
  \cline{3-5}
  lexer
& \(\longleftarrow\)
& \multicolumn{1}{|@{\,}c@{\,}|}{\exc{B}}
& \multicolumn{1}{@{\,}c@{\,}|}{\texttt{\char`\ }}
& \multicolumn{1}{@{\,}c@{\,}|}{\exc{C}}

& \(\longleftarrow\)
& file\\
  \cline{3-5}
&
&
&
& \multicolumn{1}{@{\,}c@{\,}}{\(\upharpoonright\)}
\end{tabular}
\end{center}
Now, if the lexer needs a character, \exc{C} is output and some
routine automatically reads some more characters from the disk and
fill them in order into the buffer. This happens when we output the
rightmost character. Assuming the next character in the file is
\exc{D}, after outputting \exc{C} we get
\begin{center}
\begin{tabular}{rcc@{\,}@{\,}c@{\,}@{\,}ccl}
  \cline{3-5}
  lexer
& \(\longleftarrow\)
& \multicolumn{1}{|@{\,}c@{\,}|}{\texttt{\char`\ }}
& \multicolumn{1}{@{\,}c@{\,}|}{\exc{C}}
& \multicolumn{1}{@{\,}c@{\,}|}{\exc{D}}
& \(\longleftarrow\)
& file\\
  \cline{3-5}
&
&
&
& \multicolumn{1}{@{\,}c@{\,}}{\(\upharpoonright\)}
\end{tabular}
\end{center}
If the buffer only contains the \emph{end-of-file} character (noted
here \eof), it means that no more characters are available from the
file. So, if we have the situation
\begin{center}
\begin{tabular}{rcc@{\,}@{\,}ccl}
  \cline{3-4}
  lexer
& \(\longleftarrow\)
& \multicolumn{1}{@{\,}c@{\,}|}{\(\cdots\)}
& \multicolumn{1}{@{\,}c@{\,}|}{\eof}
& \(\longleftarrow\)
& empty file\\
  \cline{3-4}
&
&
& \multicolumn{1}{@{\,}c@{\,}}{\(\upharpoonright\)}
\end{tabular}
\end{center}
in which the lexer requests a character, it would get \eof and
subsequent requests would fail, because both the buffer and the file
would be empty.

\mypar{Transition diagrams}

As an intermediary step in the construction of a lexical analyser, we
introduce another concept, called \emph{transition
  diagram}. Transition diagrams depict the actions that take place
when a lexer is called by a parser to get the next token.
\emph{States} in a transition diagram are drawn as circles. Some
states have double circles, with or without an
asterisk~\textsc{*}. States are connected by arrows, called
\emph{edges}, each one carrying an input character as \emph{label}, or
the special label~\other. An example of such transition diagram is
given in \fig~\vref{fig:dfa_geq}.
\begin{figure}[b]
\centering
\includegraphics[bb=60 660 190 730]{dfa_geq}
\caption{A transition diagram}
\label{fig:dfa_geq}
\end{figure}
Double-circled states are called \emph{final states}. The special
arrow which does not connect two states points to the \emph{initial
  state}. A state in the transition diagram corresponds to the state
of the input buffer, \emph{i.e.,} its contents and the output pointer
at a given moment. At the initial state, the buffer contains at least
one character. If the only one remaining character is \eof, the lexer
returns a special token \term{\$} to the parser and stops. Let us
assume that the character~\(c\) is pointed by~\(\upharpoonright\) in
the input buffer and that \(c\)~is not \eof, depicted as follows:
\begin{center}
\begin{tabular}{rcc@{\,}@{\,}c@{\,}@{\,}ccl}
  \cline{3-5}
  lexer
& \(\longleftarrow\)
& \multicolumn{1}{@{\,}c@{\,}|}{\(\cdots\)}
& \multicolumn{1}{@{\,}c@{\,}|}{\(c\)}
& \multicolumn{1}{@{\,}c@{\,}}{\(\cdots\)}
& \(\longleftarrow\)
& file\\
  \cline{3-5}
&
&
& \multicolumn{1}{@{\,}c@{\,}}{\(\upharpoonright\)}
\end{tabular}
\end{center}
When the parser requests a token, if an edge to the state~\(s\) has a
label with character~\(c\), then the current state in the transition
diagram becomes~\(s\), and \(c\)~is removed from the buffer. This is
repeated until a final state is reached or we get stuck. If a final
state is reached, it means the lexer recognised a token --~which is in
turn returned to the parser. Otherwise a lexical error occurred.

Let us consider again the diagram in \fig~\vref{fig:dfa_geq} and let
us assume that the initial input buffer is
\begin{center}
\begin{tabular}{rcc@{\,}@{\,}c@{\,}@{\,}c@{\,}@{\,}c@{\,}@{\,}ccl}
  \cline{3-7}
  lexer
& \(\longleftarrow\)
& \multicolumn{1}{|@{\,}c@{\,}|}{\phantom{=}}
& \multicolumn{1}{@{\,}c@{\,}|}{\exc{>}}
& \multicolumn{1}{@{\,}c@{\,}|}{\exc{=}}
& \multicolumn{1}{@{\,}c@{\,}|}{\texttt{\char`\ }}
& \multicolumn{1}{@{\,}c@{\,}|}{\exc{1}}
& \(\longleftarrow\)
& file\\
  \cline{3-7}
&
&
& \multicolumn{1}{@{\,}c@{\,}}{\(\upharpoonright\)}
\end{tabular}
\end{center}
From the initial state~\(1\) to the state~\(2\) there is an arrow with
the label `\exc{>}'. Because this label is present at the output
position of the buffer, we can change the diagram state to~\(2\) and
remove `\exc{<}' from the buffer, which becomes
\begin{center}
\begin{tabular}{rcc@{\,}@{\,}c@{\,}@{\,}c@{\,}@{\,}c@{\,}@{\,}ccl}
  \cline{3-7}
  lexer
& \(\longleftarrow\)
& \multicolumn{1}{|@{\,}c@{\,}|}{\phantom{=}}
& \multicolumn{1}{@{\,}c@{\,}|}{\exc{>}}
& \multicolumn{1}{@{\,}c@{\,}|}{\exc{=}}
& \multicolumn{1}{@{\,}c@{\,}|}{\texttt{\char`\ }}
& \multicolumn{1}{@{\,}c@{\,}|}{\exc{1}}
& \(\longleftarrow\)
& file\\
  \cline{3-7}
&
&
&
& \multicolumn{1}{@{\,}c@{\,}}{\(\upharpoonright\)}
\end{tabular}
\end{center}
From state~\(2\) to state~\(3\) there is an arrow with label `\exc{=}',
so we remove it:
\begin{center}
\begin{tabular}{rcc@{\,}@{\,}c@{\,}@{\,}c@{\,}@{\,}c@{\,}@{\,}ccl}
  \cline{3-7}
  lexer
& \(\longleftarrow\)
& \multicolumn{1}{|@{\,}c@{\,}|}{\phantom{=}}
& \multicolumn{1}{@{\,}c@{\,}|}{\exc{>}}
& \multicolumn{1}{@{\,}c@{\,}|}{\exc{=}}
& \multicolumn{1}{@{\,}c@{\,}|}{\texttt{\char`\ }}
& \multicolumn{1}{@{\,}c@{\,}|}{\exc{1}}
& \(\longleftarrow\)
& file\\
  \cline{3-7}
&
&
&
&
& \multicolumn{1}{@{\,}c@{\,}}{\(\upharpoonright\)}
\end{tabular}
\end{center}
and we move to state~\(3\). Since state~\(3\) is a final state, we are
done: we recognised the token \token{relop}{>=}. Let us imagine now
the input buffer is
\begin{center}
\begin{tabular}{rcc@{\,}@{\,}c@{\,}@{\,}c@{\,}@{\,}c@{\,}@{\,}ccl}
  \cline{3-7}
  lexer
& \(\longleftarrow\)
& \multicolumn{1}{|@{\,}c@{\,}|}{\phantom{=}}
& \multicolumn{1}{@{\,}c@{\,}|}{\exc{>}}
& \multicolumn{1}{@{\,}c@{\,}|}{\exc{1}}
& \multicolumn{1}{@{\,}c@{\,}|}{\exc{+}}
& \multicolumn{1}{@{\,}c@{\,}|}{\exc{2}}
& \(\longleftarrow\)
& file\\
  \cline{3-7}
&
&
& \multicolumn{1}{@{\,}c@{\,}}{\(\upharpoonright\)}
\end{tabular}
\end{center}
In this case, we will move from the initial state to state~\(2\):
\begin{center}
\begin{tabular}{rcc@{\,}@{\,}c@{\,}@{\,}c@{\,}@{\,}c@{\,}@{\,}ccl}
  \cline{3-7}
  lexer
& \(\longleftarrow\)
& \multicolumn{1}{|@{\,}c@{\,}|}{\phantom{=}}
& \multicolumn{1}{@{\,}c@{\,}|}{\exc{>}}
& \multicolumn{1}{@{\,}c@{\,}|}{\exc{1}}
& \multicolumn{1}{@{\,}c@{\,}|}{\exc{+}}
& \multicolumn{1}{@{\,}c@{\,}|}{\exc{2}}
& \(\longleftarrow\)
& file\\
  \cline{3-7}
&
&
&
& \multicolumn{1}{@{\,}c@{\,}}{\(\upharpoonright\)}
\end{tabular}
\end{center}
We cannot use the edge with label `\exc{=}', but we can use the one
with `\other'. Indeed, the `\other' label refers to any character that
is not indicated by any of the edges leaving the state. Hence, we move
to state~\(4\) and the input buffer becomes
\begin{center}
\begin{tabular}{rcc@{\,}@{\,}c@{\,}@{\,}c@{\,}@{\,}c@{\,}@{\,}ccl}
  \cline{3-7}
  lexer
& \(\longleftarrow\)
& \multicolumn{1}{|@{\,}c@{\,}|}{\phantom{=}}
& \multicolumn{1}{@{\,}c@{\,}|}{\exc{>}}
& \multicolumn{1}{@{\,}c@{\,}|}{\exc{1}}
& \multicolumn{1}{@{\,}c@{\,}|}{\exc{+}}
& \multicolumn{1}{@{\,}c@{\,}|}{\exc{2}}
& \(\longleftarrow\)
& file\\
  \cline{3-7}
&
&
&
&
& \multicolumn{1}{@{\,}c@{\,}}{\(\upharpoonright\)}
\end{tabular}
\end{center}
and the lexer emits the token \token{relop}{>}. But there is a problem
here: if the parser requests another token, we have to start again
with this buffer but we already skipped the character
\begin{tabular}{|@{\,}c@{\,}|}
\hline
\exc{1}\\
\hline
\end{tabular}
and we forgot where the recognised lexeme starts. The idea is to use
another arrow to mark the starting position when we try to recognise a
token. Let~\(\upharpoonleft\) be this new pointer. Then the initial
buffer of our previous example would be depicted as
\begin{center}
\begin{tabular}{rcc@{\,}@{\,}c@{\,}@{\,}c@{\,}@{\,}c@{\,}@{\,}ccl}
  \cline{3-7}
  lexer
& \(\longleftarrow\)
& \multicolumn{1}{|@{\,}c@{\,}|}{\phantom{=}}
& \multicolumn{1}{@{\,}c@{\,}|}{\exc{>}}
& \multicolumn{1}{@{\,}c@{\,}|}{\exc{1}}
& \multicolumn{1}{@{\,}c@{\,}|}{\exc{+}}
& \multicolumn{1}{@{\,}c@{\,}}{\(\cdots\)}
& \(\longleftarrow\)
& file\\
  \cline{3-7}
&
&
& \multicolumn{1}{@{}c@{}}{\(\upharpoonleft \upharpoonright\)}
\end{tabular}
\end{center}
When the lexer reads the next available character, the
pointer~\(\upharpoonright\) is shifted to the right of one position.
\begin{center}
\begin{tabular}{rcc@{\,}@{\,}c@{\,}@{\,}c@{\,}@{\,}c@{\,}@{\,}ccl}
  \cline{3-7}
  lexer
& \(\longleftarrow\)
& \multicolumn{1}{|@{\,}c@{\,}|}{\phantom{=}}
& \multicolumn{1}{@{\,}c@{\,}|}{\exc{>}}
& \multicolumn{1}{@{\,}c@{\,}|}{\exc{1}}
& \multicolumn{1}{@{\,}c@{\,}|}{\exc{+}}
& \multicolumn{1}{@{\,}c@{\,}}{\(\cdots\)}
& \(\longleftarrow\)
& file\\
  \cline{3-7}
&
&
& \multicolumn{1}{@{\,}c@{\,}}{\(\upharpoonleft\)}
& \multicolumn{1}{@{\,}c@{\,}}{\(\upharpoonright\)}
\end{tabular}
\end{center}
We are now at state~\(2\) and the current character, that is, pointed
by~\(\upharpoonright\), is~\exc{1}. The only way to continue is to go
to state~\(4\), using the special label \other. The pointer of the
secondary buffer shifts to the right and, since it points to the last
position, we input one character from the primary buffer:
\begin{center}
\begin{tabular}{rcc@{\,}@{\,}c@{\,}@{\,}c@{\,}@{\,}c@{\,}@{\,}ccl}
  \cline{3-7}
  lexer
& \(\longleftarrow\)
& \multicolumn{1}{|@{\,}c@{\,}|}{\phantom{=}}
& \multicolumn{1}{@{\,}c@{\,}|}{\exc{>}}
& \multicolumn{1}{@{\,}c@{\,}|}{\exc{1}}
& \multicolumn{1}{@{\,}c@{\,}|}{\exc{+}}
& \multicolumn{1}{@{\,}c@{\,}}{\(\cdots\)}
& \(\longleftarrow\)
& file\\
  \cline{3-7}
&
&
& \multicolumn{1}{@{\,}c@{\,}}{\(\upharpoonleft\)}
&
& \multicolumn{1}{@{\,}c@{\,}}{\(\upharpoonright\)}
\end{tabular}
\end{center}
State~\(4\) is a final state a bit special: it is marked with
`\emph{\textsc{*}}'. This means that before emitting the recognised
lexeme we have to shift the current pointer by one position \emph{to
  the left}:
\begin{center}
\begin{tabular}{rcc@{\,}@{\,}c@{\,}@{\,}c@{\,}@{\,}c@{\,}@{\,}ccl}
  \cline{3-7}
  lexer
& \(\longleftarrow\)
& \multicolumn{1}{|@{\,}c@{\,}|}{\phantom{=}}
& \multicolumn{1}{@{\,}c@{\,}|}{\exc{>}}
& \multicolumn{1}{@{\,}c@{\,}|}{\exc{1}}
& \multicolumn{1}{@{\,}c@{\,}|}{\exc{+}}
& \multicolumn{1}{@{\,}c@{\,}}{\(\cdots\)}
& \(\longleftarrow\)
& file\\
  \cline{3-7}
&
&
& \multicolumn{1}{@{\,}c@{\,}}{\(\upharpoonleft\)}
& \multicolumn{1}{@{\,}c@{\,}}{\(\upharpoonright\)}
\end{tabular}
\end{center}
\emph{This allows to recover the character}
\begin{tabular}{|@{\,}c@{\,}|}
\hline
\exc{1}\\
\hline
\end{tabular}
\emph{as current character.} Moreover, the recognised lexeme now
always starts at the pointer~\(\upharpoonleft\) and ends one position
before the pointer~\(\upharpoonright\). So, here, the lexer outputs
the lexeme `\exc{>}'. Actually, we can complete our token
specification by adding some extra information that are useful for the
recognition process, as just described. First, it is convenient for
some tokens, like \tokenName{relop}, not to carry the lexeme verbatim,
but a symbolic name instead, which is independent of the actual size
of the lexeme. For instance, we shall write \itoken{relop}{gt} instead
of \token{relop}{>}. Second, it is useful to write the recognised
token and the lexeme close to the final state in the transition
diagram itself. See \fig~\vref{fig:dfa_geq_completed}.
\begin{figure}[b]
\centering
\includegraphics[bb=55 660 224 730]{dfa_geq_completed}
\caption{Completion of \fig~\vref{fig:dfa_geq}}
\label{fig:dfa_geq_completed}
\end{figure}
Similarly, \fig~\vref{fig:dfa_relop}
\begin{figure}
\centering
\includegraphics[bb=50 600 265 745]{dfa_relop}
\caption{Relational operators}
\label{fig:dfa_relop}
\end{figure}
shows all the relational operators.

\mypar{Identifiers and longest prefix match}

A transition diagram for specifying identifiers is given in \fig~\vref{fig:dfa_ident}.
\begin{figure}
\centering
\includegraphics[bb=48 684 260 755]{dfa_ident}
\caption{Transition diagram for identifiers}
\label{fig:dfa_ident}
\end{figure}
\texttt{lexeme} is a function call which returns the recognised
lexeme, as found in the \texttt{buffer}. The \other label on the last
step to final state force the identifier to be of \emph{maximal
  length}. For instance, given \texttt{counter+1}, the lexer will
recognise \texttt{counter} as identifier and not just
\texttt{count}. This is called \emph{the longest prefix} property.

\mypar{Keywords}

Since keywords are sequences of letters, they are exceptions to the
rule that a sequence of letters and digits starting with a letter is
an identifier. One solution for specifying keywords is to use
dedicated transition diagrams, one for each keyword. For example, the
\term{if} keyword is simply specified in \fig~\vref{fig:dfa_if}.
\begin{figure}[b]
\centering
\includegraphics[bb=47 711 192 730]{dfa_if}
\caption{Transition diagram for \term{if}}
\label{fig:dfa_if}
\end{figure}
If one keyword diagram succeeds, \emph{i.e.,} the lexer reaches a
final state, then the corresponding keyword is transmitted to the
parser; otherwise, another keyword diagram is tried after shifting the
current pointer~\(\upharpoonright\) in the input buffer back to the
starting position, \emph{i.e.,} pointed by~\(\upharpoonleft\).

There is a problem, though. Consider the \OCaml language, where there
are two keywords \term{fun} and \term{function}. If the diagram of
\term{fun} is tried successfully on the input \texttt{function} and
then the diagram for identifiers, the lexer outputs the lexemes
\term{fun} and \token{id}{ction} instead of one keyword
\term{function}. As for identifiers, we want the longest prefix
property to hold for keywords too and this is simply achieved by
\emph{ordering the transition diagrams}. For example, the diagram of
\term{function} must be tried before the one for \term{fun} because
\term{fun} is a prefix of \term{function}. This strategy implies that
the diagram for the identifiers (given in \fig~\ref{fig:dfa_ident} on
page~\pageref{fig:dfa_ident}) must appear \emph{after} the diagrams
for the keywords.

There are still several drawbacks with this technique, though. The
first problem is that if we indeed have the longest prefix property
among keywords, it does not hold with respect to the identifiers. For
instance, \texttt{iff} would lead to the keyword \term{if} and the
identifier \texttt{f}, instead of the (longest and sole) identifier
\texttt{iff}. This can be remedied by forcing the keyword diagram to
recognise a keyword and not an identifier. This is done by failing if
the keyword is followed by a letter or a digit (remember we try the
longest keywords first, otherwise we would miss some keywords --- the
ones which have prefix keywords). The way to specify this is to use a
special label \compl such that \compl \(c\) denotes the set of
characters which are \emph{not} \(c\). Actually, the special label
\other can always be represented using this \compl label because
\other means `not the others labels.' Therefore, the completed
\term{if} transition diagram would be as found in
\fig~\vref{fig:dfa_if_completed}.
\begin{figure}
\centering
\includegraphics[bb=48 711 261 738]{dfa_if_completed}
\caption{Completion of \fig~\vref{fig:dfa_if}}
\label{fig:dfa_if_completed}
\end{figure}
where \term{alpha} (which stands for alpha-numerical') is defined by
the following regular definition:
\begin{equation*}
\term{alpha} \rightarrow \text{\term{letter} \disj \term{digit}}
\end{equation*}
The second problem with this approach is that we have to create a
transition diagram for each keyword and a state for each of their
letters. In real programming languages, this means that we get
hundreds of states only for the keywords. This problem can be avoided
if we change our technique and give up the specification of keywords
with transition diagrams.

\begin{wrapfigure}[8]{r}[0pt]{0pt}
% 8 vertical lines
% right placement
% 0pt of margin overhang
\begin{tabular}{>{\tt}ll}
\toprule
  \multicolumn{2}{c}{Keywords}\\
\midrule
  \multicolumn{1}{c}{Lexeme}
& \multicolumn{1}{c}{Token}\\
\hline \hline
if   & \tokenName{if}\\
then & \tokenName{then}\\
else & \tokenName{else}\\
\bottomrule
\end{tabular}
\end{wrapfigure}
Since keywords are a strict subset of identifiers, let us use only the
identifier diagram but \emph{we change the action at the final state},
\emph{i.e.,} instead of always returning a \tokenName{id} token, we
make some computations first to decide whether it is either a keyword
or an identifier. Let us call \texttt{switch} the function which makes
this decision based on the buffer (equivalently, the current diagram
state) and a \emph{table of keywords}. The specification is shown in \fig~\vref{fig:dfa_id_kwd}.
\begin{figure}
\centering
\includegraphics[bb=48 684 285 755]{dfa_id_kwd}
\caption{Transition diagram for keywords}
\label{fig:dfa_id_kwd}
\end{figure}
The table of keywords is a two-column table whose first column (the
entry) contains the keyword lexemes and the second column the
corresponding token. Let us write the code for \texttt{switch} in
pseudo-language, in \fig~\ref{fig:switch} on
page~\pageref{fig:switch}.
\begin{figure}
\begin{codebox}
\Procname{\(\proc{Switch} (\id{buffer}, \id{keywords})\)}
\zi \(\id{str} \gets \proc{Lexeme} (\id{buffer})\)
\zi \If \(\id{str} \in \mathcal{D} (\id{keywords})\)
\zi \Then \(\proc{Switch} \gets \id{keywords}[\id{str}]\)
\zi \Else \(\proc{Switch} \gets \polytoken{id}{\id{str}}\)
\zi \End
\end{codebox}
\caption{Pseudo-code for function \texttt{switch}}
\label{fig:switch}
\end{figure}
Function names are in uppercase, like \proc{Lexeme}. Writing \(\id{x}
\gets a\) means that we \emph{assign} the value of the
expression~\(a\) to the variable~\id{x}. Then the value of~\id{x} is
the value of~\(a\). The value \(\mathcal{D}(\id{t})\) is the first
column of table~\id{t}. The value \(\id{t}[\id{e}]\) is the value
corresponding to~\id{e} in table~\id{t}. \proc{Switch} is also used as
a special variable whose value becomes the result of the function
\proc{Switch} when it finishes.

\mypar{Numbers}

Let us consider now the numbers as specified by the regular definition
\begin{align*}
\term{num} & \rightarrow \text{\term{digit}\plus{} \lparen\exc{.}
  \term{digit}\plus\rparen\opt{} \lparen\exc{E} \lparen\exc{+} \disj
  \exc{-}\rparen\opt{} \term{digit}\plus\rparen\opt}
\end{align*}
and propose a transition diagram in \fig~\vref{fig:dfa_num} as an
intermediary step to their recognition.
\begin{figure}
\centering
\includegraphics[bb=47 627 320 746]{dfa_num}
\caption{Transition diagram for numbers}
\label{fig:dfa_num}
\end{figure}

\mypar{White spaces}

The only remaining issue concerns white spaces as specified by the
regular definition
\begin{align*}
\term{white\_space} & \rightarrow \text{\term{delim}\plus}
\end{align*}
which is equivalent to the transition diagram in \fig~\vref{fig:dfa_ws}.
\begin{figure}
\centering
\includegraphics[bb=48 711 190 755]{dfa_ws}
\caption{Transition diagram for white spaces}
\label{fig:dfa_ws}
\end{figure}
The specificity of this diagram is that there is no action
associated to the final state: no token is emitted.

\paragraph{A simplification}

There is a simple away to reduce the size of the diagrams used to
specify the tokens while retaining the longest prefix property: allow
to pass through several final states. This way, we can actually also
get rid of the `\emph{\textsc{*}}' marker on final states. Coming back
to the first example in \fig~\vref{fig:dfa_geq_completed}, we would
alternatively make up \fig~\vref{fig:dfa_geq_opt}.
\begin{figure}
\centering
\includegraphics[bb=48 702 220 730]{dfa_geq_opt}
\caption{Alternative to \fig~\vref{fig:dfa_geq_completed}}
\label{fig:dfa_geq_opt}
\end{figure}
But we have to change the recognition process a little bit here in
order to keep the longest prefix match: we do not want to stop at
state~\(2\) if we could recognise `\texttt{>=}'.

The simplified complete version with respect to the one given in
\fig~\vref{fig:dfa_relop} is found in \fig~\vref{fig:dfa_relop_opt}.
\begin{figure}
\centering
\includegraphics[bb=25 600 200 745]{dfa_relop_opt}
\caption{Simplification of \fig~\vref{fig:dfa_relop}}
\label{fig:dfa_relop_opt}
\end{figure}
The transition diagram for specifying identifiers \emph{and} keywords
looks now like \fig~\vref{fig:dfa_id_kwd_opt}.
\begin{figure}
\centering
\includegraphics[bb=48 682 208 756]{dfa_id_kwd_opt}
\caption{Simplification of \fig~\vref{fig:dfa_id_kwd}}
\label{fig:dfa_id_kwd_opt}
\end{figure}
The transition diagram for specifying numbers is simpler now, as seen
in \fig~\vref{fig:dfa_num_opt}.
\begin{figure}
\centering
\includegraphics[bb=48 632 320 758]{dfa_num_opt}
\caption{Simplification of \fig~\vref{fig:dfa_num}}
\label{fig:dfa_num_opt}
\end{figure}

How do we interpret these new transition diagrams, where the final
states may have out\hyp{}going edges (and the initial state have
incoming edges)? For example, let us consider the recognition of a
number:
\begin{center}
\begin{tabular}{rcc@{\,}@{\,}c@{\,}@{\,}c@{\,}@{\,}c@{\,}@{\,}c@{\,}@{\,}c@{\,}@{\,}c@{\,}@{\,}ccl}
  \cline{3-10}
  lexer
& \(\longleftarrow\)
& \multicolumn{1}{|@{\,}c@{\,}|}{\exc{a}}
& \multicolumn{1}{@{\,}c@{\,}|}{\exc{=}}
& \multicolumn{1}{@{\,}c@{\,}|}{\exc{1}}
& \multicolumn{1}{@{\,}c@{\,}|}{\exc{5}}
& \multicolumn{1}{@{\,}c@{\,}|}{\exc{3}}
& \multicolumn{1}{@{\,}c@{\,}|}{\exc{+}}
& \multicolumn{1}{@{\,}c@{\,}|}{\exc{6}}
& \multicolumn{1}{@{\,}c@{\,}}{\(\cdots\)}
& \(\longleftarrow\)
& file\\
  \cline{3-10}
&
&
&
& \multicolumn{1}{@{\,}c@{\,}}{\(\upharpoonleft\upharpoonright\)}
\end{tabular}
\end{center}
As usual, if there is a label of an edge going out of the current
state which matches the current character in the buffer, the
pointer~\(\upharpoonright\) is shifted to the right of one
position. The new feature here is about final states. When the current
state is final
\begin{enumerate*}

  \item the current position in the buffer is pointed to with a new
  pointer~\(\Uparrow\);

  \item if there is an out-going edge which carries a matching
  character, we try to recognise a longer lexeme;
  \begin{enumerate*}

    \item if we fail, \emph{i.e.,} if we cannot go further in the
      diagram and the current state is not final, then we shift back
      the current pointer~\(\upharpoonright\) to the position pointed
      by~\(\Uparrow\);

    \item and return the then-recognised token and lexeme;

  \end{enumerate*}

  \item if not, we return the recognised token and lexeme associated
  to the current final state.

\end{enumerate*}
Following our example of number recognition:
\begin{itemize}

  \item The label \term{digit} matches the current character in the
  buffer, \emph{i.e.,} the one pointed by~\(\upharpoonright\), so we move to
  state~\(2\) and we shift right by one the
  pointer~\(\upharpoonright\).
\begin{center}
\begin{tabular}{rcc@{\,}@{\,}c@{\,}@{\,}c@{\,}@{\,}c@{\,}@{\,}c@{\,}@{\,}c@{\,}@{\,}c@{\,}@{\,}ccl}
  \cline{3-10}
  lexer
& \(\longleftarrow\)
& \multicolumn{1}{|@{\,}c@{\,}|}{\exc{a}}
& \multicolumn{1}{@{\,}c@{\,}|}{\exc{=}}
& \multicolumn{1}{@{\,}c@{\,}|}{\exc{1}}
& \multicolumn{1}{@{\,}c@{\,}|}{\exc{5}}
& \multicolumn{1}{@{\,}c@{\,}|}{\exc{3}}
& \multicolumn{1}{@{\,}c@{\,}|}{\exc{+}}
& \multicolumn{1}{@{\,}c@{\,}|}{\exc{6}}
& \multicolumn{1}{@{\,}c@{\,}}{\(\cdots\)}
& \(\longleftarrow\)
& file\\
  \cline{3-10}
&
&
&
& \multicolumn{1}{@{\,}c@{\,}}{\(\upharpoonleft\)}
& \multicolumn{1}{@{\,}c@{\,}}{\(\upharpoonright\)}
\end{tabular}
\end{center}
  \item The state~\(2\) is final, so we set the pointer~\(\Uparrow\)
  to the current position in the buffer
\begin{center}
\begin{tabular}{rcc@{\,}@{\,}c@{\,}@{\,}c@{\,}@{\,}c@{\,}@{\,}c@{\,}@{\,}c@{\,}@{\,}c@{\,}@{\,}ccl}
  \cline{3-10}
  lexer
& \(\longleftarrow\)
& \multicolumn{1}{|@{\,}c@{\,}|}{\exc{a}}
& \multicolumn{1}{@{\,}c@{\,}|}{\exc{=}}
& \multicolumn{1}{@{\,}c@{\,}|}{\exc{1}}
& \multicolumn{1}{@{\,}c@{\,}|}{\exc{5}}
& \multicolumn{1}{@{\,}c@{\,}|}{\exc{3}}
& \multicolumn{1}{@{\,}c@{\,}|}{\exc{+}}
& \multicolumn{1}{@{\,}c@{\,}|}{\exc{6}}
& \multicolumn{1}{@{\,}c@{\,}}{\(\cdots\)}
& \(\longleftarrow\)
& file\\
  \cline{3-10}
&
&
&
& \multicolumn{1}{@{\,}c@{\,}}{\(\upharpoonleft\)}
& \multicolumn{1}{@{}c@{}}{\(\Uparrow\upharpoonright\)}
\end{tabular}
\end{center}

  \item We shift right by one the current pointer and stay in
  state~\(2\) because the matching edge is a loop (notice that we did
  not stop here).
\begin{center}
\begin{tabular}{rcc@{\,}@{\,}c@{\,}@{\,}c@{\,}@{\,}c@{\,}@{\,}c@{\,}@{\,}c@{\,}@{\,}c@{\,}@{\,}ccl}
  \cline{3-10}
  lexer
& \(\longleftarrow\)
& \multicolumn{1}{|@{\,}c@{\,}|}{\exc{a}}
& \multicolumn{1}{@{\,}c@{\,}|}{\exc{=}}
& \multicolumn{1}{@{\,}c@{\,}|}{\exc{1}}
& \multicolumn{1}{@{\,}c@{\,}|}{\exc{5}}
& \multicolumn{1}{@{\,}c@{\,}|}{\exc{3}}
& \multicolumn{1}{@{\,}c@{\,}|}{\exc{+}}
& \multicolumn{1}{@{\,}c@{\,}|}{\exc{6}}
& \multicolumn{1}{@{\,}c@{\,}}{\(\cdots\)}
& \(\longleftarrow\)
& file\\
  \cline{3-10}
&
&
&
& \multicolumn{1}{@{\,}c@{\,}}{\(\upharpoonleft\)}
& \multicolumn{1}{@{\,}c@{\,}}{\(\Uparrow\)}
& \multicolumn{1}{@{\,}c@{\,}}{\(\upharpoonright\)}
\end{tabular}
\end{center}

  \item The state~\(2\) is final so we set pointer~\(\Uparrow\) to point to
    the current position:
\begin{center}
\begin{tabular}{rcc@{\,}@{\,}c@{\,}@{\,}c@{\,}@{\,}c@{\,}@{\,}c@{\,}@{\,}c@{\,}@{\,}c@{\,}@{\,}ccl}
  \cline{3-10}
  lexer
& \(\longleftarrow\)
& \multicolumn{1}{|@{\,}c@{\,}|}{\exc{a}}
& \multicolumn{1}{@{\,}c@{\,}|}{\exc{=}}
& \multicolumn{1}{@{\,}c@{\,}|}{\exc{1}}
& \multicolumn{1}{@{\,}c@{\,}|}{\exc{5}}
& \multicolumn{1}{@{\,}c@{\,}|}{\exc{3}}
& \multicolumn{1}{@{\,}c@{\,}|}{\exc{+}}
& \multicolumn{1}{@{\,}c@{\,}|}{\exc{6}}
& \multicolumn{1}{@{\,}c@{\,}}{\(\cdots\)}
& \(\longleftarrow\)
& file\\
  \cline{3-10}
&
&
&
& \multicolumn{1}{@{\,}c@{\,}}{\(\upharpoonleft\)}
&
& \multicolumn{1}{@{}c@{}}{\(\Uparrow\upharpoonright\)}
\end{tabular}
\end{center}

  \item The \term{digit} label of the loop matches again the current
  character (here \exc{3}), so we shift right by one the current
  pointer.
\begin{center}
\begin{tabular}{rcc@{\,}@{\,}c@{\,}@{\,}c@{\,}@{\,}c@{\,}@{\,}c@{\,}@{\,}c@{\,}@{\,}c@{\,}@{\,}ccl}
  \cline{3-10}
  lexer
& \(\longleftarrow\)
& \multicolumn{1}{|@{\,}c@{\,}|}{\exc{a}}
& \multicolumn{1}{@{\,}c@{\,}|}{\exc{=}}
& \multicolumn{1}{@{\,}c@{\,}|}{\exc{1}}
& \multicolumn{1}{@{\,}c@{\,}|}{\exc{5}}
& \multicolumn{1}{@{\,}c@{\,}|}{\exc{3}}
& \multicolumn{1}{@{\,}c@{\,}|}{\exc{+}}
& \multicolumn{1}{@{\,}c@{\,}|}{\exc{6}}
& \multicolumn{1}{@{\,}c@{\,}}{\(\cdots\)}
& \(\longleftarrow\)
& file\\
  \cline{3-10}
&
&
&
& \multicolumn{1}{@{\,}c@{\,}}{\(\upharpoonleft\)}
&
& \multicolumn{1}{@{\,}c@{\,}}{\(\Uparrow\)}
& \multicolumn{1}{@{\,}c@{\,}}{\(\upharpoonright\)}
\end{tabular}
\end{center}

  \item Because state~\(2\) is final we set the pointer~\(\Uparrow\)
    to the current pointer~\(\upharpoonright\):
\begin{center}
\begin{tabular}{rcc@{\,}@{\,}c@{\,}@{\,}c@{\,}@{\,}c@{\,}@{\,}c@{\,}@{\,}c@{\,}@{\,}c@{\,}@{\,}ccl}
  \cline{3-10}
  lexer
& \(\longleftarrow\)
& \multicolumn{1}{|@{\,}c@{\,}|}{\exc{a}}
& \multicolumn{1}{@{\,}c@{\,}|}{\exc{=}}
& \multicolumn{1}{@{\,}c@{\,}|}{\exc{1}}
& \multicolumn{1}{@{\,}c@{\,}|}{\exc{5}}
& \multicolumn{1}{@{\,}c@{\,}|}{\exc{3}}
& \multicolumn{1}{@{\,}c@{\,}|}{\exc{+}}
& \multicolumn{1}{@{\,}c@{\,}|}{\exc{6}}
& \multicolumn{1}{@{\,}c@{\,}}{\(\cdots\)}
& \(\longleftarrow\)
& file\\
  \cline{3-10}
&
&
&
& \multicolumn{1}{@{\,}c@{\,}}{\(\upharpoonleft\)}
&
&
& \multicolumn{1}{@{}c@{}}{\(\Uparrow\upharpoonright\)}
\end{tabular}
\end{center}

  \item State~\(2\) is final, so it means that we succeeded in
    recognising the token associated with state~\(2\):
    \token{num}{lexeme(buffer)}, whose lexeme is between
    \(\upharpoonleft\)~included and~\(\upharpoonright\) excluded,
    \emph{i.e.,} \texttt{153}.

\end{itemize}
Let us consider the following initial buffer:
\begin{center}
\begin{tabular}{rcc@{\,}@{\,}c@{\,}@{\,}c@{\,}@{\,}c@{\,}@{\,}c@{\,}@{\,}c@{\,}@{\,}c@{\,}@{\,}ccl}
  \cline{3-10}
  lexer
& \(\longleftarrow\)
& \multicolumn{1}{|@{\,}c@{\,}|}{\exc{a}}
& \multicolumn{1}{@{\,}c@{\,}|}{\exc{=}}
& \multicolumn{1}{@{\,}c@{\,}|}{\exc{1}}
& \multicolumn{1}{@{\,}c@{\,}|}{\exc{5}}
& \multicolumn{1}{@{\,}c@{\,}|}{\exc{.}}
& \multicolumn{1}{@{\,}c@{\,}|}{\exc{+}}
& \multicolumn{1}{@{\,}c@{\,}|}{\exc{6}}
& \multicolumn{1}{@{\,}c@{\,}}{\(\cdots\)}
& \(\longleftarrow\)
& file\\
  \cline{3-10}
&
&
&
& \multicolumn{1}{@{\,}c@{\,}}{\(\upharpoonleft\upharpoonright\)}
\end{tabular}
\end{center}
Character~\texttt{1} is read and we arrive at state~\(2\) with the
following situation:
\begin{center}
\begin{tabular}{rcc@{\,}@{\,}c@{\,}@{\,}c@{\,}@{\,}c@{\,}@{\,}c@{\,}@{\,}c@{\,}@{\,}c@{\,}@{\,}ccl}
  \cline{3-10}
  lexer
& \(\longleftarrow\)
& \multicolumn{1}{|@{\,}c@{\,}|}{\exc{a}}
& \multicolumn{1}{@{\,}c@{\,}|}{\exc{=}}
& \multicolumn{1}{@{\,}c@{\,}|}{\exc{1}}
& \multicolumn{1}{@{\,}c@{\,}|}{\exc{5}}
& \multicolumn{1}{@{\,}c@{\,}|}{\exc{.}}
& \multicolumn{1}{@{\,}c@{\,}|}{\exc{+}}
& \multicolumn{1}{@{\,}c@{\,}|}{\exc{6}}
& \multicolumn{1}{@{\,}c@{\,}}{\(\cdots\)}
& \(\longleftarrow\)
& file\\
  \cline{3-10}
&
&
&
& \multicolumn{1}{@{\,}c@{\,}}{\(\upharpoonleft\)}
& \multicolumn{1}{@{}c@{}}{\(\Uparrow\upharpoonright\)}
\end{tabular}
\end{center}
Then \texttt{5}~is read and we arrive again at state~\(2\) but we
encounter a different situation:
\begin{center}
\begin{tabular}{rcc@{\,}@{\,}c@{\,}@{\,}c@{\,}@{\,}c@{\,}@{\,}c@{\,}@{\,}c@{\,}@{\,}c@{\,}@{\,}ccl}
  \cline{3-10}
  lexer
& \(\longleftarrow\)
& \multicolumn{1}{|@{\,}c@{\,}|}{\exc{a}}
& \multicolumn{1}{@{\,}c@{\,}|}{\exc{=}}
& \multicolumn{1}{@{\,}c@{\,}|}{\exc{1}}
& \multicolumn{1}{@{\,}c@{\,}|}{\exc{5}}
& \multicolumn{1}{@{\,}c@{\,}|}{\exc{.}}
& \multicolumn{1}{@{\,}c@{\,}|}{\exc{+}}
& \multicolumn{1}{@{\,}c@{\,}|}{\exc{6}}
& \multicolumn{1}{@{\,}c@{\,}}{\(\cdots\)}
& \(\longleftarrow\)
& file\\
  \cline{3-10}
&
&
&
& \multicolumn{1}{@{\,}c@{\,}}{\(\upharpoonleft\)}
&
& \multicolumn{1}{@{}c@{}}{\(\Uparrow\upharpoonright\)}
\end{tabular}
\end{center}
The label on the edge from state~\(2\) to~\(3\) matches `\exc{.}' so we
move to state~\(3\), shift by one the current pointer in the buffer:
\begin{center}
\begin{tabular}{rcc@{\,}@{\,}c@{\,}@{\,}c@{\,}@{\,}c@{\,}@{\,}c@{\,}@{\,}c@{\,}@{\,}c@{\,}@{\,}ccl}
  \cline{3-10}
  lexer
& \(\longleftarrow\)
& \multicolumn{1}{|@{\,}c@{\,}|}{\exc{a}}
& \multicolumn{1}{@{\,}c@{\,}|}{\exc{=}}
& \multicolumn{1}{@{\,}c@{\,}|}{\exc{1}}
& \multicolumn{1}{@{\,}c@{\,}|}{\exc{5}}
& \multicolumn{1}{@{\,}c@{\,}|}{\exc{.}}
& \multicolumn{1}{@{\,}c@{\,}|}{\exc{+}}
& \multicolumn{1}{@{\,}c@{\,}|}{\exc{6}}
& \multicolumn{1}{@{\,}c@{\,}}{\(\cdots\)}
& \(\longleftarrow\)
& file\\
  \cline{3-10}
&
&
&
& \multicolumn{1}{@{\,}c@{\,}}{\(\upharpoonleft\)}
&
& \multicolumn{1}{@{\,}c@{\,}}{\(\Uparrow\)}
& \multicolumn{1}{@{\,}c@{\,}}{\(\upharpoonright\)}
\end{tabular}
\end{center}
Now we are stuck at state~\(3\). Because this is not a final state, we
should fail, \emph{i.e.,} report a lexical error, but because the
pointer~\(\Uparrow\) has been set (\emph{i.e.,} we met a final state),
we shift the current pointer back to the position of the
pointer~\(\Uparrow\) and return the corresponding lexeme~\texttt{15}:
\begin{center}
\begin{tabular}{rcc@{\,}@{\,}c@{\,}@{\,}c@{\,}@{\,}c@{\,}@{\,}c@{\,}@{\,}c@{\,}@{\,}c@{\,}@{\,}ccl}
  \cline{3-10}
  lexer
& \(\longleftarrow\)
& \multicolumn{1}{|@{\,}c@{\,}|}{\exc{a}}
& \multicolumn{1}{@{\,}c@{\,}|}{\exc{=}}
& \multicolumn{1}{@{\,}c@{\,}|}{\exc{1}}
& \multicolumn{1}{@{\,}c@{\,}|}{\exc{5}}
& \multicolumn{1}{@{\,}c@{\,}|}{\exc{.}}
& \multicolumn{1}{@{\,}c@{\,}|}{\exc{+}}
& \multicolumn{1}{@{\,}c@{\,}|}{\exc{6}}
& \multicolumn{1}{@{\,}c@{\,}}{\(\cdots\)}
& \(\longleftarrow\)
& file\\
  \cline{3-10}
&
&
&
& \multicolumn{1}{@{\,}c@{\,}}{\(\upharpoonleft\)}
&
& \multicolumn{1}{@{}c@{}}{\(\Uparrow\upharpoonright\)}
\end{tabular}
\end{center}
