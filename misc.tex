%%-*-latex-*-

% ------------------------------------------------------------------------
%
\begin{slide}
\titre{Subset construction (continued)}

\raggedslides[0pt]

Why is \(2^n\) the number of subsets of a finite set of cardinal
\(n\)?

Let us note \(\dbinom{n}{k}\) the number of \myblue{combinations} of
\(k\) elements taken among \(n\).

For instance, \(\dbinom{5}{3} = 10\) because, assuming \(\{a, b, c, d,
e\}\), we can form the subsets
\[abc, \quad adb, \quad abe, \quad acd, \quad ace, \quad ade, \quad
bcd, \quad bce, \quad bde, \quad cde
\]

In general, the choice of the first element can be made among \(n\),
so \[\dbinom{n}{k} = n \dots\]

\end{slide}

% ------------------------------------------------------------------------
%
\begin{slide}
\titre{Subset construction (continued)}

\raggedslides[0pt]

The second element can be chosen among \(n-1\) remining ones, but the
choice of the second element does not depend on the choice of the
first one, so
\[\dbinom{n}{k} = n \times (n-1) \times \dots\]
Until we choose the \(k\)-th element:
\[\dbinom{n}{k} = n (n-1) \dots (n - k + 1) \dots\]
But since there is no order among the \(k\) elements (it is a set), we
have to not count the repeated \myblue{permutations}.

In the example above, \(abc, bca, cab\) should be counted as one
combination, not three.

\end{slide}

% ------------------------------------------------------------------------
%
\begin{slide}
\titre{Subset construction (continued)}

\raggedslides[0pt]

How many permutations of \(k\) elements are they? The first element
can be chosen among \(k\), the second among the \(k-1\) remaining ones
etc. So the number of permutations of \(k\) elements is \(k (k-1)
\dots 1 = k!\)

So, finally, the number of combinations of \(k\) elements chosen among
\(n\), without order is:
\[
\dbinom{n}{k} = \frac{n (n-1) \dots (n - k + 1)}{k (k-1) \dots 1} =
  \frac{n!}{k! (n-k)!}
\quad \text{where} \; n \geqslant k \geqslant 0
\]
So the number of subsets of a set of size \(n\) is the number of
subsets of size 0 plus the number of subsets of size 1 plus
etc. Formally, it is
\[
\binom{n}{0} + \binom{n}{1} + \dots + \binom{n}{n}=
\sum_{k=0}^{n}{\binom{n}{k}}
\]

\end{slide}





% ------------------------------------------------------------------------
%
\begin{slide}
\titre{Recognition of tokens / Transition diagrams and double-buffering}

\raggedslides[0pt]

One solution is to put back the discarded character into the buffer,
but this is not compatible with a pure queue data structure (the input
side is different from the output side).

So a better solution is to have a second buffer: when a character is
removed from the (primary) input buffer, it is not discarded but put into
this secondary buffer.
\begin{center}
\begin{tabular}{rcc@{\,}@{\,}c@{\,}@{\,}ccc@{\,}@{\,}c@{\,}@{\,}ccl}
  \cline{3-5}\cline{7-9}
  lexer
& \(\longleftarrow\)
& \multicolumn{1}{@{\,}c@{\,}|}{\(\cdots\)}
& \multicolumn{1}{@{\,}c@{\,}|}{\(c\)}
& \multicolumn{1}{@{\,}c@{\,}}{\(\cdots\)}
& \(\longleftarrow\)
& \multicolumn{1}{@{\,}c@{\,}|}{\(\cdots\)}
& \multicolumn{1}{@{\,}c@{\,}|}{\(d\)}
& \multicolumn{1}{@{\,}c@{\,}}{\(\cdots\)}
& \(\longleftarrow\)
& file\\
  \cline{3-5}\cline{7-9}
&
&
& \multicolumn{1}{@{\,}c@{\,}}{\(\upharpoonright\)}
&
&
&
& \multicolumn{1}{@{\,}c@{\,}}{\(\upharpoonright\)}\\
&
& \multicolumn{4}{@{}l@{}}{secondary}
& \multicolumn{4}{@{}l@{}}{primary}\\
&
& \multicolumn{4}{@{}l@{}}{buffer}
& \multicolumn{4}{@{}l@{}}{buffer}\\

\end{tabular}
\end{center}
The lexer now only operates on the secondary buffer (it has no more
access to the primary buffer).


\end{slide}

% ------------------------------------------------------------------------
%
\begin{slide}
\titre{Recognition of tokens / Transition diagrams and double-buffering
(continued)}

\raggedslides[0pt]

So, initially, the current state is \(1\) and the secondary buffer is
not empty.
\begin{center}
\begin{tabular}{rcc@{\,}@{\,}c@{\,}@{\,}c@{\,}@{\,}c@{\,}@{\,}c
c@{\,}c@{\,}@{\,}c@{\,}@{\,}c@{\,}@{\,}c@{\,}@{\,}ccl}
  \cline{3-7}\cline{9-13}
  lexer
& \(\longleftarrow\)
& \multicolumn{1}{|@{\,}c@{\,}|}{\phantom{=}}
& \multicolumn{1}{@{\,}c@{\,}|}{\phantom{=}}
& \multicolumn{1}{@{\,}c@{\,}|}{\phantom{=}}
& \multicolumn{1}{@{\,}c@{\,}|}{\phantom{=}}
& \multicolumn{1}{@{\,}c@{\,}|}{\exc{>}}
& \(\longleftarrow\)
& \multicolumn{1}{|@{\,}c@{\,}|}{\phantom{=}}
& \multicolumn{1}{@{\,}c@{\,}|}{\phantom{=}}
& \multicolumn{1}{@{\,}c@{\,}|}{\exc{1}}
& \multicolumn{1}{@{\,}c@{\,}|}{\exc{+}}
& \multicolumn{1}{@{\,}c@{\,}|}{\exc{2}}
& \(\longleftarrow\)
& file\\
  \cline{3-7}\cline{9-13}
&
&
&
&
&
& \multicolumn{1}{@{\,}c@{\,}}{\(\upharpoonright\)}
&
&
&
& \multicolumn{1}{@{\,}c@{\,}}{\(\upharpoonright\)}
\end{tabular}
\end{center}
When the lexer reads the next available character, it is \textbf{not}
immediately removed from the secondary buffer: instead, the
pointer \(\upharpoonright\) is shifted to the right of one position.

If the pointer was already at the last position, then some characters
from the primary buffer are input into the secondary buffer.

If the primary buffer becomes empty, some characters are read from the
file, until \eof is found (and read).

\end{slide}


% ------------------------------------------------------------------------
%
\begin{slide}
\titre{Recognition of tokens / Transition diagrams and double-buffering
(continued)}

There are two special features in the secondary buffer, compared to the
primary buffer:
\begin{enumerate}

  \item we use a new pointer, noted \(\upharpoonleft\), to mark the
  current character when the current state is initial;

  \item if there is no position to the right, some characters are
  input in the secondary buffer from the primary one (at least one).

\end{enumerate}
So, our example show be now depicted as
\begin{center}
\begin{tabular}{rcc@{\,}@{\,}c@{\,}@{\,}c@{\,}@{\,}c@{\,}@{\,}c
c@{\,}c@{\,}@{\,}c@{\,}@{\,}c@{\,}@{\,}c@{\,}@{\,}ccl}
  \cline{3-7}\cline{9-13}
  lexer
& \(\longleftarrow\)
& \multicolumn{1}{|@{\,}c@{\,}|}{\phantom{=}}
& \multicolumn{1}{@{\,}c@{\,}|}{\phantom{=}}
& \multicolumn{1}{@{\,}c@{\,}|}{\phantom{=}}
& \multicolumn{1}{@{\,}c@{\,}|}{\phantom{=}}
& \multicolumn{1}{@{\,}c@{\,}|}{\exc{>}}
& \(\longleftarrow\)
& \multicolumn{1}{|@{\,}c@{\,}|}{\phantom{=}}
& \multicolumn{1}{@{\,}c@{\,}|}{\phantom{=}}
& \multicolumn{1}{@{\,}c@{\,}|}{\exc{1}}
& \multicolumn{1}{@{\,}c@{\,}|}{\exc{+}}
& \multicolumn{1}{@{\,}c@{\,}|}{\exc{2}}
& \(\longleftarrow\)
& file\\
  \cline{3-7}\cline{9-13}
&
&
&
&
&
& \multicolumn{1}{@{\,}c@{\,}}{\(\upharpoonleft \upharpoonright\)}
&
&
&
& \multicolumn{1}{@{\,}c@{\,}}{\(\upharpoonright\)}
\end{tabular}
\end{center}

\end{slide}

% ------------------------------------------------------------------------
%
\begin{slide}
\titre{Recognition of tokens / Transition diagrams and double-buffering
(continued)}

\raggedslides[0pt]

The character \exc{>} in the buffer is compared to the labels of the
out-going edges of the current state, i.e. the initial state. Such
label exists on an edge to state \(2\), hence the current state
becomes state \(2\).

In the buffer, the current pointer \(\upharpoonright\) is shifted to
the right. But is is already pointing to the last position, so we must
input some characters from the primary buffer.

Let us assume we input one character at a time. We get
\begin{center}
\begin{tabular}{rcc@{\,}@{\,}c@{\,}@{\,}c@{\,}@{\,}c@{\,}@{\,}c
c@{\,}c@{\,}@{\,}c@{\,}@{\,}c@{\,}@{\,}c@{\,}@{\,}ccl}
  \cline{3-7}\cline{9-13}
  lexer
& \(\longleftarrow\)
& \multicolumn{1}{|@{\,}c@{\,}|}{\phantom{=}}
& \multicolumn{1}{@{\,}c@{\,}|}{\phantom{=}}
& \multicolumn{1}{@{\,}c@{\,}|}{\phantom{=}}
& \multicolumn{1}{@{\,}c@{\,}|}{\exc{>}}
& \multicolumn{1}{@{\,}c@{\,}|}{\exc{1}}
& \(\longleftarrow\)
& \multicolumn{1}{|@{\,}c@{\,}|}{\phantom{=}}
& \multicolumn{1}{@{\,}c@{\,}|}{\phantom{=}}
& \multicolumn{1}{@{\,}c@{\,}|}{\phantom{=}}
& \multicolumn{1}{@{\,}c@{\,}|}{\exc{+}}
& \multicolumn{1}{@{\,}c@{\,}|}{\exc{2}}
& \(\longleftarrow\)
& file\\
  \cline{3-7}\cline{9-13}
&
&
&
&
& \multicolumn{1}{@{\,}c@{\,}}{\(\upharpoonleft\)}
& \multicolumn{1}{@{\,}c@{\,}}{\(\upharpoonright\)}
&
&
&
&
& \multicolumn{1}{@{\,}c@{\,}}{\(\upharpoonright\)}
\end{tabular}
\end{center}

\end{slide}

% ------------------------------------------------------------------------
%
\begin{slide}
\titre{Recognition of tokens / Transition diagrams and double-buffering
  (continued)}

\raggedslides[0pt]

We are now at state \(2\) and the next character is \exc{1}.

The only way to continue is to go to state \(4\), using the special
label \other.

We shift the pointer of the secondary buffer to the right and,
since it points to the last position, we input one character from the
primary buffer:
\begin{center}
\begin{tabular}{rcc@{\,}@{\,}c@{\,}@{\,}c@{\,}@{\,}c@{\,}@{\,}c
c@{\,}c@{\,}@{\,}c@{\,}@{\,}c@{\,}@{\,}c@{\,}@{\,}ccl}
  \cline{3-7}\cline{9-13}
  lexer
& \(\longleftarrow\)
& \multicolumn{1}{|@{\,}c@{\,}|}{\phantom{=}}
& \multicolumn{1}{@{\,}c@{\,}|}{\phantom{=}}
& \multicolumn{1}{@{\,}c@{\,}|}{\exc{>}}
& \multicolumn{1}{@{\,}c@{\,}|}{\exc{1}}
& \multicolumn{1}{@{\,}c@{\,}|}{\exc{+}}
& \(\longleftarrow\)
& \multicolumn{1}{|@{\,}c@{\,}|}{\phantom{=}}
& \multicolumn{1}{@{\,}c@{\,}|}{\phantom{=}}
& \multicolumn{1}{@{\,}c@{\,}|}{\phantom{=}}
& \multicolumn{1}{@{\,}c@{\,}|}{\phantom{=}}
& \multicolumn{1}{@{\,}c@{\,}|}{\exc{2}}
& \(\longleftarrow\)
& file\\
  \cline{3-7}\cline{9-13}
&
&
&
& \multicolumn{1}{@{\,}c@{\,}}{\(\upharpoonleft\)}
&
& \multicolumn{1}{@{\,}c@{\,}}{\(\upharpoonright\)}
&
&
&
&
&
& \multicolumn{1}{@{\,}c@{\,}}{\(\upharpoonright\)}
\end{tabular}
\end{center}

\end{slide}


% ------------------------------------------------------------------------
%
\begin{slide}
\titre{Recognition of tokens / Transition diagrams and double-buffering
(continued)}

\raggedslides[0pt]

State \(4\) is a final state a bit special: it is
marked with \emph{\textsc{*}}. This means that before emitting the
recognised lexeme we have to shift the current pointer by one position
\emph{to the left}:
\begin{center}
\begin{tabular}{rcc@{\,}@{\,}c@{\,}@{\,}c@{\,}@{\,}c@{\,}@{\,}c
c@{\,}c@{\,}@{\,}c@{\,}@{\,}c@{\,}@{\,}c@{\,}@{\,}ccl}
  \cline{3-7}\cline{9-13}
  lexer
& \(\longleftarrow\)
& \multicolumn{1}{|@{\,}c@{\,}|}{\phantom{=}}
& \multicolumn{1}{@{\,}c@{\,}|}{\phantom{=}}
& \multicolumn{1}{@{\,}c@{\,}|}{\exc{>}}
& \multicolumn{1}{@{\,}c@{\,}|}{\exc{1}}
& \multicolumn{1}{@{\,}c@{\,}|}{\exc{+}}
& \(\longleftarrow\)
& \multicolumn{1}{|@{\,}c@{\,}|}{\phantom{=}}
& \multicolumn{1}{@{\,}c@{\,}|}{\phantom{=}}
& \multicolumn{1}{@{\,}c@{\,}|}{\phantom{=}}
& \multicolumn{1}{@{\,}c@{\,}|}{\phantom{=}}
& \multicolumn{1}{@{\,}c@{\,}|}{\exc{2}}
& \(\longleftarrow\)
& file\\
  \cline{3-7}\cline{9-13}
&
&
&
& \multicolumn{1}{@{\,}c@{\,}}{\(\upharpoonleft\)}
& \multicolumn{1}{@{\,}c@{\,}}{\(\upharpoonright\)}
&
&
&
&
&
&
& \multicolumn{1}{@{\,}c@{\,}}{\(\upharpoonright\)}
\end{tabular}
\end{center}
\emph{This allows to recover the}
\begin{tabular}{|@{\,}c@{\,}|}
\hline
\exc{1}\\
\hline
\end{tabular}
\emph{character as next available character.}

The recognised lexeme always starts at the \(\upharpoonleft\) pointer
and ends one position before the \(\upharpoonright\). So, here, we
get \exc{>}.

We do not have to remove the recognised lexeme from the secondary
buffer: if a character (here \texttt{>}) reaches the leftmost position
of the buffer, it will be automatically discarded at the following
input.

\end{slide}

% ------------------------------------------------------------------------
%
\begin{slide}
\titre{Recognition of tokens / Transition diagrams and double-buffering
(continued)}

\raggedslides[0pt]

As a simplification, we can now consider that the lexer is connected
to only one buffer which behaves as the secondary buffer for outputs
and as the primary buffer for inputs and write
\begin{center}
\begin{tabular}{rcc@{\,}@{\,}c@{\,}@{\,}c@{\,}@{\,}c@{\,}@{\,}ccl}
  \cline{3-7}
  lexer
& \(\longleftarrow\)
& \multicolumn{1}{|@{\,}c@{\,}|}{\phantom{=}}
& \multicolumn{1}{@{\,}c@{\,}|}{\exc{>}}
& \multicolumn{1}{@{\,}c@{\,}|}{\exc{1}}
& \multicolumn{1}{@{\,}c@{\,}|}{\exc{+}}
& \multicolumn{1}{@{\,}c@{\,}}{\(\cdots\)}
& \(\longleftarrow\)
& file\\
  \cline{3-7}
&
&
& \multicolumn{1}{@{\,}c@{\,}}{\(\upharpoonleft \upharpoonright\)}
\end{tabular}
\end{center}

\end{slide}


The source language is often more abstract (e.g. the Ada programming
language) than the target language (e.g. assembly or machine
language).

Compilers are sometimes called \myblue{translators} when they
map programs to programs whose language is of the same level of
abstraction.

Usually, a language is considered more \myblue{abstract} than another
if its underlying concepts (or `its paradigms') include the other
ones or if they are `more distant' from the underlying machine model
(including control stacks, registers, memory etc.).

So this notion of abstraction is relative.
