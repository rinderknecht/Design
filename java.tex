\chapter{Translation to \Java}
\index{Java@\Java!translation to $\sim$|(}

In this chapter we show how to translate simple functional programs to
\Java, illustrating what may be called a functional style in an
object\hyp{}oriented language. We will make no use of
side\hyp{}effects, so all variables are assigned only once. Objects
programmed in this manner are sometimes called \emph{functional
  objects}. The difficulties we face have to do with the type system
of \Java, since our functional language is untyped. As a consequence,
some programs cannot be translated and others require intermediary
translations, called \emph{refinements}, before the equivalent \Java
program is produced.

In the introduction, \vpageref{par:java}, we laid out the design
pattern for modelling stacks: an abstract class \texttt{Stack<Item>}
parameterised over the type of its contents \texttt{Item} and two
extensions, \texttt{EStack<Item>} for empty stacks and
\texttt{NStack<Item>} for non\hyp{}empty stacks. It is important to
understand the rationale behind this pattern, because it is perhaps
more likely to find an imperative implementation in the following
lines:
\begin{alltt}
\public \class Stack<Item> \{\hfill// Stack.java
  \private Item head;
  \private Stack<Item> tail;

  \public Stack() \{ head = \nullX; tail = \nullX; \}

  \public \booleanX empty() \{ \return head == \nullX; \}

  \public Item pop() \throws EmptyStack \{
    \ifX (empty()) \throw \new EmptyStack();
    \final Item orig = head;
    \ifX (tail.empty()) head = \nullX; \elseX head = tail.pop();
    \return orig; \}

  \public \void push(\final Item item) \{
    Stack<Item> next = \new Stack<Item>();
    next.head = head;
    next.tail = tail;
    head = item;
    tail = next; \}
\}

// EmptyStack.java
\public \class EmptyStack \extends Exception \{\}
\end{alltt}
This encoding has several defaults. Firstly, it is incorrect if the
top of a stack (\texttt{head}) is a \texttt{null} reference. This can
be remedied by adding a level of indirection, that is, by creating a
private class holding \texttt{head} and \texttt{tail} or modelling the
empty stack with \texttt{null}, which entails to check for
\texttt{null} before calling any method. Secondly, pervasive use of
\texttt{null} references increases the risk of an invalid
access. Thirdly, the code for \texttt{pop} and \texttt{push} already
suggests that further operations will lead to lengthy reference
manipulations. Fourthly, persistence, for example, keeping successive
versions of a stack, is not easy.

A thorough study of the issues of null references in programming
language design is given by \cite{ChalinJames_2007,Cobbe_2008} and
\cite{Hoare_2009}. One practical inconvenience of such references is
that they render the composition of methods cumbersome and, for
example, instead of writing \texttt{s.cat(t).push(x).rev()}, we would
have to check if each call returns \texttt{null} or not.

The design we presented in the introduction avoids all the problems
and limitations previously mentioned. Of course, this comes with a
cost, which we will discuss in a couple of occasions. For now, let us
recall the \Java program of the introduction, which will be completed
further in this chapter:
\begin{alltt}
// Stack.java
\public \abstractX \class Stack<Item> \{
  \public \final NStack<Item> push(\final Item item) \{
    \return \new NStack<Item>(item,\this); \}
  \public \abstractX Stack<Item> cat(\final Stack<Item> t); \}

// EStack.java
\public \final \class EStack<Item> \extends Stack<Item> \{
  \public Stack<Item> cat(\final Stack<Item> t) \{ \return t; \}
\}

// NStack.java
\public \final \class NStack<Item> \extends Stack<Item> \{
  \private \final Item head;
  \private \final Stack<Item> tail;

  \public NStack(\final Item item, \final Stack<Item> stack) \{
    head = item; tail = stack;
  \}

  \public NStack<Item> cat(\final Stack<Item> t) \{
    \return tail.cat(t).push(head);
  \}
\}
\end{alltt}
Notice that we eschewed defining any method \texttt{pop}. The reason
is that it is intrinsically a partial function, undefined when its
argument is the empty stack, and this is why we had to resort to the
exception \texttt{EmptyStack} above. In practice, anyway, this method
is hardly useful as it would be implicit in an algorithm, for example,
in the definition of \texttt{cat} in class \texttt{NStack<Item>}. More
fundamentally, because \texttt{head} and \texttt{tail} are part of the
data structure itself, there is no need for a \texttt{pop} method.

\section{Single dispatch}
\label{sec:single_dispatch}
% \extends Comparable<? \super Item>

\paragraph{Stack reversal}
\index{stack!reversal!$\sim$ in \Java|(}

As we have seen in equation~\eqref{def:rev} \vpageref{def:rev}, the
efficient manner to reverse a stack in our functional language
consists in using an accumulator as follows:
\begin{equation*}
\fun{rev}(s) \xrightarrow{\smash{\epsilon}}
             \fun{rcat}(s,\el).\quad\;\;
\fun{rcat}(\el,t) \xrightarrow{\smash{\zeta}} t;\quad\;\;
\fun{rcat}(\cons{x}{s},t) \xrightarrow{\smash{\eta}}
                          \fun{rcat}(s,\cons{x}{t}).
\end{equation*}
Function \fun{rev/1} is defined with one rule which does not
discriminate on the nature of its parameter. Therefore, its
translation will be one method in the abstract class
\texttt{Stack<Item>}:
\begin{alltt}
// Stack.java
\public \abstractX \class Stack<Item> \{
  ...
  \public Stack<Item> rev() \{
    \return rcat(\new EStack<Item>()); \}
\}
\end{alltt}
An examination of the patterns defining the rules of function
\fun{rcat/2} shows that only the first parameter is constrained, more
precisely, it is either expected to be an empty stack or a
non\hyp{}empty stack. This simple kind of functional definition is
easily translated by relying upon the \emph{single dispatch} feature
of \Java, that is, the dynamic class of an object determines the
method being called. When, in a rule, exactly one parameter is
constrained, it is thus the natural candidate for single dispatch. For
example,
\begin{alltt}
// Stack.java
\public \abstractX \class Stack<Item> \{
  ...
  \public \abstractX Stack<Item> rcat(\final Stack<Item> t);
\}
\end{alltt}
Class \texttt{EStack<Item>} contains the translation of rule~\(\zeta\):
\begin{alltt}
// EStack.java
\public \final \class EStack<Item> \extends Stack<Item> \{
  ...
  \public Stack<Item> rcat(\final Stack<Item> t) \{ \return t; \}
\}
\end{alltt}
Class \texttt{NStack<Item>} holds the translation of rule~\(\eta\)
(\texttt{tail} is~\(s\), \texttt{head} is~\(x\)):
\begin{alltt}
// NStack.java
\public \final \class NStack<Item> \extends Stack<Item> \{
  ...
  \public Stack<Item> rcat(\final Stack<Item> t) \{
    \return tail.rcat(t.push(head));
  \}
\}
\end{alltt}
\index{stack!reversal!$\sim$ in \Java|)}

\paragraph{Skipping}
\index{stack!skipping an item!$\sim$ in \Java|(}

Let us recall the function \fun{sfst/2} in
section~\vref{sec:skipping}:
\begin{equation*}
\fun{sfst}(\el,x)          \xrightarrow{\smash{\theta}} \el;\quad\;\;
\fun{sfst}(\cons{x}{s},x)  \xrightarrow{\smash{\iota}}  s;\quad\;\;
\fun{sfst}(\cons{y}{s},x)  \xrightarrow{\smash{\kappa}}
                                         \cons{y}{\fun{sfst}(s,x)}.
\end{equation*}
This definition cannot be translated as it is because it contains an
implicit equality test in its patterns \(\iota\)~and~\(\kappa\), hence
it cannot solely rely on single dispatch. Perhaps the easiest way to
proceed is to extend our functional language with a conditional
expression \(\fun{if} \dots \fun{then} \dots \fun{else} \dots\) and
refine the original definition to use it. Here, we obtain
\begin{equation*}
\fun{sfst}(\el,x)          \xrightarrow{\smash{\theta}} \el;\quad\;\;
\fun{sfst}(\cons{y}{s},x)  \xrightarrow{\smash{\iota+\kappa}}
\fun{if} \; x = y \; \fun{then} \; s \;
\fun{else} \; \cons{y}{\fun{sfst}(s,x)}.
\end{equation*}
Now that pattens do not overlap anymore, we can translate to
\Java. First, we must not forget to expand the abstract class
\texttt{Stack<Item>}:
\begin{alltt}
// Stack.java
\public \abstractX \class Stack<Item> \{
  ...
  \public \abstractX Stack<Item> sfst(\final Item x);
\}
\end{alltt}
The translation of rule~\(\theta\) goes into the class
\texttt{EStack<Item>}:
\begin{alltt}
// EStack.java
\public \final \class EStack<Item> \extends Stack<Item> \{
  ...
  \public EStack<Item> sfst(\final Item x) \{ \return \this; \}
\}
\end{alltt}
The joint translation of \(\iota\)~and~\(\kappa\) is
\begin{alltt}
// NStack.java
\public \final \class NStack<Item> \extends Stack<Item> \{
  ...
  \public Stack<Item> sfst(\final Item x) \{
    \return head.compareTo(x) == 0 ?
           tail : tail.sfst(x).push(head); \}
\}
\end{alltt}
This last step reveals a mistake and a limitation of our method: in
order to compare two values of class \texttt{Item} like \texttt{head}
and~\texttt{x}, we need to specify that \texttt{Item} extends the
predefined class \texttt{Comparable}. Therefore, we must rewrite our
class definitions as follows:
\begin{alltt}
\public \abstractX
\class Stack<Item \extends Comparable<? \super Item>> \{
  ...
\}
\public \class EStack<Item \extends Comparable<? \super Item>>
       \extends Stack<Item> \{
  ...
\}
\public \class NStack<Item \extends Comparable<? \super Item>>
       \extends Stack<Item> \{
  ...
\}
\end{alltt}
It is a limitation because we must constrain \texttt{Item} to be
comparable even if some methods do not require this, like
\texttt{rev}, which is purely structural. Also, the abstract class
must be updated every time a new operation is added. (For a detailed
understanding of generics in \Java, see \cite{NaftalinWadler_2006}.)
\index{stack!skipping an item!$\sim$ in \Java|)}

\paragraph{Insertion sort}
\index{insertion sort!$\sim$ in \Java|(}
\index{Java@\Java|see{insertion sort}}

Insertion sort was defined in section~\vref{sec:straight_ins} as
\begin{equation*}
\begin{array}{@{}r@{\;}l@{\;}lr@{\;}l@{\;}l@{}}
  \fun{ins}(\cons{y}{s},x)
& \xrightarrow{\smash{\kappa}}
& \cons{y}{\fun{ins}(s,x)}, \,\text{if \(x \succ y\)};
& \fun{isrt}(\el)
& \xrightarrow{\smash{\mu}}
& \el;\\
  \fun{ins}(s,x)
& \xrightarrow{\smash{\lambda}}
& \cons{x}{s}.
& \fun{isrt}(\cons{x}{s})
& \xrightarrow{\smash{\nu}}
& \fun{ins}(\fun{isrt}(s),x).
\end{array}
\end{equation*}
Function \fun{isrt/1} is easy to translate because it only needs
single dispatch. Function \fun{ins/2} too, but first requires a refinement
introducing a conditional. Easy things first:
\begin{alltt}
// Stack.java
\public \abstractX
\class Stack<Item \extends Comparable<? \super Item>> \{
  ...
  \public \abstractX Stack<Item> isrt();
\}
// EStack.java
\public \class EStack<Item \extends Comparable<? \super Item>>
       \extends Stack<Item> \{
  ...
  \public EStack<Item> isrt() \{ \return \this; \}
\}
// NStack.java
\public \class NStack<Item \extends Comparable<? \super Item>>
       \extends Stack<Item> \{
  ...
  \public NStack<Item> isrt() \{\return tail.isrt().ins(head);\}
\}
\end{alltt}
Notice how \(\fun{ins}(\fun{isrt}(s),x)\) became
\texttt{tail.isrt().ins(head)}. The general method consists, firstly,
in translating \(x\)~and~\(s\) into \texttt{x}~and~\texttt{tail};
secondly, in finding a possible order of evaluation (there may be more
than one) and laying out the translations of each function call from
left to right, separated by full stops. In the case at hand,
\(\fun{isrt}(s)\) is evaluated first and becomes \texttt{tail.isrt()},
then \(\fun{ins}(\texttt{\textvisiblespace},x)\), which yields
\texttt{\textvisiblespace.ins(head)}.

The refinement of \fun{ins/2} is
\begin{equation*}
\begin{array}{@{}r@{\;}l@{\;}l@{}}
\fun{ins}(\cons{y}{s},x) & \rightarrow & \fun{if} \; x \succ y \;
\fun{then} \; \cons{y}{\fun{ins}(s,x)} \; \fun{else} \; \cons{x}{s};\\
\fun{ins}(\el,x) & \rightarrow & [x].
\end{array}
\end{equation*}
Note how we split rule~\(\lambda\) in two cases: \(\el\) and
\(\cons{y}{s}\), so the latter can be merged with rule~\(\kappa\) and
partake to the conditional. We can now translate to \Java:
\begin{alltt}
// Stack.java
\public \abstractX
\class Stack<Item \extends Comparable<? \super Item>> \{
  ...
  \protectedX \abstractX NStack<Item> ins(\final Item x);
\}
// EStack.java
\public \class EStack<Item \extends Comparable<? \super Item>>
       \extends Stack<Item> \{
  ...
  \protectedX NStack<Item> ins(\final Item x) \{\return push(x);\}
\}
// NStack.java
\public \class NStack<Item \extends Comparable<? \super Item>>
       \extends Stack<Item> \{
  ...
  \protectedX NStack<Item> ins(\final Item x) \{
    \return head.compareTo(x) < 0 ?
           tail.ins(x).push(head) : push(x); \}
\}
\end{alltt}
The method \texttt{ins} is declared \texttt{protected}, because it is
erroneous to insert an item into a non\hyp{}ordered stack.
\index{insertion sort!$\sim$ in \Java|)}

\paragraph{Testing}
\index{testing|(}

To test these definitions, we need a method \texttt{print}:
\begin{alltt}
\public \abstractX \void print();\hfill// Stack.java
\public \void print() \{ System.out.println(); \}\hfill// EStack.java
\public \void print() \{\hfill// NStack.java
  System.out.print(head + " "); tail.print(); \}
\end{alltt}
Finally, the \texttt{Main} class would look like
\begin{alltt}
\public \class Main \{\hfill // Main.java
  \public \static \void \main (String[] args) \{
    Stack<Integer> nil = \new EStack<Integer>();
    Stack<Integer> s = nil.push(5).push(2).push(7);
    s.print();\hfill// 7 2 5
    s.rev().print();\hfill// 5 2 7
    Stack<Integer> t = nil.push(4).push(1);\hfill// 1 4
    s.cat(t).print();\hfill// 7 2 5 1 4
    t.cat(s).isrt().print(); \}\hfill// 1 2 4 5 7
\}
\end{alltt}
The properties we proved earlier on the functional source codes are
transferred to the \Java target codes. In particular, the costs are
left invariant and they count the number of method calls needed to
compute the value of a method call, \emph{assuming that we do not
  account for the calls to \texttt{push}}: this method, which was
intended as a commodity, was declared \texttt{final} in the hope that
the compiler might inline its definition.  \index{testing|)}

\paragraph{Cutting}
\index{stack!cutting!$\sim$ in \Java|(}

In section~\vref{sec:cutting}, we saw how to cut a stack in two stacks
by specifying the length of the prefix:
\begin{equation*}
\begin{array}{@{}r@{\;}l@{\;}lr@{\;}l@{\;}l@{}}
\fun{cut}(s,0) & \rightarrow & \pair{\el}{s};
& \fun{push}(x,\pair{t}{u}) & \rightarrow & \pair{\cons{x}{t}}{u}.\\
\fun{cut}(\cons{x}{s},k) & \rightarrow
& \fun{push}(x,\fun{cut}(s,k-1)).
\end{array}
\end{equation*}
Here, we have another instance of a rule which covers two different
cases: in the first rule of \fun{cut/2}, either \(s\)~is empty or
not. If the latter, the translation of the rule must be merged with
the translation of the second rule. There is an additional difficulty
in the fact that \Java does not provide native pairs to translate
\(\pair{t}{u}\) immediately. Fortunately, it is not difficult to
devise a class for pairs if we realise that, abstractly, a pair is a
thing that has two properties: one informing about its `first
component' and another about its `second component'. We have
\begin{alltt}
// Pair.java
\public \class Pair<Fst,Snd> \{
  \protectedX \final Fst fst;
  \protectedX \final Snd snd;
  \public Pair(\final Fst f, \final Snd s) \{fst = f; snd = s;\}

  \public Fst fst() \{ \return fst; \}
  \public Snd snd() \{ \return snd; \}
\}
\end{alltt}
Now we can proceed as follows:
\begin{alltt}
// Stack.java
\public \abstractX
\class Stack<Item \extends Comparable<? \super Item>> \{
  ...
  \public \abstractX
  Pair<Stack<Item>,Stack<Item>> cut(\final int k);
\}
\end{alltt}
Notice that, as usual, a function with \(n\)~parameters is translated
into a method with \(n-1\) parameters because one of them is used to
perform the single dispatch or, in terms of the type system, it
supports \emph{subtype polymorphism} \citep{Pierce_2002}.
\begin{alltt}
// EStack.java
\public \class EStack<Item \extends Comparable<? \super Item>>
       \extends Stack<Item> \{
  ...
  \public Pair<Stack<Item>,Stack<Item>> cut(\final int k) \{
    \return \new Pair<Stack<Item>,Stack<Item>>(\this,\this);
  \}
\}
// NStack.java
\public \class NStack<Item \extends Comparable<? \super Item>>
       \extends Stack<Item> \{
  ...
  \public Pair<Stack<Item>,Stack<Item>> cut(\final int k) \{
    \ifX (k == 0)
       \return \new Pair<Stack<Item>,Stack<Item>>
                      (\new EStack<Item>(),\this);
    Pair<Stack<Item>,Stack<Item>> p = tail.cut(k-1);
    \return \new Pair<Stack<Item>,Stack<Item>>
                   (p.fst().push(head),p.snd());
  \}
\}
\end{alltt}
Finally, the~\texttt{Main} class could look like
\begin{alltt}
// Main.java
\public \class Main \{
  \public \static \void \main (String[] args) \{
    Stack<Integer> nil = \new EStack<Integer>();
    Stack<Integer> s = nil.push(5).push(2).push(7);\hfill// 7 2 5
    Stack<Integer> t = nil.push(4).push(1);\hfill// 1 4
    Pair<Stack<Integer>,Stack<Integer>> u = s.cat(t).cut(2);
    u.fst().print();\hfill// 7 2
    u.snd().print();\hfill// 5 1 4
  \}
\}
\end{alltt}
A translation may or must enjoy different interesting properties. For
example, it must be correct in the sense that the result of the
evaluation of a call in the source language is, in a certain sense,
equal to the result of the evaluation of the translated call. This is
what we wanted for our translation from our functional language to
\Java. Additionally, it could be required that erroneous behaviours
also translate into erroneous behaviours. For example, it may be
desirable that infinite loops translate into infinite loops and that
run\hyp{}time errors in evaluations in the source langage correspond
to similar errors in the target. (The translation of \fun{cut/2} does
not preserve all errors. Why?)\index{stack!cutting!$\sim$ in \Java|)}


\section{Binary dispatch}
\index{Java@\Java|see{merge sort}}

Functions defined by matching two or more stacks in the same pattern
call for \emph{multiple dispatch} in a target object\hyp{}oriented
language. Unfortunately, \Java only features single dispatch, in other
words, only one parameter of the source function supports subtype
polymorphism in the target. In case of \emph{binary
  dispatch}\index{Java@\Java!binary dispatch}, we can refine the
source definition into an equivalent definition which can, in turn, be
translated with single dispatch alone. This technique was proposed a
long time ago by \cite{Ingalls_1986} and a practical overview was
published by \cite{MuscheviciPotaninTemperoNoble_2008}.

\paragraph{Merge sort}

For example, let us consider again the definition of \fun{mrg/2} in
\fig~\vref{fig:tms}:
\begin{equation*}
\begin{array}{@{}r@{\;}l@{\;}l@{}}
\fun{mrg}(\el,t)         & \rightarrow & t;\\
\fun{mrg}(s,\el)         & \rightarrow & s;\\
\fun{mrg}(\cons{x}{s},\cons{y}{t}) & \rightarrow
                         & \cons{y}{\fun{mrg}(\cons{x}{s},t)},\;
                           \text{if \(x \succ y\)};\\
\fun{mrg}(\cons{x}{s},t) & \rightarrow & \cons{x}{\fun{mrg}(s,t)}.
\end{array}
\end{equation*}
We need to rewrite this definition so only one stack is inspected in
the patterns. In order to do so, we choose arbitrarily one stack
parameter, let us say the first, and match it against the empty and
non\hyp{}empty patterns as follows:
\begin{equation*}
\begin{array}{@{}r@{\;}l@{\;}l@{}}
\fun{mrg}(\el,t)         & \rightarrow & t;\\
\fun{mrg}(\cons{x}{s},t) & \rightarrow & \fun{mrg}_0(x,s,t).\\
\\
\fun{mrg}_0(x,s,\el)     & \rightarrow & \cons{x}{s};\\
\fun{mrg}_0(x,s,\cons{y}{t}) & \rightarrow
& \fun{if} \; x \succ y \; \fun{then} \;
  \cons{y}{\fun{mrg}(\cons{x}{s},t)} \;
  \fun{else} \; \cons{x}{\fun{mrg}(s,\cons{y}{t})}.
\end{array}
\end{equation*}
Note how we had to introduce an auxiliary function \fun{mrg\(_0\)/3}
to handle the matching of the second argument, \(t\), of
\fun{mrg/2}. We also made use of a conditional construct in the last
rule of \fun{mrg\(_0\)/3}, where we can also avoid the
memory\hyp{}consuming pushes \(\cons{x}{s}\) and \(\cons{y}{t}\) by,
respectively, calling \(\fun{mrg}_0(x,s,t)\) and
\(\fun{mrg}_0(y,t,s)\). The latter call is a consequence of the
theorem that merging is a symmetric function: \(\fun{mrg}(s,t) \equiv
\fun{mrg}(t,s)\). We have now
\begin{equation*}
\begin{array}{@{}r@{\;}l@{\;}l@{}}
\fun{mrg}(\el,t)         & \rightarrow & t;\\
\fun{mrg}(\cons{x}{s},t) & \rightarrow & \fun{mrg}_0(x,s,t).\\
\\
\fun{mrg}_0(x,s,\el)     & \rightarrow & \cons{x}{s};\\
\fun{mrg}_0(x,s,\cons{y}{t}) & \rightarrow
& \fun{if} \; x \succ y \; \fun{then} \;
  \cons{y}{\fun{mrg}_0(x,s,t)} \;
  \fun{else} \; \cons{x}{\fun{mrg}_0(y,t,s)}.
\end{array}
\end{equation*}
Then we have two functions that can be translated to \Java only using
single dispatch:\index{merge sort!merging!$\sim$ in \Java|(}
\begin{alltt}
// Stack.java
\public \abstractX
\class Stack<Item \extends Comparable<? \super Item>> \{
  ...
  \public \abstractX Stack<Item> mrg(\final Stack<Item> t);
  \public \abstractX
  Stack<Item> mrg0(\final Item x, \final Stack<Item> s);
\}
// EStack.java
\public \class EStack<Item \extends Comparable<? \super Item>>
       \extends Stack<Item> \{
  ...
  \public Stack<Item> mrg(\final Stack<Item> t) \{ \return t; \}
  \public Stack<Item> mrg0(\final\! Item x,\final Stack<Item> s)\{
    \return s.push(x); \}
\}
// NStack.java
\public \class NStack<Item \extends Comparable<? \super Item>>
       \extends Stack<Item> \{
  ...
  \public Stack<Item> mrg(\final Stack<Item> t) \{
    \return t.mrg0(head,tail); \}

  \public Stack<Item> mrg0(\final Item x,
                          \final Stack<Item> s) \{
    \return x.compareTo(head) > 0 ?
           tail.mrg0(x,s).push(head)
         : s.mrg0(head,tail).push(x); \}
\}
\end{alltt}
Keep in mind that the pattern \(\fun{mrg}_0(x,s,\cons{y}{t})\) means
that \(\cons{y}{t}\) translates as \texttt{this}, so the target code
should go in the class \texttt{NStack<Item>}, where \texttt{head} is
the translation of~\(y\) and \texttt{tail} is the image of~\(t\).
Finally,\index{merge sort!merging!$\sim$ in \Java|)}
\begin{alltt}
\public \class Main \{\hfill // Main.java
  \public \static \void \main (String[] args) \{
    Stack<Integer> nil = \new EStack<Integer>();
    Stack<Integer> s = nil.push(5).push(2).push(7);\hfill// 7 2 5
    Stack<Integer> t = nil.push(4).push(1);\hfill// 1 4
    s.isrt().mrg(t.isrt()).print();\hfill// 1 2 4 5 7
    t.isrt().mrg(s.isrt()).print();\hfill// 1 2 4 5 7
  \}
\}
\end{alltt}
At this point, it could be argued that our translation scheme leads to
cryptic \Java programs. But this critique is valid only because it
forgets to take into account any specification. We propose that the
functional program \emph{is} the specification of the \Java program
and should accompany it. A translation to \Erlang could be performed
first, due to its simplicity, and it would in turn be translated into
\Java, while remaining as a comment in the target code and its
documentation. For example, we would write
\begin{alltt}
// NStack.java
\public \class NStack<Item \extends Comparable<? \super Item>>
       \extends Stack<Item> \{
  ...
  // mrg0(X,S,[Y|T]) -> case X > Y of
  //                      true  -> [Y|mrg0(X,S,T)];
  //                      false -> [X|mrg0(Y,T,S)]
  //                    end.
  //
  \public Stack<Item> mrg0(\final Item x,
                          \final Stack<Item> s) \{
    \return x.compareTo(head) > 0 ?
           tail.mrg0(x,s).push(head)
         : s.mrg0(head,tail).push(x); \}
\}
\end{alltt}
(We could even go a step further and rename the \Erlang parameters
\texttt{Y} and~\texttt{T}, respectively, as \texttt{Head} and
\texttt{Tail}.) This has the additional advantage that the
specification is executable, so the result of running the \Erlang
program can be expected to be equal to the result of the translated
run in \Java, which is a significant help for the test phase.

Let us finish now the translation of top\hyp{}down merge sort in
\fig~\vref{fig:tms}. We have
\begin{equation*}
\fun{tms}(\cons{x,y}{t}) \xrightarrow{\smash{\alpha}}
 \fun{cutr}([x],\cons{y}{t},t);\qquad
\fun{tms}(t) \xrightarrow{\smash{\beta}} t.
\end{equation*}
A refinement is necessary because rule~\clause{\beta} covers the case
of the empty stack and the singleton stack. Let us make all these
cases explicit and reorder them as follows:
\begin{equation*}
\fun{tms}(\el)            \rightarrow  \el;\quad
\fun{tms}([x])            \rightarrow  [x];\quad
\fun{tms}(\cons{x,y}{t})  \rightarrow
                          \fun{cutr}([x],\cons{y}{t},t).
\end{equation*}
Now, we must fuse the two last rules because they are covered by the
case 'non\hyp{}empty stack' and introduce an auxiliary function
\fun{tms\(_0\)/2} whose role is to distinguish between the singleton
and longer stacks. Whence
\begin{equation*}
\begin{array}{@{}r@{\;}l@{\;}l@{\qquad}r@{\;}l@{\;}l@{}}
  \fun{tms}(\el)             & \xrightarrow{\smash{\gamma}} & \el;
& \fun{tms}_0(x,\el)         & \xrightarrow{\smash{\epsilon}} & [x];\\
\fun{tms}(\cons{x}{t})       & \xrightarrow{\smash{\delta}}
                             & \fun{tms}_0(x,t).
& \fun{tms}_0(x,\cons{y}{t}) & \xrightarrow{\smash{\zeta}}
                             & \fun{cutr}([x],\cons{y}{t},t).
\end{array}
\end{equation*}
At this point, both \fun{tms/1} and \fun{tms\(_0\)/2} can be
translated into \Java using single dispatch but, before doing so, we
must check whether the final refinement is indeed equivalent to the
original program. We can test some calls whose partial
evaluations\index{functional language!evaluation!partial $\sim$} make
use, as a whole, of all the rules in the refined program and then
compare them with the partial evaluations in the original
program. (This is an instance of \emph{structural
  testing}\index{testing!structural $\sim$}, more precisely,
\emph{path testing}\index{testing!path $\sim$}.) Here, we found out
that three cases are distinguished: empty stack, singleton and longer
stacks. In the refined program, we have the interpretations
\begin{equation*}
\begin{array}{@{}r@{\;}l@{\;}l@{}}
\fun{tms}(\el) & \xrightarrow{\smash{\gamma}} & \el.\\
\fun{tms}([x]) & \xrightarrow{\smash{\delta}} & \fun{tms}_0(x,\el)
                 \xrightarrow{\smash{\epsilon}} [x].\\
\fun{tms}(\cons{x,y}{t})
  & \xrightarrow{\smash{\delta}} & \fun{tms}_0(x,\cons{y}{t})
    \xrightarrow{\smash{\zeta}} \fun{cutr}([x],\cons{y}{t},t).
\end{array}
\end{equation*}
We can now compare with the same calls being partially computed with
the original program:
\begin{equation*}
\begin{array}{@{}r@{\;}l@{\;}l@{}}
\fun{tms}(\el) & \xrightarrow{\smash{\beta}} & \el.\\
\fun{tms}([x]) & \xrightarrow{\smash{\beta}} & [x].\\
\fun{tms}(\cons{x,y}{t}) & \xrightarrow{\smash{\alpha}}
                         & \fun{cutr}([x],\cons{y}{t},t).
\end{array}
\end{equation*}
These calls agree and cover all the arrows in all the definitions. We
can now translate the refined program:\index{merge sort!top-down
  $\sim$!$\sim$ in \Java|(}
\begin{alltt}
// Stack.java
\public \abstractX
\class Stack<Item \extends Comparable<? \super Item>> \{
  ...
  \public \abstractX Stack<Item> tms();
  \protectedX \abstractX Stack<Item> tms0(\final Item x);
\}
// EStack.java
\public \class EStack<Item \extends Comparable<? \super Item>>
       \extends Stack<Item> \{
  ...
  \public Stack<Item> tms() \{ \return \this; \}

  \protectedX Stack<Item> tms0(\final Item x) \{
    \return push(x); \}
\}

// NStack.java
\public \class NStack<Item \extends Comparable<? \super Item>>
       \extends Stack<Item> \{
  ...
  \public Stack<Item> tms() \{ \return tail.tms0(head); \}

  \protectedX Stack<Item> tms0(\final Item x) \{
    \return tail.cutr(\new EStack<Item>().push(x),\this); \}
\}
\end{alltt}
\index{merge sort!top-down $\sim$!$\sim$ in \Java|)} Notice again how
auxiliary functions result in \texttt{protected} methods and how they
add to the overall cost, compared to the functional program: this is a
limitation on how properties on the source transfer to the
target. However, the asymptotic cost is the same as in the original
functional program. A closer look at \texttt{tms0} shows that we
translated \(\cons{y}{t}\) by \texttt{this} instead of
\texttt{tail.push(head)}, as an optimisation. Moreover, we dispatch
\texttt{cutr} on \texttt{tail} because we already know that this is
the parameter we are going to distinguish when translating
\fun{cutr/3} defined in \fig~\vref{fig:tms}:
\begin{equation*}
\begin{array}{@{}r@{\;}l@{\;}l@{}}
\fun{cutr}(s,\cons{y}{t},\cons{a,b}{u})
                       & \rightarrow & \fun{cutr}(\cons{y}{s},t,u);\\
\fun{cutr}(s,t,u)      & \rightarrow
                       & \fun{mrg}(\fun{tms}(s),\fun{tms}(t)).
\end{array}
\end{equation*}
We remark that two stacks are matched by the first pattern so some
refinement is called for and we must choose one parameter to start
with. A little attention reveals that the third one is the best
choice, because if it does not contain at least two items, it is
simply discarded. We need to introduce two auxiliary functions,
\fun{cutr\(_0\)/3} and \fun{cutr\(_1\)/3}, that check, respectively,
whether the third parameter contains at least two items and whether
the second contains at least one. The result is shown in
\fig~\ref{fig:cutr}.
\begin{figure}
\begin{equation*}
\boxed{
\begin{array}{@{}r@{\;}l@{\;}l@{}}
\fun{cutr}(s,t,\el)     & \rightarrow
                        & \fun{mrg}(\fun{tms}(s),\fun{tms}(t));\\
\fun{cutr}(s,t,\cons{a}{u})
                        & \rightarrow & \fun{cutr}_0(s,t,u).\\
\\
\fun{cutr}_0(s,t,\el)   & \rightarrow
                        & \fun{mrg}(\fun{tms}(s),\fun{tms}(t));\\
\fun{cutr}_0(s,t,\cons{b}{u})
                        & \rightarrow & \fun{cutr}_1(s,t,u).\\
\\
\fun{cutr}_1(s,\el,u)   & \rightarrow & \fun{tms}(s);\\
\fun{cutr}_1(s,\cons{y}{t},u)
                        & \rightarrow
                        & \fun{cutr}(\cons{y}{s},t,u).
\end{array}}
\end{equation*}
\caption{Cutting and merging top-down}
\label{fig:cutr}
\end{figure}

Note how \(\fun{cutr}_1(s,\el,u) \rightarrow
\fun{mrg}(\fun{tms}(s),\fun{tms}(\el))\) was simplified using the
theorems \(\fun{tms}(\el) \equiv \el\) and \(\fun{mrg}(s,\el) \equiv
s\). The translation is now direct:\index{merge sort!top-down
  $\sim$!$\sim$ in \Java|(}
\begin{alltt}
// Stack.java
\public \abstractX
\class Stack<Item \extends Comparable<? \super Item>> \{
  ...
  \public \abstractX
  Stack<Item> cutr(\final Stack<Item> s, \final Stack<Item> t);

  \protectedX \abstractX
  Stack<Item> cutr0(\final Stack<Item> s, \final Stack<Item> t);

  \protectedX \abstractX
  Stack<Item> cutr1(\final Stack<Item> s, \final Stack<Item> u);
\}

// EStack.java
\public \class EStack<Item \extends Comparable<? \super Item>>
\extends Stack<Item> \{
  ...
  \public
  Stack<Item> cutr(\final Stack<Item> s, \final Stack<Item> t) \{
    \return s.tms().mrg(t.tms()); \}

  \protectedX
  Stack<Item> cutr0(\final Stack<Item> s, \final Stack<Item> t) \{
    \return s.tms().mrg(t.tms()); \}

  \protectedX
  Stack<Item> cutr1(\final Stack<Item> s, \final Stack<Item> u) \{
    \return s.tms(); \}
\}

// NStack.java
\public \class NStack<Item \extends Comparable<? \super Item>>
       \extends Stack<Item> \{
  ...
  \public
  Stack<Item> cutr(\final Stack<Item> s, \final Stack<Item> t) \{
    \return tail.cutr0(s,t); \}

  \protectedX
   Stack<Item> cutr0(\final Stack<Item> s, \final Stack<Item> t) \{
    \return t.cutr1(s,tail); \}

  \protectedX
  Stack<Item> cutr1(\final Stack<Item> s, \final Stack<Item> u) \{
    \return u.cutr(s.push(head),tail); \}
\}
\end{alltt}
Note that it is of the utmost importance to constantly mind the
parameter in the source function whose translation will be the base
for dispatch. Finally, the \texttt{Main} class might look like
\begin{alltt}
// Main.java
\public \class Main \{
  \public \static \void \main (String[] args) \{
    Stack<Integer> nil = \new EStack<Integer>();
    Stack<Integer> s = nil.push(5).push(2).push(7);\hfill// 7 2 5
    Stack<Integer> t = nil.push(4).push(1);\hfill// 1 4
    s.tms().print();\hfill// 2 5 7
    s.cat(t).tms().print();\hfill// 1 2 4 5 7
  \}
\}
\end{alltt}
\index{merge sort!top-down $\sim$!$\sim$ in \Java|)}
\index{Java@\Java!translation to $\sim$|)}
