Let us start by presenting a subset of the functional core of \OCaml
(previously known as \emph{Objective Caml}). Contrary to previous
chapters, the following introduction will be more formal, relying on a
systematic classification of the syntactic constructs and their
semantics. The reader is advised to keep in mind that what follows is
a very concise and limited introduction to \OCaml, with the aim to
show what a semi\hyp{}formal definition of a programming language
looks like. Afterwards, we will show how \OCaml can be used to parse
other languages, presenting along the way the basic theoretical
concepts borrowed from compiler construction and formal language
theory.

A \emph{sentence}, or global definition, is defined by the following
cases, where~\(e\) denotes an expression, \(x\)~and~\(f\) are
variables and \Xlet and~\Xrec are keywords:

\bigskip

\noindent
\begin{tabular}{rll}
    $\bullet$ 
  & \emph{global definition}
  & \phrase{$\Xlet \,\, x = e$}\\
    $\bullet$
  & \emph{global recursive definition}
  & \phrase{$\Xlet \,\, \Xrec \,\, f = e$}
\end{tabular}

\bigskip

Global definitions can be optionally followed by two semicolons: these
are necessary when inputting the sentences in the toplevel loop
(prompted after running the command \texttt{ocaml} from a Unix
shell). In contrast with \Erlang, note that the keyword \Xrec is
necessary to enable recursion. A program is a sequence of
sentences. When we write that \(e\)~is an expression, we mean that
\(e\)~denotes a part of a sentence which is classified as an
expression according to its syntax. We say that \(e\)~is a
\emph{metavariable} because, being a name, it is a variable, but that
variable does not belong to the language being described (\OCaml) and,
instead, exists only in the descriptive language. In other words, it
is not an \OCaml program but a notation to refer to fragments of
\OCaml programs, perhaps an infinity. For instance, the metavariable
\(x\)~denotes a set of \OCaml variables and we must not confuse it
with the \OCaml variable~\ident{x}. Similarly, the metavariable
\(e_1\)~denotes an infinity of expressions.

\paragraph{Expressions}

An expression \(e\)~is recursively defined by the following cases:

\bigskip

\noindent
\begin{tabular}{rll}
    $\bullet$
  & \emph{variable}
  & $f, g, h$ (function); $x, y, z$ (other). \\
    $\bullet$
  & \emph{function} (or \emph{abstraction})
  & $\Xfun \,\, x \rightarrow e$\\
    $\bullet$
  & \emph{call} (or \emph{application})
  & $e_1 \,\, e_2$ \\ 
    $\bullet$
  & \emph{arithmetic operator}
  & \lpar\texttt{+}\rpar \ \lpar\texttt{-}\rpar \ \lpar\texttt{/}\rpar
    \ \lpar\texttt{*}\rpar\\
    $\bullet$
  & \emph{arithmetic operation}
  & $e_1$ \texttt{+} $e_2$ or $e_1$ \texttt{-} $e_2$
    or $e_1$ \texttt{/} $e_2$ or $e_1$ \texttt{*} $e_2$\\
    $\bullet$
  & \emph{unit or integer constant}
  & \unit \ or \textsf{0} or \textsf{1} or \textsf{2} etc.\\
    $\bullet$
  & \emph{parentheses}
  & $\lpar{e}\rpar$\\
    $\bullet$
  & \emph{local definition}
  & $\Xlet \,\, x = e_1 \,\, \Xin \,\, e_2$
\end{tabular}

\bigskip

Note that what we print nicely as `\(\rightarrow\)' is actually
written \texttt{->} in the source code. Here is an example of a
program, where the keywords are set in bold for greater clarity:
\begin{alltt}
\textbf{let} x = 0
\textbf{let} id = \textbf{fun} x \(\rightarrow\) x
\textbf{let} y = 2 \textbf{in} id y
\textbf{let} x = (\textbf{fun} x \(\rightarrow\) \textbf{fun} y \(\rightarrow\) x + y) 1 2
\textbf{let} z = x+1
\end{alltt}
\noindent Let us remark the following:
\begin{itemize}

  \item Variables must start with a lowercase character.

  \item The arrow is right-associative, so the expression
    \begin{equation*}
      \Xfun \,\, x_1 \rightarrow \Xfun \,\, x_2 \rightarrow \ldots
      \rightarrow \Xfun \,\, x_n \rightarrow e
    \end{equation*}
    is equivalent to
     \(\Xfun \,\, x_1 \rightarrow (\Xfun \,\, x_2 \rightarrow
     (\ldots \rightarrow (\Xfun \,\, x_n \rightarrow e)) \ldots)\).

  \item Function calls are left\hyp{}associative, so the
  expression \(e_1 \, e_2 \, e_3 \, \ldots \, e_n\) is equivalent
  to \((((\ldots (e_1 \, e_2) \, e_3) \ldots) \, e_n)\).

  \item Function calls have higher priority than operator calls, for
  example, \(f \, 3 + 4\) is equivalent to \((f \, 3) + 4\).

  \item Operator calls have higher syntactic priority than
  abstractions, for instance, \(\Xfun \, x \rightarrow x + y \) is
  equivalent to \(\Xfun \, x \rightarrow (x + y)\).

\end{itemize}
A program, that is, a series of global definitions, can always be
rewritten into a single sentence by means of nested local
definitions. The previous program is thus equivalent to
\begin{alltt}
\textbf{let} x = 0 \textbf{in}
  \textbf{let} id = \textbf{fun} x \(\rightarrow\) x \textbf{in}
  \textbf{let} _ = \textbf{let} y = 2 \textbf{in} id y \textbf{in}
  \textbf{let} x = (\textbf{fun} x \(\rightarrow\) \textbf{fun} y \(\rightarrow\) x + y) 1 2
\textbf{in} x+1
\end{alltt}
The symbol `\texttt{\_}' denotes a variable whose name is globally
unique, that is, it is different from any other, but purposefully
omitted. Without loss of generality, we will consider programs reduced
to a single expression. The meaning of a program is
the \emph{evaluation} of \texttt{x+1}.

\section*{Abstract syntax}

As with any programming language, before considering the evaluation of
\OCaml, we must specify the notion of \emph{scope of variables}, that
is to say, what is denoted by a given variable at a particular
location in the source file. With this aim, a graphical representation
of programs (let us remember that expressions are enough) as trees
proves very handy.
\begin{center}
\begin{tabular}{l|c}
\toprule
  \multicolumn{1}{c}{Expression}
& \multicolumn{1}{c}{Tree}\\
\toprule
    $x$
  & $x$\\
    $\Xfun \,\, x \rightarrow e$
  & \includegraphics[bb=72 697 93 721]{fun_tree}\\
    $e_1 \,\, e_2$ 
  & \includegraphics[bb=72 695 101 720]{app_tree}\\
\bottomrule
\end{tabular}
\end{center}
Intuitively, the method for constructing abstract syntax trees
consists in first parenthesising fully the expression making up the
program. Each parenthesis corresponds to a subexpression and each
subexpression corresponds to a subtree. The tree is built from the
root, down to the leaves, by traversing subexpressions from the
outermost to the innermost, that is, the most embedded ones. For
instance, consider
\fig~\vref{fig:abs_trees}.
\begin{figure}
\centering
\subfloat[\texttt{(1+2)*(5/1)}]{\includegraphics{arith_tree1}}
\qquad
\subfloat[\texttt{let x = 1 in (1+2)*(5/1)}]{%
  \includegraphics{arith_tree2}}
\qquad
\subfloat[\texttt{let x = 1 in ((let x = 2 in x) + x)}]{
  \includegraphics[bb=71 658 160 721]{arith_tree3}
}
\caption{Concrete and abstract syntax\label{fig:abs_trees}}
\end{figure}
Again, note that $x$~and~\texttt{x} are different: the former is a
metavariable, standing for \emph{any} \OCaml variable, whilst the
latter is a particular \OCaml variable.

\paragraph{Static scoping and environment}

A sentence associates an expression~\(e\) to a variable~\(x\): we
speak of \emph{binding}, which we write \(x \mapsto e\). A subprogram
then defines a set of bindings called an \emph{environment}. A binding
is \emph{static} if we can determine at compile\hyp{}time, that is, by
examining the source code, what expression a given variable refers
to. For instance, in
\begin{alltt}
\textbf{let} x = 0 \textbf{in}
  \textbf{let} id = \textbf{fun} x \(\rightarrow\) x \textbf{in}
  \textbf{let} y = id x \textbf{in}
  \textbf{let} x = (\textbf{fun} x \(\rightarrow\) \textbf{fun} y \(\rightarrow\) x + y) 1 2 
\textbf{in} x+1
\end{alltt}
the variable~\texttt{x} in the expression \texttt{x+1} denotes the
expression bound to the variable~\texttt{x} in the previous line, not
the first. Bindings are ordered in the environment \emph{by the order
of their definitions}. Thus
\begin{enumerate}

  \item the environment is initially empty: \{\};

  \item with \texttt{\textbf{let} x = 0 \textbf{in} \ldots}, it
    becomes $\{\ident{x} \mapsto 0\}$;

  \item with \texttt{\textbf{let} id = \textbf{fun} x $\rightarrow$ x
    \textbf{in} \ldots}, it is
    $\{\ident{id} \mapsto \Xfun \,\, \ident{x} \rightarrow \ident{x}; \ident{x} \mapsto
    0\}$;

  \item \texttt{\textbf{let} y = id x \textbf{in} \ldots} yields
    $\{\ident{y} \mapsto \ident{id} \, \ident{x}; \ident{id} \mapsto \Xfun \,\, \ident{x} \rightarrow \ident{x}; \ident{x} \mapsto
    0\}$;

  \item with \texttt{\textbf{let} x = \ldots}, it is
      $\{\underline{\ident{x} \mapsto \ldots}; \ident{y} \mapsto \ident{id} \, \ident{x}; \ident{id} \mapsto \Xfun \,\, \ident{x} \rightarrow \ident{x}; \underline{\ident{x} \mapsto
      0}\}$.

\end{enumerate}
The binding $\ident{x} \mapsto 0$ is therefore hidden, or out of
scope, in \texttt{x+1}. We will write~\(\rho(x)\) the first binding
of~\(x\) in the environment~\(\rho\), if any.

\paragraph{Free variables}

The local definition \Xlet $x$ = $e_1$ \Xin $e_2$ binds $e_1$~to~$x$,
which we write $x \mapsto e_1$, in~$e_2$. It may be the case that,
in~\(e_2\), another local definition binds the same variable. To
clearly understand that situation, we proceed as follows on the
abstract syntax tree of the program. From each variable occurrence, we
move up, towards the root. If we encounter a \Xlet binding that
variable, we create an oriented edge between the variable location and
that \Xlet, and we stop. Otherwise, if we reach the root (no such
\Xlet has been found), and the variable is said to be \emph{free} in the
expression. We note \(\mathcal{F}(e)\) the set of the variables free
in~\(e\). See for instance \fig~\vref{fig:free_var1}.
\begin{figure}
\centering
\subfloat[\texttt{\textbf{let} x = 1 \textbf{in} (\textbf{let} x = 2
  \textbf{in} x) + x}\label{fig:free_var1}]{
\includegraphics[bb=71 658 160 721]{free_var1}}
\qquad\qquad
\subfloat[\texttt{\textbf{fun} y \(\rightarrow\) x + (\textbf{fun} x \(\rightarrow\) x)
  y}\label{fig:free_var2}]{
  \includegraphics{free_var2}}
\caption{Graphical representation of bindings}
\end{figure}
A similar situation arises with functions. Indeed, in the body~\(e\)
of \Xfun $x \rightarrow e$, the parameter~\(x\) may shadow (hide)
another variable~\(x\) bound at an upper level in the tree. We must
then consider that \Xfun is a binder, just like \Xlet. See
\fig~\vref{fig:free_var2} for an example.

\paragraph{Closed expressions}

A \emph{closed} expression is an expression without free
variables. Only a closed program can be evaluated (executed). Indeed,
what would be the value of the program reduced to the simple
expression~\texttt{x}? That is why the first static analysis performed
by compilers consists in determining the variables which are free in
expressions. If the program is not closed, it is rejected. In the case
of~\texttt{x}, the \OCaml compiler prints
\begin{center}
\texttt{Unbound value x}
\end{center}
and stops. It is useful that this open expression is rejected at
compile\hyp{}time and does not cause a run\hyp{}time error.

\section{Evaluation}

The evaluation of an expression, therefore of a program, is a partial
function from expressions to \emph{values}. A partial function models
the fact that an evaluation may not terminate or can be interrupted
because of an error. The values~\(v\) of \OCaml are almost a strict
subset of the expressions, defined recursively by the following cases:

\bigskip

\begin{tabular}{rll}
    $\bullet$
  & \emph{unit or integer constant}
  & \unit \ or \textsf{0} or \textsf{1} or \textsf{2} etc.\\
    $\bullet$
  & \emph{closure}
  & $\langle\Xfun \,\, x \rightarrow e, \rho\rangle$,\\
  && where $\rho$ is an environment.\\
  && For the operators: $\langle\lpar{\texttt{+}}\rpar, \rho\rangle$
  etc.
\end{tabular}

\bigskip

A \emph{closure} is a pair made of a function and an environment. This
means in particular that functions are values, that is, they can be
the result of the evaluation of an \OCaml program --~indeed, this is
the characteristic feature of a functional language.

In order to define the evaluation, we must modify the concept of
binding: a binding will now associate a variable to a \emph{value}
--~and not an expression anymore. Evaluation, and perhaps the
resulting value, will depend on the environment at the occurrence of
the expression. Let us note \((x \mapsto v) \oplus \rho\) the
environment resulting from extending the environment~\(\rho\) with the
binding \(x \mapsto v\), shadowing any other binding for~\(x\) that
might exist in~\(\rho\).

For each expression~\(e\), let us define an evaluation rule in an
environment~\(\rho\), yielding the value~\(v\), as follows:

\bigskip

\begin{tabular}{@{}rll@{}}
    $\bullet$
  & $x$ 
  & Look up the \emph{first} value bound to~\(x\) in~\(\rho\).\\[2mm]
    $\bullet$
  & $\Xfun \,\, x \rightarrow e$
  & The value is $\langle\Xfun \,\, x \rightarrow e, \rho\rangle$.\\[2mm]
    $\bullet$
  & \texttt{+} \ \texttt{-} \ \texttt{/} \ \texttt{*}
  & The value is $\langle\lpar\texttt{+}\rpar, \rho\rangle$ etc.\\[2mm]
    $\bullet$
  & $e_1$ \texttt{+} $e_2$ \ etc.
  & Evaluate $e_1$ and $e_2$ in $\rho$, and sum, etc.\\[2mm]
    $\bullet$
  & \unit \ or \textsf{0} or \textsf{1} or \textsf{2} etc.
  & The value is \unit or \textsf{0} or \textsf{1} etc.\\[2mm]
    $\bullet$
  & $\lpar{e}\rpar$
  & Evaluate $e$ into $v$ in $\rho$.\\[2mm]
    $\bullet$
  & $\Xlet \,\, x = e_1 \,\, \Xin \,\, e_2$
  & Evaluate $e_1$ into $v_1$ in $\rho$;\\
  & 
  & evaluate $e_2$ into $v$ in $(x \mapsto v_1) \oplus \rho$.\\[2mm]
    $\bullet$
  & $e_1 \,\, e_2$
  & Evaluate $e_1$ and $e_2$ into $v_1$ and $v_2$ in $\rho$\\
  &
  & (order left unspecified);\\
  &
  & $v_1$ must be of the form $\langle\Xfun \,\, x \rightarrow
    e,\rho'\rangle$;\\
  &
  & evaluate $e$ in $(x \mapsto v_2) \oplus \rho'$:\\
  && the value is $v$.
\end{tabular}

\bigskip

\paragraph{Example}

The environment is initially empty: \(\rho = \{\}\). Let us evaluate
\begin{alltt}
\textbf{let} x = 0 \textbf{in}
  \textbf{let} id = \textbf{fun} x \(\rightarrow\) x \textbf{in}
  \textbf{let} y = id x \textbf{in}
  \textbf{let} x = (\textbf{fun} x \(\rightarrow\) \textbf{fun} y \(\rightarrow\) x + y) 1 2 
\textbf{in} x+1
\end{alltt}
\begin{itemize}

\item The evaluation of \texttt{\textbf{let} x = 0 \textbf{in} \ldots}
  imposes the prior evaluation of~\texttt{0}, which, obviously,
  yields~\(0\). Then, we create the binding $\ident{x} \mapsto 0$, we
  add it to~\(\rho\), which results in $\{\ident{x} \mapsto 0\}$, and
  we evaluate the subexpression elided with this new environment.

\item The evaluation of \texttt{\textbf{let} id = \textbf{fun} x \(\rightarrow\) x
    \textbf{in} \ldots} is performed within the environment
  $\{\ident{x} \mapsto 0\}$. The value~\(v\) is then the closure
  $\langle\Xfun \,\, \ident{x} \rightarrow \ident{x}, \{\ident{x} \mapsto
  0\}\rangle$. We extend the current environment with
  $\ident{id} \mapsto v$ and we evaluate the subexpression with it.

\item The evaluation of \texttt{\textbf{let} y = id x \textbf{in}
    \ldots} is done within the environment $\{\ident{id} \mapsto
  \langle\Xfun \,\, \ident{x} \rightarrow \ident{x}, \{\ident{x}
  \mapsto 0\}\rangle; \ident{x} \mapsto 0\}$.
    \begin{itemize}

      \item We evaluate first (\texttt{id x}) in the current
        environment (we used parentheses for better readability). In
        order to do so, we evaluate~\texttt{id} and~\texttt{x}
        separately. These are both variables, thus we look them up in
        the environment to retrieve the first corresponding
        binding: \(\ident{id} \mapsto \langle\Xfun \,\, \ident{x} \rightarrow \ident{x}, \{\ident{x} \mapsto
        0\}\rangle\), and \(\texttt{x} \mapsto 0\). We need to
        evaluate~\texttt{x} in the environment $(\ident{x} \mapsto
        0) \oplus \{\ident{x} \mapsto 0\}$, which yields~\(0\).

      \item We create the binding $\ident{y} \mapsto 0$, we add it to
        the current environment and we evaluate the elided
        subexpression with the new environment.
 
    \end{itemize}

  \item The evaluation of \texttt{\textbf{let} x = (\textbf{fun} x \(\rightarrow\)
      \textbf{fun} y \(\rightarrow\) x + y) 1 2 \textbf{in} \ldots}
      uses the environment $\{\ident{y} \mapsto 0; \ident{id} \mapsto
    \langle\Xfun \,\, \ident{x} \rightarrow \ident{x}, \{\ident{x}
    \mapsto 0\}\rangle; \ident{x} \mapsto 0\}$.

\end{itemize}

\paragraph{Formal definition}

If we note \(\evalf{e}{\rho}\) the value obtained by evaluating the
expression~\(e\) in the environment~\(\rho\), then we can resume the
evaluation of \OCaml programs as follows:
\begin{align*}
\evalf{\overline{n}}{\rho} &= \dot{n}, \;\, \text{with
  $\overline{n}$ an \OCaml integer and $\dot{n} \in \mathbb{N}$};\\
\evalf{e_1 \texttt{+} \, e_2}{\rho} &= \evalf{e_1}{\rho} +
\evalf{e_2}{\rho},\;\, \text{etc.};\\
\evalf{\lpar{e}\rpar}{\rho} &= \evalf{e}{\rho};\\
\evalf{x}{\rho} &= \rho(x), \;\, \text{(the first binding of~$x$
  in~$\rho$)};\\
\evalf{\Xfun \,\, x \rightarrow e}{\rho} &= \langle\Xfun \,\, x
\rightarrow e, \rho\rangle;\\
\evalf{\Xlet \,\, x = e_1 \,\, \Xin \,\, e_2}{\rho} &= \evalf{e_2}{((x
  \mapsto \evalf{e_1}{\rho}) \oplus \rho)};\\
\evalf{e_1 \,\, e_2}{\rho} &= \evalf{e}{((x \mapsto \evalf{e_2}{\rho})
  \oplus \rho')},\\
&\phantom{=\llbracket} \textnormal{where} \ \ \evalf{e_1}{\rho} =
\langle\Xfun \,\, x \rightarrow e,\rho'\rangle.
\end{align*}
The evaluation consists in applying the equations from left to right,
until termination or an error occurs.

We establish directly that $\evalf{\lpar\Xfun \; x \rightarrow
  e_2\rpar \; e_1}{\rho} = \evalf{\Xlet \,\, x = e_1 \,\, \Xin \,\,
  e_2}{\rho}$, that is, the local binding is not necessary, at least
  as long as there is no type system.

\paragraph{Applications}

If closures are values, they can be the value of a function call. For
instance,
\begin{alltt}
\textbf{let} add = \textbf{fun} x \(\rightarrow\) \textbf{fun} y \(\rightarrow\) x + y \textbf{in}
  \textbf{let} incr = add 1
\textbf{in} incr 5
\end{alltt}
The call (\texttt{add 1}) is a \emph{partial application}, as opposed
to a \emph{complete application} like (\texttt{add 1 5}): a partial
application results in a closure. Like all applications, operations
can also be partially evaluated:
\begin{alltt}
\textbf{let} incr = (+) 1 \textbf{in} incr 5
\end{alltt}
where the parentheses in~\texttt{(+)} mean that the operator has to be
considered in prefix position, not infix, as usual.

\paragraph{Termination}

We can already compute with our tiny subset of \OCaml everything that
is computable with the underlying computer. For instance, we can write
the following non\hyp{}terminating program:
\begin{center}
\texttt{\textbf{let} omega = \textbf{fun} f \(\rightarrow\) f f \textbf{in} omega omega}
\end{center}
We have the evaluation
\begin{gather*}
\evalf{\texttt{\textbf{let} omega = \textbf{fun} f \(\rightarrow\) f f \textbf{in} omega omega}}{\rho}\\
\begin{align*}
&= \evalf{\texttt{omega omega}}{((\texttt{omega} \mapsto
\evalf{\texttt{\textbf{fun} f \(\rightarrow\) f f}}{\rho}) \oplus \rho)}\\
&= \evalf{\texttt{f f}}{((\texttt{\small f} \mapsto
    \langle\texttt{\textbf{fun} f \(\rightarrow\) f f},\rho\rangle) \oplus \rho)}\\
&= \text{\emph{idem.}}
\end{align*}
\end{gather*}

\paragraph{Recursion}

To demonstrate the expressive power of \OCaml, let us see how we can
simulate recursive function by means of the auto\hyp{}applicative
function \texttt{omega}. First, let us define a function \texttt{fix},
traditionally called the \emph{Y combinator}:
\begin{alltt}
\textbf{let} omega = \textbf{fun} f \(\rightarrow\) f f
\textbf{let} fix = \textbf{fun} g \(\rightarrow\) omega (\textbf{fun} h \(\rightarrow\) \textbf{fun} x \(\rightarrow\) g (h h) x)
\end{alltt}

\noindent It is tedious to show that, for all function~\(f\) and
variable~\(x\),
\begin{equation*}
\evalf{\lpar\ident{fix} \; f\rpar{} \, x}{\rho} = \evalf{f \;
  \lpar\ident{fix} \; f\rpar{} \, x}{\rho}.
\end{equation*}
In other words, we have $\lpar\ident{fix} \; f\rpar{} \equiv f \;
\lpar\ident{fix} \; f\rpar$. Moreover, by definition, the fixed
point~\(p\) of a function~\(f\) satisfies the equation \(p =
f(p)\). Therefore, the fixed point of a function~\ident{f}, \emph{if
  it exists}, is the value of \lpar\ident{fix} \ident{f}\rpar{}.

It is possible to define the semantics (the evaluation) of a family of
fixed\hyp{}point operators (and not just one as previously) by
declaring that such operators must satisfy
\begin{equation*}
\evalf{\ident{fix} \,\, e}{\rho} = \evalf{e_1}{(f \mapsto
  \evalf{\ident{fix} \, \lpar\Xfun \, f \rightarrow e_1\rpar}{\rho}
  \oplus \rho')},
\end{equation*}
where $\evalf{e}{\rho} = \langle\Xfun \, f \rightarrow
e_1,\rho'\rangle$.

Let us consider the following definitions:
\begin{alltt}
\textbf{let} pre\_fact =
  \textbf{fun} f \(\rightarrow\) \textbf{fun} n \(\rightarrow\) \textbf{if} n=1 \textbf{then} 1 \textbf{else} n * f(n-1)
\textbf{let} fact = fix pre\_fact
\end{alltt}
We see that \texttt{fact} is the fixed point of \texttt{pre\_fact}, if
it exists, that is,
\begin{align*}
\evalf{\texttt{fact}}{\rho} &= \evalf{\texttt{pre\_fact fact}}{\rho}\\
&= \evalf{\texttt{\textbf{fun} n \(\rightarrow\) \textbf{if} n=1 \textbf{then} 1 \textbf{else} n * fact(n-1)}}{\rho}
\end{align*}
Therefore, \texttt{fact} is the factorial function. We can then
predefine the fixed\hyp{}point operator \texttt{fix} (not necessarily
the one we used above) and allow the programmer to use it directly,
but, instead, we are going to extend our description of the subset of
\OCaml with a native recursive binder:
\begin{equation*}
\evalf{\Xlet \; \Xrec \; f = e_1 \, \Xin \, e_2}{\rho} =
\evalf{\Xlet \; f = \textsf{fix} \, \lpar\Xfun \, f \rightarrow
e_1\rpar \, \Xin \, e_2}{\rho}.
\end{equation*}

\section*{Extensions}

Let us add the following expressions:

\bigskip

\begin{tabular}{rll} 
    $\bullet$
  & \emph{Boolean constants}
  & \Xtrue{} or \Xfalse\\
    $\bullet$
  & \emph{Boolean operators}
  & \texttt{(\&\&)} or \texttt{(||)} or \textsf{\textbf{not}}\\
    $\bullet$
  & \emph{$n$-tuple}
  & $e_1, \ldots, e_n$\\
    $\bullet$
  & \emph{conditional}
  & \Xif{} $e_0$ \Xthen{} $e_1$ \Xelse{} $e_2$\\
    $\bullet$
  & \emph{local recursive binding}
  & \Xlet{} \Xrec{} $f$ = $e_1$ \Xin{} $e_2$
\end{tabular}

\bigskip

\noindent Note that parentheses are recommended around tuples.

\paragraph{Irrefutable patterns}

Let us distinguish the variables occurring after \Xlet{} and \Xfun{},
because they are \emph{irrefutable patterns}, noted $\ipat{p}$:

\bigskip

\begin{tabular}{rll}
    $\bullet$
  & \emph{function}
  & $\Xfun \,\, \ipat{p} \rightarrow e$\\
    $\bullet$
  & \emph{local definition}
  & \(\Xlet \,\, \textrm{[}\Xrec\textrm{]} \,\, \ipat{p} = e_1 \,\,
    \Xin \,\, e_2\)
\end{tabular}

\bigskip

\noindent An irrefutable pattern $\ipat{p}$ is defined recursively by
the following cases:

\bigskip

\begin{tabular}{rll}
    $\bullet$
  & \emph{variable}
  & $f, g, h$ (functions) and $x, y, z$ (others). \\
    $\bullet$
  & \emph{unit}
  & \unit\\
    $\bullet$
  & \emph{$n$-tuple}
  & $\ipat{p}_1, \ldots, \ipat{p}_n$\\
    $\bullet$
  & \emph{parentheses}
  & $\lpar\ipat{p}\rpar$\\
    $\bullet$
  & \emph{underscore}
  & {\Large \_}
\end{tabular}

\bigskip

Note that, from the syntactic point of view, irrefutable patterns are
special cases of expressions, except the underscore, which is a
special case avoiding the creation of a binding.

\paragraph{Additional rules for expressions}

The comma takes priority over the arrow: $(\Xfun \,\, x \rightarrow x,
y)$ is equivalent to $\Xfun \,\, x \rightarrow (x,y)$. In order to
alleviate the notation
$\Xfun \,\, \ipat{p}_1 \rightarrow \ldots \rightarrow \Xfun \,\, \ipat{p}_n \rightarrow
e$, we define the equivalent constructs:

    \medskip

    \begin{tabular}{rll}
      $\bullet$
      & \phrase{$\Xlet \,\,
        \textrm{[}\Xrec\textrm{]} \,\, f 
        = \Xfun \,\, \ipat{p}_1 \; \ipat{p}_2 \; \ldots \; \ipat{p}_n \rightarrow e$}
      & (new expression)\\
        $\bullet$
      & \phrase{$\Xlet \,\,
        \textrm{[}\Xrec\textrm{]} \,\, f 
        \,\, \ipat{p}_1 \; \ipat{p}_2 \; \ldots \; \ipat{p}_n = e$}
      & (new sentence)
    \end{tabular}

    \medskip

\noindent For example,  we would write
\begin{alltt}
\textbf{let} mult = \textbf{fun} x y \(\rightarrow\) x * y \textbf{in}
  \textbf{let} eq x y = (x = y) \textbf{in}
  \textbf{let rec} fact n = \textbf{if} eq n 1 \textbf{then} 1 \textbf{else} mult n (fact(n-1))
\textbf{in} fact 9
\end{alltt}
We also extend the syntax to alleviate certain expressions. By
definition,
\begin{equation*}
\Xlet \,\, \ipat{p}_1 = e_1 \,\,
    \Xand \,\, \ipat{p}_2 = e_2 \, \ldots \,\,
    \Xand \,\, \ipat{p}_n = e_n \,\, \Xin \,\, e
\end{equation*}
is equivalent to
\begin{equation*}
\Xlet \,\, \ipat{p}_1, \ldots, \ipat{p}_n = e_1, \ldots, e_n \,\, \Xin \,\, e.
\end{equation*}
Similarly, let us introduce mutually recursive definitions:
\begin{equation*}
\Xlet \,\, \Xrec \,\, \ipat{p}_1 = e_1 \,\,
    \Xand \,\, \ipat{p}_2 = e_2 \, \ldots \,\, \Xand \,\, \ipat{p}_n =
      e_n \,\, \Xin \,\, e.
\end{equation*}
Furthermore, the sentence~\(e\) is equivalent to \(\Xlet \,\, \_ =
e\). Let us consider the case where the irrefutable patterns are
variables:
\begin{equation*}
\Xlet \,\, x = e_1 \,\, \Xand \,\, y = e_2 \,\, \Xin \,\,
e, \; \text{where} \, x \neq y.
\end{equation*}
If~\(x\) is free in~\(e_2\), that is, $x \in \mathcal{F} (e_2)$, then
the previous construct is defined as being equivalent to
\begin{center}
\begin{minipage}{0.3\linewidth}
\begin{tabbing}
\underline{\Xlet} \= \underline{$z$ \equal{} $x$ \Xin}\\
\> \Xlet{} $x$ \equal{} $e_1$ \Xin\\
\> \Xlet{} $y$ \equal{} \underline{\Xlet{} $x$ \equal{} $z$ \Xin} $e_2$\\
\Xin $e$
\end{tabbing}
\end{minipage}
\end{center}
where \(z \not\in \mathcal{F}(e_1) \cup \mathcal{F}(e_2) \cup
\mathcal{F}(e)\), in order to avoid being captured by~\(e_1\),
\(e_2\)~or~\(e\). It is thus not an elementary construct.

The multiple \Xlet{} \Xrec{} (with \Xand) can always be reduced to a
simple \Xlet{} \Xrec{} (with \Xin) by parameterising one of the
definitions by the other. Consider the following sentence
\begin{equation*}
\Xlet \; \Xrec \; x \; \equal \; e_1 \; \Xand \; y \; \equal \; e_2 \;
\Xin \; e
\end{equation*}
where $x \neq y$. It is equivalent, by definition, to
\begin{center}
\begin{minipage}{0.3\linewidth}
\begin{tabbing}
\Xlet{} \= \Xrec{} $x$ \equal{} \underline{\Xfun{} $y$
  $\rightarrow$} $e_1$ \Xin\\
\> \Xlet{} \Xrec{} $y$ \equal{} \underline{\Xlet{} $x$ \equal{} $x \,
y$ \Xin} $e_2$ \Xin\\
\> \underline{\Xlet{} $x$ \equal{} $x \, y$}\\
\underline{\Xin} $e$
\end{tabbing}
\end{minipage}
\end{center}
Therefore, it is not an elementary construct either.

The addition of new expressions to the language forces us to extend
the values as well, which are now defined by

\bigskip

\begin{tabular}{rll}
    $\bullet$
  & \emph{unit or constants}
  & \unit \ or \textsf{0} or \Xtrue{} etc.\\
    $\bullet$
  & \emph{closure}
  & $\langle\Xfun \,\, x \rightarrow e, \rho\rangle$, where~$\rho$ is
    an environment.\\
  & 
  & For the operators: $\langle\lpar{\texttt{+}}\rpar, \rho\rangle$
    etc.\\
    $\bullet$
  & \emph{$n$-tuple}
  & $v_1, \ldots, v_n$
\end{tabular}

\bigskip

\paragraph{Curryfied functions}

A function is \emph{curryfied} --~after the name of the logician
Curry~-- if it returns a closure, thus enabling partial
applications. By the way, let us not forget that an \OCaml function
always takes exactly one argument: if we wish to simultaneously pass
multiple values, we have to employ a data structure, for example, a
tuple. In the following, the sentences are input one by one into
the \OCaml toplevel, which runs on top of a virtual machine, like
the \Erlang shell does, therefore they are terminated with a special
token, two semicolons:

\bigskip

\noindent\topin{\textbf{let} add x y = x + y}

\noindent\topout{\textbf{val} add~:~int $\rightarrow$ int $\rightarrow$ int}

\noindent\topin{\textbf{let} add' (x,y) = x + y}

\noindent\topout{\textbf{val} add'~:~int $\times$ int $\rightarrow$ int}

\bigskip

\noindent The function \texttt{add} is curryfied and \texttt{add'} is
not.

Earlier, we saw a static analysis which yields the variables that are
free in a given expression. Moreover, the evaluation of a closed
expression cannot fail for lack of a necessary binding. Most compilers
reject open programs, but, by doing so, they sometimes reject innocent
programs, like \Xif{} \Xtrue{} \Xthen{} \num{1} \Xelse{} \ident{x},
where \ident{x} is unbound. In order to accept this kind of example,
the compiler should be able to predict the control flow, here, which
branch of the conditional is used for all possible evaluations. In the
case above, this is trivial, but, in general, the problem is
undecidable.

\section{Types}

Types are terms associated to expressions, either by the programmer or
the compiler, and they summarise some invariant properties which can
be automatically composed and checked. A type~$t$ is a term defined
recursively by the following cases:

\bigskip

\begin{tabular}{rll}
    $\bullet$
  & \emph{simple}
  & \textsf{char}, \textsf{bool}, \textsf{int}, \textsf{string},
    \textsf{float}, \textsf{unit}\\
    $\bullet$
  & \emph{Cartesian product}
  & $t_1 \times \ldots \times t_n$\\
    $\bullet$
  & \emph{functional}
  & $t_1 \rightarrow t_2$\\
    $\bullet$
  & \emph{parenthesised}
  & $(t)$\\
    $\bullet$
  & \emph{free variable}
  & $\alpha$, $\beta$, $\gamma$ etc. \\
    $\bullet$
  & \emph{parameterised}
  & $\alpha$ \textsf{list}
\end{tabular}

\bigskip

Until now, we have not encountered values of the type \textsf{float},
\textsf{char} or \textsf{string}, but they are fairly obvious.

We write $\times$, $\alpha$, $\beta$ etc. what is
written \textsf{*}, \textsf{'a}, \textsf{'b} etc. in \textsc{ascii}.

The Cartesian product takes multiple arguments, instead of being
binary like in mathematics, because $(\times)$~is not associative
in \OCaml: $t_1 \times t_2 \times t_3 \neq (t_1 \times t_2) \times
t_3 \neq t_1 \times (t_2 \times t_3)$.

The arrow is used in types too, where it is also
right\hyp{}associative: $t_1 \rightarrow
t_2 \rightarrow \ldots \rightarrow t_n$ is equivalent to
$t_1 \rightarrow (t_2 \rightarrow (\ldots (t_{n-1} \rightarrow
t_n)) \ldots)$.

The Cartesian product has priority over the arrow: $t_1 \times
t_2 \rightarrow t_3$ is equivalent to $(t_1 \times t_2) \rightarrow
t_3$.

The \OCaml compiler associates a type to each expression in a program:
this is called \emph{static type inference}. For the simple constants,
we have
\begin{center}
\begin{tabular}{l|l|c}
\toprule
\multicolumn{1}{c}{Type} & \multicolumn{1}{c}{Values}
& \multicolumn{1}{c}{Some functions}\\
\midrule
    \textsf{unit}
  & \unit
  & \\
    \textsf{bool}
  & \Xtrue \ \Xfalse
  & \texttt{\&\&} \ \texttt{||} \ \texttt{not} \\
    \textsf{int}
  & \textsf{1} \, \textsf{2} \ \textsf{max\_int} \ etc.
  & \texttt{+} \ \texttt{-} \ \texttt{*} \ \texttt{/} \ etc.\\
    \textsf{float}
  & \textsf{1.0} \, \textsf{2.} \ \textsf{1e4} \ etc.
  & \texttt{+.} \ \texttt{-.} \ \texttt{*.} \ \texttt{/.} \
    \texttt{cos} \ etc.\\
    \textsf{char}
  & \textsf{'a'} \ \textsf{'}\verb+\+\textsf{n'} \
    \textsf{'}\verb+\+\textsf{097'} \ etc.
  & \textsf{Char.code} \, \textsf{Char.chr} \ etc.\\
    \textsf{string}
  & \textsf{"a}\verb+\+\textsf{tb}\verb+\+\textsf{010c}\verb+\+\textsf{n"}
    \ etc.
  & \verb+(^)+ \, \textsf{s.[i]} \ etc.\\
\bottomrule
\end{tabular}
\end{center}
Operations on floating point numbers are written differently from
their equivalent counterparts on integers. Moreover, we extend the
syntax of the sentences to allow the binding of a type to a name,
or \emph{type aliasing}, just as we have aliasing on expressions:

\bigskip

\begin{tabular}{rll}
    $\bullet$
  & \emph{type binding}
  & \(\textsf{type} \,\, q = t\)\\
    $\bullet$
  & \emph{recursive types}
  & $\textsf{type} \,\, q_1 = t_1 \,\, \textrm{[}\Xand
     \,\, q_2 = t_2 \,\, \ldots\textrm{]}$\\
    $\bullet$
  & \emph{type variable}
  & $q$
\end{tabular}

\bigskip

\noindent We can now write, for instance, the following:
\begin{alltt}
\textbf{type} abscissa = float
\textbf{type} ordinate = float
\textbf{type} point = abscissa \(\times\) ordinate
\end{alltt}

\paragraph{Type inference}

Tuples contain data of the same type and their arity is part of their
type, for instance, the pair \textsf{(1,2)} has type \textsf{int}
$\times$ \textsf{int}, and the triplet \textsf{(1,2,3)} has type
\textsf{int} $\times$ \textsf{int} $\times$ \textsf{int}, thus
are incompatible. Consider also the following different types:

\bigskip

\noindent\topin{\textbf{let} middle x y = (x+y)/2}

\noindent\topout{\textbf{val} middle~:~int $\rightarrow$ int $\rightarrow$ int}

\noindent\topin{\textbf{let} middle (x,y) = (x+y)/2}

\noindent\topout{\textbf{val} middle~:~int $\times$ int $\rightarrow$ int}

\paragraph{Polymorphism}

Projections are polymorphic on tuples of same arity, so \Xfun $(x, y,
z) \rightarrow x$ has type $(\alpha \times \beta \times \gamma)
\rightarrow \alpha$. Consider the session:

\bigskip

\noindent\topin{\textbf{let} compose f g = \textbf{fun} x \(\rightarrow\) f (g x)}

\noindent\topout{\textbf{val} compose~: $(\alpha \rightarrow \beta) \rightarrow
  (\gamma \rightarrow \alpha) \rightarrow \gamma \rightarrow \beta =
  \topclos$}
\begin{alltt}
\textbf{\# let rec} power f n = \textbf{if} n \(\leqslant\) 0 \textbf{then fun} x \(\rightarrow\) x
                                \textbf{else} compose f (power f (n-1));;
\end{alltt}
\vspace{-4mm}
\noindent\topout{\textbf{val} power~:~($\alpha \rightarrow \alpha$)
  $\rightarrow$ int $\rightarrow$ ($\alpha \rightarrow \alpha$) =
  $\topclos$}

\bigskip

\noindent The type of the function \textsf{compose} is inferred as follows:
\begin{itemize}

  \item the first argument~\textsf{f} is an arbitrary function, hence
    of type \(\alpha \rightarrow \beta\);

  \item the second argument~\textsf{g} is a function whose result is
    the argument of~\textsf{f}, therefore of type~\(\alpha\);

  \item the domain of~\textsf{g} is arbitrary, hence \textsf{g}~has
    type \(\gamma \rightarrow \alpha\);

  \item the function \textsf{compose} has an argument~\textsf{x} which
    is passed to~\textsf{g}, hence has type~\(\gamma\); finally, the
    result of \textsf{compose} is returned by~\textsf{f}, hence of
    type~\(\beta\).

\end{itemize}
\noindent The equality operator in \OCaml is polymorphic and thus is
built in:

\noindent\topin{(=)}

\noindent\topout{- : $\alpha \rightarrow \alpha \rightarrow \texttt{bool} = \topclos$}

\bigskip

\noindent We must be very careful that it coincides with our
intended notion of equality, because the compiler will always accept
it on non\hyp{}functional values. It consists in the syntactical
equality: two values are equal if they have the same structure and if
their respective parts are equal, recursively. In particular, it will
not apply to functions --~this is different from
\Erlang, where the abstract syntax trees of the functions are compared
at run\hyp{}time.

\bigskip

\noindent\topin{1 = 1 \&\& [1;2] = [1;2] \&\& \str{yes} = \str{yes}}

\noindent\topout{- : bool = true}

\noindent\topin{(\textbf{fun} x \(\rightarrow\) x) = (\textbf{fun} x \(\rightarrow\) x)}

\error{Exception: Invalid\_argument "equal: functional value".}

%We note \texttt{<>} the negation of equality.


\section{Pattern matching}

Let us extend further the expressions with \emph{matchings} against
\emph{patterns}:
\begin{center}
  \Xmatch{} $e$ \Xwith{} $p_1 \rightarrow e_1 \mid \ldots \mid
    p_n \rightarrow e_n$, where the~$p_i$ are patterns.
\end{center}

\noindent Patterns are recursively defined by the following cases:

\bigskip

\begin{tabular}{rll}
    $\bullet$
  & \emph{variable}
  & $f, g, h$ (functions) and $x, y, z$ (others). \\
    $\bullet$
  & \emph{unit or constant}
  & \unit \ or \textsf{0} or \Xtrue{} etc.\\
    $\bullet$
  & \emph{$n$-tuple}
  & $p_1, \ldots, p_n$\\
    $\bullet$
  & \emph{parentheses}
  & $\lpar{p}\rpar$\\
    $\bullet$
  & \emph{underscore}
  & {\Large \_}
\end{tabular}

\bigskip

\noindent Note that irrefutable patterns are patterns. Typically,
pattern matching is used to define functions case by case, as commonly
done in mathematics. For example,
\begin{alltt}
\textbf{let rec} fib n = \textbf{match} n \textbf{with} 0 \(\rightarrow\) 1
                           | 1 \(\rightarrow\) 1
                           | \_ \(\rightarrow\) fib(n-1) + fib(n-2)
\end{alltt}
As in mathematics, the cases are ordered by the order of writing and
the previous definition is read as follows: `If the value
of~\texttt{n} has the shape~\texttt{0}, then \texttt{fib(n)} has the
value~\texttt{1}; else, if the value of~\texttt{b} has the
shape~\texttt{1}, then \texttt{fib(n)} has the value~\texttt{1};
otherwise, \texttt{fib(n)} has the value of \texttt{fib(n-1) +
fib(n-2)}.' What is the meaning of the relation `\(v\)~has the
shape~\(p\)'?  Or, equivalently, `\(p\)~matches~\(v\)'?
\begin{itemize}

  \item A constant, including \unit, has the shape of itself in a
    pattern;

  \item a \(n\)-tuple has the shape of a \(n\)-tuple in a pattern;

  \item any value has the shape of a variable in a pattern or
    of the underscore `{\Large \_}'.

\end{itemize}
Note that patterns do not match closures, which means that the~\(e\)
in `\Xmatch{} $e$ \Xwith{}' must not have a closure as a value. In the
case of constants, including \unit, matching coincides with equality.

Matching is the evaluation of a \Xmatch construct. Informally, the
evaluation of `\Xmatch{} $e$ \Xwith{} $p_1 \rightarrow e_1 \mid
\ldots \mid p_n \rightarrow e_n$' begins with that of~$e$
in~$v$. Next, \(v\)~is compared to the patterns~\(p_i\) in the order
of writing. If~\(p_i\) is the first pattern to match~\(v\), then the
value of the matching is the value of~\(e_i\). For instance, here is
the curryfied definition of the logical disjunction:
\begin{alltt}
\textbf{let} disj = \textbf{fun} (x,y) \(\rightarrow\) \textbf{match} (x,y) \textbf{with} (\textbf{false}, \textbf{false}) \(\rightarrow\) \textbf{false}
                                       | \_ \(\rightarrow\) \textbf{true}
\end{alltt}

\paragraph{Variant types}

Variant types are a generalisation of enumerations. For instance,
Boolean values can be (re)defined as
\begin{alltt}
\textbf{type} boolean = True | False
\textbf{let} t = True \textbf{and} f = False
\end{alltt}
Note that data constructors, like \texttt{True} or
\texttt{False}, must begin with an uppercase letter. Pattern matching
enables the examination of the values of a variant type:
\begin{alltt}
\textbf{let} int\_of\_boolean b = \textbf{match} b \textbf{with} True  \(\rightarrow\) 1 | False \(\rightarrow\) 0
\end{alltt}
Data constructors can also carry information beyond their mere name,
for instance, a pack of playing cards can be defined as
\begin{alltt}
\textbf{type} card = Card \textbf{of} ordinary | Joker
\textbf{and} ordinary = suit \(\times\) face
\textbf{and} suit = Heart | Diamond | Spade | Club
\textbf{and} face = Ace | King | Queen | Jack | Simple \textbf{of} int
\end{alltt}

\noindent Let us define the cards and functions creating them:

\bigskip

\noindent\topin{\textbf{let} jack\_of\_spade = Card (Spade, Jack)}

\noindent\topout{\textbf{val} jack\_of\_spade : card = Card (Spade, Jack)}

\medskip

\noindent\topin{\textbf{let} card f c = Card (c,f)}

\noindent\topout{\textbf{val} card : face $\rightarrow$ suit $\rightarrow$ card
  = $\topclos$}

\medskip

\noindent\topin{\textbf{let} king = card King}

\noindent\topout{\textbf{val} king : suit $\rightarrow$ card = $\topclos$}

\begin{alltt}
\textbf{let} value c = \textbf{match} c \textbf{with} Card Ace        \(\rightarrow\) 14
                         | Card King       \(\rightarrow\) 13
                         | Card Queen      \(\rightarrow\) 12
                         | Card Jack       \(\rightarrow\) 11
                         | Card (Simple k) \(\rightarrow\) k
                         | Joker           \(\rightarrow\) 0
\end{alltt}
A pattern can capture several cases at once:
\begin{alltt}
\textbf{let} is\_smaller c = \textbf{match} c \textbf{with} Card (Simple \_) \(\rightarrow\) \textbf{true}
                              | \_ \(\rightarrow\) \textbf{false}
\end{alltt}

\paragraph{Incomplete patterns}

We have already seen that, when a value~\(v\) is matched with \(p_1
\rightarrow e_1 \mid \ldots \mid p_n \rightarrow e_n\), the
expression~\(e_i\) of the first pattern~\(p_i\) which matches~\(v\) is
evaluated in the current environment extended with the bindings
introduced by~\(p_i\), if any. The pattern matching is incomplete if
there exists at least a value which cannot be matched by any
pattern. In that case, a warning is printed at compile\hyp{}time
because it is strongly advised to avoid incomplete pattern matchings.

\bigskip

\noindent\topin{\textbf{let} simple c = \textbf{match} c \textbf{with} Card (\_, Simple k) $\rightarrow$ k}

\noindent\error{Characters 15-51}

\noindent\error{Warning: this pattern-matching is not exhaustive.}

\noindent\error{Here is an example of a value that is not matched:
  Joker}

\noindent\topout{\textbf{val} simple : card $\rightarrow$ int = $\topclos$}

\bigskip

\paragraph{Linearity}

Contrary to \Erlang, a variable cannot be bound twice or more in the
same pattern, because this would imply a default equality to be used
during pattern matching. Such a pattern is said to be
\emph{non\hyp{}linear}.

\medskip

\noindent\topin{\textbf{fun} (x,y) $\rightarrow$ \textbf{match} (x,y) \textbf{with} (Card z, z) $\rightarrow$ \textbf{true}}

\noindent\error{Characters 40-41}

\noindent\error{This variable is bound several times in this
  matching.}

\paragraph{Lists}

Lists can be defined as a polymorphic, recursive variant type:
\begin{alltt}
\textbf{type} \(\alpha\) mylist = Nil | Cons \textbf{of} \(\alpha\) \(\times\) \(\alpha\) mylist
\end{alltt}
The data constructors \texttt{Nil} and \texttt{Cons} are traditional
in the community of functional languages. The former denotes the empty
list; the latter represents a non\hyp{}empty list. A non\hyp{}empty
list is then modelled as a pair whose first component is an element
(of type~\(\alpha\)) of the list and the second component is the
remainder of the list, or a sublist (of
type \(\alpha\)~\texttt{mylist}). For instance:
\begin{alltt}
\textbf{let} empty= Nil
\textbf{let} singleton1 = Cons ('a', empty)
\textbf{let} singleton2 = Cons (7, empty)
\textbf{let} long = Cons (1, Cons (2, Cons (3, singleton2)))
\end{alltt}
By default, the \OCaml system predefines a
type \(\alpha\)~\texttt{list}, whose constructor for the empty list
is~\texttt{[]}, and those of the non\hyp{}empty list is~\texttt{::},
used in infix position. The function catenating two lists is also
predefined and noted~\texttt{(@)}. Note that it is \emph{not} a
constructor. Here are equivalent lists:
\begin{alltt}
\textbf{let} l = 1::(2::3::(4::[]))
\textbf{let} l = 1::2::3::4::[]
\textbf{let} l = [1;2;3;4]
\end{alltt}
The standard library supplies a module \textsf{List} with useful
functions on lists. For example, the reversal of a list:
\begin{alltt}
\textbf{let} \textbf{rec} reverse l = \textbf{match} l \textbf{with}
                      [] \(\rightarrow\) []
                    | h::l \(\rightarrow\) (reverse l) @ [h]
\end{alltt}
or, more efficiently, using an accumulator:
\begin{alltt}
\textbf{let} reverse l = 
  \textbf{let rec} rev\_aux acc l = \textbf{match} l \textbf{with}
                            [] \(\rightarrow\) acc
                          | h::t \(\rightarrow\) rev\_aux (h::acc) t
\textbf{in} rev\_aux [] l
\end{alltt}

\section{Exceptions}

Exceptions are values of the same type \textsf{exn}. They behave in
the following manner.
\begin{itemize*}

  \item The predefined type \textsf{exn} is an \emph{open} variant
    type, that is, it is extensible;

  \item throwing an exception stops the current evaluation and returns
    an exceptional value of type \textsf{exn};

  \item an exception can be matched if the expression where it was
    raised is embedded in a construct `\textsf{\textbf{try}
    $e$ \textbf{with} $m$}', or if any pending function call was
    embedded in such a way.
    \begin{itemize*}
 
      \item If the evaluation of~\(e\) results in a normal value, that
        value is the value of the \textsf{\textbf{try}} block, without
        considering the patterns in~\(m\);

      \item otherwise, the exception is matched against the patterns
        in~\(m\); if one of those patterns, say~\(p_i\), matches the
        exception, then \(e_i\)~is evaluated, else the exception is
        propagated backwards in the control flow (call graph) until a
        handler matches it or the execution stops with an error. Note
        that the patterns for exceptions need not be complete, whence
        a source of potential problems because the programmer has to
        keep track of the set of exceptions potentially raised by a
        given expression. (Contrary to \Java, exceptions in \OCaml are
        no part of the type of functions, because this would be deemed
        cumbersome in the presence of higher\hyp{}order functions,
        like \textsf{List.map}.)
    
    \end{itemize*}

  \item It is possible to observe an exception, that is, to match it
    and reraise it: \texttt{\textbf{try} f x \textbf{with} Failure
    s \textbf{as} x $\rightarrow$ prerr\_string s; raise x}.

\end{itemize*}
Let us consider an example, the linear search in a list:
\begin{alltt}
\textbf{exception} Not_found

\textbf{let rec} find key l =
  \textbf{match} l \textbf{with}
    (h,v)::t \(\rightarrow\) \textbf{if} h = key \textbf{then} v \textbf{else} find key t
  |       [] \(\rightarrow\) raise Not_found
  
\textbf{let} k = \textbf{try}
          find "Louis" [("Georges",14); ("Louis",5)]
        \textbf{with} Not_found \(\rightarrow\) 10
\end{alltt}

\noindent Formally, we write

\bigskip

\begin{tabular}{ll}
\toprule
Definition (sentence) & \phrase{\textbf{exception} $C$ [\textbf{of} $t$]}\\
\midrule
Raising (expression) & \phrase{raise $e$}\\
Matching (expression) & \phrase{\textbf{try} $e$ \textbf{with}
$p_1 \rightarrow e_1 \mid \ldots \mid p_n \rightarrow e_n$}\\
\bottomrule
\end{tabular}

\bigskip

\noindent Notice the analogy with the pattern matching on
  non\hyp{}exceptional values.
