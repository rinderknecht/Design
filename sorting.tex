\section{Optimal sorting}
\label{sec:opt_sort}
\index{sorting}

Sorting is the process of rearranging a series of objects, called
\emph{keys}\index{sorting!key}, to conform with a predefined
order. According to \cite{Knuth_1998}, the first sorting algorithms
were invented and automated as tabulating machines in the late
nineteenth century, in order to support the establishment of the
census of the United States of America.

\paragraph{Permutations}
\label{par:permutations}
\index{permutation}

We saw \vpageref{par:mean_sort} that the average cost of a
comparison\hyp{}based sorting algorithm is defined as the arithmetic
mean of the costs of sorting all permutations of a given length. A
permutation of \((1,2,\dots,n)\) is another tuple
\((a_1,a_2,\dots,a_n)\) such that \(a_i \in \{1,\dots,n\}\) and \(a_i
\neq a_j\) for all~\(i \neq j\). For example, all the permutations
of~\((1,2,3)\) are
\begin{equation*}
(1,2,3) \quad (1,3,2) \quad (2,1,3) \quad (2,3,1) \quad (3,1,2) \quad
(3,2,1).
\end{equation*}
Given all the permutations of~\((1,2,\dots,n-1)\), let us build
inductively all the permutations of
\((1,2,\dots,n)\). If~\((a_1,a_2,\dots,a_{n-1})\) is a permutation of
\((1,2,\dots,n-1)\), then we can construct \(n\)~permutations
of~\((1,2,\dots,n)\) by inserting~\(n\) at all possible places
in~\((a_1,a_2,\dots,a_{n-1})\):
\begin{equation*}
(\boldsymbol{n},a_1,a_2,\dots,a_{n-1})\quad
(a_1,\boldsymbol{n},a_2,\dots,a_{n-1})\quad \ldots \quad
(a_1,a_2,\dots,a_{n-1},\boldsymbol{n}).
\end{equation*}
For example, it is obvious that all the permutations of \((1,2)\) are
\((1,2)\) and \((2,1)\). The method leads from \((1,2)\) to
\((\boldsymbol{3},1,2)\), \((1,\boldsymbol{3},2)\) and
\((1,2,\boldsymbol{3})\); and from \((2,1)\) to
\((\boldsymbol{3},2,1)\), \((2,\boldsymbol{3},1)\) and
\((2,1,\boldsymbol{3})\). If we name~\(p_n\) the number of
permutations on \(n\)~elements, we draw from this the recurrence \(p_n
= n \cdot p_{n-1}\), which, with the additional obvious \(p_1 = 1\),
leads to \({p_n = n!}\), for all \({n > 0}\), exactly as expected. If
the \(n\)~objects to permute are not \((1,2,\dots,n)\) but, for
example, \((\textsf{b},\textsf{d},\textsf{a},\textsf{c})\), simply
associate each of them to their index in the tuple, for example,
\textsf{b}~is represented by~\(1\), \textsf{d}~by~\(2\),
\textsf{a}~by~\(3\) and \textsf{c}~by~\(4\), so the tuple is then
associated to \((1,2,3,4)\) and, for instance, the permutation
\((4,1,2,3)\) means~\((\textsf{c},\textsf{b},\textsf{d},\textsf{a})\).

\paragraph{Factorial}\index{factorial}

We encountered the factorial function in the introduction and here
again. There is a simple derivation enabling the characterisation of
its asymptotic growth, proposed by
\cite{GrahamKnuthPatashnik_1994}. We start by squaring the factorial
and regrouping the factors as follows:
\begin{equation*}
n!^2 = (1 \cdot 2 \cdot \ldots \cdot n) (n \cdot \ldots \cdot 2 \cdot
1) = \prod_{k=1}^{n}{k(n+1-k)}.
\end{equation*}
The parabola \(P(k) := k(n+1-k) = -k^2 + (n+1)k\) reaches its maximum
where its derivative is zero: \(P'(k_{\max}) = 0 \Leftrightarrow
k_{\max}=(n+1)/2\). The corresponding ordinate is \(P(k_{\max}) =
((n+1)/2)^2 = k_{\max}^2\). When \(k\)~ranges from~\(1\) to~\(n\), the
minimal ordinate, \(n\), is reached at absissas \(1\)~and~\(n\), as
shown in \fig~\vref{fig:parabola}.
\begin{figure}
\centering
\includegraphics[bb=55 563 205 730]{parabola}
\caption{Parabola \(P(k) := k(n+1-k)\)}
\label{fig:parabola}
\end{figure}
Hence, \(1 \leqslant k \leqslant k_{\max}\) implies
\begin{equation*}
P(1) \leqslant P(k) \leqslant P(k_{\max}),\quad \text{that is,} \quad
n \leqslant k(n+1-k) \leqslant \left(\frac{n+1}{2}\right)^2.
\end{equation*}
Multiplying the sides by varying~\(k\) over the discrete interval
\([1..n]\) yields
\begin{gather*}
n^n = \prod_{k=1}^{n}{n} \leqslant n!^2
\leqslant
\prod_{k=1}^{n}{\left(\!\frac{n+1}{2}\!\right)^2} \!\!=
\left(\!\frac{n+1}{2}\!\right)^{2n}
\!\!\!\Rightarrow\!
n^{n/2} \leqslant n! \leqslant \left(\!\frac{n+1}{2}\!\right)^n.
\end{gather*}
It is clear now that \(n!\)~is \emph{exponential}, so it
asymptotically outgrows any polynomial. Concretely, a function whose
cost is proportional to a factorial is useless even for small
inputs. For the cases where an equivalence is preferred, Stirling's
formula\index{Stirling's formula} states that
\begin{equation}
n! \sim n^n e^{-n} \sqrt{2\pi n}.\label{eq:Stirling}
\end{equation}

\paragraph{Enumerating all permutations}

Let us write a program computing all the permutations of a given
stack. We define the function~\fun{perm/1}\index{perm@\fun{perm/1}}
such that \(\fun{perm}(s)\) is the stack of all permutations of the
items in stack~\(s\). We implement the inductive method presented
above, which worked by inserting a new object into all possible places
of a shorter permutation.
\begin{equation*}
\fun{perm}(\el)         \xrightarrow{\smash{\alpha}} \el;\quad
\fun{perm}([x])         \xrightarrow{\smash{\beta}} [[x]];\quad
\fun{perm}(\cons{x}{s}) \xrightarrow{\gamma}
                          \fun{dist}(x,\fun{perm}(s)).
\end{equation*}
The function~\fun{dist/2}\index{dist@\fun{dist/2}} (\emph{distribute})
is such that \(\fun{dist}(x,s)\) is the stack of all stacks obtained
by inserting the item~\(x\) at all different places in
stack~\(s\). Because such an insertion into a permutation of
length~\(n\) yields a permutation of length~\(n+1\), we must join the
new permutations to the previously found others of same length:
\begin{equation*}
\fun{dist}(x,\el)         \xrightarrow{\smash{\delta}} \el;\quad
\fun{dist}(x,\cons{p}{t}) \xrightarrow{\smash{\epsilon}}
                            \fun{cat}(\fun{ins}(x,p),\fun{dist}(x,t)).
\end{equation*}
The call~\(\fun{ins}(x,p)\)\index{ins@\fun{ins/2}} computes the stack
of permutations resulting from inserting~\(x\) at all places in the
permutation~\(p\). We thus derive
\begin{equation*}
\fun{ins}(x,\el) \xrightarrow{\smash{\zeta}} [[x]];\quad
\fun{ins}(x,\cons{j}{s}) \xrightarrow{\smash{\eta}}
 \cons{\cons{x,j}{s}}{\fun{push}(j,\fun{ins}(x,s))}.
\end{equation*}
where the function~\fun{push/2}\index{push@\fun{push/2}} (not to be
confused with the function of same name and arity in
section~\ref{sec:persistence}) is such that any
call~\(\fun{push}(x,t)\) pushes item~\(x\) on all the permutations of
the stack of permutations~\(t\). The order is left unchanged:
\begin{equation*}
\fun{push}(x,\el) \xrightarrow{\smash{\theta}} \el;\quad
\fun{push}(x,\cons{p}{t}) \xrightarrow{\smash{\iota}}
 \cons{\cons{x}{p}}{\fun{push}(x,t)}.
\end{equation*}
Now we can compute all the permutations of \((4,1,2,3)\) or
\((\fun{c}, \fun{b}, \fun{d}, \fun{a})\) by calling
\(\fun{perm}([4,1,2,3])\) or \(\fun{perm}([\fun{c}, \fun{b}, \fun{d},
  \fun{a}])\). Note that, after computing the permutations of
length~\(n+1\), the permutations of length~\(n\) are not needed
anymore, which would allow an implementation to reclaim the
corresponding memory for further uses (a process called \emph{garbage
  collection}\index{garbage collection}). As far as the costs are
concerned, the definition of \fun{push/2} yields
\begin{equation*}
\C{\fun{push}}{0} \eqn{\theta} 1;\qquad
\C{\fun{push}}{n+1} \eqn{\iota} 1 + \C{\fun{push}}{n},\quad
\text{with \(n \geqslant 0\)}.
\end{equation*}
We easily deduce \(\C{\fun{push}}{n} = n +
1\).\index{push@$\C{\fun{push}}{n}$} We know that the result of
\(\fun{ins}(x,p)\) is a stack of length~\(n+1\) if~\(p\) is a
permutation of \(n\)~objects into which we insert one more
object~\(x\). Hence, the definition of \fun{ins/2} leads to
\begin{equation*}
\C{\fun{ins}}{0}   \eqn{\zeta} 1;\quad
\C{\fun{ins}}{k+1} \eqn{\eta} 1 + \C{\fun{push}}{k+1} +
\C{\fun{ins}}{k} = 3 + k + \C{\fun{ins}}{k}, \quad \text{with \(k
                   \geqslant 0\)},
\end{equation*}
where~\(\C{\fun{ins}}{k}\) is the cost of \(\fun{ins}(x,p)\)
with~\(p\) of length~\(k\). By summing for all~\(k\) from
\(0\)~to~\(n-1\), for~\(n>0\), on both sides we draw
\begin{equation*}
\sum_{k=0}^{n-1}{\C{\fun{ins}}{k+1}}
  = \sum_{k=0}^{n-1}{3} + \sum_{k=0}^{n-1}{k}
     + \sum_{k=0}^{n-1}{\C{\fun{ins}}{k}}.
\end{equation*}
By cancelling identical terms in the sums (telescoping)
\(\sum_{k=0}^{n-1}{\C{\fun{ins}}{k+1}}\) and
\(\sum_{k=0}^{n-1}{\C{\fun{ins}}{k}}\), we draw\index{ins@$\C{\fun{ins}}{n}$}
\begin{equation*}
\C{\fun{ins}}{n}
  = 3n + \tfrac{1}{2}n(n-1) + \C{\fun{ins}}{0}
  = \tfrac{1}{2}{n^2} + \tfrac{5}{2}{n} + 1.
\end{equation*}
This last equation is actually valid even if \(n =
0\). Let~\(\C{\fun{dist}}{n!}\) be the cost for distributing an item
among \(n!\)~permutations of length~\(n\). The definition of
\fun{dist/2}\index{dist@\fun{dist/2}} shows that it repeats calls to
\fun{cat/2}\index{cat@\fun{cat/2}} and
\fun{ins/2}\index{ins@\fun{ins/2}} whose arguments are always of
length \(n+1\)~and~\(n\), respectively, because all processed
permutations here have the same length. We deduce, for~\(k \geqslant
0\), that
\begin{equation*}
\C{\fun{dist}}{0} \eqn{\delta} 1;\quad
\C{\fun{dist}}{k+1}
  \eqn{\epsilon} 1 + \C{\fun{cat}}{n+1} + \C{\fun{ins}}{n}
                    + \C{\fun{dist}}{k}
  = \tfrac{1}{2}{n^2} + \tfrac{7}{2}{n} + 4 + \C{\fun{dist}}{k},
\end{equation*}
since we already know that \(\C{\fun{cat}}{n} = n +
1\)\index{cat@$\C{\fun{cat}}{n}$} and the value
of~\(\C{\fun{ins}}{n}\). By summing both sides of the last equation
for all~\(k\) from \(0\)~to~\(n!-1\), we can eliminate most of the
terms and find a non\hyp{}recursive definition of
\(\C{\fun{dist}}{n!}\)\index{dist@$\C{\fun{dist}}{n}$}:
\begin{align*}
\sum_{k=0}^{n!-1}{\C{\fun{dist}}{k+1}}
   &= \sum_{k=0}^{n!-1}{\left(\frac{1}{2}{n^2}
      + \frac{7}{2}{n} + 4\right)}
      + \sum_{k=0}^{n!-1}{\C{\fun{dist}}{k}},\\
\C{\fun{dist}}{n!}
  &= \left(\frac{1}{2}{n^2}+\frac{7}{2}{n}+4\right)n! +
\C{\fun{dist}}{0} = \frac{1}{2}(n^2 + 7n + 8)n! + 1.
\end{align*}
Let us finally compute the cost of \(\fun{perm}(s)\), noted
\(\C{\fun{perm}}{k}\)\index{perm@$\C{\fun{perm}}{n}$|(}, where \(k\)~is
the length of the stack~\(s\). From rules \clause{\alpha}
to~\clause{\gamma}, we deduce the following equations, where \(k >
0\):
\begin{align*}
\C{\fun{perm}}{0}   &\eqn{\alpha} 1;\qquad
\C{\fun{perm}}{1}   \eqn{\beta} 1;\\
\C{\fun{perm}}{k+1}
  &\eqn{\gamma} 1 + \C{\fun{perm}}{k} + \C{\fun{dist}}{k!}
   = \tfrac{1}{2}(k^2 + 7k + 8)k! + 2 + \C{\fun{perm}}{k}.
\intertext{Again, summing both sides, most of the terms cancel out:}
\sum_{k=1}^{n-1}{\C{\fun{perm}}{k+1}}
  &= \frac{1}{2}\sum_{k=1}^{n-1}{(k^2+7k+8)k!} + \sum_{k=1}^{n-1}{2}
     + \sum_{k=1}^{n-1}{\C{\fun{perm}}{k}},\\
\C{\fun{perm}}{n}
  &= \frac{1}{2}\sum_{k=1}^{n-1}{(k^2 + 7k + 8)k!}
     + 2(n-1) + \C{\fun{perm}}{1}\\
  &= \frac{1}{2}\sum_{k=1}^{n-1}{((k+2)(k+1)+6+4k)k!} + 2n - 1\\
  &= \frac{1}{2}\sum_{k=1}^{n-1}{(k+2)(k+1)k!}
     + 3 \sum_{k=1}^{n-1}{k!} + 2 \sum_{k=1}^{n-1}{kk!} + 2n - 1\\
  &= \frac{1}{2}\sum_{k=1}^{n-1}{(k+2)!}
     + 3 \sum_{k=1}^{n-1}{k!} + 2 \sum_{k=1}^{n-1}{kk!} + 2n - 1\\
  &= \frac{1}{2}\sum_{k=3}^{n+1}{k!}
     + 3 \sum_{k=1}^{n-1}{k!} + 2 \sum_{k=1}^{n-1}{kk!} + 2n - 1\\
%  &= \left(\frac{1}{2}((n+1)! + n! - 2! - 1!)
%     + \frac{1}{2}\sum_{k=1}^{n-1}{k!}\right)
%     + 3 \sum_{k=1}^{n-1}{k!}\\
%  &\phantom{= x} + 2 \sum_{k=1}^{n-1}{kk!} + 2n - 1.\\
  &= \frac{1}{2}{(n+2)n!} + \frac{7}{2}\sum_{k=1}^{n-1}{k!}
     + 2 \sum_{k=1}^{n-1}{kk!} + 2n - \frac{5}{2}.
\end{align*}
This last equation is actually valid even if~\(n = 1\). One sum has a
simple closed form:
\begin{equation*}
\sum_{k=1}^{n-1}{kk!} = \sum_{k=1}^{n-1}((k+1)! - k!) =
\sum_{k=2}^{n}{k!} - \sum_{k=1}^{n-1}{k!} = n! - 1.
\end{equation*}
Resuming our previous derivation,
\begin{align*}
\C{\fun{perm}}{n}
  &= \frac{1}{2}{nn!} + n! + \frac{7}{2}\sum_{k=1}^{n-1}{k!}
     + 2(n! - 1) + 2n - \frac{5}{2}\\
  &= \frac{1}{2}{nn!} + 3n!
     + 2n - \frac{9}{2} + \frac{7}{2}\sum_{k=1}^{n-1}{k!},\quad
     \text{with \(n > 0\)}.
\end{align*}
The remaining sum is called the \emph{left factorial}
\citep{Kurepa_1971}\index{factorial!left $\sim$} and is usually
defined as
\begin{equation*}
!n := \sum_{k=1}^{n-1}{k!},\quad \text{with \(n > 0\)}.
\end{equation*}
Unfortunately, no closed expression of the left factorial is
known. This is actually a common situation when determining the cost
of relatively complex functions. The best course of action is then to
study the asymptotic approximation of the cost. Obviously, \(n!
\leqslant !(n+1)\). Also,
\begin{equation*}
!(n+1) - n! \leqslant (n-2) \cdot (n-2)! + (n-1)! \leqslant
(n-1) \cdot (n-2)! + (n-1)! = 2 (n-1)!
\end{equation*}
Therefore,
\begin{equation*}
1 \leqslant \frac{!(n+1)}{n!} \leqslant \frac{n! + 2(n-1)!}{n!} = 1 +
\frac{2}{n} \;\Rightarrow\; !n \sim (n-1)!
\end{equation*}
Also, \((n+1)! = (n+1)n!\), so \((n+1)!/(nn!) = 1 + 1/n\), hence \(nn!
\sim (n+1)!\). Consequently,
\begin{equation*}
\abovedisplayshortskip=0pt
\abovedisplayskip=0pt
%\belowdisplayskip=0pt
\C{\fun{perm}}{n} \sim \frac{1}{2}(n+1)! + 3n! + \frac{7}{2}(n-1)!
+ 2n - \frac{9}{2} \sim \frac{1}{2}(n+1)!
\end{equation*}
This is an unbearably slow program, as expected. We should not hope to
compute \(\C{\fun{perm}}{11}\) \index{perm@$\C{\fun{perm}}{n}$|)}
easily and there is no way to improve significantly the cost because
the number of permutations it computes is inherently exponential, so
it would even suffice to spend only one function call per permutation
to obtain an exponential cost. In other words, the memory necessary to
hold the result has a size which is exponential in the size of the
input, therefore, the cost is at least exponential, because at least
one function call is necessary to allocate some memory. For a deep
study on the enumeration of all permutations, refer to the survey of
\cite{Knuth_2011}.

\paragraph{Permutations and sorting}

Permutations\index{permutation|(} are worth studying in detail because
of their intimate relationship with sorting. A permutation can be
thought of as scrambling originally ordered keys and a sorting
permutation puts them back to their place. A slightly different
notation for permutations is helpful here, one which shows the indexes
together with the keys. For example, instead of writing \(\pi_1 =
(2,4,1,5,3)\), we write
\begin{equation*}
\pi_1 =
\begin{pmatrix}
1 & 2 & 3 & 4 & 5\\
2 & 4 & 1 & 5 & 3
\end{pmatrix}.
\end{equation*}
The first line is made of the ordered indexes and the second line
contains the keys. In general, a permutation \(\pi =
(a_1,a_2,\dots,a_n)\) is equivalent to
\begin{equation*}
\pi =
\begin{pmatrix}
     1 &      2 & \dots &     n\\
\pi(1) & \pi(2) & \dots & \pi(n)
\end{pmatrix},
\end{equation*}
where~\(a_i = \pi(i)\), for all~\(i\) from \(1\)~to~\(n\). The
following permutation~\(\pi_s\) sorts the keys of~\(\pi_1\):
\begin{equation*}
\pi_s =
\begin{pmatrix}
1 & 2 & 3 & 4 & 5\\
3 & 1 & 5 & 2 & 4
\end{pmatrix}.
\end{equation*}
To see why, we define the composition\index{permutation!composition}
of two permutations\index{permutation} \(\pi_a\)~and~\(\pi_b\):
\begin{equation*}
\pi_b \circ \pi_a :=
\begin{pmatrix}
              1 &               2 & \dots & n\\
\pi_b(\pi_a(1)) & \pi_b(\pi_a(2)) & \dots & \pi_b(\pi_a(n))
\end{pmatrix}.
\end{equation*}
Then \(\pi_s~\circ~\pi_1 = \mathcal{I}\), where the \emph{identity
  permutation}~\(\mathcal{I}\)\index{permutation!identity} is such
that \(\mathcal{I}(k) = k\), for all indexes~\(k\). In other words,
\(\pi_s = \pi_1^{-1}\), that is, \emph{sorting a permutation consists
  in building its inverse}\index{permutation!inverse}:
\begin{equation*}
\begin{pmatrix}
1 & 2 & 3 & 4 & 5\\
3 & 1 & 5 & 2 & 4
\end{pmatrix}
\circ
\begin{pmatrix}
1 & 2 & 3 & 4 & 5\\
2 & 4 & 1 & 5 & 3
\end{pmatrix}
=
\begin{pmatrix}
1 & 2 & 3 & 4 & 5\\
1 & 2 & 3 & 4 & 5
\end{pmatrix}.
\end{equation*}

An alternative representation of permutations and their
composition\index{permutation!composition} is based on considering
them as bijections from an interval onto itself, denoted by
\emph{bipartite graphs}\index{graph!bipartite
  $\sim$}\index{permutation!bigraph}, also called
\emph{bigraphs}\index{bigraph|see{graph, bipartite}}. Such
graphs are made of two disjoint, ordered sets of vertices of same
cardinal, the indexes and the keys, and the edges always go from an
index to a key, without sharing the vertices with other edges. For
example, permutation~\(\pi_1\) is shown in \fig~\vref{fig:perm1}
\begin{figure}
\centering
\subfloat[Bigraph of \(\pi_1 = (2,4,1,5,3)\)\label{fig:perm1}]{%
\includegraphics[bb=79 662 206 721]{perm1}
}
\qquad\qquad
\subfloat[Bigraph of \(\pi_1^{-1} = (3,1,5,2,4)\)\label{fig:perm3}]{%
\includegraphics[bb=77 662 208 721]{perm3}
}
\caption{Permutation \(\pi_1\) and its inverse \(\pi_1^{-1}\)}
\label{fig:pi_1}
\index{permutation!inverse}
\end{figure}
and its inverse \(\pi_1^{-1}\) is displayed in
\fig~\vref{fig:perm3}. The composition\index{permutation!composition}
of~\(\pi_1^{-1}\) and~\(\pi_1\) is then obtained by identifying the
key vertices of~\(\pi_1\) with the index vertices of~\(\pi_1^{-1}\),
as shown in \fig~\vref{fig:perm4}.
\begin{figure}[t]
\centering
\subfloat[\(\pi_1^{-1} \circ \pi_1\)\label{fig:perm4}]{%
\includegraphics[bb=81 634 204 721]{perm4}
}%
\qquad\qquad
\subfloat[\(\mathcal{I} = \pi_1^{-1} \circ \pi_1\)\label{fig:perm2}]{%
\includegraphics[bb=81 634 204 721]{perm2}
}
\caption{Applying \(\pi_1\) to \(\pi_1^{-1}\).}
\end{figure}
The identity permutation\index{permutation!identity} is obtained by
replacing two adjacent edges by their transitive
closure\index{transitive closure} and erasing the intermediate
vertices, as shown in \fig~\ref{fig:perm2}
page~\pageref{fig:perm2}. Note that a permutation may equal its
inverse, like
\begin{equation*}
\pi_3 =
\begin{pmatrix}
1 & 2 & 3 & 4 & 5\\
3 & 4 & 1 & 2 & 5
\end{pmatrix}.
\end{equation*}
In \fig~\ref{fig:involution}
\begin{figure}
\centering
\subfloat[\(\pi_3 \circ \pi_3\)\label{fig:inv1}]{%
\includegraphics[bb=81 634 204 721]{inv1}
}
\qquad\qquad
\subfloat[\(\pi_3 \circ \pi_3\)]{%
\includegraphics[bb=81 634 204 721]{inv2}
}
\caption{Involution \(\pi_3\) sorts itself}
\label{fig:involution}
\end{figure}
is shown that \(\pi_3~\circ~\pi_3 = \pi_3\), so \(\pi_3\) is an
\emph{involution}\index{permutation!involution}\index{permutation!inverse}.

Studying permutations and their basic properties helps understanding
sorting algorithms, particularly their average cost. They also provide
a way to quantify disorder. Given \((1,3,5,2,4)\), we can see that
only the pairs of keys \((3,2)\), \((5,2)\) and~\((5,4)\) are out of
order. In general, given \((a_1, a_2, \dots, a_n)\), the pairs
\((a_i,a_j)\) such that~\(i < j\) and~\(a_i > a_j\) are called
\emph{inversions}\index{permutation!inversion}. The more inversions,
the greater the disorder. As expected, the identity
permutation\index{permutation!identity} has no inversions and the
previously studied permutation \(\pi_1 = (2,4,1,5,3)\) has~\(4\). When
considering permutations as represented by
bigraphs\index{graph!bipartite $\sim$}\index{permutation!bigraph}, an
inversion\index{permutation!inversion} corresponds to an intersection
of two edges, more precisely, it is the pair made of the keys pointed
at by two arrows. Therefore, the number of
inversions\index{permutation!inversion} is the number of edge
crossings, so, for instance, \(\pi_1^{-1}\)~has \(4\)~inversions. In
fact, \emph{the inverse\index{permutation!inverse} of a permutation
  has the same number of inversions\index{permutation!inversion} as
  the permutation\index{permutation} itself}. This can be clearly seen
when comparing the bigraphs\index{permutation!bigraph} of~\(\pi_1\)
and~\(\pi_1^{-1}\) in \fig~\vref{fig:pi_1}: in order to deduce the
bigraph of~\(\pi_1^{-1}\) from the one corresponding to~\(\pi_1\), let
us reverse each edge, that is, the direction in which the arrows are
pointing, then swap the indexes and the keys, that is, exchange the
two lines of vertices. Alternatively, we can imagine that we fold down
the paper along the key line, then look through and reverse the
arrows. Anyhow, the crossings are invariant. The horizontal symmetry
is obvious in \figs~\ref{fig:perm4} and~\ref{fig:inv1}.

\paragraph{Minimax}
\label{par:minimax}

After analysing the cost of a sorting algorithm based on comparisons,
we will need to know how close it is to an optimal sorting
algorithm\index{sorting!optimality|(}. The first theoretical problem
we examine is that of the best worst case, so called the
\emph{minimax}\index{cost!minimax}\index{sorting!minimax|(}. \Fig~\vref{fig:cmp_tree}
features the tree of all possible comparisons \index{tree!comparison
  $\sim$|see{sorting}} for sorting three numbers.
\begin{figure}[t]
\centering
\includegraphics[bb=71 636 327 723]{cmp_tree}
\caption{A comparison tree for sorting three keys}
\label{fig:cmp_tree}
\end{figure}
The \emph{external nodes}\index{binary tree!external
  node}\label{def:external_node} are all the
permutations\index{permutation|)} of \((a_1,a_2,a_3)\). The
\emph{internal nodes}\index{binary tree!internal node} are comparisons
between two keys \(a_i\)~and~\(a_j\), noted \(a_i?a_j\). Note that
leaves, in this setting, are internal nodes with two external nodes as
children.
\begin{wrapfigure}[8]{r}[0pt]{0pt}
% [8] vertical lines
% {r} mandatory right placement
% [0pt] of margin overhang
\centering
\includegraphics[bb=71 638 217 723]{red_cmp}
\caption{Useless \(a_1 > a_3\)}
\label{fig:red_cmp}
\end{wrapfigure}
If \(a_i < a_j\), then this property holds everywhere in the left
subtree, otherwise \(a_i > a_j\) holds in the right subtree. This tree
is one possible amongst many: it corresponds to an algorithm which
starts by comparing \(a_1\)~and~\(a_2\) and there are, of course, many
other strategies. But it does not contain redundant comparisons: if a
path from the root to a leaf includes \(a_i < a_j\) and \(a_j < a_k\),
we do not expect the useless \(a_i < a_k\). \Fig~\vref{fig:red_cmp}
shows an excerpt of a comparison tree with such a useless
comparison. The special external node~\(\bot\) corresponds to no
permutation because the comparison \(a_1 < a_3\) cannot fail as it is
implied by transitivity of the previous comparisons on the path from
the root.
\begin{center}
  \emph{A comparison tree for \(n\)~keys without redundancy has
    \(n!\)~external nodes.}
\end{center}
Because we are investigating minimum\hyp{}comparison sorting, we shall
consider henceforth optimal comparison trees with \(n!\)~external
nodes. Furthermore, amongst them, we want to determine the trees such
that the maximum number of comparisons is minimum.

An \emph{external path}\index{binary tree!external path} is a path
from the root to an external node. Let the \emph{height}\index{binary
  tree!height} of a tree be the length (that is, the number of edges)
of its longest external path. In \fig~\ref{fig:cmp_tree}, the height
is~\(3\) and there are~\(4\) external paths of length~\(3\), like
\((a_1 < a_2) \rightarrow (a_2 > a_3) \rightarrow (a_1 > a_3)
\rightarrow (a_3,a_1,a_2)\).

The maximality constraint means that we must consider the height of
the comparison tree because the number of internal nodes (comparisons)
along the maximum external paths is an upper bound for the number of
comparisons needed for sorting \emph{all} the
permutations\index{permutation}\index{sorting}.

The minimality constraint in the problem statement above then
signifies that \emph{we want a lower bound on the height of a
  comparison tree with \(n!\)~external nodes.}

% Wrapping figure better declared before a paragraph
%
\setlength{\intextsep}{0pt}
\begin{wrapfigure}[]{r}[0pt]{0pt}
% {r} mandatory right placement (better because of a list)
\centering
\includegraphics{comp_tree}
\caption{Perfect binary tree of height~\(3\)}
\label{fig:comp_tree}
\end{wrapfigure}
A \emph{perfect binary tree}\index{binary tree!perfect $\sim$} is a
binary tree whose internal nodes have children which are either two
internal nodes or two external nodes. If such a tree has height~\(h\),
then it has \(2^h\)~external nodes. For instance,
\fig~\vref{fig:comp_tree} shows the case where the height~\(h\)
is~\(3\) and there are indeed \(2^h=8\) external nodes, figured as
squares. Since, by definition, minimum\hyp{}comparison trees have
\(n!\)~external nodes and height~\(S(n)\), they must contain fewer
external nodes than a perfect binary tree of identical height, that
is, \(n! \leqslant 2^{S(n)}\), therefore
\begin{equation*}
S(n) \geqslant \ceiling{\lg n!},
\end{equation*}
where \(\ceiling{x}\) (\textsl{ceiling of~\(x\)})
\index{ceiling@$\ceiling{x}$|see{ceiling function}} \index{ceiling
  function} is the least integer greater than or equal to~\(x\). To
deduce a good lower bound on~\(S(n)\), we need the following theorem.
\setlength{\intextsep}{12pt}
\begin{thm}[Sum and integral]
\label{thm:integral_bounds}
Let \(f\colon [a,b] \rightarrow \mathbb{R}\) be an integrable,
monotonically increasing function. Then
\begin{equation}
\abovedisplayskip=0pt
\sum_{k=a}^{b-1}f(k) \leqslant \int_{a}^{b}{\!\!f(x)} \,dx
                   \leqslant \sum_{k=a+1}^{b}{\!\!\!f(k)}.
\tag*{\(\blacksquare\)}
\end{equation}
\end{thm}
\noindent Let us take \(a:=1\), \(b:=n\) and \(f(x) := \lg x\). The
theorem implies
\begin{equation*}
n\lg n - \frac{n}{\ln 2} + \frac{1}{\ln 2}
= \int_{1}^{n}{\!\!\lg x} \,dx \leqslant \sum_{k=2}^{n}{\lg k}
= \lg n! \leqslant S(n).
\end{equation*}
A more powerful but more complex approach in real analysis, known as
Euler-Maclaurin summation, yields Stirling's formula
\citep[chap.~4]{SedgewickFlajolet_1996}\index{Stirling's formula},
which is a very precise lower bound for \(\lg n!\):
\begin{equation}
\Big(n + \frac{1}{2}\Big)\lg n - \frac{n}{\ln 2} + \lg\sqrt{2\pi}
< \lg n! \leqslant S(n).
\label{ineq:S_lower}
\end{equation}

\paragraph{Minimean}
\label{par:opt_sort:minimean}
\index{cost!minimean|(}
\index{sorting!minimean|(}
\index{binary tree!external path length|(}

We investigate here the best mean case, or \emph{minimean}. Let us
call the sum of the lengths of all the external paths the
\emph{external path length of the
  tree}\label{sorting__external_path_length}\index{binary
  tree!external path length}. Then, the average number of comparisons
is the mean external path length. In \fig~\vref{fig:cmp_tree}, this is
\((2+3+3+3+3+2)/3!=8/3\). Our problem here therefore consists in
determining the shape of the optimal comparison trees of minimum
external path length.

These trees have their external nodes on one or two successive levels
and are thus \emph{almost perfect}.\index{binary tree!almost perfect
  $\sim$} Let us consider a binary tree where this is not the case, so
the topmost external nodes are located at level~\(l\) and the
bottommost at level~\(L\), with \(L \geqslant l + 2\). If we exchange
a leaf at level~\(L-1\) with an external node at level~\(l\), the
external path length is decreased by \((l+2L) - (2(l+1) + (L-1)) = L -
l - 1 \geqslant 1\). Repeating these exchanges yields the expected
shape.

The external paths are made up of \(p\)~paths ending at the
penultimate level~\(h-1\) and \(q\)~paths ending at the bottommost
level~\(h\). (The root has level~\(0\).) Let us find two equations
whose solutions are \(p\)~and~\(q\).
\begin{itemize}

  \item From the minimax problem\index{sorting!minimax|)}, we know
    that an optimal comparison tree for \(n\)~keys has \(n!\)~external
    nodes: \(p+q=n!\).

  \item If we replace the external nodes at level~\(h-1\) by leaves,
    the level~\(h\) becomes full with~\(2^h\) external nodes:
    \(2p+q=2^h\).

\end{itemize}
We now have two linear equations satisfied by~\(p\) and~\(q\), whose
solutions are \(p=2^h-n!\) and \(q=2n!-2^h\). Furthermore, we can now
express the minimal external path length as follows: \((h-1)p + hq =
(h+1)n! - 2^h\). Finally, we need to determine the height~\(h\) of the
tree in terms of~\(n!\). This can be done by remarking that, by
contruction, the last level may be incomplete, so \(0 < q \leqslant
2^h\), or, equivalently, \(h = \ceiling{\lg n!}\). We conclude that
the minimum external path length is
\begin{equation*}
(\ceiling{\lg n!} + 1) n! - 2^{\ceiling{\lg n!}}.
\end{equation*}
Let \(M(n)\) \label{def:Mn} be the minimum average number of
comparisons of an optimal sorting algorithm. We have
\begin{equation*}
M(n) = \ceiling{\lg n!} + 1 - \frac{1}{n!}2^{\ceiling{\lg n!}}.
\end{equation*}
We have \(\ceiling{\lg n!} = \lg n! + x\), with \(0 \leqslant x < 1\),
therefore, if we set the function \(\theta(x) := 1 + x - 2^x\), we
draw
\begin{equation*}
M(n) = \lg n! + \theta(x).
\end{equation*}
We have \(\max_{0\leqslant x < 1}\theta(x) = 1 - (1 + \ln\ln 2)/\!\ln 2
\simeq 0.08607\), therefore,
\begin{equation}
\lg n! \leqslant M(n) < \lg n! + 0.09.
\label{ineq:Mn}
\end{equation}
\index{sorting!optimality|)}
\index{cost!minimean|)}
\index{sorting!minimean|)}
\index{binary tree!external path length|)}
