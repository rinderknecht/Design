\chapter{Insertion Sort}
\label{chap:insertion}
\index{tri!$\sim$ par insertions|(}

If we have a stack of totally ordered, distinct keys, it is easy to
insert one more key so the stack remains ordered by comparing it with
the first on top, then, if necessary, with the second, the third
etc. For example, inserting~\(1\) into \([3,5]\) requires
comparing~\(1\) with~\(3\) and results in \([1,3,5]\), without
relating~\(1\) to~\(5\). The algorithm called \emph{insertion sort}
\citep{Knuth_1998}\index{sorting|see{insertion sort}}\index{insertion
  sort} consists in inserting thusly keys one by one in a stack
originally empty. The playful analogy is that of sorting a hand in a
card game: each card, from left to right, is moved leftwards until it
reaches its place.

\section{Straight insertion}
\label{sec:straight_ins}
\index{insertion sort!straight $\sim$|(}

Let~\(\fun{ins}(s,x)\)\index{ins@\fun{ins/2}} (not to be confused with
the function of same name and arity in section~\ref{sec:opt_sort}) be
the increasingly ordered stack resulting from the straight insertion
of~\(x\) into the stack~\(s\). Function \fun{ins/2} can be defined
assuming a minimum and a maximum function,
\fun{min/2}\index{min@\fun{min/2}} and
\fun{max/2}\index{max@\fun{max/2}}:
\begin{equation*}
\fun{ins}(x,\el)         \rightarrow [x];\qquad
\fun{ins}(x,\cons{y}{s}) \rightarrow
   \cons{\fun{min}(x,y)}{\fun{ins}(\fun{max}(x,y),s)}.
\end{equation*}
Temporarily, let us restrict ourselves to sorting natural numbers in
increasing order. We need to provide definitions to calculate the
minimum and maximum:
\begin{equation*}
\begin{array}{@{}r@{\;}l@{\;}lr@{\;}l@{\;}l@{}}
  \fun{max}(0,y) & \rightarrow & y; & \fun{min}(0,y) & \rightarrow & 0;\\
  \fun{max}(x,0) & \rightarrow & x; & \fun{min}(x,0) & \rightarrow & 0;\\
  \fun{max}(x,y) & \rightarrow & 1 + \fun{max}(x-1,y-1).
& \fun{min}(x,y) & \rightarrow & 1 + \fun{min}(x-1,y-1).
\end{array}
\end{equation*}
While this approach fits in our functional language, it is both
inefficient and bulky, hence it is worth extending our language so
rewrite rules are selected by pattern matching only if some optional
associated comparison holds. We can then
define~\fun{isrt/1}\index{isrt@\fun{isrt/1}|(} (\emph{insertion sort})
and redefine~\fun{ins/2}\index{ins@\fun{ins/2}|(} as
\begin{equation*}
\begin{array}{@{}r@{\;}l@{\;}lr@{\;}l@{\;}l@{}}
  \fun{ins}(\cons{y}{s},x)
& \smashedrightarrow{\kappa}
& \cons{y}{\fun{ins}(s,x)}, \,\text{if \(x \succ y\)};
& \fun{isrt}(\el)
& \smashedrightarrow{\mu}
& \el;\\
  \fun{ins}(s,x)
& \smashedrightarrow{\lambda}
& \cons{x}{s}.
& \fun{isrt}(\cons{x}{s})
& \smashedrightarrow{\nu}
& \fun{ins}(\fun{isrt}(s),x).
\end{array}
\end{equation*}
Let us consider a short example in \fig~\vref{fig:isrt_312}.
\begin{figure}
\begin{equation*}
\boxed{%
\begin{array}{r@{\;}l@{\;}l}
\fun{isrt}([3,1,2])
& \xrightarrow{\smash{\nu}} & \fun{ins}(\fun{isrt}([1,2]),3)\\
& \xrightarrow{\smash{\nu}}
& \fun{ins}(\fun{ins}(\fun{isrt}([2]),1),3)\\
& \xrightarrow{\smash{\nu}}
& \fun{ins}(\fun{ins}(\fun{ins}(\fun{isrt}(\el),2),1),3)\\
& \xrightarrow{\smash{\mu}}
& \fun{ins}(\fun{ins}(\fun{ins}(\el,2),1),3)\\
& \xrightarrow{\smash{\lambda}}
& \fun{ins}(\fun{ins}([2],1),3)\\
& \xrightarrow{\smash{\lambda}}
& \fun{ins}([1,2],3)\\
& \xrightarrow{\smash{\kappa}}
& [1|\fun{ins}([2],3)]\\
& \xrightarrow{\smash{\kappa}}
& [1,2|\fun{ins}(\el,3)]\\
& \xrightarrow{\smash{\lambda}}
& [1,2,3].
\end{array}}
\end{equation*}
\caption{\(\fun{isrt}([3,1,2]) \twoheadrightarrow [1,2,3]\)}
\label{fig:isrt_312}
\index{insertion sort!straight $\sim$!example}
\end{figure}
\index{isrt@\fun{isrt/1}|)}
\index{ins@\fun{ins/2}|)}

Let \(\C{\fun{isrt}}{n}\)\index{isrt@$\C{\fun{isrt}}{n}$|(} be the cost
of sorting by straight insertion \(n\)~keys, and \(\C{\fun{ins}}{i}\)
the cost of inserting one key in a stack of length~\(i\). We directly
derive from the functional program the following recurrences:
\begin{equation*}
\C{\fun{isrt}}{0}   \eqn{\smash{\mu}} 1;\qquad
\C{\fun{isrt}}{i+1} \eqn{\smash{\nu}} 1 + \C{\fun{ins}}{i} +
  \C{\fun{isrt}}{i}.
\end{equation*}
The latter equation assumes that the length of \(\fun{isrt}(s)\) is
the same as~\(s\) and the length of \(\fun{ins}(s,x)\) is the same as
\(\cons{x}{s}\). We deduce
\begin{equation}
\C{\fun{isrt}}{n} = 1 + n + \sum_{i=0}^{n-1}\C{\fun{ins}}{i}.
\label{eq:cost_isrt}
\end{equation}
A look up at the definition of~\fun{ins/2} reveals that
\(\C{\fun{ins}}{i}\) cannot be expressed only in terms of~\(i\)
because it depends on the relative order of all the keys. Instead, we
resort to the minimum, maximum and average
costs.\index{isrt@$\C{\fun{isrt}}{n}$|)}

\addcontentsline{toc}{subsection}{Cost}
\paragraph{Minimum cost}
\index{insertion sort!straight $\sim$!minimum cost}

The best case does not exert rule~\(\kappa\), which is recursive,
whilst \(\lambda\)~rewrites to a value. In other words, in
rule~\(\nu\), each key~\(x\) to be inserted in a non\hyp{}empty,
sorted stack \(\fun{isrt}(s)\)\index{isrt@\fun{isrt/1}} would be lower
than or equal to the top of the stack. This rule also inserts the keys
in reverse order:
\begin{equation}
\fun{isrt}([x_1,\dots,x_n]) \twoheadrightarrow
\fun{ins}(\fun{ins}(\dots(\fun{ins}(\el,x_n)\dots),x_2),x_1).
\label{inv_isrt}
\end{equation}
As a consequence, \emph{the minimum cost results from the input stack
  being already increasingly sorted,} that is, the keys in the result
are increasing, but may be repeated. Then \(\B{\fun{ins}}{n} =
\len{\lambda} = 1\)\index{ins@$\B{\fun{ins}}{n}$} and
  equation~\eqref{eq:cost_isrt} implies that straight insertion sort
  has a linear cost in the best case:
\begin{equation*}
\B{\fun{isrt}}{n} = 2n+1 \sim 2n.
\end{equation*}

\paragraph{Maximum cost}
\index{insertion sort!straight $\sim$!maximum cost}

The worst case must exert rule~\(\kappa\) as much as possible, which
implies that \emph{the worst case happens when the input is a stack
  decreasingly sorted}. We have\index{ins@$\W{\fun{ins}}{n}$|(}
\begin{equation*}
\W{\fun{ins}}{n} = \len{\kappa^n\lambda} = n + 1.
\end{equation*}
Substituting maximum costs in equation~\eqref{eq:cost_isrt} then
implies that straight insertion sort has a quadratic cost in the worst
case:
\begin{equation*}
\W{\fun{isrt}}{n} = \frac{1}{2}{n^2} + \frac{3}{2}{n} + 1
\sim \frac{1}{2}{n^2}.
\end{equation*}
Another way is to take the length of the longest trace:
\begin{equation*}
\W{\fun{isrt}}{n}
 = \left\lvert\nu^n\mu \prod_{i=0}^{n-1}\kappa^i\lambda\right\rvert
 = \len{\nu^n\mu} + \sum_{i=0}^{n-1}\len{\kappa^i\lambda}
 = \frac{1}{2}{n^2} + \frac{3}{2}{n} + 1.
\end{equation*}
This cost should not be surprising because
\fun{isrt/1}\index{isrt@\fun{isrt/1}} and
\fun{rev\(_0\)/1}\index{rev0@\fun{rev\(_0\)/1}}, in
section~\ref{sec:reversal}, yield the same kind of partial rewrite, as
seen by comparing~\eqref{inv_isrt}, \vpageref{inv_isrt},
and~\eqref{eq:rev0} \vpageref{eq:rev0}, and also \(\C{\fun{cat}}{n} =
\W{\fun{ins}}{n}\),\index{cat@$\C{\fun{cat}}{n}$}\index{ins@$\W{\fun{ins}}{n}$}
where \(n\)~is the size of their first argument. Hence
\(\W{\fun{isrt}}{n} = \W{\fun{rev}_0}{n}\).
\index{rev0@$\W{\fun{rev}_0}{n}$}\index{ins@$\W{\fun{ins}}{n}$|)}
\index{isrt@$\W{\fun{isrt}}{n}$}

\paragraph{Average cost}
\label{par:ave_isrt}
\index{insertion sort!straight $\sim$!average cost}

The average cost obeys equation~\eqref{eq:cost_isrt} because all
permutations \((x_1,\dots,x_n)\) are equally likely, hence
\begin{equation}
\M{\fun{isrt}}{n} = 1 + n + \sum_{i=0}^{n-1}\M{\fun{ins}}{i}.
\label{eq:mean_isrt}
\end{equation}
Without loss of generality,
\(\M{\fun{ins}}{i}\)\index{ins@$\M{\fun{ins}}{n}$|(} is the cost of
inserting the key \(i+1\) into all permutations of \((1,\dots,i)\),
divided by \(i+1\). (This is how the set of
permutations\index{permutation} of a given length is inductively built
on page~\pageref{par:permutations}.)  The partial
evaluation~\eqref{inv_isrt}\index{functional
  language!evaluation!partial $\sim$} on page~\pageref{inv_isrt} has
length \(\len{\nu^n\mu}=n+1\). The trace for inserting in an empty
stack is~\(\mu\). If the stack has length~\(i\), inserting on top has
trace~\(\lambda\); just after the first key, \(\kappa\lambda\)
etc. until after the last key, \(\kappa^i\lambda\). Therefore, the
average cost for inserting one key is, for \(i \geqslant 0\),
\begin{equation}
\M{\fun{ins}}{i} = \frac{1}{i+1}\sum_{j=0}^{i}\len{\kappa^j\lambda}
                 = \frac{i}{2}+1.
\label{eq:ins}
\end{equation}
The average cost for inserting \(n\)~keys in an empty stack is
therefore \(\sum_{i=0}^{n-1}\M{\fun{ins}}{i} = \frac{1}{4}n^2 +
\frac{3}{4}n\). Finally, from equation~\eqref{eq:mean_isrt}, the
average cost of sorting \(n\)~keys by straight insertion is
\begin{equation*}
\M{\fun{isrt}}{n} = \frac{1}{4}n^2 + \frac{7}{4}n + 1.
\end{equation*}
\index{ins@$\M{\fun{ins}}{n}$|)}

\paragraph{Assessment}
\index{permutation!inversion|(}
\index{insertion sort!inversion|see{permutation}}

Despite the average cost being asymptotically equivalent to~\(50\%\)
of the maximum cost, it is nevertheless quadratic. On a positive note,
straight insertion is quite efficient when the data is short or nearly
sorted \citep{CookKim_1980}. It is a typical example of an
\emph{adaptive sorting algorithm}
\citep{EstivillWood_1992,MoffatPetersson_1992}\index{insertion
  sort!adaptive}. The natural measure of sortedness for insertion sort
is the number of inversions. Indeed, the partial
evaluation~\eqref{inv_isrt}, \vpageref{inv_isrt} shows that the keys
are inserted in reverse order. Thus, in rule~\(\kappa\), we know that
the key \(x\)~was originally before~\(y\), but \(x \succ
y\). Therefore, one application of \emph{rule~\(\kappa\) removes one
  inversion from the input.} As a corollary, the average number of
inversions in a random permutation of \(n\)~objects is
\begin{equation*}
\belowdisplayskip=-4pt
\sum_{j=0}^{n-1}\frac{1}{j+1}\sum_{i=0}^{j}\len{\kappa^i} =
\frac{n(n-1)}{4}.
\end{equation*}
\index{permutation!inversion|)}

\paragraph{Exercises}
\begin{enumerate*}

  \item A sorting algorithm that preserves the relative order of equal
    keys is said \emph{stable}. Is
    \fun{isrt/1}\index{isrt@\fun{isrt/1}} stable?

  \item Prove \(\fun{len}(\cons{x}{s}) \equiv
    \fun{len}(\fun{ins}(x,s))\).\index{len@\fun{len/1}}

  \item Prove \(\fun{len}(s) \equiv \fun{len}(\fun{isrt}(s))\).

  \item Traditionally, textbooks about the analysis of algorithms
    assess the cost of sorting procedures by counting the comparisons,
    not the function calls. Doing so allows one to compare with the
    same measure different sorting algorithms as long as they perform
    comparisons, even if they are implemented in different programming
    languages. (There are sorting techniques that do not rely on
    comparisons.) Let
    \(\OB{\fun{isrt}}{n}\)\index{isrt@$\OB{\fun{isrt}}{n}$},
    \(\OW{\fun{isrt}}{n}\)\index{isrt@$\OW{\fun{isrt}}{n}$} and
    \(\OM{\fun{isrt}}{n}\)\index{isrt@$\OM{\fun{isrt}}{n}$} be the
    minimum, maximum and average numbers of comparisons needed to sort
    by straight insertion a stack of length~\(n\). Establish that
    \begin{equation*}
      \OB{\fun{isrt}}{n} = n - 1; \quad
      \OW{\fun{isrt}}{n} = \frac{1}{2}n(n - 1); \quad
      \OM{\fun{isrt}}{n} = \frac{1}{4}n^2 + \frac{3}{4}n - H_n,
    \end{equation*}
    where \(H_n := \sum_{k=1}^n{1/k}\) is the \(n\)th \emph{harmonic
      number}\index{harmonic@$H_n$|see{harmonic
        number}}\index{harmonic number} and, by convention, \(H_0 :=
    0\). \emph{Hint:} the use of rule~\(\lambda\) implies a comparison
    if, and only if, \(s\)~is not empty.

\end{enumerate*}

\addcontentsline{toc}{subsection}{Soundness}
\paragraph{Ordered stacks}

As we did for the proof of the soundness\index{insertion
  sort!soundness|(} of \fun{cut/2}\index{cut@\fun{cut/2}}
\vpageref{par:cut_sound}, we need to express the characteristic
properties we expect on the output of \fun{isrt/1}, and relate them to
some assumptions on the input, first informally, then formally. We
would say that `The stack \(\fun{isrt}(s)\)\index{isrt@\fun{isrt/1}}
is totally, increasingly ordered and contains all the keys in~\(s\),
but no more.'  This captures all what is expected from any sorting
program.

Let us name \(\pred{Ord}{s}\)\index{Ord@\predName{Ord}|(} the
proposition `The stack~\(s\) is sorted increasingly.' To define
formally this concept, let us use \emph{inductive logic
  definitions}\index{induction!definition by $\sim$}. We employed this
technique to formally define stacks, on page~\pageref{def:stack}, in a
way that generates a simple well\hyp{}founded order used by structural
induction, to wit, \(\cons{x}{s} \succ x\) and \(\cons{x}{s} \succ
s\). Here, we similarly propose that, for \(\pred{Ord}{s}\) to hold,
\(\pred{Ord}{t}\) must hold with~\(s \succ t\). We defined \fun{cut/2}
in section~\ref{sec:cutting}, \vpageref{sec:cutting}, using the same
technique and relying on inference rules. All three cases, namely,
data structure, proposition and function, are instances of inductive
definitions. Let us constructively define \(\predName{Ord}\) by the
axioms \(\TirName{Ord}_0\) and \(\TirName{Ord}_1\), and the inference
rule \(\TirName{Ord}_2\):\label{def:Ord}
\begin{mathpar}
\inferrule*{}{\pred{Ord}{\el}}
\;\TirName{Ord}_0
\qquad
\inferrule*{}{\pred{Ord}{[x]}}
\;\TirName{Ord}_1
\qquad
\inferrule
  {x \prec y \and \pred{Ord}{\cons{y}{s}}}
  {\pred{Ord}{\cons{x,y}{s}}}
\,\TirName{Ord}_2
\end{mathpar}
Note that this system is parameterised by the well\hyp{}founded order
(\(\prec\)) on the keys such that \(x \prec y :\Leftrightarrow y \succ
x\). Rule \(\TirName{Ord}_2\) could equivalently be \((x \prec y
\mathrel{\wedge} \pred{Ord}{\cons{y}{s}}) \Rightarrow
\pred{Ord}{\cons{x,y}{s}}\) or \(x \prec y \Rightarrow
(\pred{Ord}{\cons{y}{s}} \Rightarrow \pred{Ord}{\cons{x,y}{s}})\) or
\(x \prec y \Rightarrow \pred{Ord}{\cons{y}{s}} \Rightarrow
\pred{Ord}{\cons{x,y}{s}}\). Because the set of ordered stacks is
exactly generated by this system, if the statement
\(\pred{Ord}{\cons{x,y}{s}}\) holds, then, necessarily,
\(\TirName{Ord}_2\) has been used to produce it, hence \(x \prec y\)
and \(\pred{Ord}{\cons{y}{s}}\) are true as well. This usage of an
inductive definition is called an \emph{inversion
  lemma}\index{induction!inversion lemma} and can be understood as
inferring necessary conditions for a putative formula, or as
\emph{case analysis on inductive definitions}.\index{Ord@\predName{Ord}|)}


\paragraph{Equivalent stacks}
\index{stack!equivalence}

The second part of our informal definition above was: `The stack
\(\fun{isrt}(s)\)\index{isrt@\fun{isrt/1}} contains all the keys
in~\(s\), but no more.'  In order to provide a general criterion
matching this concept, we should abstract it as: `The stack~\(s\)
contains all the keys in the stack~\(t\), but no more.' Permutations
allow us to clarify the meaning with a mathematical phrasing: `Stacks
\(s\)~and~\(t\) are permutations of each other,' which we note \(s
\approx t\) and \(t \approx s\). Since the role of~\(s\) do not differ
in any way from that of~\(t\), we expect the relation \((\approx)\) to
be symmetric: \(s \approx t \Rightarrow t \approx s\). Moreover, we
expect it to be \emph{transitive} as well: \(s \approx u\) and \(u
\approx t\) imply \(s \approx t\). Also, we want (\(\approx\)) to be
\emph{reflexive}, that is, \(s \approx s\). By definition, a binary
relation which is reflexive, symmetric and transitive is an
\emph{equivalence relation}\index{equivalence!$\sim$ relation}.

The relation (\(\approx\)) can be defined in different ways. The idea
here consists in defining a permutation as a series of
\emph{transpositions}\index{transposition}, namely, exchanges of
adjacent keys. This approach is likely to work here because insertion
can be thought of as adding a key on top of a stack and then
performing a series of transpositions until the total order is
restored.
\begin{mathpar}
\inferrule*{}{\el \approx \el}
\;\TirName{Pnil}
\qquad
\inferrule*{}{\cons{x,y}{s} \approx \cons{y,x}{s}}
\;\TirName{Swap}\\
\inferrule
  {s \approx t}
  {\cons{x}{s} \approx \cons{x}{t}}
\,\TirName{Push}
\qquad
\inferrule
  {s \approx u \and u \approx t}
  {s \approx t}
\,\TirName{Trans}
\end{mathpar}
We recognise \TirName{Pnil} and \TirName{Swap} as axioms, the latter
being synonymous with transposition. Rule \TirName{Trans} is
transitivity and offers an example with two premises, so a derivation
using it becomes a \emph{binary tree}\index{tree!binary
  $\sim$|see{binary tree}}\index{binary tree}, as shown in
\fig~\vref{fig:proof_trees}.
\begin{figure}
\centering
\subfloat[Extended tree\label{fig:perm_proof}]{
  \includegraphics[bb=135 672 369 723]{proof}}
\qquad
\subfloat[Pruned tree\label{fig:312eq231}]{
  \includegraphics[bb=70 670 131 723]{312eq231}}
\caption{Proof tree of \({[}3,1,2{]} \protect\approx {[}2,3,1{]}\)}
\label{fig:proof_trees}
\end{figure}
As an exception to the convention about tree layouts, proof
trees\index{tree!proof $\sim$} have their root at the bottom of the
figure.

We must now prove the \emph{reflexivity} of~\((\approx)\),
namely\index{Refl@\predName{Refl}|(} \(\pred{Refl}{s} \colon s \approx
s\), by induction\index{induction!example|(} on the structure of the
proof. The difference with the proof of the soundness of
\fun{cut/2}\index{cut@\fun{cut/2}}, \vpageref{par:cut_sound}, is that,
due to \TirName{Trans} having two premises, the induction hypothesis
applies to both of them. Also, the theorem is not explicitly an
implication. First, we start by proving \predName{Refl} on the axioms
(the leaves of the proof trees) and proceed with the induction on the
proper inference rules, to wit, reflexivity is conserved while moving
towards the root of the proof tree.
\begin{itemize}

  \item Axiom \TirName{Pnil} proves \(\pred{Refl}{\el}\).
    Axiom~\TirName{Swap} proves \(\pred{Refl}{\cons{x,x}{s}}\).

  \item Let us assume now that \predName{Refl}~holds for the premise
    in \TirName{Push}, namely, \(s = t\). Clearly, the conclusion
    implies \(\pred{Refl}{\cons{x}{s}}\).

  \item Let us assume that \predName{Refl}~holds for the \emph{two}
    antecedents in \TirName{Push}, that is, \(s=u=t\). The conclusion
    leads to
    \(\pred{Refl}{s}\)\index{Refl@\predName{Refl}|)}.\hfill\(\Box\)

\end{itemize}

Let us prove the \emph{symmetry} of~\((\approx)\) using the same
technique. Let\index{Sym@\predName{Sym}} \(\pred{Sym}{s,t}\colon s
\approx t \Rightarrow t \approx s\). We deal with an implication here,
so let us suppose that \(s \approx t\), that is, we have a proof
tree~\(\Delta\) whose root is \(s \approx t\), and let us establish
\(t \approx s\). In the following, we overline variables from the
inference system.
\begin{itemize}

  \item If \(\Delta\) ends with \TirName{Pnil}, then \(\el = s = t\),
    which trivially implies \(t \approx s\).

  \item If \(\Delta\) ends with \TirName{Swap}, then
    \(\cons{\overline{x},\overline{y}}{\overline{s}} = s\) and
    \(\cons{\overline{y},\overline{x}}{\overline{s}} = t\), so \(t
    \approx s\).

  \item If \(\Delta\) ends with \TirName{Push}, then
    \(\cons{\overline{x}}{\overline{s}} = s\) and
    \(\cons{\overline{x}}{\overline{t}} = t\). The induction
    hypothesis applies to the premise, \(\overline{s} \approx
    \overline{t}\), hence \(\overline{t} \approx \overline{s}\)
    holds. An application of \TirName{Push} with the latter premise
    implies \(\cons{\overline{x}}{\overline{t}} \approx
    \cons{\overline{x}}{\overline{s}}\), that is, \(t \approx s\).

  \item If \(\Delta\) ends with \TirName{Trans}, then the induction
    hypothesis applies to its prem\-ises and we deduce \(u \approx s\)
    and \(t \approx u\), which can be premises for \TirName{Trans}
    itself and lead to \(t \approx s\).\hfill\(\Box\)

\end{itemize}

\paragraph{Soundness}

Let us now turn our attention to our main objective, which we may call
\(\pred{Isrt}{s}\colon \pred{Ord}{\fun{isrt}(s)} \mathrel{\wedge}
\fun{isrt}(s) \approx s\).\index{Isrt@\predName{Isrt}} Let us tackle
its proof by structural induction on~\(s\).
\begin{itemize}

  \item The basis is~\(\pred{Isrt}{\el}\) and it is showed to hold
    twofold. First, we have \(\fun{isrt}(\el)
    \xrightarrow{\smash{\mu}} \el\) and \(\pred{Ord}{\el}\) is axiom
    \(\TirName{Ord}_0\). The other conjunct holds as well since
    \(\fun{isrt}(\el) \approx \el \Leftrightarrow \el \approx \el\),
    which is axiom \TirName{Pnil}.

  \item Let us assume \(\pred{Isrt}{s}\) and establish
    \(\pred{Isrt}{\cons{x}{s}}\). In other words, let us assume
    \(\pred{Ord}{\fun{isrt}(s)}\) and \(\fun{isrt}(s) \approx s\). We
    have
    \begin{equation}
      \fun{isrt}(\cons{x}{s}) \xrightarrow{\smash{\nu}}
      \fun{ins}(\fun{isrt}(s),x).\label{eq:B}
    \end{equation}
    Since we want \(\pred{Ord}{\fun{isrt}(\cons{x}{s})}\), assuming
    \(\pred{Ord}{\fun{isrt}(s)}\), we realise that we need the lemma
    \(\pred{Ord}{s} \Rightarrow \pred{Ord}{\fun{ins}(s,x)}\), which we
    may name \(\pred{InsOrd}{s}\)\index{InsOrd@\predName{InsOrd}}. To
    prove the conjunct \(\fun{isrt}(\cons{x}{s}) \approx
    \cons{x}{s}\), assuming \(\fun{isrt}(s) \approx s\), we need the
    lemma \(\pred{InsCmp}{s} \colon \fun{ins}(s,x) \approx
    \cons{x}{s}\). In particular, \(\pred{InsCmp}{\fun{isrt}(s)}\) is
    \(\fun{ins}(\fun{isrt}(s),x) \approx \cons{x}{\fun{isrt}(s)}\). We
    have
    \begin{itemize*}

      \item Rule \TirName{Push} and the induction hypothesis
        \(\fun{isrt}(s) \approx s\) imply \(\cons{x}{\fun{isrt}(s)}
        \approx \cons{x}{s}\).

      \item By the transitivity of~\((\approx)\), we draw
        \(\fun{ins}(\fun{isrt}(s),x) \approx \cons{x}{s}\),

      \item by rewrite~\eqref{eq:B}, \(\fun{isrt}(\cons{x}{s})
        \approx \cons{x}{s}\) and \(\pred{Isrt}{\cons{x}{s}}\)
        follow.

    \end{itemize*}
    By the induction principle, we conclude \(\forall s \in
    S.\pred{Isrt}{s}\).\hfill\(\Box\)

\end{itemize}

\paragraph{Insertion adds a key}

To complete the previous proof, we prove the
lemma\index{InsCmp@\predName{InsCmp}} \(\pred{InsCmp}{s} \colon
\fun{ins}(s,x) \approx \cons{x}{s}\) by structural induction on~\(s\).
\begin{itemize*}

  \item The basis \(\pred{InsCmp}{\el}\) stands because
    \(\fun{ins}(\el,x) \xrightarrow{\smash{\lambda}} [x] \approx
          [x]\), by composing rules \TirName{Pnil} and \TirName{Push}.

  \item Let us assume \(\pred{InsCmp}{s}\) and deduce
    \(\pred{InsCmp}{\cons{y}{s}}\), that is,
    \begin{equation*}
      \fun{ins}(s,x) \approx \cons{x}{s} \Rightarrow
      \fun{ins}(\cons{y}{s},x) \approx \cons{x,y}{s}.
    \end{equation*}
    There are two cases to analyse.
    \begin{itemize*}

    \item If \(y \succ x\), then \(\fun{ins}(\cons{y}{s},x)
      \xrightarrow{\smash{\lambda}} \cons{x,y}{s} \approx
      \cons{x,y}{s}\), by rule~\TirName{Swap}.

    \item Otherwise, \(x \succ y\) and \(\fun{ins}(\cons{y}{s},x)
      \xrightarrow{\smash{\kappa}} \cons{y}{\fun{ins}(s,x)}\).
      \begin{itemize*}

        \item We deduce \(\cons{y}{\fun{ins}(s,x)} \approx
          \cons{y,x}{s}\) by using the induction hypothesis as the
          premise of rule \TirName{Push}.

        \item Furthermore, \TirName{Swap}~yields \(\cons{y,x}{s}
          \approx \cons{x,y}{s}\).

        \item Transitivity of~\((\approx)\) applied to the two last
          statements leads to \(\fun{ins}(\cons{y}{s},x) \approx
          \cons{x,y}{s}\).

      \end{itemize*}
      Note that we do not need to assume that \(s\)~is sorted: what
      matters here is that \(\fun{ins/2}\) loses no key it inserts,
      but misplacement is irrelevant.\hfill\(\Box\)

    \end{itemize*}

\end{itemize*}


\paragraph{Insertion preserves order}

To complete the soundness proof, we must prove the
lemma~\(\pred{InsOrd}{s} \colon \pred{Ord}{s} \Rightarrow
\pred{Ord}{\fun{ins}(s,x)}\) by induction on the
structure\index{induction!example} of~\(s\), meaning that insertion
preserves order.

\begin{itemize*}

  \item The basis \(\pred{InsOrd}{\el}\) is easy to check:
    \(\TirName{Ord}_0\)~states \(\pred{Ord}{\el}\); we have the
    rewrite \(\fun{ins}(\el,x) \xrightarrow{\smash{\lambda}} [x]\) and
    \(\TirName{Ord}_1\)~is \(\pred{Ord}{[x]}\).

  \item Let us prove \(\pred{InsOrd}{s} \Rightarrow
    \pred{InsOrd}{\cons{x}{s}}\) by assuming
    \begin{equation*}
      (H_0) \;\; \pred{Ord}{s},\qquad
      (H_1) \;\; \pred{Ord}{\fun{ins}(s,x)},\qquad
      (H_2) \;\; \pred{Ord}{\cons{y}{s}},
    \end{equation*}
    and deriving \(\pred{Ord}{\fun{ins}(\cons{y}{s},x)}\).

    \noindent Two cases arise from comparing \(x\)~to~\(y\):
    \begin{itemize*}

      \item If \(y \succ x\), then \(H_2\)~implies
        \(\pred{Ord}{\cons{x,y}{s}}\), by rule
        \(\TirName{Ord}_2\). Since \(\fun{ins}(\cons{y}{s},x)
        \xrightarrow{\smash{\lambda}} \cons{x,y}{s}\), we have
        \(\pred{Ord}{\fun{ins}(\cons{y}{s},x)}\).

      \item Otherwise, \(x \succ y\) and we derive
        \begin{equation}
          \fun{ins}(\cons{y}{s},x) \xrightarrow{\smash{\kappa}}
          \cons{y}{\fun{ins}(s,x)}.\label{eq:A}
        \end{equation}
        Here, things get more complicated because we need to consider
        the structure of~\(s\).
        \begin{itemize*}

        \item If \(s=\el\), then \(\cons{y}{\fun{ins}(s,x)}
          \xrightarrow{\smash{\lambda}} [y,x]\). Furthermore, \(x
          \succ y\), \(\TirName{Ord}_1\) and \(\TirName{Ord}_2\) imply
          \(\pred{Ord}{[y,x]}\), so
          \(\pred{Ord}{\fun{ins}(\cons{y}{s},x)}\).

          \item Else, there exists a key~\(z\) and a stack~\(t\) such
            that \(s = \cons{z}{t}\).
            \begin{itemize*}

              \item If \(z \succ x\), then
                \begin{equation}
                  \cons{y}{\fun{ins}(s,x)} =
                  \cons{y}{\fun{ins}(\cons{z}{t},x)}
                  \xrightarrow{\smash{\lambda}} \cons{y,x,z}{t} =\!
                  \cons{y,x}{s}.\label{eq:C}
                \end{equation}
                \(H_0\)~is \(\pred{Ord}{\cons{z}{t}}\), which, with
                \(z \succ x\) and rule \(\TirName{Ord}_2\), implies
                \(\pred{Ord}{\cons{x,z}{t}}\). Since~\(x \succ\! y\),
                another application of~\(\TirName{Ord}_2\) yields
                \(\pred{Ord}{\cons{y,x,z}{t}}\), that is,
                \(\pred{Ord}{\cons{y,x}{s}}\). This and
                rewrite~\eqref{eq:C} entail that
                \(\pred{Ord}{\cons{y}{\fun{ins}(s,x)}}\). Finally, due
                to rewrite~\eqref{eq:A},
                \(\pred{Ord}{\fun{ins}(\cons{y}{s},x)}\) holds.

              \item The last remaining case to examine is when~\(x
                \succ z\):
                \begin{equation}
                  \cons{y}{\fun{ins}(s,x)} \!=\!
                  \cons{y}{\fun{ins}(\cons{z}{t},x)}
                  \!\xrightarrow{\smash{\kappa}}\!
                  \cons{y,z}{\fun{ins}(t,x)}.\label{eq:D}
                \end{equation}
                Hypothesis~\(H_2\) is \(\pred{Ord}{\cons{y,z}{t}}\),
                which, by means of the inversion lemma of
                rule~\(\TirName{Ord}_2\), leads to~\(y \succ z\). By
                the last rewrite, hypothesis~\(H_1\) is equivalent to
                \(\pred{Ord}{\cons{z}{\fun{ins}(t,x)}}\), which,
                with~\(y \succ z\), enables the use of
                rule~\(\TirName{Ord}_2\) again, leading to
                \(\pred{Ord}{\cons{y,z}{\fun{ins}(t,x)}}\).
                Rewrite~\eqref{eq:D} then yields
                \(\pred{Ord}{\cons{y}{\fun{ins}(s,x)}}\), which,
                together with rewrite~\eqref{eq:A} yields
                \(\pred{Ord}{\fun{ins}(\cons{y}{s},x)}\).\index{induction!example|)}\hfill\(\Box\)

      \end{itemize*}
    \end{itemize*}
  \end{itemize*}
\end{itemize*}

\paragraph{Assessment}

Perhaps the most striking feature of the soundness proof is its
length. More precisely, two aspects may give rise to questions. First,
since the program is four lines long and the specification (the
\(S_i\)~and~\(P_j\)) consists in a total of seven cases, it may be
unclear how the proof raises our confidence in the program. Second,
the proof itself is rather long, which leads us to wonder whether any
error is hiding in it. The first concern can be addressed by noting
that the two parts of the specification are disjoint and thus as easy
to comprehend as the program. Moreover, specifications, being logical
and not necessarily computable, are likely to be more abstract and
composable than programs, so a larger proof may reuse them in
different instances. For example, the predicate~\(\predName{Isrt}\)
can easily be abstracted (higher\hyp{}order) over the sorting function
as \(\pred{Isrt}{f,s} \colon \pred{Ord}{f(s)} \mathrel{\wedge} f(s)
\approx s\)\index{Isrt@\predName{Isrt}}\index{Ord@\predName{Ord}} and
thus applies to many sorting algorithms, with the caveat
that~\((\approx)\) is probably not always suitably defined by
transpositions. The second concern can be completely taken care of by
relying on a \emph{proof assistant}, like\index{Coq@\textsf{Coq}}
\textsf{Coq} \citep{BertotCasteran_2004}. For instance, the formal
specification of~\((\approx)\) and the automatic proofs (by means
of~\texttt{eauto}) of its reflexivity and symmetry consists in the
following script, where \verb|x::s| stands for~\(\cons{x}{s}\),
(\verb|->|) is~\((\Rightarrow)\), \verb|perm s t| is~\(s \approx t\)
and \verb|List| is synonymous with stack:
\begin{verbatim}
Set Implicit Arguments.
Require Import List.
Variable A: Type.

Inductive perm: list A -> list A -> Prop :=
  Pnil  : perm nil nil
| Push  : forall x s t, perm s t -> perm (x::s) (x::t)
| Swap  : forall x y s, perm (x::y::s) (y::x::s)
| Trans : forall s t u, perm s u -> perm u t -> perm s t.

Hint Constructors perm.

Lemma reflexivity: forall s, perm s s.
Proof. induction s; eauto. Qed.

Lemma symmetry: forall s t, perm s t -> perm t s.
Proof. induction 1; eauto. Qed.
\end{verbatim}
\index{insertion sort!soundness|)}

\mypar{Termination}
\index{termination!insertion sort|(}
\index{insertion sort!straight $\sim$!termination|(}

Informally, what soundness means is that, if some program terminates,
then the result is what was expected. This property is called
\emph{partial correctness} when it is relevant to distinguish it from
\emph{total correctness}\index{correctness!total
  $\sim$|see{termination}}, which is partial correctness and
termination. Let us prove the termination of \fun{isrt/1} by the
dependency pairs\index{termination!dependency pair} method
(section~\ref{flattening:termination},
page~\pageref{flattening:termination}). The pairs to order are
\((\fun{ins}(\cons{y}{s},x), \fun{ins}(s,x))_\kappa\),
\((\fun{isrt}(\cons{x}{s}), \fun{isrt}(s))_\nu\) and
\((\fun{isrt}(\cons{x}{s}), \fun{ins}(\fun{isrt}(s), x))_\nu\). By
using the proper subterm\index{induction!proper subterm order}
relation on the first parameter of \fun{ins/2}\index{ins@\fun{ins/2}},
we order the first pair:
\begin{equation*}
\fun{ins}(\cons{y}{s},x) \succ \fun{ins}(s,x) \Leftrightarrow
\cons{y}{s} \succ s.
\end{equation*}
This is enough to prove that \fun{ins/2} terminates. The second pair
is similarly oriented:
\begin{equation*}
\fun{isrt}(\cons{x}{s}) \succ \fun{isrt}(s) \Leftrightarrow
\cons{x}{s} \succ s.
\end{equation*}
The third pair is not worth considering after all, because we already
know that \fun{ins/2} terminates, so the second pair is enough to
entail the termination of \fun{isrt/1}\index{isrt@\fun{isrt/1}}. In
other words, since \fun{ins/2} terminates, it can be considered, as
far as termination analysis is concerned, as a data constructor, so
the third pair becomes useless:
\begin{equation*}
\fun{isrt}(\cons{x}{s}) \succ \underline{\fun{ins}}(\fun{isrt}(s), x)
\Leftrightarrow \fun{isrt}(\cons{x}{s}) \succ \fun{isrt}(s)
\Leftrightarrow \cons{x}{s} \succ s,
\end{equation*}
where \(\fun{\ufun{ins}/2}\) stands for
\fun{ins/2}\index{ins@\fun{ins/2}} considered as a constructor. (We
have used this notation in \fig~\vref{fig:ver}.)\index{insertion
  sort!straight $\sim$!termination|)}\index{termination!insertion
  sort|)}\index{insertion sort!straight $\sim$|)}\hfill\(\Box\)

\section{2-way insertion}
\label{sec:2-way}
\index{insertion sort!2-way $\sim$|(}

Let us recall the definition of sorting by straight insertion:
\begin{equation*}
\begin{array}{@{}r@{\;}l@{\;}l@{\qquad}r@{\;}l@{\;}l@{}}
  \fun{ins}(\cons{y}{s},x)
& \xrightarrow{\smash{\kappa}}
& \cons{y}{\fun{ins}(s,x)}, \,\text{if \(y \succ x\)};
& \fun{isrt}(\el)
& \xrightarrow{\smash{\mu}}
& \el;\\
  \fun{ins}(s,x)
& \xrightarrow{\smash{\lambda}}
& \cons{x}{s}.
& \fun{isrt}(\cons{x}{s})
& \xrightarrow{\smash{\nu}}
& \fun{ins}(\fun{isrt}(s),x).
\end{array}
\end{equation*}
The reason why \fun{ins/2}\index{ins@\fun{ins/2}} is called straight
insertion is because keys are compared in one direction only: from the
top of the stack towards its bottom. We may wonder what would happen
if we could move up or down, \emph{starting from the previously
  inserted key}. Conceptually, this is like having a finger pointing
at the last inserted key and the next insertion resuming from that
point, up or down the stack. Let us call it \emph{two\hyp{}way
  insertion} and name \fun{i2w/1}\index{i2w@\fun{i2w/1}} the sorting
function based upon it. The stack with finger can be simulated by
having two stacks, \(t\)~and~\(u\), such that
\(\fun{rcat}(t,u)\)\index{rcat@\fun{rcat/2}} stands for the currently
sorted stack, corresponding to
\(\fun{isrt}(s)\)\index{isrt@\fun{isrt/1}} in rule~\(\nu\). (In
section~\ref{sec:queueing}, \vpageref{sec:queueing}, we used two
stacks to simulate a queue.) Let us call
\(\fun{rcat}(t,u)\)\index{rcat@\fun{rcat/2}} the \emph{simulated
  stack}\index{stack!simulated $\sim$}; stack~\(t\) is a
\emph{reversed prefix}\index{stack!reversed prefix} of the simulated
stack and stack~\(u\) is a \emph{suffix}\index{stack!suffix}. For
example, a finger pointing at~\(5\) in the simulated stack
\([0,2,4,5,7,8,9]\) would be represented by \([4,2,0]\) and
\([5,7,8,9]\). The reversing of the first stack is best visually
understood by drawing it with the top facing the \emph{right} side of
the page:
\begin{equation*}
\abovedisplayskip=0pt
%\abovedisplayshortskip=0pt
%\belowdisplayskip=0pt
\begin{array}{@{}r|c|c|c|ccc|c|c|c|c|l@{}}
  \multicolumn{1}{c}{} & \multicolumn{1}{c}{} & \multicolumn{1}{c}{}
& \multicolumn{1}{c}{} & \multicolumn{1}{c}{} & \multicolumn{1}{c}{}
& \multicolumn{1}{c}{} & \multicolumn{1}{c}{\downarrow}\\
\cline{2-5}\cline{7-11}
t = & 0 & 2 & 4 & & & & 5 & 7 & 8 & 9 & = u\\
\cline{2-5}\cline{7-11}
\end{array}
\end{equation*}
Given some key~\(x\), it is straightly inserted either in~\(t\)
(minding it is sorted in reverse order) or in~\(u\). If we want to
insert~\(1\), we should pop~\(4\) and push it on the right stack, same
for~\(2\) and then push~\(1\) on the right stack, as, by convention,
the finger always points to the top of the right stack, where the last
inserted key is:
\begin{equation*}
\abovedisplayskip=0pt
\begin{array}{@{}|c|ccc|c|c|c|c|c|c|c|@{}}
  \multicolumn{1}{c}{} & \multicolumn{1}{c}{} & \multicolumn{1}{c}{}
& \multicolumn{1}{c}{} & \multicolumn{1}{c}{\downarrow}\\
\cline{1-2}\cline{4-11}
0 & & & & 1 & 2 & 4 & 5 & 7 & 8 & 9\\
\cline{1-2}\cline{4-11}
\end{array}
\end{equation*}

Let \(\fun{i2w}(s)\)\index{i2w@\fun{i2w/1}} (\emph{insertion going two
  ways}) be the sorted stack corresponding to stack~\(s\). Let
\(\fun{i2w}(s,t,u)\)\index{i2w@\fun{i2w/3}} be the sorted stack
containing all the keys from~\(s\), \(t\) and~\(u\), where \(s\)~is a
suffix of the original (probably unsorted) stack and
\(\fun{rcat}(t,u)\)\index{rcat@\fun{rcat/2}} is the current simulated
stack, that is, \(t\)~is the left stack (reversed prefix) and~\(u\)
the right stack (suffix). The function
\fun{i2w/1}\index{i2w@\fun{i2w/1}} is defined in
\fig~\vref{fig:i2w_def}.
\begin{figure}
\begin{equation*}
\boxed{%
\begin{array}{r@{\;}l@{\;}ll}
\fun{i2w}(s)         & \xrightarrow{\smash{\xi}}
                     & \fun{i2w}(s,\el,\el).\\
\fun{i2w}(\el,\el,u) & \xrightarrow{\smash{\pi}}
                     & u;\\
\fun{i2w}(\el,\cons{y}{t},u)
                     & \xrightarrow{\smash{\rho}}
                     & \fun{i2w}(\el,t,\cons{y}{u});\\
\fun{i2w}(\cons{x}{s},t,\cons{z}{u})
                     & \xrightarrow{\smash{\sigma}}
                     & \fun{i2w}(\cons{x}{s},\cons{z}{t},u),
                     & \text{if \(x \succ z\)};\\
\fun{i2w}(\cons{x}{s},\cons{y}{t},u)
                     & \xrightarrow{\smash{\tau}}
                     & \fun{i2w}(\cons{x}{s},t,\cons{y}{u}),
                     & \text{if \(y \succ x\)};\\
\fun{i2w}(\cons{x}{s},t,u)
                     & \xrightarrow{\smash{\upsilon}}
                     & \fun{i2w}(s,t,\cons{x}{u}).
\end{array}}
\end{equation*}
\caption{Sorting with 2-way insertion \fun{i2w/1}}
\label{fig:i2w_def}
\end{figure}
Rule~\(\xi\) introduces the two stacks used for insertion. Rules
\(\pi\)~and~\(\rho\) could be replaced by \(\fun{i2w}(\el,t,u)
\rightarrow \fun{rcat}(t,u)\), but we opted for a self\hyp{}contained
definition. Rule~\(\sigma\) is used to move keys from the right stack
to the left stack. Rule~\(\tau\) moves them the other
way. Rule~\(\upsilon\) performs the insertion itself, namely, on top
of the right stack. \Fig~\vref{fig:i2w2314} shows the evaluation of
\(\fun{i2w}([2,3,1,4])\), whose trace \index{functional
  language!evaluation!trace} is then \((\xi)(\upsilon)(\sigma\upsilon)
(\tau\upsilon)(\sigma^3\upsilon)(\rho^3\pi)\). The number of times
rule~\(\rho\) is used is the number of keys on the left
stack after there are no more keys to sort. Rule~\(\pi\) is used once.
\begin{figure}[h]
\centering
\includegraphics[bb=71 523 257 721]{i2w2314}
\caption{\(\fun{i2w}([2,3,1,4]) \twoheadrightarrow [1,2,3,4]\)}
\label{fig:i2w2314}
\end{figure}

\mypar{Extremal costs}
\index{insertion sort!2-way $\sim$!minimum cost|(}

Let us find the minimum and maximum costs for an input stack of
\(n\)~keys. The best case will exert minimally rules
\clause{\sigma}~and~\clause{\tau}, and this minimum number of calls
turns out to be zero when the two comparisons are false. The first key
inserted does not use rules \clause{\sigma}~and~\clause{\tau}, but
only rule~\clause{\upsilon}, so, right after, the reversed prefix is
empty and the suffix contains this key. If we want to insert the
second key without moving the first key, and go straight to use
rule~\clause{\upsilon}, the second key must be smaller than the
first. Based on the same argument, the third key must be smaller than
the second etc. In the end, this means that \emph{the input, in the
  best case, is a stack sorted non\hyp{}increasingly.} The last steps
consisting in the reversal of the prefix, such prefix being empty, we
do not even use rule~\clause{\rho} at all --~only rule~\clause{\pi}
once. In other words, the evaluation trace is
\(\zeta\epsilon^n\alpha\) and, if we note \(\B{\fun{i2w}}{n}\) the
cost when the stack contains \(n\)~keys in non\hyp{}increasing order,
then we have
\begin{equation*}
\B{\fun{i2w}}{n} = \len{\zeta\epsilon^n\alpha} = n + 2.
\end{equation*}
\index{insertion sort!2-way $\sim$!minimum cost|)}

\index{insertion sort!2-way $\sim$!maximum cost|(} Let us assume that
the input stack is noted \([x_0, x_1, \dots, x_{n-1}]\) and \(x \prec
y\) means \(y \succ x\). The worst case must exert maximally rules
\clause{\sigma}~and~\clause{\tau}, on the one hand, and rules
\clause{\pi}~and~\clause{\rho}, on the other hand. Let us focus first
on maximising the use of \clause{\sigma}~and~\clause{\tau}. Since
\(x_0\)~is the first key, it is always pushed on the suffix stack by
rule~\clause{\upsilon}. The second key, \(x_1\), in order to travel
the furthest, has to be inserted below~\(x_0\). By doing so,
rule~\clause{\sigma} is used once and then~\clause{\upsilon},
therefore, as a result, \(x_0\)~is on the left (the reversed prefix)
and \(x_1\)~on the right (the suffix). In other words: we have
\([x_0]\) and \([x_1]\). Because of this symmetry, in pursuit of the
worst case, we can now move either \(x_0\)~or~\(x_1\) to the facing
stack, that is, choose either to set \(x_2 \prec x_0\) or \(x_1 \prec
x_2\).
\begin{itemize}

\item If \(x_2 \prec x_0\), rule~\clause{\tau} is used once, then
  rule~\clause{\upsilon}. As a result, we have the configuration
  \(\el\) and \([x_2, x_0, x_1]\). This translates as \(x_2 \prec x_0
  \prec x_1\). The fourth key, \(x_3\), must be inserted at the bottom
  of the right stack, which must be first reversed on top of the left
  stack by rule~\clause{\sigma}: we then obtain \([x_1, x_0, x_2]\)
  and \([x_3]\), that is, \(x_2 \prec x_0 \prec x_1 \prec
  x_3\). Finally, the left stack is reversed on top of the second by
  rule~\clause{\rho} and rule~\clause{\pi} is last. The evaluation
  trace is
  \((\xi)(\upsilon)(\sigma\upsilon)(\tau\upsilon)(\sigma^3\upsilon)
  (\rho^3\pi)\), whose length is~\(14\).

\item If \(x_1 \prec x_2\), we would have \([x_1, x_0]\) and
  \([x_2]\), then the stacks \(\el\) and \([x_3, x_0, x_1, x_2]\),
  that is, \(x_3 \prec x_0 \prec x_1 \prec x_2\). The complete
  evaluation trace is
  \((\xi)(\upsilon)(\sigma\upsilon)(\sigma\upsilon)(\tau^2\upsilon)
  (\pi)\). The length of this trace is~\(10\), which is shorter than
  the previous trace.

\end{itemize}
As a conclusion, the choice \(x_2 \prec x_0\) leads to a worse
case. But what if the input stack contains an odd number~\(n\) of
keys? To guess what happens, let us insert~\(x_4\) assuming either
\(x_1 \prec x_2\) or \(x_2 \prec x_0\).
\begin{itemize}

\item If \(x_2 \prec x_0\), we move all the keys out of the left
  stack, yielding the configuration \([x_4]\) and \([x_2, x_0, x_1,
  x_3]\), so \(x_4 \prec x_2 \prec x_0 \prec x_1 \prec x_3\),
  corresponding to the trace
  \((\xi)(\upsilon)(\sigma\upsilon)(\tau\upsilon)(\sigma^3\upsilon)
  (\tau^3\upsilon)(\rho\pi)\), whose length is~\(16\).

\item If \(x_1 \prec x_2\), we want to insert~\(x_4\) at the bottom of
  the right stack, thus obtaining \([x_2, x_1, x_0, x_3]\) and
  \([x_4]\): \(x_3 \prec x_0 \prec x_1 \prec x_2 \prec x_4\),
  corresponding to the trace \((\xi)(\upsilon)
  (\sigma\upsilon)(\sigma\upsilon)(\tau^2\upsilon)(\sigma^4\upsilon)
  (\rho^4\pi)\), whose length is~\(19\). It is perhaps better
  visualised by means of oriented edges, revealing a spiral in
  \fig~\vref{fig:spiral}.
    \begin{figure}
      \centering
      \includegraphics[bb=71 671 185 716]{a3a0a1a2a4}
      \caption{Worst case for \fun{i2w/1} if \(n=5\) (\(x_1 \prec x_2\))}
      \label{fig:spiral}
    \end{figure}

\end{itemize}
Therefore, it seems that when the number of keys is odd, having \(x_1
\prec x_2\) leads to the maximum cost, whilst \(x_2 \prec x_0\) leads
to the maximum cost when the number of keys is even. Let us determine
these costs for any~\(n\) and find out which is the greater. Let us
note \(\W{x_1 \prec x_2}{2p+1}\) the former cost and \(\W{x_2 \prec
  x_0}{2p}\) the latter.
\begin{itemize}

\item If \(n = 2p+1\) and \(x_1 \prec x_2\), then the evaluation trace
  is
    \begin{equation*}
    (\xi)(\upsilon)(\sigma\upsilon)(\sigma\upsilon)
    (\tau^2\upsilon)(\sigma^4\upsilon)(\tau^4\upsilon) \ldots
    (\sigma^{2p-2}\upsilon)(\tau^{2p-2}\upsilon)(\sigma^{2p}\upsilon)
    (\rho^{2p}\pi),
    \end{equation*}
    as a partial evaluation with~\(p=3\) suggests:
    \begin{equation*}
      \!\begin{array}{@{}r@{\;}c@{\;}l@{}}
        \fun{i2w}([x_0,x_1,x_2,x_3,x_4,x_5,x_6])
        & \xrightarrow{\smash{\xi}}
        & \fun{i2w}(\el,\el,[x_0,x_1,x_2,x_3,x_4,x_5,x_6])\\
        & \xrightarrow{\smash{\upsilon}}
        & \fun{i2w}(\el,[x_0],[x_1,x_2,x_3,x_4,x_5,x_6])\\
        & \xrightarrow{\smash{\sigma}}
        & \fun{i2w}([x_0],\el,[x_1,x_2,x_3,x_4,x_5,x_6])\\
        & \xrightarrow{\smash{\upsilon}}
        & \fun{i2w}([x_0],[x_1],[x_2,x_3,x_4,x_5,x_6])\\
        & \xrightarrow{\smash{\sigma}}
        & \fun{i2w}([x_1,x_0],\el,[x_2,x_3,x_4,x_5,x_6])\\
        & \xrightarrow{\smash{\upsilon}}
        & \fun{i2w}([x_1,x_0],[x_2],[x_3,x_4,x_5,x_6])\\
        & \stackrel{\smash{\tau^{\smash{2}}}}{\twoheadrightarrow}
        & \fun{i2w}(\el,[x_0,x_1,x_2],[x_3,x_4,x_5,x_6])\\
        & \xrightarrow{\smash{\upsilon}}
        & \fun{i2w}(\el,[x_3,x_0,x_1,x_2],[x_4,x_5,x_6])\\
        & \stackrel{\smash{\sigma^{\smash{4}}}}{\twoheadrightarrow}
        & \fun{i2w}([x_2,x_1,x_0,x_3],\el,[x_4,x_5,x_6])\\
        & \xrightarrow{\smash{\upsilon}}
        & \fun{i2w}([x_2,x_1,x_0,x_3],[x_4],[x_5,x_6])\\
        & \stackrel{\smash{\tau^{\smash{4}}}}{\twoheadrightarrow}
        & \fun{i2w}(\el,[x_3,x_0,x_1,x_2,x_4],[x_5,x_6])\\
        & \xrightarrow{\smash{\upsilon}}
        & \fun{i2w}(\el,[x_5,x_3,x_0,x_1,x_2,x_4],[x_6])\\
        & \stackrel{\smash{\sigma^{\smash{6}}}}{\twoheadrightarrow}
        & \fun{i2w}([x_4,x_2,x_1,x_0,x_3,x_5],\el,[x_6])\\
        & \xrightarrow{\smash{\upsilon}}
        & \fun{i2w}([x_4,x_2,x_1,x_0,x_3,x_5],[x_6],\el).
      \end{array}
    \end{equation*}
    If we omit rules~\clause{\xi}, \clause{\upsilon}, \clause{\pi}
    and~\clause{\rho}, we can see a pattern emerge from the subtrace
    \((\sigma^2\tau^2)(\sigma^4\tau^4)(\sigma^6\tau^6) \ldots
    (\sigma^{2p-2}\tau^{2p-2})(\sigma^{2p})\). Rule~\clause{\upsilon}
    is used \(n\)~times because it inserts the key in the right
    place. So the total cost is
    \begin{align*}
      \W{x_1 \prec x_2}{2p+1}
        &= \len{\xi} + \len{\upsilon^{2p+1}}
           + \sum_{k=1}^{p-1}{\left(\len{\sigma^{2k}} + \len{\tau^{2k}}\right)}
           + \len{\sigma^{2p}} + \len{\rho^{2p}\pi}\\
        &= 1 + (2p + 1) + \sum_{k=1}^{p-1}{2(2k)} + (2p) + (2p + 1)\\
        &= 2p^2 + 4p + 3.
    \end{align*}

  \item If \(n = 2p\) and \(x_2 \prec x_0\), then the evaluation trace is
    \begin{equation*}
      (\xi)(\upsilon)(\sigma\upsilon)(\tau\upsilon)
      (\sigma^3\upsilon)(\tau^3\upsilon)
      \ldots (\sigma^{2p-1}\upsilon)(\rho^{2p-1}\pi),
    \end{equation*}
    as the following partial evaluation with~\(p=3\) suggests (first
    difference with the previous case is in boldface type):
\begin{equation*}
\begin{array}{@{}r@{\;}c@{\;}l@{}}
\fun{i2w}([x_{0},x_{1},x_{2},x_{3},x_{4},x_{5}])
& \rightarrow
& \fun{i2w}(\el,\el,[x_{0},x_{1},x_{2},x_{3},x_{4},x_{5}])\\
& \xrightarrow{\smash{\upsilon}}
& \fun{i2w}(\el,[x_{0}],[x_{1},x_{2},x_{3},x_{4},x_{5}])\\
& \xrightarrow{\smash{\sigma}}
& \fun{i2w}([x_{0}],\el,[x_{1},x_{2},x_{3},x_{4},x_{5}])\\
& \xrightarrow{\smash{\upsilon}}
& \fun{i2w}([x_{0}],[x_{1}],[x_{2},x_{3},x_{4},x_{5}])\\
& \xrightarrow{\smash{\tau}}
& \boldsymbol{\fun{i2w}(\el,[x_{0},x_{1}],[x_{2},x_{3},x_{4},x_{5}])}\\
& \xrightarrow{\smash{\upsilon}}
& \fun{i2w}(\el,[x_{2},x_{0},x_{1}],[x_{3},x_{4},x_{5}])\\
& \stackrel{\smash{\sigma^{\smash{3}}}}{\twoheadrightarrow}
& \fun{i2w}([x_{1},x_{0},x_{2}],\el,[x_{3},x_{4},x_{5}])\\
& \xrightarrow{\smash{\upsilon}}
& \fun{i2w}([x_{1},x_{0},x_{2}],[x_{3}],[x_{4},x_{5}])\\
& \stackrel{\smash{\tau^{\smash{3}}}}{\twoheadrightarrow}
& \fun{i2w}(\el,[x_{2},x_{0},x_{1},x_{3}],[x_{4},x_{5}])\\
& \xrightarrow{\smash{\upsilon}}
& \fun{i2w}(\el,[x_{4},x_{2},x_{0},x_{1},x_{3}],[x_{5}])\\
& \stackrel{\smash{\sigma^{\smash{5}}}}{\twoheadrightarrow}
& \fun{i2w}([x_{3},x_{1},x_{0},x_{2},x_{4}],\el,[x_{5}])\\
& \xrightarrow{\smash{\upsilon}}
& \fun{i2w}([x_{3},x_{1},x_{0},x_{2},x_{4}],[x_{5}],\el)
\end{array}
\end{equation*}
If we omit rules~\clause{\xi}, \clause{\upsilon}, \clause{\pi}
and~\clause{\rho}, we can see a pattern emerge from the subtrace
\((\sigma^1\tau^1)(\sigma^3\tau^3)(\sigma^5\tau^5) \ldots
(\sigma^{2p-3}\tau^{2p-3})(\sigma^{2p-1})\). Rule~\clause{\upsilon} is
used \(n\)~times because it inserts the key in the right place. So the
total cost is
\begin{align*}
\W{x_2 \prec x_0}{2p}
   &= \len{\xi} + \len{\upsilon^{2p}}
           + \sum_{k=1}^{p-1}{\left(\len{\sigma^{2k-1}} + \len{\tau^{2k-1}}\right)}
           + \len{\sigma^{2p-1}} + \len{\rho^{2p-1}\pi}\\
   &= 1 + (2p) + \sum_{k=1}^{p-1}{2(2k-1)} + (2p-1) + ((2p - 1) + 1)\\
   &= 2p^2 + 2p + 2.
\end{align*}
\end{itemize}
These formulas hold for all \(p \geqslant 0\). We can now conclude
this discussion about the worst case of \fun{i2w/1}:
\begin{itemize}

\item If \(n = 2p\), the worst case happens when the keys satisfy the
  total order \(x_{2p} \prec x_{2p-2} \prec \dots \prec x_0 \prec x_1
  \prec x_3 \prec \dots \prec x_{2p-3} \prec x_{2p-1}\) and
  \(\W{\fun{i2w}}{2p} = 2p^2 + 2p + 2\), that is, \(\W{\fun{i2w}}{n} =
  \frac{1}{2}{n^2} + n + 2\).

\item If \(n = 2p+1\), the worst case happens when the keys satisfy
  the order \(x_{2p-1} \prec x_{2p-3} \prec \dots \prec x_3 \prec x_0
  \prec x_1 \prec x_2 \prec \dots \prec x_{2p-2} \prec x_{2p}\) and
  \(\W{\fun{i2w}}{2p+1} = 2p^2 + 4p + 3\), that is, \(\W{\fun{i2w}}{n}
  = \frac{1}{2}{n^2} + n + \frac{3}{2}\).

\end{itemize}
The first case yields the maximum cost:
\begin{equation*}
%\abovedisplayskip=0pt
\belowdisplayskip=0pt
  \W{\fun{i2w}}{n} = \frac{1}{2}{n^2} + n + 2 = \W{\fun{isrt}}{n} - n
  + 1 \sim
  \W{\fun{isrt}}{n} \sim \frac{1}{2}{n^2}.
\end{equation*}
\index{insertion sort!2-way $\sim$!maximum cost|)}


\mypar{Average cost}
\index{insertion sort!2-way $\sim$!average cost|(}

Let \(\M{\fun{i2w}}{n}\)\index{i2w@$\M{\fun{i2w}}{n}$} be the average
cost of the call \(\fun{i2w}(s)\)\index{i2w@\fun{i2w/1}}, where the
stack~\(s\) has length~\(n\). We are going to use the same assumption
as with \(\M{\fun{isrt}}{n}\), namely, we look for the cost for
sorting all permutations of \((1,2,\dots,n)\), divided by~\(n!\). The
insertions are illustrated by the \emph{evaluation
  tree}\index{tree!evaluation $\sim$} in \fig~\ref{fig:2way_unbal},
\begin{figure}[t]
\centering
\includegraphics{2way_unbal}
\caption{Sorting \([a,b,c]\) (first argument of \fun{i2w/3} hidden)}
\label{fig:2way_unbal}
\end{figure}
where the keys \(a\), \(b\) and~\(c\) are inserted in this order in a
stack originally empty, with all possible total orders. Note how all
permutations are attained exactly once at the external
nodes\index{tree!node!external $\sim$} (see
page~\pageref{def:external_node}). For example, \([a,b,c]\) and
\([c,b,a]\) are external nodes. The total cost is the \emph{external
  path length}\index{tree!external path
  length}\label{external_path_length} of the tree, that is, the sum of
the lengths of the paths from the root to all the external nodes,
which is the same as the sum of the lengths of all possible
traces\index{functional
  language!evaluation!trace}\index{tree!evaluation $\sim$}:
\(\len{\xi\upsilon\sigma\upsilon\sigma\upsilon\rho^2\pi} +
\len{\xi\upsilon\sigma\upsilon\tau\upsilon\pi} +
\len{\xi\upsilon\sigma\upsilon^2\rho\pi} +
\len{\xi\upsilon^2\sigma^2\upsilon\rho^2\pi} +
\len{\xi\upsilon^2\sigma\upsilon\rho\pi} + \len{\xi\upsilon^3\pi} =
44\), so the average cost of sorting \(3\)~keys is \(44/3! = 22/3\).

Given a left stack of \(p\)~keys and a right stack of \(q\)~keys, let
us characterise all the possible traces for the insertion of one more
key, stopping before another key is inserted or the final stack is
made. In the left stack, an insertion is possible after the first key,
after the second etc. until after the last. After the \(k\)th key,
with \(1 \leqslant k \leqslant p\), the trace is thus
\(\tau^k\upsilon\). In the right stack, an insertion is possible on
top, after the first key, after the second etc. until after the
last. After the \(k\)th key, with \(0 \leqslant k \leqslant q\), the
trace is hence \(\sigma^k\upsilon\). All the possible traces are thus
\begin{equation*}
\sum_{k=1}^{p}{\tau^k\upsilon} + \sum_{k=0}^{q}{\sigma^k\upsilon},
\end{equation*}
whose cumulated lengths amount to
\begin{equation*}
C_{p,q} := \sum_{k=1}^{p}\len{\tau^k\upsilon} +
\sum_{k=0}^{q}\len{\sigma^k\upsilon}
= (p+q+1) + \frac{1}{2}p(p+1) + \frac{1}{2}q(q+1).
\end{equation*}
There are \(p+q+1\) insertion loci, so the average cost of one
insertion in the configuration \((p,q)\) is
\begin{equation}
\M{}{p,q} := \frac{C_{p,q}}{p+q+1}
           = 1 + \frac{p^2 + q^2 + p + q}{2p + 2q + 2}.
\label{eq:Mpq}
\end{equation}
By letting \(k := p + q\), we can re\hyp{}express this cost as
\begin{equation*}
\M{}{k-q,q} = \frac{1}{k+1}{q^2} - \frac{k}{k+1}{q} + \frac{k+2}{2}.
\end{equation*}
The left stack is reversed after the last insertion, so the subsequent
traces are \(\rho^{p-k}\pi\), with \(1 \leqslant k \leqslant p\), if
the last insertion took place on the left, otherwise
\(\rho^{p+k}\pi\), with \(0 \leqslant k \leqslant q\), that is,
\(\rho^0\pi\), \(\rho^1\pi\), \ldots, \(\rho^{p+q}\pi\). In other
words, after an insertion, all possible configurations are uniquely
realised (only the right stack being empty is invalid, due to
rule~\(\upsilon\)). As a consequence, we can average the average costs
of inserting one key over all the partitions of~\(k\) into~\(p + q\),
with \(q \neq 0\), so the average cost of inserting one key in a
simulated stack of \(k\)~keys is
\begin{equation*}
\M{}{0} := 1;\quad
\M{}{k} := \frac{1}{k}\!\sum_{p+q=k}\M{}{p,q}
         = \frac{1}{k}\sum_{q=1}^{k}\M{}{k-q,q}
         = \frac{1}{3}k + \frac{7}{6},
\end{equation*}
minding that \(\sum_{q=1}^{k}{q^2} = k(k+1)(2k+1)/6\) (see
equation~\eqref{eq:sum_of_squares} \vpageref{eq:sum_of_squares}). The
cost of the final reversal is also averaged over all possible
configurations, here of \(n>0\) keys:
\begin{equation*}
\M{\curvearrowright}{n} = \frac{1}{n}\sum_{k=0}^{n-1}\len{\rho^k\pi}
                        = \frac{n+1}{2}.
\end{equation*}
Finally, we know that all traces start with~\(\xi\), then proceed with
all the insertions and conclude with a reversal. This means that the
average cost \(\M{\fun{i2w}}{n}\)\index{i2w@$\M{\fun{i2w}}{n}$} of
sorting \(n\)~keys is defined by the following equations:
\begin{equation*}
\M{\fun{i2w}}{0} = 2;\quad
\M{\fun{i2w}}{n} = 1 + \sum_{k=0}^{n-1}{\M{}{k}} + \M{\curvearrowright}{n}
                 = \frac{1}{6}n^2 + \frac{3}{2}n + \frac{4}{3}
                 \sim \frac{1}{6}n^2.
\end{equation*}
We can check that \(\M{\fun{i2w}}{3} = 22/3\), as expected. As a
conclusion, in average, sorting with 2-way insertions is faster than
with straight insertions, but the cost is still asymptotically
quadratic.\index{insertion sort!2-way $\sim$!average
  cost|)}\index{insertion sort!2-way $\sim$|)}


\paragraph{Exercises}
\begin{enumerate}

  \item When designing~\fun{i2w/3}\index{i2w@\fun{i2w/3}}, we chose to
    always push the key to be inserted on top of the right stack in
    rule~\(\upsilon\). Let us modify slightly this strategy and push
    instead onto the left stack when it is empty. See
    rule~(\(\leadsto\)) in \fig~\vref{fig:i2w1}.
    \begin{figure}[b]
    \begin{equation*}
      \boxed{%
      \begin{array}{r@{\;}l@{\;}ll}
        \fun{i2w}_1(s) & \rightarrow
                      & \fun{i2w}_1(s,\el,\el).\\
        \fun{i2w}_1(\el,\el,u) & \rightarrow & u;\\
        \fun{i2w}_1(\el,\cons{y}{t},u)
                     & \rightarrow
                     & \fun{i2w}_1(\el,t,\cons{y}{u});\\
        \fun{i2w}_1(\cons{x}{s},t,\cons{z}{u})
                     & \rightarrow
                     & \fun{i2w}_1(\cons{x}{s},\cons{z}{t},u),
                     & \text{if \(x \succ z\)};\\
        \fun{i2w}_1(\cons{x}{s},\el,u)
                     & \leadsto
                     & \fun{i2w}_1(s,[x],u);\\
        \fun{i2w}_1(\cons{x}{s},\cons{y}{t},u)
                     & \rightarrow
                     & \fun{i2w}_1(\cons{x}{s},t,\cons{y}{u}),
                     & \text{if \(y \succ x\)};\\
        \fun{i2w}_1(\cons{x}{s},t,u)
                     & \rightarrow
                     & \fun{i2w}_1(s,t,\cons{x}{u}).
      \end{array}}
    \end{equation*}
    \caption{Variation \fun{i2w\(_1\)/1} on \fun{i2w/1} (see
      (\(\leadsto\)))}
    \label{fig:i2w1}
    \end{figure}
    Prove that the average cost satisfies now the equation
    \index{i2w1@\fun{i2w\(_1\)/1}} \index{i2w1@\fun{i2w\(_1\)/3}}
    \index{i2w1@$\M{\fun{i2w}_1}{n}$}
    \begin{equation*}
      \M{\fun{i2w}_1}{n} = \M{\fun{i2w}}{n} - H_n + 2.
    \end{equation*}
    \emph{Hint:} write down the examples similar to the ones in
    \fig~\vref{fig:2way_unbal}, determine the average path length and
    observe that the difference with
    \fun{i2w/1}\index{i2w@\fun{i2w/1}} is that the configuration with
    an empty left stack is replaced with a configuration with a
    singleton left stack, to wit, \(\M{}{0,k}\) is replaced
    with~\(\M{}{1,k-1}\) in the definition of~\(\M{}{k}\).

  \item In rule~\(\upsilon\) of~\fun{i2w/3}\index{i2w@\fun{i2w/3}},
    the key~\(x\) is pushed on the right stack. Consider in
    \fig~\vref{fig:i2w2} (rule~(\(\leadsto\))) the variant where it is
    pushed on the left stack instead. Show very simply that the
    average cost satisfies\index{i2w2@\fun{i2w\(_2\)/1}}
    \index{i2w2@\fun{i2w\(_2\)/3}} \index{i2w2@$\M{\fun{i2w}_2}{n}$}
    \begin{equation*}
      \M{\fun{i2w}_2}{n} = \M{\fun{i2w}}{n} + 1.
    \end{equation*}
    \begin{figure}[h]
    \begin{equation*}
      \boxed{%
      \begin{array}{r@{\;}l@{\;}ll}
        \fun{i2w}_2(s)         & \rightarrow
                               & \fun{i2w}_2(s,\el,\el).\\
        \fun{i2w}_2(\el,\el,u) & \rightarrow
                               & u;\\
        \fun{i2w}_2(\el,\cons{y}{t},u)
                               & \rightarrow
                               & \fun{i2w}_2(\el,t,\cons{y}{u});\\
        \fun{i2w}_2(\cons{x}{s},t,\cons{z}{u})
                               & \rightarrow
                               & \fun{i2w}_2(\cons{x}{s},\cons{z}{t},u),
                               & \text{if \(x \succ z\)};\\
        \fun{i2w}_2(\cons{x}{s},\cons{y}{t},u)
                               & \rightarrow
                               & \fun{i2w}_2(\cons{x}{s},t,\cons{y}{u}),
                               & \text{if \(y \succ x\)};\\
        \fun{i2w}_2(\cons{x}{s},t,u)
                               & \leadsto
                               & \fun{i2w}_2(s,\cons{x}{t},u).
      \end{array}}
    \end{equation*}
    \caption{Variation \fun{i2w\(_2\)/1} on \fun{i2w/1} (see
      (\(\leadsto\)))}
    \label{fig:i2w2}
    \end{figure}

\end{enumerate}

\section{Balanced 2-way insertion}
\index{insertion sort!2-way $\sim$!balanced $\sim$|(}

When sorting with 2-way insertions, keys are inserted from whence the
finger is on the simulated stack. We could maintain the finger at the
middle of the stack, leading to what we call \emph{balanced 2-way
  insertions}. The adjective `balanced' refers to the shape of the
comparison tree.

Our best effort to keep the two stacks about the same length must lead
to two cases: either (\textsl{a})~they are exactly of the same length,
or (\textsl{b})~one of them, say the right one, contains one more
key. Let us envisage how to maintain this invariant through
insertions. Let us suppose we are in case~(\textsl{b}). Then, if the
key has to be inserted in the left stack, the resulting stacks will
have equal lengths, which means case~(\textsl{a}); otherwise, we move
the top of the right stack to the top of the left stack, in addition
to the insertion itself, and we are back to case~(\textsl{a}) as
well. If we are in case~(\textsl{a}) and the insertion takes place in
the right stack, no rebalancing has to be done; otherwise, the top of
the left stack is moved to the top of the right: in both events, we
are in case~(\textsl{b}). What if the key has to be inserted at the
finger position?  If the two stacks have same length, that is,
case~(\textsl{a}), we push the key on top of the right one and go back
to case~(\textsl{b}); otherwise, it means that the right stack exceeds
the left by one, that is, case~(\textsl{b}), so it is best to push it
on the left stack: as a result, the stacks end having equal lengths
and we are back to case~(\textsl{a}).

To program this algorithm, we need a variant
\fun{idn/2}\index{idn@\fun{idn/2}} (\emph{insert downwardly})
of~\fun{ins/2}\index{ins@\fun{ins/2}} because the left stack is sorted
decreasingly. Let us rename~\fun{ins/2}
into~\fun{iup/2}\index{iup@\fun{iup/2}} (\emph{insert upwardly}).
\begin{equation*}
\begin{array}{@{}r@{\;}l@{\;}lr@{\;}l@{\;}l@{}}
  \fun{iup}(\cons{y}{s},x)
& \xrightarrow{\smash{\kappa_0}}
& \cons{y}{\fun{iup}(s,x)},\,\text{if \(y \succ x\)};
& \fun{iup}(s,x)
& \xrightarrow{\smash{\lambda_0}} & \cons{x}{s}.\\
  \fun{idn}(\cons{y}{s},x)
& \xrightarrow{\smash{\kappa_1}}
& \cons{y}{\fun{idn}(s,x)},\,\text{if \(x \succ y\)};
& \fun{idn}(s,x)
& \xrightarrow{\smash{\lambda_1}} & \cons{x}{s}.
\end{array}
\end{equation*}
Furthermore, we need an additional parameter that represents the
difference in length between the two stacks: \(0\)~if they have the
same length and \(1\)~if the right contains one more key. Let us call
the new function
\fun{i2wb/1}\index{i2wb@\fun{i2wb/1}}\index{i2wb@\fun{i2wb/3}}, whose
definition is displayed in \fig~\vref{fig:i2wb}.
\begin{figure}[t]
\begin{equation*}
\boxed{%
\begin{array}{r@{\;}l@{\;}ll}
\fun{i2wb}(s)
& \xrightarrow{\smash{\xi}} & \fun{i2wb}(s,\el,\el,0).\\
\fun{i2wb}(\el,\el,u,d)
& \xrightarrow{\smash{\pi}} & u;\\
\fun{i2wb}(\el,\cons{y}{t},u,d)
& \xrightarrow{\smash{\rho}} & \fun{i2wb}(\el,t,\cons{y}{u},d);\\
\fun{i2wb}(\cons{x}{s},t,\cons{z}{u},0)
& \xrightarrow{\smash{\sigma}} &
\fun{i2wb}(s,t,\cons{z}{\fun{iup}(u,x)},1),
& \text{if \(x \succ z\)};\\
\fun{i2wb}(\cons{x}{s},\cons{y}{t},u,0)
& \xrightarrow{\smash{\tau}} &
\fun{i2wb}(s,\fun{idn}(t,x),\cons{y}{u},1),
& \text{if \(y \succ x\)};\\
\fun{i2wb}(\cons{x}{s},t,u,0)
& \xrightarrow{\smash{\upsilon}} & \fun{i2wb}(s,t,\cons{x}{u},1);\\
\fun{i2wb}(\cons{x}{s},t,\cons{z}{u},1)
& \xrightarrow{\smash{\phi}}
& \fun{i2wb}(s,\cons{z}{t},\fun{iup}(u,x),0),
& \text{if \(x \succ z\)};\\
\fun{i2wb}(\cons{x}{s},\cons{y}{t},u,1)
& \xrightarrow{\smash{\chi}}
& \fun{i2wb}(s,\cons{y}{\fun{idn}(t,x)},u,0),
& \text{if \(y \succ x\)};\\
\fun{i2wb}(\cons{x}{s},t,u,1) & \xrightarrow{\smash{\psi}} &
\fun{i2wb}(s,\cons{x}{t},u,0).
\end{array}}
\end{equation*}
\caption{Balanced 2-way insertion}
\label{fig:i2wb}
\end{figure}
In \fig~\vref{fig:2way_bal}
\begin{figure}
\centering
\includegraphics{2way_bal}
\caption{Sorting \([a,b,c]\) by balanced 2-way insertions}
\label{fig:2way_bal}
\end{figure}
are shown all the possible traces\index{tree!evaluation $\sim$} and
outcomes of sorting \([a,b,c]\). Note that the tree is not
perfect\index{tree!perfect $\sim$}, but balanced, as some arrows
correspond to two rewrites. The \emph{internal path
  length}\label{insertion__internal_path_length}
\index{binary tree!internal path length} of the tree is~\(43\), that
is the sum of the lengths of the paths from the root to each internal
node, so the average cost is~\(43/6\).

\mypar{Minimum cost}

Let us continue by finding what is the minimum cost of
\fun{i2wb/1}. Let us assume that we have the input \([x_0, x_1, x_2,
x_3, x_4]\) and we want it to minimise the rewrites, which means not
to use rules~\clause{\sigma}, \clause{\tau}, \clause{\phi}
and~\clause{\chi}; also, the usage of rule~\clause{\rho} should be
minimum. The latter rule is not an issue because it reverses the left
stack and, by design, the right stack has the same length as the left,
or exceeds it at most by one key. A simple diagram with the two stacks
initially empty suffices to convince us that the keys must go
alternatively to the right and then to the left, leading, for example,
to \([x_3 , x_1]\) and \([x_4, x_2, x_0]\). This is perhaps better
visualised by means of oriented edges revealing a whirlpool in
\fig~\vref{fig:whirlpool},
\begin{figure}[b]
\centering
\includegraphics[bb=71 671 185 716]{a1a3a4a2a0}
\caption{Best case for \fun{i2wb/1} if \(n=5\)}
\label{fig:whirlpool}
\end{figure}
to be contrasted with the spiral in \fig~\vref{fig:spiral} for
\fun{i2w/1}.

The rule definining \fun{i2w/1} has to be used first. Then each key is
inserted, alternatively by means of rule~\clause{\upsilon}
and~\clause{\psi}. Finally, the left stack is reversed by
rules~\clause{\pi} and~\clause{\rho}, so the question hinges on
determining the length of the left stack in the best case. By design,
if the total number of keys is even, then the two stacks will end up
containing, before using rule~\clause{\rho}, exactly half of them,
because the stacks have the same length. If the total is odd, the left
stack contains the integral part of this number halved. Technically,
let us note \(\B{\fun{i2wb}}{n}\) the cost of any call
\(\fun{i2wb}(s)\), where the stack~\(s\) contains \(n\)~keys. If \(p
\geqslant 0\), then
\begin{equation*}
\B{\fun{i2wb}}{2p}   = 1 +     2p + p = 3p + 1,\quad
\B{\fun{i2wb}}{2p+1} = 1 + (2p+1) + p = 3p + 2.
\end{equation*}
Another, more compact, way to put it is: \(\B{\fun{i2wb}}{n} = 1 + n
+ \floor{n/2} \sim \tfrac{3}{2}{n}\). The equivalence is correct
because \(n/2-1 < \floor{n/2} \leqslant n/2\).

\paragraph{Exercise}

The worst case occurs when insertions are repeatedly performed at the
bottom of the longest stack. Find \(\W{\fun{i2wb}}{n}\) and
characterise the worst case.

\mypar{Average cost}

Let us consider the average cost when \(n=2p\). Then\index{i2wb@$\M{\fun{i2wb}}{n}$}
\begin{equation*}
\M{\fun{i2wb}}{2p} =
  1 + \sum_{k=0}^{2p-1}{\M{}{k}} + \M{\curvearrowright}{p},
\end{equation*}
where \(\M{}{k}\)~is the average cost of inserting a key into a
simulated stack of \(k\)~keys and
\(\M{\curvearrowright}{p}\)~is\index{rev@$\M{\curvearrowright}{n}$}
the cost of reversing \(p\)~keys from left to right. The variable~\(p\)
in~\(\M{\curvearrowright}{p}\) is correct since there are
\(\floor{n/2}\)~keys in the left stack after all insertions are
over. Clearly,
\begin{equation*}
\M{\curvearrowright}{p} = p + 1.
\end{equation*}
The determination of~\(\M{}{k}\) requires the consideration of only
two cases: \(k\)~is even or not. When analysing the average cost of
\fun{i2w/1}, there were much more configurations to take into account
because not all the insertions lead to balanced stacks. If \(k\)~is
even, then there exists an integer~\(j\) such that \(k=2j\) and
\begin{equation*}
\M{}{2j} = \M{}{j,j},
\end{equation*}
where \(\M{}{j,j}\) is the average number of rewrites to insert a
random number into a configuration of two stacks of length~\(j\). We
already computed~\(\M{}{p,q}\) in equation~\eqref{eq:Mpq}
\vpageref{eq:Mpq}. Consequently,
\begin{equation*}
\M{}{2j} = \frac{j^2 + 3j + 1}{2j+1}
         = \frac{1}{2}{j} - \frac{1}{4} \cdot \frac{1}{2j+1} +
         \frac{5}{4}.
\end{equation*}
The case \(k=2j+1\) is similarly derived: \(\M{}{2j+1} =
(j+3)/2\). Hence:
\begin{align}
\M{\fun{i2wb}}{2p}
  &= 1 + \sum_{k=0}^{2p-1}{\M{}{k}} + (p+1)
   = 2 + p + \sum_{j=0}^{p-1}{(\M{}{2j} + \M{}{2j+1})}\notag\\
  &= \frac{1}{2}{p^2} + \frac{13}{4}{p} + 2 -
             \frac{1}{4}\sum_{j=0}^{p-1}{\frac{1}{2j+1}}.
\label{eq:i2wb_2p}
\end{align}
We need to find the value of this sum. Let \(H_n :=
\sum_{k=1}^n{1/k}\) be the \(n\)th \emph{harmonic
  number}\index{harmonic number}. Then
\begin{equation*}
H_{2p} = \sum_{j=0}^{p-1}{\frac{1}{2j+1}} + \sum_{j=1}^{p}{\frac{1}{2j}}
      = \sum_{j=0}^{p-1}{\frac{1}{2j+1}} + \frac{1}{2}{H_{p}}.
\end{equation*}
We can now replace our sum by harmonic numbers in
equation~\eqref{eq:i2wb_2p}:
\begin{equation*}
\M{\fun{i2wb}}{2p}
  = \frac{1}{2}{p^2} + \frac{13}{4}{p} - \frac{1}{4}{H_{2p}}
    + \frac{1}{8}{H_p} + 2.
\end{equation*}
The remaining case is to find \(\M{\fun{i2wb}}{2p+1}\), which, by the
same reckoning, is
\begin{equation*}
\M{\fun{i2wb}}{2p+1}
  = 1 + \sum_{k=0}^{2p}{\M{}{k}} + \M{\curvearrowright}{p}.
\end{equation*}
Let us reuse previous calculations:
\begin{equation*}
\M{\fun{i2wb}}{2p+1}
   = \M{\fun{i2wb}}{2p} + \M{}{2p}
   = \frac{1}{2}{p^2} + \frac{15}{4}{p} - \frac{1}{4}{H_{2p+1}}
     + \frac{1}{8}{H_p} + \frac{13}{4}.
\end{equation*}
We have \(1 + x < e^x\), for all real \(x \neq 0\). In particular,
\(x=1/i\), for~\(i>0\) integer, leads to \(1 + 1/i < e^{1/i}\). Both
sides being positive, we deduce \(\prod_{i=1}^{n}(1+1/i) <
\prod_{i=1}^{n}{e^{1/i}} \Leftrightarrow n+1 < \exp(H_n)\). Finally,
\(\ln(n+1) < H_n\). An upper bound of~\(H_n\) can be similarly derived
by replacing~\(x\) by~\(-1/i\):
\begin{equation}
\ln(n+1) < H_n < 1 + \ln n.\label{ineq:Hn}
\end{equation}
We can now express the bounds on
\(\M{\fun{i2wb}}{n}\)\index{i2wb@$\M{\fun{i2wb}}{n}$} without~\(H_n\):
\begin{align*}
\ln(p+1) - 2\ln(2p) + 14
&< 8 \cdot \M{\fun{i2wb}}{2p} - 4{p^2} - 26{p}\\
& < \ln p - 2\ln(2p+1) + 17;\\
\ln(p+1) - 2\ln(2p+1) + 24
&< 8 \cdot \M{\fun{i2wb}}{2p+1} - 4{p^2} - 30{p}\\
&< \ln{p} - 2\ln(2p+2) + 27.
\end{align*}
Setting \(n=2p\) and \(n=2p+1\) leads to the respective bounds
\begin{align*}
-2\ln n + \ln(n+2) + 4 &< \varphi(n) < -2\ln(n+1) + \ln n + 7,\\
-2\ln n + \ln(n+1)     &< \varphi(n) < -2\ln(n+1) + \ln(n-1) + 3,
\end{align*}
where \(\varphi(n) := 8 \cdot \M{\fun{i2wb}}{n} - n^2 - 13n - 10 + \ln
2\). We retain the minimum of the lower bounds and the maximum of the
upper bounds of~\(\varphi(n)\) so,
\begin{equation*}
\ln(n+1) - 2\ln n < \varphi(n) < -2\ln(n+1) + \ln n + 7.
\end{equation*}
We can weaken the bounds a little bit with \(\ln n < \ln(n+1)\) and
simplify:
\begin{equation*}
0 < 8 \cdot \M{\fun{i2wb}}{n} - n^2 - 13n + \ln 2n - 10 < 7.
\end{equation*}
Therefore, for all \(n > 0\), there exists~\(\epsilon_n\) such that
\(0 < \epsilon_n < 7/8\) and
\begin{equation}
\M{\fun{i2wb}}{n}
  = \frac{1}{8}(n^2 + 13n - \ln 2n + 10) + \epsilon_n.
\label{eq:ave_i2wb}
\end{equation}
\index{insertion sort!2-way $\sim$!balanced $\sim$|)}
\index{tri!$\sim$ par insertions|)}
