\chapter{Introduction to \XSLT}

Recursion is a programming technique often neglected in undergraduate
curricula, rushed at the end of the semester, except when the
programming language is functional, that is, if it prominently
features immutable data and a control flow mostly defined by the
composition of mathematical functions. Examples of such languages are
\Scheme, \Haskell, \OCaml, \Erlang and \XSLT~\citep{Kay_2008}. Amongst
them, \XSLT is rarely taught in college, so professionals without
prior exposure to functional programming are likely to face the double
challenge of learning a new programming paradigm and using \XML: while
the former put forth recursion, the latter obscures it because of its
inherent verbosity. The syntactic difficulty is inevitable with \XSLT
because its grammar \emph{is} \XML, as well as it is the grammar of
its input and, usually, of its output.

This is why this chapter introduces the basics of \XSLT by relying on
our understanding of our abstract functional programming language or,
concretely, a small subset of \Erlang, a language chosen for its plain
and regular syntax, as well as its native pattern matching, yielding
very compact programs. Hopefully, the mental model of an \Erlang
programmer will ease the transition to thinking recursively in \XSLT,
having then to overcome only the obstacle of \XML. Accordingly, a tiny
subset of \XSLT is presented and previous examples in \Erlang are
systematically stripped of their use of pattern matching so they
become easily translatable into \XSLT, which lacks that feature. At
the same time, new exercises on unranked trees, that is, trees where
internal nodes can have a variable number of children, are worked out
directly in \XSLT, aiming at a transfer of skills from \Erlang.

The purpose of this chapter is not to present as many features of
\XSLT as possible, but to think recursively in \XSLT, which can be
conceived as a functional language specialised for the processing of
\XML documents.

\section{Documents}

We start our journey with a very brief presentation of the basics of
\XML, \HTML and \DTD, based on our lecture notes. The reader already a
bit familiar with these languages may skip it and use it as a
reference for the following chapters.

\mypar{\XML}

The acronym \XML stands for \emph{eXtensible Markup Language}. It is a
language for defining unranked trees with plain text, with a minimum
number of syntactic constructs. These trees are used to model
\emph{structured documents}. Database programmers would perhaps call
them \emph{semi\hyp{}structured data} because they are then conceived
in opposition to data that fit well into tables, the fundamental
structure of relational databases. These tables implement a
mathematical model of \emph{relations} satisfying \emph{schemas},
whilst \XML represents \emph{unranked trees} and \emph{formal
  grammars}. (To add to the confusion, \XML also can be adjoined
schemas of their own.)  Anyway, to understand what \XML is and how
this modelling works, it is probably easier to start with a small
example, like an email.  What are the different \emph{elements} and
what is the \emph{structure}, that is, how are the elements related to
each other? As far as the elements are concerned, an email contains at
least
\begin{itemize*}

  \item the sender's address,

  \item a subject or title,

  \item the recipient's address,

  \item a body of plain text.

\end{itemize*}
The elements correspond to the tree nodes and the structure is modelled
by the shape of the tree itself (its topology). For example:

\noindent\rule{\linewidth}{0.5pt}
\begin{alltt}
From: Me
Subject: Homework
To: You

  A \textbf{deadline} is a due date for a \emph{homework}.
\end{alltt}
\rule{\linewidth}{0.5pt}
\medskip
This email can be modelled by a tree in \fig~\vref{fig:mail}.
\begin{figure}
\centering
\includegraphics[scale=1,bb=71 620 389 723]{mail}
\caption{An email viewed as an \XML tree}
\label{fig:mail}
\end{figure}
Note that the (boxed) leaves, called \emph{text nodes}, contain text
whereas the inner nodes contain information \emph{about} their
subtrees, in particular the leaves. Since the information in the inner
nodes describes the information actually laid out, it is called
\emph{metadata} or \emph{mark\hyp{}up}, which explains part of the
acronym `\XML'. The corresponding \XML document is:
\begin{verbatim}
<email>
  <from>Me</from>
  <subject>Homeworks</subject>
  <to>You</to>
  <body>
  A <definition>deadline</definition> is a due date for a
<emphasis>homework</emphasis>.
  </body>
</email>
\end{verbatim}

\paragraph{Elements}

Each subtree is denoted by an opening and a closing \emph{tag}. An
opening tag is a name enclosed between \texttt{<} and \texttt{>}. A
closing tag is a name enclosed between \texttt{</} and \texttt{>}. A
pair of opening and closing tags constitute an \emph{element}; in
other words, a subtree corresponds to an element. In particular, the
element including all the others is called the \emph{root element}
(here, it is named \texttt{email}). The \emph{element name} is not
part of the text, it is metadata, so it suggests the meaning of the
data contained in the subtree. For example, the whole \XML document is
an element whose name is \texttt{email} because the document describes
a email. A preorder traversal of the \XML tree (see
page~\pageref{preorder} for binary trees) yields nodes in the same
order as their corresponding elements, when read in the \XML
document. (We shall comment on this.) The data (as opposed to the
metadata) is always contained in the leaves, and is always text. Note,
in particular, how the contents of the text nodes who are children of
elements \texttt{definition} and \texttt{emphasis} have been
respectively typeset with a bold typeface and in italics, but other
interpretations would have been possible. It is important to
understand that visual interpretations of mark\hyp{}up are \emph{not}
defined in \XML. This is why we wrote earlier that \XML is purely a
formal grammar, without semantics.

Actually, our example is not a correct \XML document because it lacks
a special element which says that the document is indeed \XML, and,
more precisely, what is the version of \XML used here, for instance:
\begin{verbatim}
<?xml version="1.0"?>
\end{verbatim}
This special element is actually not an element per se, as the special
markers \texttt{<?} and \texttt{?>} show. It is more a declaration,
carrying some information about the current file to destination of the
reader, whether it is a parsing software, usually called an \emph{\XML
  processor}, or a human. As such, it is a \emph{processing
  instruction} (more to come on this).

For now, consider the following element:
\begin{verbatim}
<axiom>
The empty set <varnothing/> contains no elements.
</axiom>
\end{verbatim}
which could be interpreted as\\
\noindent\rule{\linewidth}{0.5pt}
\textsf{
\textbf{Axiom}: The empty set \(\varnothing\) contains no elements.}\\
\noindent\rule{\linewidth}{0.5pt}

This \verb|<varnothing/>| is an \emph{empty element}, it features a
special tag terminator, \verb|/>|, which is absent in normal opening
and closing tags. It is useful for denoting things, as symbols, that
cannot be written with the Roman alphabet and need to be distinguished
from plain text. The associated tree is
\begin{center}
\includegraphics{axiom}
\end{center}
An empty element corresponds to a leaf in the \XML tree, despite it is
mark\hyp{}up and not data.

Nodes do not need to be unique amongst siblings. For instance, if we
want to send an email to several recipients, we would write:
\begin{alltt}
<email>
  <from>Me</from>
  <subject>Homeworks</subject>
  <to>You</to>
  \textbf{<to>Me</to>}
  <body>
  A <definition>deadline</definition> is a due date for a
<emphasis>homework</emphasis>.
  </body>
</email>
\end{alltt}
The \XML tree associated to this \XML document is
\begin{center}
\includegraphics[bb=72 622 412 723]{mailtoto}
\end{center}
Note that there are two nodes~\texttt{to} and that their order must be
the same as in the \XML document.

\paragraph{Attributes}

It is possible to annotate each mark\hyp{}up node with some labelled
strings, called \emph{attributes}. For example, we may want to specify
that our email is urgent, which is a property of the email as a whole,
not a part of the contents per se:
\begin{alltt}
<email \textbf{priority="urgent"}>
  <from>Me</from>
  <subject>Homeworks</subject>
  <to>You</to>
  <body>
  A <definition>deadline</definition> is a due date for a
<emphasis>homework</emphasis>.
  </body>
</email>
\end{alltt}
That \XML document may be represented by the following annotated tree:
\begin{center}
\includegraphics[bb=71 652 322 721]{urgent}
\end{center}
Note the symbol \textsf{@} preceding the attribute name, which
distinguishes it from element nodes. Amongst siblings, attribute nodes
are found \emph{before}\label{attr_before} element nodes. We may
attach several attributes to a given element:
\begin{alltt}
<email \textbf{priority="urgent"} \textbf{ack="yes"}>
  <from>Me</from>
  <subject>Homeworks</subject>
  <to>You</to>
  <body>
  A <definition>deadline</definition> is a due date for a
<emphasis>homework</emphasis>.
  </body>
</email>
\end{alltt}
\begin{center}
\includegraphics[bb=70 651 367 723]{attr}
\end{center}
The order of the attributes matters. Any element can have attributes,
including empty elements. Attributes are considered to be a special
kind of node, although they are not often represented in the \XML tree
for lack of room.

The \texttt{xml} processing instruction too can hold other predefined
attributes besides \verb|version|:
\begin{verbatim}
<?xml version="1.0" encoding="UTF-8"?>
<?xml version='1.1' encoding="US-ASCII"?>
<?xml version='1.0' encoding='iso-8859-1'?>
\end{verbatim}
The encoding is the \emph{character encoding} of the \XML document,
which is particularly useful when using \Unicode or some dedicated
Asian font, for instance. Note that the attribute names must be in
lowercase and the attribute \emph{values} must be enclosed in single
or double quotes. In the case of \verb|version| and \verb|encoding|,
only some standardised values are valid.

\paragraph{Escaping}

Most programming languages offer strings of characters to the
programmer to use. For instance, in \Clang, the strings are enclosed
between double quotes, like \verb|"abc"|. Thus, if the string contains
double quotes, we must take care of distinguishing, or
\emph{escaping}, them, so the compiler (or, more precisely, the
parser) can recognise the double quotes in the contents from the
enclosing double quotes. In~\Clang, escaping a character is achieved
by adding a backslash just before it. This way, the following is a
valid \Clang string: \verb|"He said: \"Hello!\"."|

In \XML, there is a similar problem. The attribute values can either
be enclosed by single or double quotes. If the latter, the double
quotes in the contents need escaping; if the former, the quotes need
escaping. Problems also stem from the characters used for the
mark\hyp{}up. For example, the following element
\begin{verbatim}
<problem>For all integers n, we have n < n + 1.</problem>
\end{verbatim}
is not valid because the text between the tags contains the character
`\texttt{<}', which is confused by the \XML parsers with the
(expected) start of a tag:
\begin{alltt}
<problem>For all integers n, we have n \fbox{<\,} n + 1.</problem>
\end{alltt}
The way in which \XML escapes this character is by using the special
sequence of characters \verb|&lt;| so the previous element, once
corrected, is
\begin{alltt}
<valid>For all integers n, we have n \textbf{&lt;} n + 1.</valid>
\end{alltt}

\paragraph{Predefined named entities}

The sequence \verb|&lt;| is called a \emph{predefined named entity}.
Such entities always
\begin{enumerate*}

  \item start with an ampersand (\verb|&|),

  \item continue with a predefined name (here, \texttt{lt}),

  \item end with a semicolon (\texttt{;}).

\end{enumerate*}
Of course, the use of the ampersand to mark the start of an entity
entails that this very character must itself be escaped if used to
only denote itself. In that case, we should use \verb|&amp;|
instead. There are some other characters which can \emph{sometimes}
cause a problem to \XML parsers (as opposed to always create a
problem, as \verb|<| and \verb|&| do).  A summary of all the
predefined named entities is given in the following table.
\begin{center}
\begin{tabular}{cll}
\toprule
\textsf{Character} & \textsf{Entity} & \textsf{Mandatory}\\
\midrule
\verb|&| & \verb|&amp;|  & always\\
\verb|<| & \verb|&lt;|   & always\\
\verb|>| & \verb|&gt;|   & in attribute values\\
\verb|"| & \verb|&quot;| & between double quotes\\
\verb|'| & \verb|&apos;| & between single quotes\\
\bottomrule
\end{tabular}
\end{center}

\noindent As an illustration, consider the following document:
\XMLinUnchecked{entities.xml}

\noindent The two last entities are \emph{predefined numbered
  entities} because they denote characters by using their \Unicode
point (\url{http://www.unicode.org/}). If the code is given in
decimal, it is introduced by \verb|&#|, for instance, \verb|&#100|. If
the code is given in hexadecimal, it is introduced by \verb|&#x|, for
example, \verb|&#x00E7|.

\paragraph{Internal entities}

It can be annoying to use numbers to refer to characters, especially
if one considers that \Unicode requires up to six digits. To make life
easier, it is possible to bind a name to an entity representing a
character, and get a \emph{user\hyp{}defined internal entity}. They
are called internal because their definition must be in the same
document where they are used. For example, it is easier to use
\verb|&n;| rather than \verb|&#241|, especially if the text is in
Spanish (this represents the letter \texttt{\~{n}}). This kind of
entity must be declared in the \emph{document type declaration}, which
is located, if any, just after the declaration \verb|<?xml ... ?>| and
before the root element.
\label{xml_intro:DOCTYPE} A document type declaration is made of the
following components:
\begin{enumerate*}

  \item the opening \verb|<!DOCTYPE|,

  \item the root element name,

  \item the opening character \verb|[|,

  \item the \emph{named character entity} declarations,

  \item the closing \verb|]>|.

\end{enumerate*}
A named character entity declaration is made of
\begin{enumerate*}

  \item the opening \verb|<!ENTITY|,

  \item the entity name,

  \item the numbered character entity between double quotes,

  \item the closing \verb|>|

\end{enumerate*}
For example: \verb|<!ENTITY n "&#241;">|
Here is a complete example:
\XMLinUnchecked{espana.xml}
\noindent One can think such an entity as being a macro in
\textsf{cpp}, the \Clang preprocessor language. It is indeed possible
to extend user\hyp{}defined internal entities to denote any character
string, not just a single character. Typically, if one wishes to
repeat a long or difficult piece of text, like a foreign company name
or the genealogy of the kings of Merina, it is best to name this text
and put an entity with that name wherever one wants its contents. The
syntax for the declaration is fundamentally the same. For example,
\begin{verbatim}
<!ENTITY univ "Konkuk University">
<!ENTITY motto "<spain>Viva Espa&n;a!</spain>">
<!ENTITY n "&#241;">
\end{verbatim}

\paragraph{External entities}

Sometimes the \XML document needs to include other \XML documents, but
copying and pasting of those is not a good strategy, since this
precludes us from automatically following their editing. Fortunately,
\XML allows us to specify the inclusion of other \XML documents by
means of \emph{external entities}. The declaration of these entities
is as follows:
\begin{enumerate*}

  \item the opening \verb|<!ENTITY|,

  \item the entity name,

  \item the keyword \verb|SYSTEM|,

  \item the full name of the \XML file between double quotes,

  \item the closing \verb|>|

\end{enumerate*}
For example,
\begin{verbatim}
<?xml version="1.0"?>
<!DOCTYPE longdoc [
  <!ENTITY part1 SYSTEM "p1.xml">
  <!ENTITY part2 SYSTEM "p2.xml">
  <!ENTITY part3 SYSTEM "p3.xml">
]>
<longdoc>
  The included files are:
  &part1;
  &part2;
  &part3;
</longdoc>
\end{verbatim}
At parsing time, the external entities are fetched by the
underlying operating system and copied into the main \XML document,
replacing their associated entity. Therefore the included parts cannot
contain any prolog, that is, no \XML declaration \verb|<?xml ... ?>|
and no document type declaration \texttt{<!DOCTYPE~...~]>}. When
reading an external entity, \XML processors are required to copy
verbatim the contents of the referenced external document and then to
parse it as if it always belonged to the main document.

\paragraph{Unparsed entities}

\emph{Unparsed entities} allow us to refer to binary objects, like
images, videos, sounds, or to some text which is not \XML, like a
program or a play by Shakespeare. They are declared by
\begin{enumerate*}

  \item the opening \verb|<!ENTITY|,

  \item the entity name,

  \item the keyword \verb|SYSTEM|,

  \item the full name of the non\hyp{}\XML external file between
    double quotes,

  \item the keyword \verb|NDATA|,

  \item a \emph{notation} (the kind of the file),

  \item the closing \verb|>|

\end{enumerate*}
The following is an example.
\XMLinUnchecked{notation.xml}
\noindent Had we used external entities, the object would have been
copied in place of the reference and parsed as \XML{} --~which it is
not. Notice the notation \texttt{gif}, which is the kind of the
unparsed entity. Notations must be defined in the document type
declarations as follows:
\begin{enumerate*}

  \item the opening \verb|<!NOTATION|,

  \item the notation name,

  \item the keyword \verb|SYSTEM|,

  \item a description of the kind of unparsed entity the notation
    refers to (it can be a \MIME type, an URL, plain English...)

  \item the closing \verb|>|

\end{enumerate*}
Notice also that unparsed entities must be used either
\begin{itemize*}

  \item as attribute values (in our example, the attribute name is
  \texttt{image}),

  \item or as names (\texttt{picture}), instead of the entity syntax
    (\texttt{\&picture;}).

\end{itemize*}
For example, the next document is \emph{not} well\hyp{}formed:
\begin{verbatim}
<?xml version="1.0"?>
<!DOCTYPE doc [
  <!NOTATION jpeg SYSTEM "image/jpeg">
  <!ENTITY pic "pictures/me.jpeg" NDATA jpeg>
]>
<doc>
  &pic;
</doc>
\end{verbatim}
\begin{figure}[b]
\centering
\includegraphics[bb=71 627 278 723]{entities}
\caption{Summary of the different kinds of entities}
\end{figure}

\paragraph{Unparsed character data}

It is sometimes tiresome to have to escape characters, that is, to use
character entities. To avoid the need of escaping, there is a special
construct: \emph{CDATA sections} (short for `Character DATA'), made of
\begin{enumerate*}

  \item the opening \verb|<!CDATA[|,

  \item some text without escaping and without the sequence
    \verb|]]>|,

  \item the closing \verb|]]>|.

\end{enumerate*}
For example
\begin{alltt}
<paragraph>An example of conditional in C:
  <c>\textbf{<!CDATA[}if (x < y) return &r;\textbf{]]>}</c>
</paragraph>
\end{alltt}

\paragraph{Internal linking}

Consider a document representing a technical book, like a textbook. It
is common to find cross\hyp{}references in such kind of books, that
is, references in some chapters to other chapters or sections, or
bibliographical entries. One easy way to achieve this is to use some
attributes as labels, that is, names unambiguously identifying a
location in the structure, and some attributes as references (to the
labels). The problem is that the writer is then in charge of checking
whether
\begin{itemize*}

  \item a given label is unique throughout the whole document,
        including external entities,

  \item every reference corresponds to an existing label (linking).

\end{itemize*}
\label{xml_intro:ATTLIST}
\XML provides a way to ensure that any validating parser will check
this kind of internal linking automatically: using the predefined
attributes \texttt{ID} and \texttt{IDREF}. The former is the kind of
all the (attribute) labels and the latter is the kind of all the
(attribute) references. The attributes used either as labels or
references must be declared in the \texttt{DOCTYPE} section using
\texttt{ATTLIST} (`attribute list').

To declare labels, we must write
\begin{enumerate*}

  \item the opening \verb|<!ATTLIST|,

  \item the name of the element being labelled,

  \item the names of the label attributes separated by spaces,

  \item the keyword \texttt{ID},

  \item the keyword \texttt{\#REQUIRED} if the element must always be
    labelled, otherwise \texttt{\#IMPLIED},

  \item the closing \verb|>|

\end{enumerate*}
For the references, we must write
\begin{enumerate*}

  \item the opening \verb|<!ATTLIST|,

  \item the name of the referring element,

  \item the names of the reference attributes separated by spaces,

  \item the keyword \texttt{IDREF},

  \item the keyword \texttt{\#REQUIRED} if the element must always
    carry a reference, otherwise \texttt{\#IMPLIED},

  \item the closing \verb|>|

\end{enumerate*}
For example,
\label{xml_intro:id_idref}
\XMLinUnchecked{id_idref.xml}

\paragraph{Comments}

It is possible to include comments in an \XML document. They are made
of
\begin{enumerate*}

  \item the opening \verb|<!--|,

  \item some text without the sequence \verb|--|,

  \item the closing \verb|-->|.

\end{enumerate*}
For example
\begin{alltt}
<p>Our store is located at</p>
\textbf{<!-- <address>Eunpyeong-gu, Seoul</address> -->}
<address>Gangnam-gu, Seoul</address>
\end{alltt}
Contrary to programming languages, comments are \emph{not} ignored
by the parsers and are nodes of the \XML tree.

\paragraph{Namespaces}

Each \XML document defines its own element names, which we
collectively call its \emph{vocabulary}. In case we use external
entities which refer to other \XML documents using, by coincidence,
the same names, we end with an ambiguity in the main document. A good
way to avoid these name clashes is to use \emph{namespaces}. A
namespace is a user\hyp{}defined annotation of each element names and
attribute names. Therefore, if two \XML documents use two different
namespaces, that is to say, two different element name annotations,
there is no way to mix their elements when importing one document into
the other, because each element name carries an extra special
annotation which is different (ideally unique in the set of documents
of interest).

The definition of a namespace can be done at the level of any element
by using a special attribute with the following syntax:
\begin{alltt}
xmlns:\textit{prefix} = "\textit{URL}"
\end{alltt}
where \texttt{\emph{prefix}} is the space name and \texttt{\emph{URL}}
(\emph{Universal Resource Location}) points to a web page describing
in natural language (for example, in English) the namespace. Consider
the namespace \texttt{course} in the following:
\begin{alltt}
<?xml version="1.0"?>
<\textbf{course:short}
  \textbf{xmlns:course}="http://konkuk.ac.kr/~rinderkn/Mirror/XML">
 <\textbf{course:}date>26 August 2006</\textbf{course:}date>
 <\textbf{course:}title>Some XML-centric languages</\textbf{course:}title>
 <\textbf{course:}topic \textbf{course:}level="advanced">
   We study XML, XPath and XSLT.</\textbf{course:}topic>
</\textbf{course:}short>
\end{alltt}
The scope of a namespace, that is, the part of the document where it
is usable, is the subtree whose root is the element declaring the
namespace. By default, if the prefix is missing, the element and all
its sub\hyp{}elements without prefix belong to the namespace. So, the
previous example could be simply rewritten
\begin{alltt}
<?xml version="1.0"?>
<short \textbf{xmlns}="http://konkuk.ac.kr/~rinderkn/Mirror/XML">
 <date>26 August 2006</date>
 <title>Some XML-centric languages</title>
 <topic level="advanced">We study XML, XPath and XSLT.</topic>
</short>
\end{alltt}
Note that the colon is missing in the namespace attribute:
`\texttt{xmlns=...}'. This example illustrates the important fact that
what ultimately defines a namespace is a URL, not a prefix (like
\texttt{course}).

As an example of a name clash and how to avoid it, let us consider a
file \texttt{fruits.xml} containing the following \HTML fragment:
\begin{verbatim}
<table>
  <tr>
    <td>Bananas</td>
    <td>Oranges</td>
  </tr>
</table>
\end{verbatim}
\HTML will be sketched in a coming subsection, but, for now, suffice
it to say that the elements have an implicit meaning if the file is
indeed interpreted as \HTML. For example, \texttt{table} refers to a
typographical layout.

Imagine now a file \texttt{furniture.xml} containing a description of
pieces of furniture, like
\begin{verbatim}
<table>
  <name>Round table</name>
  <wood>Oak</wood>
</table>
\end{verbatim}
Now, the main document \texttt{main.xml} includes both files:
\begin{verbatim}
<?xml version="1.0"?>
<!DOCTYPE eclectic [
  <!ENTITY part1 SYSTEM "fruits.xml">
  <!ENTITY part2 SYSTEM "furniture.xml">
]>
<eclectic>
  &part1;
  &part2;
</eclectic>
\end{verbatim}
The problem is that \texttt{table} has a different meaning in the two
included files, so they should not be confused: this is a clash
name. The solution consists in using two different namespaces. First:
\begin{verbatim}
<html:table xmlns:html="http://www.w3.org/TR/html5/">
  <html:tr>
    <html:td>Bananas</html:td>
    <html:td>Oranges</html:td>
  </html:tr>
</html:table>
\end{verbatim}
Second:
\begin{verbatim}
<f:table xmlns:f="http://www.e-shop.com/furnitures/">
  <f:name>Round table</f:name>
  <f:wood>Oak</f:wood>
</f:table>
\end{verbatim}
But this is a heavy solution. Fortunately, namespaces can be
defaulted:
\begin{verbatim}
<table xmlns="http://www.w3.org/TR/html5/">
  <tr>
    <td>Bananas</td>
    <td>Oranges</td>
  </tr>
</table>
\end{verbatim}
Second:
\begin{verbatim}
<table xmlns="http://www.e-shop.com/furnitures/">
  <name>Round table</name>
  <wood>Oak</wood>
</table>
\end{verbatim}
The two kinds of tables can be safely mixed now. For example
\begin{verbatim}
<mix xmlns:html="http://www.w3.org/TR/html5/"
     xmlns:f="http://www.e-shop.com/furnitures/">
<html:table>
  ...
  <f:table>
  ...
  </f:table>
...
<html:table>
</mix>
\end{verbatim}
Note that element \texttt{mix} has no associated namespace (it is
neither \texttt{html} nor~\texttt{f}). It is possible to unbind or
rebind a prefix namespace (the following examples are found at
\url{http://www.w3.org/TR/REC-xml-names/}):
\begin{alltt}
<?xml version="1.1"?>
<x xmlns:n1="http://www.w3.org">
  <n1:a/> <!-- valid; the prefix n1 is bound to
               http://www.w3.org -->
    <x \textbf{xmlns:n1=""}>
      <n1:a/> <!-- invalid; the prefix n1 is not bound here -->
      <x xmlns:n1="http://www.w3.org">
        <n1:a/> <!-- valid; the prefix n1 is bound again -->
      </x>
   </x>
</x>
\end{alltt}

\XMLinUnchecked{beers.xml}

\XMLinUnchecked{scoping.xml}

\noindent An element may have children elements with the same name
(like the element~\texttt{to} in the email example above), but an
element may not have attributes with identical names or identical
namespaces (URL) and identical names. For example, each of the
\texttt{bad} empty elements is invalid in the following:
\begin{verbatim}
<!-- http://www.w3.org is bound to n1 and n2 -->
<x xmlns:n1="http://www.w3.org" xmlns:n2="http://www.w3.org" >
  <bad a="1"    a="2"/>     <!-- invalid -->
  <bad n1:a="1" n2:a="2"/>  <!-- invalid -->
</x>
\end{verbatim}
However, each of the following is valid, the second because
\emph{default namespaces never apply to attribute names}:
\begin{verbatim}
<!-- http://www.w3.org is bound to n1 and is the default -->
<x xmlns:n1="http://www.w3.org" xmlns="http://www.w3.org" >
  <good a="1" b="2"/>    <!-- valid -->
  <good a="1" n1:a="2"/> <!-- valid -->
</x>
\end{verbatim}
Namespaces will be very important when learning \XSLT. Although
namespaces are declared as attributes, they are present in the \XML
tree corresponding to the document as a special node, different from
the attribute nodes.

\paragraph{Processing instructions}

In some exceptional cases, it may be useful to include in an \XML
document some data that are targeted to a specific \XML
processor. These data are then embedded in a special element, and the
data themselves are called a \emph{processing instruction} because
they tell a particular processor, for instance, \Saxon, what to do at
this point. The syntax is
\begin{alltt}
<?\textit{target} \textit{data}?>
\end{alltt}
The \emph{target} is a string supposed to be recognised by a specific
\XML processor and the \emph{data} is then used by this
processor. Note that the data takes the shape of attribute values, and
may be absent. For example:
\begin{verbatim}
<?xml version="1.0"?>
\end{verbatim}

\paragraph{Validation}

All \XML processors must check whether the input document satisfy the
\emph{syntactical} requirements of a well\hyp{}formed \XML document.
In particular,
\begin{itemize*}

\item pairs of tags making up an element must be present, except for
  empty elements which are closed with \texttt{/>} (this has to be
  contrasted with \HTML, which is very lax in this regard),

  \item the predefined entities must really be predefined (unicodes
    are automatically checked),

  \item internal entities must be declared in the prolog, etc.

\end{itemize*}
\emph{Validating processors} also check that the external entities are
indeed found (their well\hyp{}formedness is checked after they have
been inserted in the main document). There are several \XML parsers
available free of charge over the internet, implemented in several
languages. Most of them are actually libraries (API), so an
application processing \XML would only need to be interfaced with one
of them. A good standalone, validating parser is \texttt{xmllint}.

\mypar{\HTML}

The \emph{Hyper-Text Markup Language} (\emph{\textsf{HTML}}) is used
to describe web pages. See the W3C Draft Recommendation at
\url{http://www.w3.org/TR/html5/}. Let us skim over this vast language
and illustrate it with some small examples. For instance, all \HTML
file containing English should at least follow the pattern
\begin{alltt}
<!DOCTYPE html>
<html lang="en-GB">
  <head>
    <title>\textit{the title of the window}</title>
  </head>
  <body>
     \emph{...contents and markup...}
  </body>
</html>
\end{alltt}
Elements \texttt{h1}, \texttt{h2}, ..., \texttt{h6} enable six kinds
of headers, of decreasing font sizes. Consider opening in a web
browser the following document: \XMLinUnchecked{h.html}
\noindent Other useful elements are the following:
\begin{itemize*}

  \item The empty element \texttt{<br/>} is interpreted by
    user agents as a \emph{line break};

  \item element \texttt{em} marks text to be \emph{emphasised}
    (for example, by using an italic font);

  \item element \texttt{strong} marks text to be emphasised stronger
    than with \texttt{em} (for instance, by using a bold font);

  \item element \texttt{p} delimits a \emph{paragraph}.

\end{itemize*}
\emph{Lists} are a common way to typeset closely related sentences,
like enumerations. There are three kinds of lists:
\begin{enumerate*}

  \item unordered lists;

  \item ordered lists;

  \item lists of definitions.

\end{enumerate*}
Unordered lists are the well\hyp{}known `bullet lists', where each
line is displayed after an indentation followed by a bullet, like the
following.
\begin{itemize*}

  \item element \texttt{ul} contains an unordered list;

  \item element \texttt{li} (`list item') contains an item in the
    list.

\end{itemize*}
Try the following recipe:
\begin{verbatim}
<h3>Ingredients</h3>
<ul>
  <li>100 g. flour,</li>
  <li>10 g. sugar,</li>
  <li>1 cup of water,</li>
  <li>2 eggs,</li>
  <li>salt and pepper.</li>
</ul>
\end{verbatim}
Ordered lists are lists whose items are introduced by an indentation
followed by a number, in increasing order. They require
\begin{enumerate*}

  \item the element \texttt{ol}, which contains the ordered list,

  \item elements \texttt{li}, as in unordered lists.

\end{enumerate*}
For example,
\begin{verbatim}
<h3>Procedure</h3>
<ol>
  <li>Mix dry ingredients thoroughly;</li>
  <li>Pour in wet ingredients;</li>
  <li>Mix for 10 minutes;</li>
  <li>Bake for one hour at 300 degrees.</li>
</ol>
\end{verbatim}
A list of definitions is a list whose items are introduced by a few
words in a bold font followed by the contents of the item
itself. Consider
\noindent\rule{\linewidth}{0.5pt}
\begin{description*}

  \item[hacker]\ \\
    A clever programmer.

  \item[nerd]

  \item[geek]\ \\
    A technically bright but socially misfit person.

\end{description*}
\noindent\rule{\linewidth}{0.5pt}
The elements involved are
\begin{itemize*}

  \item \texttt{dl} (`definition list'), which contains the whole
    list of definitions;

  \item \texttt{dt} (`definition term'), which contains every term
    to be defined;

  \item \texttt{dd} (`definition description'), which contains every
    definition of a term.

\end{itemize*}
The previous example corresponds to the following \HTML excerpt:
\begin{verbatim}
<dl>
  <dt><strong>hacker</strong></dt>
    <dd>A clever programmer.</dd>
  <dt><strong>nerd</strong></dt>
  <dt><strong>geek</strong></dt>
    <dd>A technically bright but socially misfit person.</dd>
</dl>
\end{verbatim}
A \emph{table} is a rectangle divided into smaller rectangles, called
\emph{cells}, which contain some text. When read vertically, cells are
said to belong to \emph{columns}, whilst horizontally, they belong to
\emph{rows}. A row or a column can have a \emph{header}, that is, a
cell at their beginning containing a name in bold face. A table can
have a \emph{caption}, which is a short text describing the contents
of the table and displayed just above it, like a title. Columns can be
divided themselves into subcolumns, when needed. The following example
is taken from \url{http://www.w3.org/TR/html4/struct/tables.html}.
\begin{center}
\emph{A test table with merged cells}\\

\begin{tabular}{|c|c|c|c|}
\hline
                 & \multicolumn{2}{|c|}{\textbf{Average}} & \textbf{Red}\\
\cline{2-3}
                 & \textbf{height}    &   \textbf{weight} & \textbf{eyes}\\
\hline
\textbf{Males}   & 1.9                & 0.003             & 40\%\\
\hline
\textbf{Females} & 1.7                & 0.002             & 43\%\\
\hline
\end{tabular}
\end{center}
\textbf{Males} and \textbf{Females} are row headers. The column
headers are \textbf{Average}, \textbf{Red eyes}, \textbf{height} and
\textbf{weight}. The column \textbf{Average} spans two columns; in
other words, it contains two subcolumns, \textbf{height} and
\textbf{weight}. The caption reads \emph{A test table with merged
  cells}. The corresponding \HTML code is
\begin{verbatim}
<table border="1">
  <caption><em>A test table with merged cells</em></caption>
  <tr>
    <th rowspan="2"/>
    <th colspan="2">Average</th>
    <th rowspan="2">Red<br/>eyes</th>
  </tr>
  <tr><th>height</th><th>weight</th></tr>
  <tr><th>Males</th><td>1.9</td><td>0.003</td><td>40%</td></tr>
  <tr>
    <th>Females</th>
    <td>1.7</td>
    <td>0.002</td>
    <td>43%</td>
  </tr>
</table>
\end{verbatim}
The meaning of the elements is as follows:
\begin{itemize*}

  \item Element \texttt{table} contains the table; its attribute
  \texttt{border} specifies the width of the table borders, that is,
  of the lines separating the cells from the rest.

  \item Element \texttt{caption} contains the caption.

  \item Element \texttt{th} (`table header') contains a row or column
    header, that is, the title of the row of column in bold type.

  \item Element \texttt{td} (`table data') contains the data of a
    cell (if not a header).

  \item Element \texttt{tr} (`table row') contains a row, that is, a
    series of \texttt{td} elements, perhaps starting with a
    \texttt{th} element.

\end{itemize*}
Notice the attributes \texttt{rowspan} and \texttt{colspan} of the
\texttt{th} element. Attribute \texttt{rowspan} allows us to specify
how many rows the current cell spans. For example, the first row, that
is, the one on the top\hyp{}left corner, is empty and covers two rows
because \verb|<th rowspan="2"/>|. Attribute \texttt{colspan} allows us
to declare how many columns the current cell spans. For example, the
second cell contains the text \textbf{Average} and covers two columns
because \verb|<th colspan="2">Average</th>|. Notice the line break
\verb|<br/>| in the third cell (first row, last column) and how
\textbf{height} and \textbf{weight} are automatically at the right
place

\emph{Hyperlinks} in \HTML are defined by the element `\texttt{a}'
with its mandatory attribute \texttt{href}
(`hyper\hyp{}reference'). For example, consider the following
hyperlink:
\begin{verbatim}
<a href="http://konkuk.ac.kr/~rinderkn/">See my web page.</a>
\end{verbatim}

\mypar{\XHTML}

The current working draft of \HTML is \HTML~5. Until it becomes a
standard supported by web browsers and \XSLT processors, it is best
for beginners to use a simpler version of \HTML, called \XHTML
(\emph{eXtensible Hyper-Text Markup Language}), whose W3C
Recommandation is found at
\url{http://www.w3.org/TR/xhtml1/}. Basically, \XHTML is \XML, but the
elements which are found in \HTML have the same meaning (instead of
none). For example, the previous examples are valid \XHTML, with the
provision that a specific \texttt{DOCTYPE} is declared. The pattern is
the following:
\begin{alltt}
<?xml version="1.0" encoding="\emph{encoding}"?>
<!DOCTYPE html
    PUBLIC "-//W3C//DTD XHTML 1.0 Strict//EN"
    "http://www.w3.org/TR/xhtml1/DTD/xhtml1-strict.dtd">
<html xmlns="http://www.w3.org/1999/xhtml"
      xml:lang="en" lang="en">
  <head>
    <title>\textit{the title of the window}</title>
  </head>
  <body>
     \emph{...contents and markup...}
  </body>
</html>
\end{alltt}
Just like \XML documents, \XHTML documents can be and should be
validated before being published on the web, for example using the
site at \url{http://validator.w3.org/}.

\mypar{\DTD}

We saw \vpageref{xml_intro:DOCTYPE} that the \emph{Document Type
  Declaration} may contain markup which constrains the \XML document
it belongs to (elements, attributes etc.). The content of a Document
Type Declaration is made of a \emph{Document Type Definition},
abbreviated \DTD. So the former is the container of the latter. It is
possible to have all or part of the \DTD in a separate file, usually
with the extension `\texttt{.dtd}'. We already saw \emph{attribute
  lists} \vpageref{xml_intro:ATTLIST} when setting up internal
linking.  In general, the special element \texttt{ATTLIST} can be used
to specify any kind of attributes, not just labels and
references. Consider the attribute declarations for element
\texttt{memo}:
\begin{verbatim}
<!ATTLIST memo ident      CDATA          #REQUIRED
               security   (high | low)   "high"
               keyword    NMTOKEN        #IMPLIED>
\end{verbatim}
\texttt{CDATA} stands for \emph{character data} and represents any
string. A \emph{named token} (\texttt{NMTOKEN}) is a string starting
with a letter and which may contain letters, numbers and certain
punctuation. In order for a document to be validated, which requires
more constraints than to be merely well\hyp{}formed, all the elements
used must be declared in the \DTD. The name of each element must be
associated a \emph{content model}, that is, a description of what it
is allowed to contain, in terms of textual data and sub\hyp{}elements
(mark\hyp{}up). This is achieved by means of the \DTD \texttt{ELEMENT}
declarations. There are five content models:
\begin{enumerate*}

  \item the \emph{empty element}:
\begin{verbatim}
<!ELEMENT padding EMPTY>
\end{verbatim}

  \item elements with \emph{no content restriction}:
\begin{verbatim}
<!ELEMENT open ALL>
\end{verbatim}

  \item elements containing \emph{only text}:
\begin{verbatim}
<!ELEMENT emphasis (#PCDATA)>
\end{verbatim}
which means \emph{parsed\hyp{}character data}.

\bigskip

  \item elements containing \emph{only elements}:
\begin{verbatim}
<!ELEMENT section (title,para+)>
<!ELEMENT chapter (title,section+)>
<!ELEMENT report (title,subtitle?,(section+ | chapter+))>
\end{verbatim}
where \texttt{title}, \texttt{subtitle} and~\texttt{para} are
elements.

\bigskip

  \item elements containing \emph{both text and elements}:
\begin{verbatim}
<!ELEMENT para (#PCDATA | emphasis | ref)+>
\end{verbatim}
where \texttt{emphasis} and~\texttt{ref} are elements.
\end{enumerate*}
The definition of a content model is akin to \emph{regular
  expressions}. Such an expression can be made up by combining the
following expressions:
\begin{itemize*}

  \item \texttt{(\(e_1\), \(e_2\), \dots, \(e_n\))} represents the
    elements represented by~\(e_1\), followed by the elements
    represented by~\(e_2\) etc. until~\(e_n\);

  \item \texttt{\(e_1\) | \(e_2\)} represents the elements represented
    by~\(e_1\) or~\(e_2\);

  \item \texttt{(\(e\))} represents the elements represented by~\(e\);

  \item \texttt{\(e\)?} represents the elements represented by~\(e\)
    or none;

  \item \texttt{\(e\)+} represents a non\hyp{}empty repetition of
    the elements represented by~\(e\);

  \item \texttt{\(e\)*} represents the repetition of the elements
    represented by~\(e\).

\end{itemize*}
\emph{Warning:} When mixing text and elements, the only possible
regular expression is either \verb|(#PCDATA)| or
\verb/(#PCDATA | ...)*/

The part of a \DTD which is included in the same file as the \XML
document it applies to is called the \emph{internal subset}. See
again the example \vpageref{xml_intro:id_idref}. The part of a
\DTD which is in an independent file (\texttt{.dtd}) is called the
\emph{external subset}. If there is no internal subset and
everything is in the external subset we have a declaration like
\begin{verbatim}
<!DOCTYPE some_root_element SYSTEM "some.dtd">
\end{verbatim}
In order to validate an \XML document, its \DTD must completely
describe the elements and attributes used.  This is not mandatory when
well\hyp{}formedness is required. Therefore, the example
\vpageref{xml_intro:id_idref} is well\hyp{}formed but not valid in
the sense above, because the elements \texttt{map} and
\texttt{country} are not declared. To validate the document, it is
enough, for example, to add
\begin{verbatim}
<!ELEMENT map (country*)>
<!ELEMENT country EMPTY>
\end{verbatim}

\section{Introduction}

\noindent Given one or several \XML documents, it may be useful
\begin{itemize}

  \item to search the documents and output what has been found in a
  format suitable for another application or reader, in particular,
  \XML (filtering in);

  \item to copy the input, perhaps without certain parts (filtering
  out), and/or adding data (updating).

\end{itemize}
When such needs arise, it is a good idea to use the functional
programming languages \XQuery or \XSLT (\emph{eXtensible Stylesheet
  Language Transformations}). Even if both languages share a great
deal of common use cases (to the point of having in common a
sublanguage, \XPath), the first application, which is more oriented
towards database management, is more commonly tackled with \XQuery,
whilst the second usage is often undertaken with \XSLT.

An \XSLT processor reads an \XML document and an \XSLT program, then
applies the \emph{transformations} defined in \XSLT to the document,
and the result is printed, usually in plain text, \XML or \HTML. The
twist is that an \XSLT file is actually an \XML document, which
enables \XSLT to transform \XSLT programs. For example, if we use an
element \verb|book| in an \XML document, \XML itself does not
necessarily imply that this element models a book, although some
application using the document perhaps will precisely do that. One can
think of \XML as merely syntactic rules, in other words, a formal
grammar, with no semantics attached to the constructs. An \XSLT
document is thus \XML with a performative declaration.

To enable the interpretation of \XML as \XSLT, \XSLT programs require
the predefined namespace \url{http://www.w3.org/1999/XSL/Transform},
which is often (but not necessarily) named \texttt{xsl}, as shown in
\XSLTinUnchecked{empty.xsl} The first line states that this is an \XML
document. The second defines the interpretation as \XSLT by declaring
the namespace for \XSLT and using the \emph{root element}
\texttt{xsl:transform} (\texttt{xsl:stylesheet} is also accepted). The
version of \XSLT is set to \texttt{2.0}, which is the version current
at the time of writing. Moreover, the element \texttt{xsl:output}
states that the output is to be plain text. Other than that, the
program says nothing else, so we might expect the transformation to do
nothing. Assume then the following \XML document,
\texttt{cookbook.xml}, to be transformed:
\XMLinUnchecked{cookbook.xml}
\noindent The application of the empty transformation to this document
yields \XMLinUnchecked{cookbook1_out.xml}

Perhaps surprisingly, something did happen: the contents of the
\emph{text nodes} of the input \XML document were extracted \emph{in
  the same order}, but not the attribute values. (Note that if
\verb|<xsl:output method="text"/>| were missing, the output would be
considered \XML and \verb|<?xml ... ?>| would be outputted by
default.) More precisely, the order corresponds to a preorder
traversal of the corresponding \XML tree: this is the implicit
traversal of \XSLT processors, also called \emph{document order}. The
rationale is that, since the aim is often to rewrite a document into
another, this traversal corresponds to the order in which a book is
read and written, cover to cover. Furthermore, the reason why the text
nodes are extracted by default is due to \XSLT favouring a filtering
style: if part of the input should be ignored or augmented, the
programmer must say so. Finally, let us remark that there is no need
for explicit printing instructions in \XSLT: the programmer assumes
that the result is \XML or text and the run\hyp{}time system ensures
automatic \emph{serialisation} of the result.

\paragraph{Matching}
\label{par:matching}

Let us complete our empty transformation as follows:
\XSLTinUnchecked{chapter.xsl}
\noindent Note the predefined element \texttt{xsl:template} defining a
\emph{template}. It bears the attribute \texttt{match}, whose value is
the name of the element we want to transform. During the preorder
traversal, if a \texttt{chapter} element is found (that is, matched),
the contents of the text node of the template (\texttt{A chapter})
becomes the result. (Let us not confuse the text node and its
contents.) Applying the previous transformation to the document in the
file named \texttt{cookbook.xml} yields
\XMLinUnchecked{cookbook2_out.xml}

\noindent Let us match now the root element with the following:
\XSLTinUnchecked{chapters.xsl}
\noindent The result is now:
\XMLinUnchecked{cookbook3_out.xml}
\noindent The reason is that when a template matches a node, called
the \emph{context node} from within the template, that node is
processed (here, the text \texttt{Chapters:} or \texttt{A chapter} is
produced) and the preorder traversal resumes \emph{without visiting the
  children of the context node}. Therefore, after the element
\texttt{cookbook} is matched and done with, the \XSLT processor
ignores every\-thing else since it is the root element.

In order to visit and try to match the children of the context node,
one must tell so the processor by using the special empty element
\begin{verbatim}
<xsl:apply-templates/>
\end{verbatim}
Let us add this element as a child of the template matching the root
element: \XSLTinUnchecked{cookbook.xsl}
\noindent The result is now:
\XMLinUnchecked{cookbook4_out.xml}
\noindent It is striking that the text `\texttt{Chapters:}' is not
aligned with the title. It is a sad fact that the handling of
whitespace and newlines can be very confusing in \XML and \XSLT, in
particular when we want or not some space if the output, like here, is
plain text. We will not discuss this thorny issue here and refer the
reader to the book by~\cite{Kay_2008}, page~141. As for the rest, the
title appears after `\texttt{Chapters:}', which is confusing.

If we simply want to get rid of the title, we could simply define an
empty template matching \texttt{title}:
\XSLTinUnchecked{cookbook1bis.xsl}
\noindent The result is now:
\XMLinUnchecked{cookbook5_out.xml}

If we want to retain the title, we need to extract the text from the
text node which is a child of the \texttt{title} element, and put it
before `\texttt{Chapters:}'. One way is to apply the templates on
\texttt{title} alone using the special attribute \texttt{select},
whose value is the child's name, then produce `\texttt{Chapters:}',
and finally apply the templates to the chapters:
\XSLTinUnchecked{cookbook1.xsl}
\noindent The result is now:
\XMLinUnchecked{cookbook6_out.xml}
\noindent This is not quite it, because we must not apply the
templates to all the children of \texttt{cookbook}, only to the
chapters:
\XSLTinUnchecked{cookbook2.xsl}
\noindent The result is now:
\XMLinUnchecked{cookbook7_out.xml}
\noindent We just hit another quirk with the newlines. What happened
is that the \emph{selection} (via the attribute \texttt{select})
gathered all the \texttt{chapter} nodes in a linear structure called a
\emph{sequence}, the templates have been applied to all of the nodes
in it, then each of the results has been serialised, concatenated to the
others and, finally, a newline character has been added. The reason
why we had newlines after each `\texttt{A chapter}' before is because
each of these texts constituted a singleton sequence. In order to
recover the newlines, we could use the special element
\texttt{xsl:text}, whose purpose is to output the content of its
unique text node \emph{as it is}, without tweaking blanks and
newlines. Here, we could force a newline after each `\texttt{A
  chapter}'. In \XML, the newline character is the numbered entity
\verb|&#10;| \XSLTinUnchecked{cookbook3.xsl}
\noindent The result is now:
\XMLinUnchecked{cookbook8_out.xml}
\noindent Still not perfect, but let us abandon this example and spend
some time understanding sequences.

\section{Transforming sequences}

As we already know from part~\ref{part:linear}, the linear structure
of choice in functional languages is the stack, also called list. In
\XSLT, it is the \emph{sequence}. The empty sequence is written
\texttt{()} and the non-empty sequence is written \texttt{(\(x_1\),
  \(x_2\), \dots, \(x_n\))}, where the~\(x_i\) are items, or
\texttt{\(x_1\), \(x_2\), \dots, \(x_n\)}. The difference with stacks
is twofold. First, sequences are always flat, so when an item of a
sequence is itself a sequence, it is replaced by its contents, if
any. For instance, \texttt{(1,(),(2,(3)),4)} is actually evaluated in
\texttt{(1,2,3,4)}. In particular, a singleton sequence is the same as
the unique item it contains: \texttt{((5))} is~\texttt{5}. Second, no
cost is incurred when concatenating two sequences (in contrast with the
linear cost in the length of the first stack when concatenating two
stacks), so the previous evaluation costs~\(0\). The reason is that
concatenation of sequences, being frequently needed, is a built\hyp{}in
operation. Therefore, in \XSLT, concatenation is the primary operation,
not pushing, as it is the case with stacks.

Just like in \Erlang, sequences can contain any kind of items, not
just integers. We will use sequences to hold \XML elements and
attributes, for instance. Another thing to know about sequences is
that if \texttt{seq} is the name of a sequence, then \texttt{\$seq}
represents the sequence: \emph{note the dollar sign}. If we write
\texttt{seq} in the value of an attribute \texttt{match} or
\texttt{select}, it means the \emph{element}
\texttt{seq}. Furthermore, the first item in \texttt{\$seq} is written
\texttt{\$seq[1]}, the second \texttt{\$seq[2]} etc. where the natural
number is the \emph{position} of the item. If we select an item
missing in a sequence, the result is the empty sequence, for example,
if \texttt{\$seq[2]} is \texttt{()}, it means that \texttt{\$seq}
contains at most one item. It is often very useful to extract a
subsequence, similarly to projecting the substack (the tail) of a
given stack. This is done with the predefined function
\texttt{position}: \texttt{\$seq[position()>1]} or
\texttt{\$seq[position() != 1]}.

\mypar{Length}

Let us recall here a functional program computing the length of a
stack:
\begin{equation*}
\fun{len}_0(\el) \rightarrow 0;\quad
\fun{len}_0(\cons{x}{s}) \rightarrow 1 + \fun{len}_0(s).
\end{equation*}
In view of a translation to \XSLT, let us add to our source language a
conditional expression and rewrite the program without pattern
matching:
\begin{equation}
\fun{len}_0(s) \rightarrow \xkwd{if} \; s = \el \; \xkwd{then} \; 0 \;
\xkwd{else} \; 1 + \fun{len}_0(\fun{tl}(s)).
\label{def:len0_xslt}
\end{equation}
where \(\fun{tl}(s)\) (\emph{tail}) returns the immediate substack
of~\(s\). Note that we cannot define \fun{tail/1} without pattern
matching, so it has to be translated into a predefined function. In
order to start writing the \XSLT program, we need to be more specific
about the input. Let us suppose that we obtain our sequence by
selecting the children nodes \texttt{chapter} of the root element
\texttt{book}. In other words, we want to count the number of chapters
in a book. For instance, \XMLinUnchecked{toc.xml}
\noindent The \DTD is as follows:
\DTDin{book.dtd}
The style we recommend in \XSLT consists in explicitly typing as much
as we can the data and the templates. In order to do so, we need to
use a tiny part of a standard called \XMLSchema~\citep{Walmsley_2002},
by means of a namespace, just like we enable \XSLT through a namespace. This explains the canvas of our program:
\begin{alltt}
<?xml version="1.0" encoding="UTF-8"?>
<xsl:transform version="2.0"
               xmlns:xsl="http://www.w3.org/1999/XSL/Transform"
               \textbf{xmlns:xs="http://www.w3.org/2001/XMLSchema"}>
  <xsl:output method="text" encoding="UTF-8"/>
  ...
</xsl:transform>
\end{alltt}
In all our \XSLT programs whose output is text, we would like to
have a newline character at the end to improve readability. Because
the root element is likely to vary, we match the \emph{document root},
noted~\texttt{/}, which is an implicit node whose only child is the
root element. We then apply any available templates to the root
element and put a newline:
\begin{verbatim}
  <xsl:template match="/">
    <xsl:apply-templates/>
    <xsl:text>&#10;</xsl:text>
  </xsl:template>
\end{verbatim}
Now we need to match the root element and call the translation of
\fun{len\(_0\)/1}. To translate functions, we will use a special kind
of template, called a \emph{named template}, which differs from the
\emph{matching templates} we saw earlier. There are two sides to their
usage: their definition and their call.

The canvas for defining a named template is
\begin{alltt}
<xsl:template name="\(f\)" as="\(t\)">
  <xsl:param name="\(x\sb{1}\)" as="\(t\sb{1}\)"/>
  ...
  <xsl:param name="\(x\sb{n}\)" as="\(t\sb{n}\)"/>
  ...
</xsl:template>
\end{alltt}
The name of the template is~\(f\), each~\(x_i\) is a parameter of
type~\(t_i\), the type of the value computed by~\(f\) is~\(t\).

The canvas for calling a named template is as follows:
\begin{alltt}
<xsl:call-template name="\(f\)">
  <xsl:with-param name="\(x\sb{1}\)" select="\(v\sb{1}\)" as="\(t\sb{1}\)"/>
  ...
  <xsl:with-param name="\(x\sb{n}\)" select="\(v\sb{n}\)" as="\(t\sb{n}\)"/>
</xsl:call-template>
\end{alltt}
The template named~\(f\) has \(n\)~parameters \(x_1\), \(x_2\),
\ldots, \(x_n\), such that the type of~\(x_i\) is~\(t_i\) and its
value is~\(v_i\). Binding values to parameters in this way is featured
in programming languages like \Ada (\emph{named association}) and
\OCaml (\emph{labels}), and it allows the programmer to forget about
the parameters order, which is particularly useful when they are
numerous.

At this point it is important to know that there exists functions in
\XSLT, defined by the element \texttt{xsl:function}, and, although we
chose named templates for the translation, plain \XSLT functions would
do as well.

Resuming our translation, we need to call the named template which
will be the translation of \fun{len\(0\)/1}:
\begin{verbatim}
  <xsl:template match="book" as="xs:integer">
    <xsl:call-template name="len0">
      <xsl:with-param name="chapters" select="chapter"
                      as="element(chapter)*"/>
    </xsl:call-template>
  </xsl:template>
\end{verbatim}
As mentioned previously, we had to name the parameter we pass a value
to, that is, \texttt{chapters}. Perhaps more puzzling is the meaning
of the attribute value \texttt{element(chapter)*}: it is the type of a
sequence (possibly empty) of \texttt{chapter} elements. Although it
may not be necessary in this instance to provide this type because it
is implicit in the selection \texttt{select="chapter"}, we recommend
to always use the attribute \texttt{as} with
\texttt{xsl:with-param}. Furthermore, note that the result of the
template matching \texttt{book} is also typed as \texttt{xs:integer}
because it is the same result as that of the called template
\texttt{len0}. (In the template matching the document root
(\texttt{/}), we did not specify the return type because we wanted
that template to work with every transformation.)

Let us focus now on the definition of the called template, that is,
the translation of \fun{len\(_0\)/1} (at last!). We expect the
following pattern:
\begin{verbatim}
  <xsl:template name="len0" as="xs:integer">
    <xsl:param name="chapters" as="element(chapter)*"/>
    ...
  </xsl:template>
\end{verbatim}
The parameter named \texttt{chapters} corresponds to~\(s\) in the
definition~\eqref{def:len0_xslt} of \fun{len\(_0\)/1}; we changed the
name to fit more closely the specialised meaning, limited to chapters
here. We need now to translate the conditional expression \(\xkwd{if}
\dots \xkwd{then} \dots \xkwd{else} \dots\) into \XSLT. Unfortunately,
tests in \XSLT are quite verbose in general. Let us introduce three
elements that allow us to write general tests in the style of the
\texttt{switch} constructs of \Java:
\begin{alltt}
<xsl:choose>
  <xsl:when test="\(b\sb{1}\)">\(e\sb{1}\)</xsl:when>
  ...
  <xsl:when test="\(b\sb{n}\)">\(e\sb{n}\)</xsl:when>
  <xsl:otherwise>\(e\sb{n+1}\)</xsl:otherwise>
</xsl:choose>
\end{alltt}
The values~\(b_i\) of the \texttt{test} attributes are evaluated (The
apparent tautology is due to the \XSLT vocabulary: an attribute value
is actually not a value in general, but an expression.) in order until
one, say~\(b_j\), results in the boolean \texttt{true}, causing the
sequence~\(e_j\) to be evaluated; otherwise the sequence~\(e_{n+1}\)
(children of \texttt{xsl:otherwise}) is processed. Resuming our
exercise, we fill a bit more the blanks:
\begin{alltt}
  <xsl:template name="len0" as="xs:integer">
    <xsl:param name="chapters" as="element(chapter)*"/>
    <xsl:choose>
      <xsl:when test="\textbf{empty(\$chapters)}">
        ... <!-- Translation of \(0\) -->
      </xsl:when>
      <xsl:otherwise>
        ... <!-- Translation of \(1 + \fun{len}\sb{0}(s)\) -->
      </xsl:otherwise>
    </xsl:choose>
  </xsl:template>
\end{alltt}
Note the built\hyp{}in \XSLT function \texttt{empty}, which returns
\texttt{true} if its argument is an empty sequence, and \texttt{false}
otherwise. The translation of~\(0\) is not so simple, though! Indeed,
the obvious~\texttt{0} would mean that we actually produce a
\emph{text} containing the character~\texttt{0}, instead of the
expected integer. There is a very useful \XSLT element which may come
handy here --~although more versatile than it seems here, as we shall
see later. Let us introduce the \emph{sequence constructor}:
\begin{verbatim}
<xsl:sequence select="..."/>
\end{verbatim}
The selected value must evaluate into a sequence which is then
substituted in place of the element \texttt{xsl:sequence}. One may
wonder why this is so convoluted, and the reason is that the value of
the \texttt{select} attributes belongs to a sublanguage of \XSLT,
called \XPath, and \XPath can only be used for selections or tests
(\texttt{empty} is an \XPath function). In a \texttt{select}
attribute, \texttt{0}~means~\(0\), not the text made of the single
character~\texttt{0}, and \texttt{xsl:sequence} allows us to inject in
\XSLT the \XPath values, so we can construct a sequence made with
\XPath. Of course, we have to keep in mind that any item is equivalent
to a singleton sequence, in particular \texttt{(0)}~is the same
as~\texttt{0} in \XPath. Therefore, the translation of~\(0\) is
\begin{alltt}
      <xsl:when test="empty(\$chapters)">
        \textbf{<xsl:sequence select="0"/>}
      </xsl:when>
\end{alltt}
The expression \(1 + \fun{len}_0(s)\) is made of three parts: the
stack~\(s\), the function call \(\fun{len}_0(s)\) and the addition
of~\(1\) to the value of the call. We already know that stacks are
translated as sequences, we also know that function calls become calls
to named templates. It is simple to add~\(1\) to a function call in
\XPath, for instance \texttt{1 + f(\$n)}, but this syntax is not valid
outside a selection or a test and, anyway, we defined a named
template, not a function in \XSLT (which must be called in
\XPath). Therefore, we have to hold temporarily the value of the
recursive call in a variable, say~\texttt{x}, then use
\texttt{xsl:sequence} to compute (in \XPath) the value of \texttt{1 +
  \$x}. The element which defines a variable in \XSLT is
\texttt{xsl:variable} and it has two possible forms: either with a
\texttt{select} attribute or with children. In the former case, we
have the pattern
\begin{alltt}
<xsl:variable name="\(x\)" select="\(v\)" as="\(t\)">
\end{alltt}
and the latter is
\begin{alltt}
<xsl:variable name="\(x\)" as="\(t\)">
  ... <!-- Children whose value is \(v\) and has type \(t\) -->
</xsl:variable>
\end{alltt}
where the value of the variable~\(x\) is~\(v\), of type~\(t\). The
reason for the dual syntax is due again to the territory delimited by
\XPath: if~\(v\) can be computed in \XPath alone, we should use the
first form, otherwise we need the second. In our problem, we need the
second form because \(v\)~is the value of a recursive call \emph{not}
expressed in \XPath, as we use \texttt{xsl:call-template}. We can now
complete the program: \XSLTinUnchecked{len0.xsl}
\noindent The result of running it on our table of contents is, as
expected:
\XMLinUnchecked{toc_out.xml}

After we recover from the sustained effort and the disappointment from
such an incredible verbosity in comparison to \Erlang, we may find out
that there exists a built\hyp{}in function in \XPath, named
\texttt{count}, which is basically a translation of
\(\fun{len}_0/1\). Nevertheless, our purpose is to address beginners,
so the didactic usefulness primes other considerations.

Let us work on the translation of a better version of
\fun{len\(_0\)/1}:
\begin{equation*}
\begin{array}{@{}r@{\;}l@{\;}lr@{\;}l@{\;}l@{}}
\fun{len}_1(s) & \rightarrow & \fun{len}_1(s,0). &
\fun{len}_1(\el,n) & \rightarrow & n;\\
&&& \fun{len}_1(\cons{x}{s},n) & \rightarrow & \fun{len}_1(s,n+1).
\end{array}
\end{equation*}
Function \fun{len\(_1\)/1} is better than \fun{len\(_0\)/1} because
its cost is the same as the latter \emph{and} it uses a constant
amount of memory, being in tail form. The tail form implies that we do
not need a variable, because the addition is performed in \XPath
(parameter of the recursive call):
\XSLTinUnchecked{len1.xsl}
\noindent Note how we did not have to define a named template for
\fun{len\(_0\)/1}.

As a last variation, consider a variant input where the chapters are
all children of a \texttt{contents} element and their names are held
in a \texttt{title} attribute, instead of a text node:
\XMLinUnchecked{toc3.xml}
\noindent Of course, the \DTD \texttt{boot\_att.dtd} has to be
changed: \DTDin{book_att.dtd}
\noindent To solve this problem, we must modify a previous \XSLT
transformation, not think from the abstract functional language.
First, we should modify the call to the template so as to select the
chapters where they are now:
\begin{alltt}
  <xsl:template match="book" as="xs:integer">
    <xsl:call-template name="len3">
      <xsl:with-param name="elm" select="\textbf{contents/chapter}"
                                 as="element(chapter)*"/>
      <xsl:with-param name="n" select="0" as="xs:integer"/>
    </xsl:call-template>
  </xsl:template>
\end{alltt}
The expression \texttt{contents/chapter} is a selection in \XPath
which means: `Gather all the children \texttt{contents} of the context
node (\texttt{book}), preserving their relative order (here, there is
only one), then select all the children named \texttt{chapter} of all
those nodes, also preserving their relative order.' Other than that,
there is no need to change the template (apart from its name, now
\texttt{len3}). Notice also that using \texttt{title} attributes made
no difference.

But let us take this opportunity to make slight variations and learn
something new. Let us say that we want the template to be able to work
on any kind of elements, not just \texttt{chapter}, and we would like
to use a \emph{default parameter}. Indeed, the type of the template
parameter \texttt{chapters} is \texttt{element(chapter)*}, so it is
not enough general. The solution is the type \texttt{element()*},
which means `A sequence, possibly empty, of elements.' Moreover, the
original value of the parameter~\texttt{n} must always be~\texttt{0},
so we could make this value a default by adding a \texttt{select}
attribute to the corresponding \texttt{xsl:param} element:
\begin{alltt}
  <xsl:template name="len3" as="xs:integer">
    <xsl:param name="elm" as="\textbf{element()*}"/>
    <xsl:param name="n" as="xs:integer" \textbf{select="0"}/>
    ...
  </xsl:template>
\end{alltt}
In passing, we renamed the parameter to the neutral \texttt{elm}. Of
course, the call to the template is now shorter:
\begin{verbatim}
  <xsl:template match="book" as="xs:integer">
    <xsl:call-template name="len3">
      <xsl:with-param name="elm" select="contents/chapter"
                                 as="element(chapter)*"/>
    </xsl:call-template>
  </xsl:template>
\end{verbatim}
Note that it is still possible to impose an initial value
to~\texttt{n} that would not be~\texttt{0}. Also, it is possible now
to reuse the template \texttt{len3} for computing the length of any
sequence of elements.

In the end, the new transformation is
\XSLTinUnchecked{len3.xsl}

\mypar{Summing}

Given a stack of integers, we can compute their sum as follows
\begin{equation*}
\begin{array}{@{}r@{\;}l@{\;}lr@{\;}l@{\;}l@{}}
\fun{sum}(\cons{x}{s}) & \rightarrow & \fun{sum}_0(\cons{x}{s},0). &
\fun{sum}_0(\el,n) & \rightarrow & n;\\
&&& \fun{sum}_0(\cons{x}{s},n) & \rightarrow & \fun{sum}_0(s,n+x).
\end{array}
\end{equation*}
Immediately, we see that there is just a small difference between
\fun{sum\(_0\)/2} and \fun{len\(_1\)/2}: instead of adding~\(1\), we
add~\(x\). Thus, we can expect a tiny modification of the
corresponding \XSLT template. Let us assume the following input:
\XMLinUnchecked{sum.xml}
\noindent with the \DTD
\DTDin{sum.dtd}
\noindent The change we would like to make to \texttt{len3} is
\begin{alltt}
        <xsl:call-template name="\textbf{sum}">
          <xsl:with-param name="elm" as="element()*"
                          select="\$elm[position()>1]"/>
          <xsl:with-param name="n" as="xs:integer"
                          select="\textbf{\$elm[1]} + \$n"/>
        </xsl:call-template>
\end{alltt}
Unfortunately, the compiler \Saxon returns the following warning about
the change:
\begin{verbatim}
The only value that can pass type-checking is an empty sequence.
Required item type of value of parameter $n is xs:integer;
supplied value has item type xs:double
\end{verbatim}
and the wrong result \texttt{18135233274}, which is just the
concatenation of the contents of the text nodes of the \texttt{num}
elements. What happened? From the message, one thing is clear: the
problem has to do with the type system, that is why we did not
anticipate it from the untyped \fun{sum\(_0\)/2}. It is also clear
that the compiler understands that \texttt{\$n} is an integer,
therefore the culprit can only be our modification,
\texttt{\$elm[1]}. We would like it to be of type \texttt{xs:integer}
too, but is it? The type of \texttt{\$elm} is \texttt{element()*}, as
declared, which means that the items it contains are elements, not
integers, hence the issue. We need to force the type of
\texttt{\$elem[1]} to become \texttt{xs:integer}, that is, we
\emph{cast} it. First, we need to select the text node of
\texttt{\$elem[1]} and then cast it by using \texttt{xs:integer} like
an \XPath \emph{function}: \texttt{xs:integer(\$elm[1]/text())}. There
is no warning now:
\begin{verbatim}
18135233274
\end{verbatim}
The wrong result is still there. It is time to understand why! It is
clearly made of the all text nodes, in document order and serialised
without any separation. We know from the start that, by default, this
is what \XSLT is meant to do, therefore we failed to specify that
\emph{we} actually want. A look back at the first call to \texttt{sum}
reveals
\begin{verbatim}
  <xsl:template match="book" as="xs:integer">
    <xsl:call-template name="sum">
      <xsl:with-param name="elm" select="contents/chapter"
                                 as="element(chapter)*"/>
    </xsl:call-template>
  </xsl:template>
\end{verbatim}
Because we do not have \texttt{chapter} elements in the input now, an
empty sequence is selected by \texttt{contents/chapter}. It should be
\texttt{num}. But this does not change the wrong result. The reason is
that there is no context node because there is no \texttt{book}
element in the document. Therefore, we should write:
\begin{verbatim}
  <xsl:template match="numbers" as="xs:integer">
    <xsl:call-template name="sum">
      <xsl:with-param name="elm" select="num"
                                 as="element(num)*"/>
    </xsl:call-template>
  </xsl:template>
\end{verbatim}
This time, the correct result comes out:
\XMLinUnchecked{sum_out.xml}
\noindent There is still a subtle error, which becomes apparent when
inputting the empty sequence. (Among other values, we recommend to
test programs with extreme values of the input.) Indeed, the result is
then~\texttt{0}, which is not what is expected if we consider the
abstract function \fun{sum/1} as a specification.
\begin{equation*}
\fun{sum}(\cons{x}{s}) \rightarrow \fun{sum}_0(\cons{x}{s},0).
\end{equation*}
In \XSLT, we forgot to forbid the empty sequence. This can be achieved
by specifying a type `non\hyp{}empty sequence of elements':
\texttt{element()+}.
\begin{alltt}
  <xsl:template match="numbers" as="xs:integer">
    <xsl:call-template name="sum">
      <xsl:with-param name="elm" select="num"
                                 as="\textbf{element(num)+}"/>
    </xsl:call-template>
  </xsl:template>
\end{alltt}
If we try the input
\XMLinUnchecked{empty_sum.xml}
\noindent we obtain now the expected error message
\begin{verbatim}
An empty sequence is not allowed as the value of parameter $elm
\end{verbatim}
The transform is complete now:
\XSLTinUnchecked{sum.xsl}

If we would rather have no output instead of an error message at
run\hyp{}time, we could check for emptiness before the first call and
do nothing. But there is a shorter way, easily understood on the
abstract program:
\begin{equation*}
\begin{array}{@{}r@{\;}l@{\;}lr@{\;}l@{\;}l@{}}
\fun{sum}_1(s) & \rightarrow & \fun{sum}_2(s,0). &
    \fun{sum}_2(\el,n) & \rightarrow & \fun{nothing}();\\
&&& \fun{sum}_2([x],n) & \rightarrow & x+n;\\
&&& \fun{sum}_2(\cons{x}{s},n) & \rightarrow & \fun{sum}_2(s,x+n).
\end{array}
\end{equation*}
The data constructor in case of empty stack, \(\fun{nothing}()\), will
be translated in \XSLT as an empty element:
\begin{alltt}
\      \textbf{<xsl:when test="empty(\$elm)"/>}
      <xsl:when test="empty(\$elm[2])">
        <xsl:sequence select="xs:integer(\$elm[1]/text()) + \$n"/>
      </xsl:when>
\end{alltt}
The case for the singleton sequence is
\texttt{empty(\$elm[2])}. Indeed, we know that \texttt{\$elm} is not
empty, because that is the previous case; therefore, all we have to do
is to check for the existence of \texttt{\$elm[2]}: if absent, this
selection results in the empty sequence and, since we know that
\texttt{\$elm[1]} exists, the sequence \texttt{\$elm} contains exactly
one element. Still, there is a problem remaining with the types: the
template matching \texttt{numbers} and the template named \texttt{sum}
have to return a value of type \texttt{xs:integer}, which is not
possible if \texttt{\$elm} is empty, in which case, as we just saw, an
empty sequence is returned (because of the empty element
\texttt{xsl:when}). There is a way in \XPath to express the type `A
sequence with no item or exactly one.' using the operator
`\texttt{?}'. If we recall that a value can always be implicitly cast
to a sequence containing that value, then \texttt{xs:integer?} means
`An integer or an empty sequence.' Therefore,
\XSLTinUnchecked{sum1.xsl}

\mypar{Skipping}

We want to make a copy of a given stack, without its last item. A way
to achieve this is to check first whether the stack contains zero, one
or more items. In the first two cases, the result is the empty stack;
in the last, we know that the first item is not the last, so we keep
it and proceed recursively with the rest:
\begin{equation*}
\fun{cutl}(\cons{x,y}{s}) \rightarrow \cons{x}{\fun{cutl}(\cons{y}{s})};\quad
\fun{cutl}(s) \rightarrow \el.
\end{equation*}
In \Erlang, this would be implemented as follows (header omitted):
\begin{verbatim}
cutl([X|S=[_|_]]) -> [X|cutl(S)];
cutl(_)           -> [].
\end{verbatim}
To see how to express it in \XSLT, we need first to set some context
of use. For example, let us say that we have a table of contents
complying with the following \DTD, \texttt{book\_bis.dtd}:
\DTDin{book_bis.dtd}
\noindent For example, the input may be
\XMLinUnchecked{toc_bis.xml}
\noindent We want a copy of that \XML document without the last
chapter:
\XMLinUnchecked{toc_bis1_out.xml}
\noindent This is the first time we use \XSLT to output \XML, acting
as a negative filter, that is, filtering out part of the input. We can
start by reusing some code from previous transforms and then work on
the translation of the function \fun{cutl/1} in \XSLT, which will be a
template named \texttt{cutl}. But first, the boilerplate and a twist:
\begin{alltt}
<?xml version="1.0" encoding="UTF-8"?>
<xsl:transform version="2.0"
               xmlns:xsl="http://www.w3.org/1999/XSL/Transform"
               xmlns:xs="http://www.w3.org/2001/XMLSchema">
               \textbf{exclude-result-prefixes="xs"}>

  \textbf{<xsl:output method="xml" version="1.0"
              encoding="UTF-8" indent="yes"/>}
\end{alltt}
Note that the output method is no longer \texttt{text}, but
\texttt{xml}, as we wish to output \XML. Of course, we then need to
state which version of \XML we want (here, \texttt{1.0}), what the
encoding of the file will be (here, \texttt{UTF-8}), and if we want
the resulting \XML to be indented (yes, because it greatly increases
legibility, but if we would expect the output to be processed by
another \XSLT program, indentation could be dropped). There is another
novelty, which is the attribute \texttt{exclude-result-prefixes="xs"}
of \texttt{xsl:transform}. For the sake of clarity, we shall come back
to it after we are finished.

On we go now with the rest of the canvas:
\begin{alltt}
  <xsl:template match="/">
    <xsl:apply-templates/>
    <xsl:text>&#10;</xsl:text>
  </xsl:template>

  <xsl:template match="book" as="element(book)">
    \textbf{<xsl:copy>}
      <xsl:sequence select="author"/>
      <xsl:sequence select="title"/>
      <contents>
        <xsl:call-template name="cutl">
          <xsl:with-param name="items" select="contents/chapter"
                          as="element(chapter)*"/>
        </xsl:call-template>
      </contents>
    \textbf{</xsl:copy>}
  </xsl:template>

  <xsl:template name="cutl" as="item()*">
    <xsl:param name="items" as="item()*"/>
    ...
  </xsl:template>

</xsl:transform>
\end{alltt}
We used a bold typeface to bring forth a new \XSLT element, namely
\texttt{xsl:copy}. Perhaps it was expected \texttt{<book>...</book>},
and this would have worked indeed. Instead, we may prefer avoid
copying the element name too often, in case it changes in future
versions. This is where \texttt{xsl:copy} comes handy: it is \emph{a
  shallow copy of the context node}. The context node is the node that
was last matched by an element \texttt{xsl:template}, so it is here
\texttt{book}, and `shallow' means that the children are not copied
(we want to copy but also modify the descendants).

Furthermore, notice how we used \texttt{xsl:sequence} with a selection
of an input element (\texttt{author} and \texttt{title}). This is
where \texttt{xsl:sequence} shines: what it does is to \emph{refer to}
the selected elements, without actually copying them. In that sense,
it acts like a pointer, as featured in some imperative programming
languages, like \Clang, and thus saves memory.

Last, but not least, note how the output is constructed by recreating
an \XML document; in particular, the juxtaposition of elements denotes
the concatenation of the singleton sequences they are (for instance, the
two aforementioned \texttt{xsl:sequence} are written one after the
other).

Now we need to translate \fun{cutl/1}. As we already know, \XSLT does
not feature pattern matching, so we should rewrite our abstract
functional program without it:
\begin{equation*}
  \fun{cutl}(t) \rightarrow \xkwd{if} \; \fun{tl}(t) \neq \el \;
  \xkwd{then} \; \cons{\fun{hd}(t)}{\fun{cutl}(\fun{tl}(t))} \;
  \xkwd{else} \; \el.
\end{equation*}
where \(\fun{hd}(t)\) (\emph{head}) \index{head@\fun{hd/1}} evaluates
in the first item of the stack~\(t\) and \(\fun{tl}(t)\) (\emph{tail})
\index{tail@\fun{tl/1}} in the immediate substack of~\(t\). (Of
course, \(\fun{hd}(\el)\) and \(\fun{tl}(\el)\) could fail, so we
always must check that their argument is not the empty stack.) We note
the two occurrences of \(\fun{tl}(t)\), so, in \XSLT, we should use a
variable to hold the value of this call to avoid recomputing it. We
start like so:
\begin{alltt}
  <xsl:template name="cutl" as="item()*">
    <xsl:param name="items" as="item()*"/>
    \textbf{<xsl:variable name="tail" select="\$items[position()>1]"
                              as="item()*"/>}
    ...
  </xsl:template>
\end{alltt}
Note that we did not specialise the template to process only
\texttt{chapter} elements, but any kind of item, including primitive
types, like integers, but also nodes and, in particular, elements.

Now we need to translate the conditional. We already have
seen the element \texttt{xsl:choose} and proceed to fill the previous ellipsis:
\begin{alltt}
    <xsl:choose>
      <xsl:when test="\textbf{not(empty(\$tail))}"> ... </xsl:when>
      <xsl:otherwise> ... </xsl:otherwise>
    </xsl:choose>
\end{alltt}
The purpose of \XPath functions \texttt{empty} and \texttt{not} is
evident. The translation of the \xkwd{else} alternative is the empty
sequence in the \texttt{xsl:otherwise} element. This is easily done
without even the \texttt{xsl:sequence} element:
\begin{alltt}
    <xsl:choose>
      <xsl:when test="not(empty(\$tail))"> ... </xsl:when>
      \textbf{<xsl:otherwise/>}
    </xsl:choose>
\end{alltt}
Indeed, an empty element can always be considered as having an empty
sequence of children. In \XSLT, conditionals which have the form of
one \texttt{xsl:when} and an empty \texttt{xsl:otherwise} are better
expressed using the element \texttt{xsl:if}. For instance, our code
becomes
\begin{alltt}
    \textbf{<xsl:if test="not(empty(\$tail))"> ... </xsl:if>}
\end{alltt}
Implicitly, if the test fails, the value of the conditional
\texttt{xsl:if} is the empty sequence. We need now to translate
\(\cons{\fun{hd}(t)}{\fun{cutl}(\fun{tl}(t))}\). \index{head@\fun{hd/1}}
\index{tail@\fun{tl/1}} We already have at our disposal the
translation of \(\fun{tl}(t)\), which is the \XSLT variable
\texttt{tail}. The translation of \(\fun{hd}(t)\) is simply the
singleton sequence \texttt{<xsl:sequence
  select="\$items[1]"/>}. Instead of pushing on a stack, we concatenate
two sequences and this concatenation is simply textual juxtaposition:
\begin{verbatim}
    <xsl:if test="not(empty($tail))">
      <xsl:sequence select="$items[1]"/>
      <xsl:call-template name="cutl">
        <xsl:with-param name="items" select="$tail"
                        as="item()*"/>
      </xsl:call-template>
    </xsl:if>
\end{verbatim}
In the end, the solution is
\XSLTinUnchecked{last.xsl}

At this point, we may wonder why we needed to set the attribute
\texttt{exclude-result-prefixes="xs"} of element
\texttt{xsl:transform}. If we remove it, we obtain the same result
except for the element \texttt{contents}:

\begin{alltt}
   <contents \textbf{xmlns:xs="http://www.w3.org/2001/XMLSchema"}>
     ...
   </contents>
\end{alltt}
The reason is that when a namespace is declared, all the descendant
elements inherit it, except the namespace associated to \XSLT, here
named \texttt{xsl}. Therefore, when we wrote
\begin{verbatim}
   <contents>
     ...
   </contents>
\end{verbatim}
in the previous transform, the element \texttt{contents}
\emph{implicitly} had the namespace child node~\texttt{xs}. The reason
why \texttt{author} and \texttt{title} did not, is that we used
\texttt{xsl:sequence} to reference the input, where that namespace is
absent. The same happens with the elements \texttt{chapter}, which are
selected in the input. The element \texttt{book} was actually copied
with \texttt{xsl:copy}, and we saw that this element does not copy
children, amongst whose the namespace nodes. The default behaviour of
the \XSLT processor is to set the inherited namespaces in case they
are of some use in the output. In the present example, \texttt{xs} is
useless, so it is best to exclude it from the (namespace) prefixes in
the result: \texttt{exclude-result-prefixes="xs"}.

\paragraph{Skipping the penultimate item}

The purpose of this exercise is to write an \XSLT transform which
takes as input a table of contents and outputs it in \XML where the
penultimate chapter is missing. If there is no chapter or only one,
the output is identical to the input. The input should conform to the
following \DTD, named \texttt{book\_bis.dtd}:
\DTDin{book_bis.dtd}
\noindent For example, the input may be
\XMLinUnchecked{toc_bis.xml}
\noindent The corresponding output is
\XMLinUnchecked{toc_bis2_out.xml}
\noindent The boilerplate \XSLT code is the same, except the template,
which we name here \texttt{cutp}. We do not start from an abstract
functional program, but from the previous transform. We will need more
cases, so \texttt{xsl:choose} is back. Perhaps the first difference is
the case when \texttt{tail} is empty. This means that we need to keep
the first item, instead of ignoring it:
\begin{verbatim}
    <xsl:choose>
      <xsl:when test="empty($tail)">
        <xsl:sequence select="$items[1]"/>
      </xsl:when>
      ...
    </xsl:choose>
\end{verbatim}
As for the complementary case, when the tail is not empty, that is to
say, when there are at least two items, we do not know whether the
first one is the penultimate or not, and the same can be said about
the second. Therefore, we need more information on the structure of
the tail, in particular, whether its tail is, in turn, empty (the tail
of the tail of the complete sequence), in other words, whether the
sequence contains at least three items or not. If it does, then we
know that the first item is not the penultimate, but we still can not
say nothing about the others, so a recursive call is in order; if it
does not, then it means that the whole sequence contains exactly two
items, so we put in the result the second one only, and ignore the
first. In the end, we have
\begin{verbatim}
  <xsl:template name="cutp" as="item()*">
    <xsl:param name="items" as="item()*"/>
    <xsl:variable name="tail" select="$items[position()>1]"
                              as="item()*"/>
    <xsl:choose>
      <xsl:when test="empty($tail)">
        <xsl:sequence select="$items[1]"/>
      </xsl:when>
      <xsl:when test="empty($tail[position()>1])">
        <xsl:sequence select="$items[2]"/>
      </xsl:when>
      <xsl:otherwise>
        <xsl:sequence select="$items[1]"/>
        <xsl:call-template name="cutp">
          <xsl:with-param name="items" select="$tail"
                                       as="item()*"/>
        </xsl:call-template>
      </xsl:otherwise>
    </xsl:choose>
  </xsl:template>
\end{verbatim}
Notice that the case \texttt{<xsl:when test="empty(\$items)"/>} is
actually missing because it is unnecessary: if \texttt{\$items} is
empty, then \texttt{\$tail} is empty as well, and the result is thus
\texttt{\$items[1]}, which is the empty sequence.

\mypar{Reversal}

The purpose of this exercise is to write an \XSLT transform which
takes as input a table of contents with chapters and outputs the same
table of contents in \XML where the chapters have been reversed with
respect to the document order (so, for example, the introduction is
listed last). In section~\vref{sec:reversal}, we saw that the
straightforward definition of \fun{rev\(_0\)}:
\begin{equation*}
\begin{array}{@{}r@{\;}l@{\;}lr@{\;}l@{\;}l@{}}
  \fun{cat}(\el,t)\index{cat@\fun{cat/2}}
& \rightarrow & t;
& \fun{rev}_0(\el)
& \rightarrow & \el;\\
  \fun{cat}(\cons{x}{s},t)
& \rightarrow & \cons{x}{\fun{cat}(s,t)}.
& \fun{rev}_0(\cons{x}{s})
& \rightarrow & \fun{cat}(\fun{rev}_0(s),[x]).
\end{array}
\end{equation*}
We saw that this definition yielded a quadratic cost and therefore
should not be used for reversing stacks. In \XSLT, the cost is linear
because concatenation has cost zero. We would then write the following
translation:
\begin{verbatim}
  <xsl:template name="rev" as="item()*">
    <xsl:param name="items" as="item()*"/>
    <xsl:if test="not(empty($items))">
      <xsl:call-template name="rev">
        <xsl:with-param name="items" as="item()*"
                        select="$items[position()>1]"/>
      </xsl:call-template>
      <xsl:sequence select="$items[1]"/>
    </xsl:if>
  </xsl:template>
\end{verbatim}
Instead of producing an \XML document, we could use this opportunity
to see how to produce an \XHTML document. Although the purpose may
seem a bit foolish (reversing a table of contents), it is appropriate
to start learning complicated languages such as \XSLT and \XHTML.

In order to instruct an \XSLT processor to produce \XHTML, we need to
set some attributes of \texttt{xsl:transform} and \texttt{xsl:output}
as follows:
\begin{verbatim}
<xsl:transform version="2.0"
               xmlns:xsl="http://www.w3.org/1999/XSL/Transform"
               xmlns:xs="http://www.w3.org/2001/XMLSchema"
               xmlns:xhtml="http://www.w3.org/1999/xhtml"
               exclude-result-prefixes="xs">

  <xsl:output method="xhtml"
              doctype-public="-//W3C//DTD XHTML 1.0 Strict//EN"
              doctype-system=
              "http://www.w3.org/TR/xhtml1/DTD/xhtml1-strict.dtd"
              indent="yes"
              omit-xml-declaration="yes"/>
\end{verbatim}
Note that we defined a namespace \texttt{xhtml} for \XHTML elements
and that the \XHTML version is~\texttt{1.0} (`strict' means that it
adheres strictly to \XML). Perhaps the real novelty is setting
\texttt{omit-xml-declaration="yes"}. Since \XHTML (strict) is \XML,
the declaration \texttt{<?xml version="1.0"?>} is to be expected, but
some web browsers are confused by this, so we prefer to be on the safe
side and not have that declaration.

Given the previous table of contents, we now would like to obtain
\begin{verbatim}
<!DOCTYPE html
  PUBLIC "-//W3C//DTD XHTML 1.0 Strict//EN"
         "http://www.w3.org/TR/xhtml1/DTD/xhtml1-strict.dtd">
<html xmlns="http://www.w3.org/1999/xhtml"
      xml:lang="en" lang="en">
   <head>
      <meta http-equiv="Content-Type"
            content="text/html; charset=UTF-8" />
      <title>Definitive XML Schema</title>
   </head>
   <body>
      <h2>Definitive XML Schema</h2>
      <p>by Priscilla Walmsley</p>
      <h3>Reversed table of contents</h3>
      <ul>
         <li>Instances and schemas</li>
         <li>Schema composition</li>
         <li>Namespaces</li>
         <li>A quick tour of XML Schema</li>
         <li>Schema: An Introduction</li>
      </ul>
   </body>
</html>
\end{verbatim}
which, interpreted by a web browser, would likely render as

\noindent\rule{\linewidth}{0.5pt}
\noindent\textbf{\large Definitive XML Schema}

\bigskip

\noindent by Priscilla Walmsley

\bigskip

\noindent\textbf{Reversed table of contents}
\begin{itemize*}

  \item Instances and schemas
  \item Schema composition
  \item Namespaces
  \item A quick tour of XML Schema
  \item Schema: An Introduction

\end{itemize*}
\noindent\rule{\linewidth}{0.5pt}

\noindent First, here is the template matching \texttt{book}:
\begin{alltt}
  <xsl:template match="book" as="element(xhtml:html)">
    <html xmlns="http://www.w3.org/1999/xhtml"
          xml:lang="en" lang="en">
      <head>
        <title><xsl:sequence select="\textbf{title/text()}"/></title>
      </head>
      <body>
        <h2>\textbf{<xsl:value-of select="title"/>}</h2>
        <p>by \textbf{<xsl:value-of select="author"/>}</p>
        <h3>Reversed table of contents</h3>
        <ul>
          <xsl:call-template name="rev">
            <xsl:with-param name="chap"
                            select="contents/chapter"/>
          </xsl:call-template>
        </ul>
      </body>
    </html>
  </xsl:template>
\end{alltt}
Note first the selection \texttt{title/text()}, which means `the text
nodes of the element \texttt{title}, which is the child of the context
node (\texttt{book}).' Second, we meet a new \XSLT element,
\texttt{xsl:value-of}, whose purpose is to create a text node from the
selected items. If we select elements, like here the unique
\texttt{title} element, its \emph{descendant} text nodes (there is
only one child here) are concatenated in document order and put in a new
text node. Therefore, \texttt{<xsl:sequence select="title/text()"/>}
has the same result as \texttt{<xsl:value-of select="title"/>},
although in the later case a new text node has been created (instead
of being shared with the input). As a remark, we decided to specialise
the types to make them fit as closely as possible the elements being
processed, like \texttt{element(xhtml:html)}, which means: `One
element \texttt{html} in the namespace \texttt{xhtml}.'

Finally, the template doing the reversal is:
\begin{alltt}
  <xsl:template name="rev" as="element(xhtml:li)*">
    <xsl:param name="chap" as="element(chapter)*"/>
    <xsl:if test="not(empty(\$chap))">
      <xsl:call-template name="rev">
        \!\!<xsl:with-param name="chap"\!\! select="\$chap[position()>1]"/>
      </xsl:call-template>
      <li \textbf{xmlns="http://www.w3.org/1999/xhtml"}>
        \textbf{<xsl:value-of select="\$chap[1]/@title"/>}
      </li>
    </xsl:if>
  </xsl:template>
\end{alltt}
Again, we specialise the types, like \texttt{element(xhtml:li)*},
meaning: `Sequence (possibly empty) of elements \texttt{li} in the
namespace \texttt{xhtml}.' And \texttt{element(chapter)*} is a
sequence of chapters, without namespace. But the two interesting
excerpts are set in a bold typeface.

The first one is the declaration of the \texttt{xhtml} namespace in
the element \texttt{li}:
\texttt{xmlns="http://www.w3.org/1999/xhtml"}. This is simply
necessary to conform with the type of the value of the
template. Indeed, this value should be, as we just saw, a sequence of
elements \texttt{li} in the namespace \texttt{xhtml}. But
\texttt{<li>} is actually outside all namespaces, because there is no
default namespace declaration in any ascendant node, contrary to the
template matching \texttt{book}, which we described previously. There,
we had the declaration \texttt{<html
  xmlns="http://www.w3.org/1999/xhtml" xml:lang="en" lang="en">}, so
any descendent element written without a namespace actually
inherits the \texttt{xhtml} namespace. This is not the case in the
template named \texttt{rev}, so an explicit declaration is necessary,
otherwise a type error is raised by the \XSLT compiler.

The second noteworthy excerpt is the selection of the
\texttt{xsl:value-of} element, \texttt{\$chap[1]/@title}, which means:
`The \texttt{title} attribute of the first element of the sequence
\texttt{\$chap}.' Here, we cannot substitute the element
\texttt{xsl:sequence}, as with the text node before. Indeed, if we try
\texttt{<xsl:sequence select="\$chap[1]/@title"/>}, the result is
\begin{verbatim}
       ...
         <li title="Instances and schemas"></li>
         <li title="Schema composition"></li>
         <li title="Namespaces"></li>
         <li title="A quick tour of XML Schema"></li>
         <li title="Schema: An Introduction"></li>
       ...
\end{verbatim}
Let us recall that \texttt{xsl:sequence} is an alias of, or a
reference to, the input, in this case an attribute node, so we should
expect a \texttt{title} \emph{attribute} in the output. But we wanted
the value of the attribute \texttt{title}, not the attribute itself,
hence the need of \texttt{xsl:value-of}. If we wonder why we really
need to create a text node, we must understand that \emph{the value of
  an attribute is not a text node}. This can be seen by changing the
selection to \texttt{<xsl:sequence select="\$chap[1]/title/text()"/>},
in which case the result is
\begin{verbatim}
       ...
         <li></li>
         <li></li>
         <li></li>
         <li></li>
         <li></li>
       ...
\end{verbatim}
Attributes are special and often a source of confusion for beginners.

\mypar{Comma-separated values}
\label{par:CSV}

The purpose of this exercise is to write an \XSLT transform which
takes as input a sequence of elements containing each one text node
and output their contents in the same order, separated by commas and
ended by a period. If the input sequence is empty, the result is the
empty sequence. More precisely, let us assume the following \DTD:
\DTDin{csv.dtd}
\noindent and the conforming input
\XMLinUnchecked{csv.xml}
\noindent Then we want
\XMLinUnchecked{csv_out.xml}

The algorithm is simple enough: if the input sequence is empty, the
result is the empty sequence; if the input is a singleton sequence,
the result is the item it contains, followed by a period; otherwise,
the first item of the result is the first in the input, followed by a
comma and the value of a recursive call on the tail. Probably, the
difficulty is to implement this scheme with \XSLT. Here is how:
\XSLTinUnchecked{csv1.xsl}
\noindent Note first that we merged the two templates that match the
document root (\texttt{/}) and the root element (\texttt{numbers})
because we do not reconstruct an \XML document. Furthermore, we may
remark on the type \texttt{text()*}, which means `A sequence (possibly
empty) of text nodes.' The remaining titbits are the
\texttt{xsl:value-of} elements, in particular a new attribute,
\texttt{separator}. Its value must be a string which is used to
separate the selected items. By default, that string is
\texttt{'\textvisiblespace'}, that is why we set it here to the empty
string. Otherwise, we would obtain:
`\texttt{0\textvisiblespace,1\textvisiblespace,A\textvisiblespace,B\textvisiblespace,C\textvisiblespace.}'. Note
that the value of the attribute \texttt{separator} is the
\emph{contents} of a string, so if we use \texttt{"''"}, we are
\emph{not} specifying the empty string and produce instead:
`\texttt{0'',1'',A'',B'',C''.}'.

At this point it is perhaps pertinent to draw the relationships
between the different types we have encountered, and learn a few
more. Consider the tree in \fig~\vref{fig:tests}.
\begin{figure}
\centering
\includegraphics{tests}
\caption{\XPath subtypes}
\label{fig:tests}
\end{figure}
The ascendant reading of an edge from a node~\(x\) to a node~\(y\) is
`\(x\)~is a [subtype of] \(y\).' For instance, an \texttt{element()}
is a \texttt{node()}. This models a subtyping relationship, which
means that in any context where a \texttt{node()} is correct, a
\texttt{element()} is also correct. This relationship is transitive,
so anywhere an \texttt{item()} is expected, a \texttt{text()} is
valid, which can be seen at work in the parameter of the template
named \texttt{csv}. As an illustration, let us consider a slight
variation of the input, where the content of interest is stored as
attribute values, like so:
\XMLinUnchecked{csv_att.xml}
\noindent The \DTD \texttt{csv\_att.dtd} is
\DTDin{csv_att.dtd}
\noindent It is then enough to change the selection of the parameter
\texttt{item} as follows:
\begin{alltt}
      <xsl:with-param name="items" select="numbers/hexa/\textbf{@val}"/>
\end{alltt}
The type of the selection is \texttt{attribute()*}, which is a
subtype of \texttt{item()*}, therefore is a suitable value for the
parameter of template \texttt{csv} and the result is the same as with
the first \XML input (no attributes).

While we are interested in the data types, let us pay attention to the
selections of the elements \texttt{xsl:value-of}, for instance
\texttt{(\$items[1],'.')}. Statically, the type of \texttt{\$items[1]}
is \texttt{item()}, although we know, from the initial template call,
that it is actually \texttt{text()}. The type of \texttt{'.'} is
\texttt{xs:string}. The sequence in question then has the type
\texttt{item()*}, because \texttt{xs:string} is a subtype of
\texttt{item()}, as seen in \fig~\vref{fig:tests}. Since the result of
the template is of type \texttt{text()*}, the strings it contains will
be cast into text nodes, thus allocating memory. The serialiser is the
back\hyp{}end of the code generated by the \XSLT compiler (the
front\hyp{}end is the \XML parser) and its purpose is to produce text
from all the values obtained. In this case, it will then destructure
all these text nodes to generate strings (either displayed on a
terminal or written in a file). If we want to avoid this boxing of
strings in text nodes and their subsequent unboxing, we could plan
ahead and opt for a return type \texttt{xs:string*}, so
\begin{alltt}
  <xsl:template match="/" as="\textbf{xs:string*}">
    ...
  </xsl:template>

  <xsl:template name="csv" as="\textbf{xs:string*}">
    ...
  </xsl:template>
\end{alltt}
As a result of this change, we obtain
`\texttt{0,\textvisiblespace{}1,\textvisiblespace{}A,\textvisiblespace{}B,\textvisiblespace{}C.}'. The
extra spaces come from the fact that we forgot that the elements
\texttt{xsl:value-of} create text nodes and the implicit serialisation
of these (by casting to \texttt{xs:string*}) yields blank separators,
here rendered as `\texttt{\textvisiblespace}'. The morale of this
excursus is to keep working with non\hyp{}primitive types, that is,
attributes and nodes, if the input contains attributes and nodes, and
let the serialiser manage the linearisation for the output. (It is
possible to use \XSLT for string processing, although this is not the
main application domain of the language, in which case working with
\texttt{xs:string*} makes sense.)

Let us give this exercise a last spin by remarking that the algorithm
may be conceived as \emph{working in parallel} on the sequence items,
as long as we know how to distinguish the last item because it must
be treated differently (it is followed by a period instead of a
comma). But `parallel' does not necessarily imply a temporal meaning,
like multithreading, and can be also thought as processing in
isolation, then fusion of the partial results, which is what a
\emph{map} does. We saw maps when introducing functional iterators in
\Erlang, \vpageref{par:maps}:
\begin{equation*}
\erlcode{map(\(F\),[\(X_1\),\(X_2\),\(\ldots\),\(X_{n}\)])}
\equiv
\erlcode{[\(F\)(\(X_1\)),\(F\)(\(X_2\)),\(\ldots\),\(F\)(\(X_{n}\))]}.
\end{equation*}
A simple definition is
\begin{verbatim}
map(_,   []) -> [];
map(F,[X|S]) -> [F(X)|map(F,S)].
\end{verbatim}
Note that \texttt{map} is a higher\hyp{}order function and \XSLT only
features first\hyp{}order templates and functions. Nevertheless, we
can implement maps by using a kind a template we already have been
using since the introduction to \XSLT, starting
\vpageref{par:matching}: the \emph{matching template}. The \XSLT
processor (by which we mean the run\hyp{}time produced by the \XSLT
compiler) implicitly performs a preorder traversal of the input \XML
tree; when it finds an element~\(e\), it evaluates the corresponding
matching template and carries on with the traversal. A matching
template works as a rewrite rule for an implicit function in our
abstract functional language, each rule being tried in turn when
encountering a node in the \XML tree.

Since we often select some children of the context node and apply to
them the same treatment (as if in parallel), we need a mechanism to
gather the results for each child into one sequence. In the running
exercise, we wish to select a sequence of \texttt{hexa} elements,
children of \texttt{numbers}, and expect a template matching
\texttt{hexa}. We also want to group the results, just like a map
would do. This is achieved twofold by the definition of a template
\begin{verbatim}
  <xsl:template match="hexa" as="text()*">
    ...
  </xsl:template>
\end{verbatim}
which is analogous to the definition of the functional parameter~\(F\)
above and by
\begin{verbatim}
    <xsl:apply-templates select="hexa"/>
\end{verbatim}
which is analogous to calling \texttt{map} in
\Erlang.\label{match_as_map}

The only problem remaining is to find out if the context node
\texttt{hexa} is the last in the sequence upon which all templates
were applied. This is where \XPath comes handy in providing a function
\texttt{last()}, which returns the position of the last item matched
in the same sequence as the context node. Here is what the transform
looks like now:
\XSLTinUnchecked{csv3.xsl}
\noindent This reminds us that any item implicitly carries information
about itself, including its position in a sequence, but also the
position of the last item in that sequence. Notice that we do not need
any parameter in the template matching \texttt{hexa}, because, inside,
the context node is one of the original elements \texttt{hexa}, and we
do not need to know which one is it or what are the others (think
parallel processing, if you feel inclined to do so). For example,
\begin{verbatim}
<xsl:sequence select="text()"/>
\end{verbatim}
means: `Reference the text node of the context node,' (as opposed to
copying it with \texttt{xsl:value-of}).

We will use matching templates in the forthcoming section about the
transformation of trees, but we need first more practice to understand
better named templates, because they are closer to the concept of
first\hyp{}order function in our abstract functional language.

\mypar{Shuffling}

The purpose of this exercise is to write an \XSLT transform which
takes as input two sequences of elements and output one sequence
containing the items of each sequence shuffled, or, more precisely,
the first item of the resulting sequence is the first item of the
first sequence, the second item is the first item of the second
sequence, the third item is the second item of the first sequence, the
fourth item is the second item of the second sequence etc. An
enlightening analogy is interleaving two hands from a deck of cards.

If the first items of both sequences are taken out at the same time,
then comes a moment when either both sequences are empty or only one
of them is. The problem is actually underspecified: nothing is said
about what to do if the two sequences are not of the same
length. Actually, in the latter case, we will ignore the remaining
items.

The \DTD we have in mind is the following
\DTDin{persons.dtd}
\noindent Then, this input:
\XMLinUnchecked{persons1.xml}
\noindent yields this output:
\XMLinUnchecked{persons1A_out.xml}
\noindent The following input:
\XMLinUnchecked{persons2.xml}
\noindent yields this output:
\XMLinUnchecked{persons2A_out.xml}
\noindent This input:
\XMLinUnchecked{persons3.xml}
\noindent yields this output:
\XMLinUnchecked{persons3A_out.xml}
Following the strategy outlined above, we expect a template named
\texttt{shuffle} to have two parameters, one for the names and one for
the notes:
\begin{verbatim}
  <xsl:template match="persons" as="element(persons)">
    <xsl:copy>
      <xsl:call-template name="shuffle">
        <xsl:with-param name="names" select="names/name"/>
        <xsl:with-param name="notes" select="notes/note"/>
      </xsl:call-template>
    </xsl:copy>
  </xsl:template>

  <xsl:template name="shuffle" as="element()*">
    <xsl:param name="names" as="element(name)*"/>
    <xsl:param name="notes" as="element(note)*"/>
    ...
  </xsl:template>
\end{verbatim}
Note that we use \texttt{xsl:copy} to copy the context node
\texttt{persons} and that we had to use the return type
\texttt{element()*} because we cannot express in \XPath: `A sequence
of mixed elements \texttt{name} and \texttt{note}.' The body of that
template follows our plan. If the parameters are both not empty, we do
something, otherwise an implicit empty sequence will be produced. This
test is performed in \XPath using the Boolean connector \texttt{and}
as follows:
\begin{alltt}
  <xsl:template name="shuffle" as="element(persons)*">
    <xsl:param name="names" as="element(name)*"/>
    <xsl:param name="notes" as="element(note)*"/>
    \textbf{<xsl:if test="not(empty(\$names)) and not(empty(\$notes))">}
      <xsl:sequence select="\$names[1]"/>
      <xsl:sequence select="\$notes[1]"/>
      <xsl:call-template name="shuffle">
        <xsl:with-param name="names"
                        select="\$names[position()>1]"/>
        <xsl:with-param name="notes"
                        select="\$notes[position()>1]"/>
      </xsl:call-template>
    </xsl:if>
  </xsl:template>
\end{alltt}

Another possibility, when confronted to dangling items, is to append
them to the already outputted items. For instance, given the first
input above yields now
\XMLinUnchecked{persons1B_out.xml}
\noindent The second input above leads to
\XMLinUnchecked{persons2B_out.xml}
\noindent And the last input above results in
\XMLinUnchecked{persons3B_out.xml}
\noindent Here is a solution:
\begin{verbatim}
  <xsl:template name="shuffle" as="element()*">
    <xsl:param name="names" as="element(name)*"/>
    <xsl:param name="notes" as="element(note)*"/>
    <xsl:choose>
      <xsl:when test="empty($notes)">
        <xsl:sequence select="$names"/>
      </xsl:when>
      <xsl:when test="empty($names)">
        <xsl:sequence select="$notes"/>
      </xsl:when>
      <xsl:otherwise>
        <xsl:sequence select="($names[1],$notes[1])"/>
        <xsl:call-template name="shuffle">
          <xsl:with-param name="names"
                          select="$names[position()>1]"/>
          <xsl:with-param name="notes"
                          select="$notes[position()>1]"/>
        </xsl:call-template>
      </xsl:otherwise>
    </xsl:choose>
  </xsl:template>
\end{verbatim}
Note how we placed \texttt{\$notes[1]} after \texttt{\$names[1]} in
\XPath with
\begin{verbatim}
        <xsl:sequence select="($names[1],$notes[1])"/>
\end{verbatim}
instead of working at the level of \XSLT, as previously:
\begin{verbatim}
      <xsl:sequence select="$names[1]"/>
      <xsl:sequence select="$notes[1]"/>
\end{verbatim}

As a variation, it is possible to achieve the same result by
extracting one element at a time, instead of two. Of course, the
program will be about twice as slow, but it is interesting
nevertheless:
\begin{verbatim}
  <xsl:template name="shuffle" as="element()*">
    <xsl:param name="names" as="element(name)*"/>
    <xsl:param name="notes" as="element(note)*"/>
    <xsl:choose>
      <xsl:when test="empty($names)">
        <xsl:sequence select="$notes"/>
      </xsl:when>
      <xsl:otherwise>
        <xsl:sequence select="$names[1]"/>
        <xsl:call-template name="shuffle">
          <xsl:with-param name="names" select="$notes"/>
          <xsl:with-param name="notes"
                          select="$names[position()>1]"/>
        </xsl:call-template>
      </xsl:otherwise>
    </xsl:choose>
  </xsl:template>
\end{verbatim}
The point is to swap the arguments in the recursive call. Of course,
the parameters' names are not pertinent anymore, and a neutral
renaming would be fitting. In our abstract functional language, we
would write
\begin{equation*}
\fun{shuffle}(\el,t) \rightarrow t;\quad
\fun{shuffle}(\cons{x}{s},t) \rightarrow \cons{x}{\fun{shuffle}(t,s)}.
\end{equation*}
This is actually almost the same definition as that of \fun{cat/1} (concatenation):
\begin{equation*}
\fun{cat}(\el,t) \rightarrow t;\quad
\fun{cat}(\cons{x}{s},t) \rightarrow \cons{x}{\fun{cat}(\underline{s},\underline{t})}.
\end{equation*}

\mypar{Maximum}

The aim of this exercise is to write an \XSLT transform which takes as
input a sequence of integers and outputs the maximum of these numbers
as plain text. If an item is not castable to \texttt{xs:integer}, a
dynamic type error is raised and the execution is stopped. If an item
is an empty text node, for example, \texttt{<num/>}, it is skipped. If
the sequence contains no integer, not text is outputted, because the
maximum is undefined. The root element is \texttt{numbers} and the
elements containing the numbers are named \texttt{num}. Any element
other than \texttt{num} is ignored.

The \DTD we have in mind is the following:
\DTDin{numbers.dtd}
\noindent If the input document is
\XMLinUnchecked{numbers.xml}
\noindent the result is
\XMLinUnchecked{numbers_out.xml}

\noindent First, let us set the types in the following schema:
\begin{verbatim}
  <xsl:template match="numbers" as="xs:integer?">
    ...
  </xsl:template>

  <xsl:template name="max" as="xs:integer?">
    <xsl:param name="int" as="xs:integer*"/>
    <xsl:param name="cur" as="xs:integer?"/>
    ...
  </xsl:template>
\end{verbatim}
Let's recall the type operator `\texttt{?}' meaning `One or none', so
\texttt{xs:integer?} is either an empty sequence or a sequence
containing one integer. This precaution is necessary because we are
not certain that the input contains at least an integer (we even
allowed for dummy elements \texttt{foo}, as a look back at the \DTD
confirms). The parameter \texttt{int} contains the remaining integers
to examine, whilst \texttt{cur} is the current maximum, if any. That
is why we should initialise the latter with the contents of the first
number:
\begin{alltt}
  <xsl:template match="numbers" as="xs:integer?">
    <xsl:call-template name="max">
      <xsl:with-param name="int"
                      select="num[position()>1]/text()"/>
      <xsl:with-param name="cur" select="\textbf{num[1]/text()}"/>
    </xsl:call-template>
  </xsl:template>
\end{alltt}
Note that had we written \texttt{\$num[1]/text()}, the selection would
have been empty, as there is no \emph{variable} \texttt{num}, but,
instead, we meant the \emph{child element} \texttt{num}. Next, we
selected the text nodes, although the expected types are
\texttt{xs:integer*} and \texttt{xs:integer?}. In fact, a cast at
run\hyp{}time will be performed. In the case of \texttt{cur}, if the
cast fails, the empty sequence will result; otherwise, an integer
(that is, a sequence containing a single integer). In the case of
\texttt{int}, a cast is attempted for each element in the sequence and
the resulting sequences are concatenated.

There are several ways to solve this problem. We could make the
following cases:
\begin{itemize}

  \item if there are no integers to examine, the result is the current
    integer, if any;

  \item if there is an actual, current integer, and if it is greater
    than the first integer to examine, we start over while discarding
    the latter;

  \item otherwise, the first integer to examine becomes the current
    maximum and we start over and discard the previous maximum.

\end{itemize}
This plan is implemented as follows:
\begin{verbatim}
  <xsl:template name="max" as="xs:integer?">
    <xsl:param name="int" as="xs:integer*"/>
    <xsl:param name="cur" as="xs:integer?"/>
    <xsl:choose>
      <xsl:when test="empty($int)">
        <xsl:sequence select="$cur"/>
      </xsl:when>
      <xsl:when test="not(empty($cur)) and $cur ge $int[1]">
        <xsl:call-template name="max">
          <xsl:with-param name="int"
                          select="$int[position()>1]"/>
          <xsl:with-param name="cur" select="$cur"/>
        </xsl:call-template>
      </xsl:when>
      <xsl:otherwise>
        <xsl:call-template name="max">
          <xsl:with-param name="int"
                          select="$int[position()>1]"/>
          <xsl:with-param name="cur" select="$int[1]"/>
        </xsl:call-template>
      </xsl:otherwise>
    </xsl:choose>
  </xsl:template>
\end{verbatim}
Notice how we checked for the presence of a current maximum with
\texttt{not(empty(\$cur))} and the \XPath Boolean operator `greater
than or equal to' is~\texttt{ge}. We may further remark that, in both
recursive calls, the parameter \texttt{int} has the same value
\texttt{\$int[position()>1]}, so we might want to share the code as
follows:
\begin{alltt}
<xsl:template name="max" as="xs:integer?">
  <xsl:param name="int" as="xs:integer*"/>
  <xsl:param name="cur" as="xs:integer?"/>
  <xsl:choose>
    <xsl:when test="empty(\$int)">
      <xsl:sequence select="\$cur"/>
    </xsl:when>
    <xsl:otherwise>
      <xsl:call-template name="max">
        <xsl:with-param name="int" select="\$int[position()>1]"/>
        <xsl:with-param name="cur">
          <xsl:choose>
            <xsl:when\!\! test="not(empty(\$cur)) and \$cur\! ge \!\$int[1]">
              <xsl:sequence select="\$cur"/>
            </xsl:when>
            <xsl:otherwise>
              <xsl:sequence select="\$int[1]"/>
            </xsl:otherwise>
          </xsl:choose>
        </xsl:with-param>
      </xsl:call-template>
    </xsl:otherwise>
  </xsl:choose>
</xsl:template>
\end{alltt}
This template contains less duplication and is more logically
structured, but it is longer, which means that the usual reflexes
gained from experience in other programming languages may be
counterproductive in \XSLT. Note in passing that this answer
illustrates that \texttt{xsl:with-param} may have children instead of
a \texttt{select} attribute.

\mypar{Reducing}

The purpose of this exercise is to write an \XSLT transform which
takes as input a flat document tree, that is, the document root has
children but no grand\hyp{}children. The children carry an attribute
and the output should be the same document without those children
being consecutively repeated. This is the same as the function
\fun{red/1} we saw in \fig~\vref{fig:red}:
\begin{equation*}
\begin{array}{@{}r@{\;}c@{\;}l@{}}
\fun{red}(\el) & \rightarrow & \el;\\
\fun{red}(\cons{x,x}{s}) & \rightarrow & \fun{red}(\cons{x}{s});\\
\fun{red}(\cons{x}{s})   & \rightarrow & \cons{x}{\fun{red}(s)}.
\end{array}
\end{equation*}
In \XSLT, we need more constraints on the input and we must be take
into account data types. Let us opt for the following \DTD:
\DTDin{numbers_bis.dtd}
\noindent For example, this valid input
\XMLinUnchecked{numbers_bis.xml}
\noindent results in
\XMLinUnchecked{numbers_bis_out.xml}
\noindent If we start from \fun{red/1}, we need to remove the pattern
matching and use instead conditionals. First, we separate the patterns
depending on the number of items:
\begin{equation*}
\begin{array}{@{}r@{\;}c@{\;}l@{}}
\fun{red}(\el) & \rightarrow & \el;\\
\fun{red}([x]) & \rightarrow & [x];\\
\fun{red}(\cons{x,x}{s}) & \rightarrow & \fun{red}(\cons{x}{s});\\
\fun{red}(\cons{x,y}{s}) & \rightarrow & \cons{x}{\fun{red}(\cons{y}{s})}.
\end{array}
\end{equation*}
Now, we can remove pattern matching:
\begin{equation*}
\begin{array}{@{}r@{\;}c@{\;}l@{}}
\fun{red}(t) & \rightarrow & \xkwd{if} \; t=\el\; \xkwd{or} \;
                             \fun{tl}(t)=\el \; \xkwd{then} \; t\\
             &             & \xkwd{else} \; \xkwd{if} \; \fun{hd}(t) = \fun{hd}(\fun{tl}(t)) \;
\xkwd{then} \; \fun{red}(\fun{tl}(t)) \; \xkwd{else} \; \cons{\fun{hd}(t)}{\fun{red}(\fun{tl}(t))}.
\end{array}
\end{equation*}
where \(\fun{hd}(t)\) \index{head@\fun{hd/1}} is the head of
stack~\(t\) and \(\fun{tl}(t)\) \index{tail@\fun{tl/1}} is the tail
of~\(t\), that is, its immediate substack. In \XSLT, we may define a
variable by means of the element \texttt{xsl:variable}, so we can
improve the translation by computing only once the translation of
\(\fun{red}(\fun{tl}(t))\). We also know that \(\fun{hd}(t)\)
translates as \texttt{\$t[1]}, where~\texttt{t} is the translation
of~\(t\), and \(\fun{tl}(t)\) translates as \texttt{\$t[position()>1]}
or \texttt{\$t[position()!=1]}. Here is the full transform:
\XSLTinUnchecked{red.xsl}
\noindent Note how we do not cast the attribute value to
\texttt{xs:integer}, for example with
\texttt{xs:integer(\$t[1]/@val) ne xs:integer(\$t[2]/@val)}, because
we want to allow any kind of value for comparison.

\mypar{Merging}
\label{par:merging}

The purpose of this exercise is to write an \XSLT transform which
takes as input two sequences of elements which are sorted by
increasing values of the same integer attribute, and it outputs one
sequence containing all the items sorted increasingly. We can reuse
for this the function \fun{mrg/2} in \fig~\vref{fig:mrg}, which merges
two ordered stacks:
\begin{equation*}
\begin{array}{r@{\;}l@{\;}l}
\fun{mrg}(\el,t)         & \rightarrow & t;\\
\fun{mrg}(s,\el)         & \rightarrow & s;\\
\fun{mrg}(\cons{x}{s},\cons{y}{t}) & \rightarrow
& \cons{y}{\fun{mrg}(\cons{x}{s},t)},\;\text{if \(x \succ y\)};\\
\fun{mrg}(\cons{x}{s},t) & \rightarrow & \cons{x}{\fun{mrg}(s,t)}.
\end{array}
\end{equation*}
We envisage the following simple \DTD \texttt{list.dtd}:
\DTDin{list.dtd}
\noindent Given the following \XML document
\XMLinUnchecked{ordered_lists.xml}
\noindent the output is
\XMLinUnchecked{ordered_lists_out.xml}
Here, we can translate \fun{mrg/2} in \XSLT without explicitly getting
rid of pattern matching:
\XSLTinUnchecked{merge1.xsl}
\noindent If we would like instead produce text, we only have the
changes
\begin{verbatim}
...
  <xsl:output method="text"/>
...
  <xsl:template match="lists" as="xs:integer*">
    <xsl:call-template name="merge">
      <xsl:with-param name="seq1" select="list[1]/item"
                                  as="xs:integer*"/>
      <xsl:with-param name="seq2" select="list[2]/item"
                                  as="xs:integer*"/>
    </xsl:call-template>
  </xsl:template>

  <xsl:template name="merge" as="xs:integer*">
    <xsl:param name="seq1" as="xs:integer*"/>
    <xsl:param name="seq2" as="xs:integer*"/>
    ...
  </xsl:template>
...
\end{verbatim}


\section{Transforming trees}

After an extensive training with the transformation of sequences, it
is time that we tackle the general case, that is, trees. As we have
mentioned at the beginning of this chapter, \XML trees are
\emph{unranked}, which means that an element node can have a variable
number of children, if not invalidated by a \DTD. This is in contrast
with binary trees, for instance, whose nodes can only have two or no
children.

\mypar{Size}

The purpose of this exercise is to write an \XSLT transform which
takes as input a table of contents and computes the number of
sections. But, contrary to a previous exercise, the table is not flat
here, more precisely, the \DTD we have in mind is as follows:
\DTDin{book_deep.dtd}
\noindent An example of valid \XML document is
\XMLinUnchecked{toc_deep.xml}
\noindent Of course, we expect the result
\saxon{tmp.txt}{toc_deep}{count1}
Instead of going back to our abstract functional language, or \Erlang,
and then translating to \XSLT, let us try to figure out the algorithm
in plain English and move to write the transform.

The idea is to match the root element, then select the first level of
sections, just below the chapters. This sequence of nodes
\texttt{section} is passed to a template named \texttt{count}, whose
job is to count all the sections in it. If this sequence of sections
is empty, the answer is~\texttt{0}. Otherwise,
\begin{enumerate}

\item we call recursively \texttt{count} on the subsections of the
  first section;

\item this number plus~\(1\) is the number of sections in the first
  section (including itself) of the sequence;

\item finally we call recursively \texttt{count} on the rest of the
  sequence (that is, the remaining sections) and add this number to
  the previous one: the total is the result.

\end{enumerate}
Note that the two recursive calls can be interchanged and the case
described (call first on the children of the first node, then on the
following siblings), is \emph{depth\hyp{}first} traversal of the tree,
which we write first:
\XSLTinUnchecked{count1.xsl}
\noindent Beware of selecting \texttt{\$sections[1]/section}, but not
\texttt{sections[1]/section}, which is empty, because \texttt{section}
is a child element of the context node, whereas \texttt{\$section} is
the content of the variable \texttt{section}. Perhaps it is wise to
avoid using variables which are also element names in the input. Note
also that because we must call named templates at the \XSLT level, not
in \XPath, we have to define variables \texttt{subsec} and
\texttt{subseq} to hold the results of the two recursive calls. Had we
use \XSLT functions (\texttt{xsl:function}), we would have called them
in \XPath. For the sake of uniformity, let us stick to named
templates, even though, in some contexts, they may add verbosity to an
already verbose language.

If we want to visit the siblings before the children, we just need to
swap the declarations of the variables:
\begin{verbatim}
      ...
        <xsl:variable name="subseq" as="xs:integer">
          ...
        </xsl:variable>
        <xsl:variable name="subsec" as="xs:integer">
          ...
        </xsl:variable>
        ...
\end{verbatim}
This makes no difference because the parts of the tree traversed by
the two calls are complementary. It is nevertheless instructive to
draw the \XML tree and follow the (descending) calls with one colour
on the left side of the nodes, and the (ascending) results with
another colour on the right side.

Instead of counting the sections in a bottom\hyp{}up fashion, we can
thread a counter during our traversal and increment it each time we
encounter a section; the final result is then the current count when
we are back at the root. (The counter is a kind of
\emph{accumulator}.) We have
\XSLTinUnchecked{count3.xsl}
\noindent Note how the case of the empty sequence \texttt{\$sections}
yields the current count instead of~\texttt{0}, as opposed to
previous versions.

For the sake of practising with the \XSLT syntax, we might remark that
the variable \texttt{subsec} is only used to initialise the parameter
\texttt{current} of second recursive call. We could avoid creating that
variable if we expand its recursive call as a child of the parameter
in question:
\begin{alltt}
  <xsl:template name="count" as="xs:integer">
    <xsl:param name="sections" as="element(section)*"/>
    <xsl:param name="current"  as="xs:integer" \textbf{select="0"}/>
    <xsl:choose>
      <xsl:when test="empty(\$sections)">
        <xsl:sequence select="\$current"/>
      </xsl:when>
      <xsl:otherwise>
        <xsl:call-template name="count">
          <xsl:with-param name="sections"
                          select="\$sections[position()>1]"/>
          \textbf{<xsl:with-param name="current" as="xs:integer">
            <xsl:call-template name="count">
              <xsl:with-param name="sections"
                              select="\$sections[1]/section"/>
              <xsl:with-param name="current"
                              select="\$current + 1"/>
            </xsl:call-template>
          </xsl:with-param>}
        </xsl:call-template>
      </xsl:otherwise>
    </xsl:choose>
  </xsl:template>
\end{alltt}
Note the use of a default value for parameter \texttt{current}, thus
avoiding its initialisation in the first call (in the template
matching element \texttt{book}).

\mypar{Summing}

The purpose of this exercise is to write an \XSLT transform which
takes as input a document made of one kind of element with one kind of
attribute whose value is a positive integer, and computes the sum of
all these numbers. More precisely, we think of the following \DTD:
\DTDin{numbers_tree.dtd}
\noindent and, for instance, of the following input:
\XMLinUnchecked{numbers_tree.xml}
\noindent The expected result is then
\saxon{tmp.txt}{numbers_tree}{sum2}
The key is to understand the difference between this exercise and the
exercise where we had to count the number of sections in a table of
contents. In the latter, we counted~\texttt{1} for each section. In
the former, we simply take the value of the attribute \texttt{val}
instead of~\texttt{1}:
\XSLTinUnchecked{sum2.xsl}

\mypar{Mirroring}

The purpose of this exercise is to write an \XSLT transform which
takes as input a table of contents with sections only and output the
same table in \XML where the sections have been reversed, level by
level, which means that the result tree is the image of the input tree
in a mirror. We already defined an abstract function \fun{mir/1} doing
exactly that appears in \fig~\vref{fig:mirror}:
\begin{equation*}
\fun{mir}(\fun{ext}()) \rightarrow \fun{ext}();
\quad
\fun{mir}(\fun{int}(x,t_1,t_2)) \rightarrow
\fun{int}(x,\fun{mir}(t_2),\fun{mir}(t_1)).
\end{equation*}
\noindent The \DTD we have in mind have in mind here is the following:
\DTDin{book_simple.dtd}
\noindent An example of valid input is the following table of contents:
\XMLinUnchecked{toc_simple.xml}
\noindent Note that each section title has been numbered in order to
better understand the corresponding output:
\XMLinUnchecked{toc_simple_out.xml}

The difference with the function \fun{mir/1} is that \XML trees are
unranked and there are no external nodes. The case
\(\fun{mir}(\fun{ext}())\) corresponds to empty sequence of
subsections and its right\hyp{}hand side \(\fun{ext}()\) translates
then as an empty sequence as well, which means that the structure of
the named template is
\begin{verbatim}
  <xsl:template name="mir" as="element(section)*">
    <xsl:param name="sections" as="element(section)*"/>
    <xsl:if test="not(empty($sections))">
      ...
    </xsl:if>
  </xsl:template>
\end{verbatim}
This is a typical use\hyp{}case for \texttt{xsl:if}, instead of the
general \texttt{xsl:choose}. Next, we focus on the second rewrite
rule, whose right\hyp{}hand side is
\(\fun{int}(x,\fun{mir}(t_2),\fun{mir}(t_1))\). In \XSLT, the
parameter is a \emph{sequence} of sections, that is, a forest, because
we are dealing with unranked trees, so the children of the root make
up a forest, not a pair \((t_1, t_2)\) like in binary
trees. Therefore, we need to generalise the mirroring to a stack. If
we simply reverse it, this is not good because the children need
reversing too, and so the grand\hyp{}children etc. In other words, we
need to traverse the whole tree, thus we should expect to perform two
recursive calls: one horizontally (to process the current level
\texttt{\$sections}), and one vertically (to process the children of a
node in the current level, usually the first).

The previous canvas then should be filled like so:
\begin{verbatim}
  <xsl:template name="mir" as="element(section)*">
    <xsl:param name="sections" as="element(section)*"/>
    <xsl:if test="not(empty($sections))">
      <xsl:call-template name="mir">
        <xsl:with-param name="sections"
                        select="$sections[position()>1]"/>
      </xsl:call-template>
      <section>
        <xsl:sequence select="$sections[1]/@title"/>
        <xsl:call-template name="mir">
          <xsl:with-param name="sections"
                          select="$sections[1]/section"/>
        </xsl:call-template>
      </section>
    </xsl:if>
  </xsl:template>
\end{verbatim}
This template can be conceived as interleaving the reversal and the
recursive mirroring of the children of the root. Note how
\texttt{<xsl:sequence select="\$sections[1]/@title"/>} is needed to
rebuild the attribute of the mirrored image
\texttt{<section>...</section>} of the first section. Recall that
attribute nodes must be defined \emph{before} the other kinds of nodes
amongst the children (see page~\pageref{attr_before}), that is,
immediately after the opening tag \texttt{<section>}. The complete
transform is \XSLTinUnchecked{flip1.xsl}
\noindent Again, we have an illustration of the necessity of the
attribute setting \texttt{exclude-result-prefixes="xs"}, or else the
rebuilt section would needlessly inherit the namespace~\texttt{xs}.

Now, let us answer the same question when the table of contents
contains chapters, which in turn contain sections, and we do not want
the chapters to be reversed, only the sections. More precisely, the
\DTD we want is \texttt{book\_deep.dtd}:
\DTDin{book_deep.dtd}
\noindent and a valid input would be
\XMLinUnchecked{toc_chap.xml}
\noindent We want the result
\XMLinUnchecked{toc_chap_out.xml}
\noindent We mentioned that we had to write
\begin{verbatim}
      <section>
        <xsl:sequence select="$sections[1]/@title"/>
        ...
      </section>
\end{verbatim}
to copy the attribute of the first section. Instead, we would like to
write \texttt{<section title="\$sections[1]/@title">}, but the
attribute value is then considered as plain text, not as a
selection. Therefore, the problem boils down to performing a selection
in an attribute which is neither \texttt{test} nor
\texttt{select}. The answer lies with an \XPath operator
\texttt{\{...\}}, which means `Consider the text between braces as
\XPath, not plain text.' In other words, we could write
\begin{alltt}
      <section title="\textbf{\{}\$sections[1]/@title\textbf{\}}">
        ...
      </section>
\end{alltt}
Clearly, we do not need to rewrite the template named \texttt{mir}
because sections are to be processed in the same way as before,
although it may be interesting to use this new \XPath operator for the
sake of learning. Other than that, all we need is a new named template
to handle the chapters by reconstructing them \emph{in the same
  order}, but with mirrored \texttt{section} children (if any). This
means that we can reuse the same structure as \texttt{mir}, but
without the reversal:
\begin{verbatim}
  <xsl:template match="book" as="element(book)">
    <xsl:copy>
      <xsl:sequence select="@title"/>
      <xsl:sequence select="author"/>
      <xsl:call-template name="mk_chap">
        <xsl:with-param name="chapters" select="chapter"/>
      </xsl:call-template>
    </xsl:copy>
  </xsl:template>

  <xsl:template name="mk_chap" as="element(chapter)*">
    <xsl:param name="chapters" as="element(chapter)*"/>
    <xsl:if test="not(empty($chapters))">
      <chapter title="{$chapters[1]/@title}">
        <xsl:call-template name="mir">
          <xsl:with-param name="sections"
                          select="$chapters[1]/section"/>
        </xsl:call-template>
      </chapter>
      <xsl:call-template name="mk_chap">
        <xsl:with-param name="chapters"
                        select="$chapters[position()>1]"/>
      </xsl:call-template>
    </xsl:if>
  </xsl:template>
\end{verbatim}
Note that the \texttt{xsl:copy} element admit no \texttt{select}
attribute: it only performs a shallow copy of the context node. Here
it is clear that the context node is \texttt{book}, because
\texttt{xsl:copy} is a child of the template matching
\texttt{book}. But what if it is in a named template? How do we know
the context node there, since we are not in a matching template? The
answer is that the context node is the last matched node in the
control flow up to the present instruction. For instance, in the
template named \texttt{mir}, the context node is the root element
\texttt{book}.

Because the order of the chapters must be left unchanged, it is
interesting to use a template matching \texttt{chapter} to process
them and call it with \texttt{<xsl:apply-templates select="chapter"/>},
instead of using the bulky template named \texttt{mk\_chap}. It means:
\begin{enumerate*}

\item select the elements \texttt{chapter} which are children of the
  context node;

\item for each element in the resulting sequence, in parallel, apply
  the first template in the stylesheet which matches \texttt{chapter};

\item when finished, gather all the results in one sequence, in the
  same order as the original chapters.

\end{enumerate*}
As we saw \vpageref{match_as_map}, a matching template is like a map,
the parallel application of a template to the items of a sequence. In
other words, when \emph{parallel} processing of elements is envisaged,
we use \texttt{xsl:apply-templates}, otherwise \emph{sequential}
processing is chosen, that is \texttt{xsl:call-template}. (Please keep
in mind that `parallel' does not imply that an implementation of an
\XSLT processor must be multi\hyp{}threaded, only that it could
be. The function \texttt{map} in \Erlang is clearly sequential, for
instance, although it could be programmed using concurrent, even
distributed, processes.) We have to rewrite the template matching
\texttt{book} and the template named \texttt{mk\_chap}, which becomes
a template matching \texttt{chapter}:
\begin{verbatim}
  <xsl:template match="book" as="element(book)">
    <xsl:copy>
      <xsl:attribute name="title" select="@title"/>
      <xsl:sequence select="author"/>
      <xsl:apply-templates select="chapter"/>
    </xsl:copy>
  </xsl:template>

  <xsl:template match="chapter" as="element(chapter)">
    <xsl:copy>
      <xsl:attribute name="title" select="@title"/>
      <xsl:call-template name="mir">
        <xsl:with-param name="sections" select="section"/>
      </xsl:call-template>
    </xsl:copy>
  </xsl:template>
\end{verbatim}
Note how the structure of the new template does not mimic anymore that
of the template named \texttt{mir}, thus is shorter. Also, we
introduced a new \XSLT element:
\begin{alltt}
 <xsl:template match="book" as="element(book)">
    <xsl:copy>
      \textbf{<xsl:attribute name="title" select="@title"/>}
      ...
\end{alltt}
This is an alternative to using \texttt{xsl:sequence} as before. Also,
we have now an element \texttt{xsl:copy} per matching template, the
context node being \texttt{book} in one case, and \texttt{chapter}, in
the other.

Comparing the contents of the template matching chapters with the
following element in the template named \texttt{mir},
\begin{verbatim}
      ...
      <section title="{$sections[1]/@title}">
        <xsl:call-template name="flip">
        <xsl:with-param name="sections"
                        select="$sections[1]/section"/>
        </xsl:call-template>
      </section>
      ...
\end{verbatim}
it becomes apparent that both actions are the same: make a shallow
copy of an element and mirror its children. Therefore it would be
advantageous if the template matching chapters also matched
sections. Because we used \texttt{xsl:copy} and
\texttt{xsl:attribute}, it becomes possible to have a common template
matching chapters and sections: \texttt{<xsl:template
  match="chapter|section">}, whose interpretation is as follows:
`Match either a \texttt{chapter} or a \texttt{section}.'

Here is the difference with the previous answer:
\begin{alltt}
  <xsl:template match="chapter|section" as="element()*">
    <xsl:copy>
      <xsl:attribute name="title" select="@title"/>
      <xsl:call-template name="mir">
        <xsl:with-param name="sections" select="section"/>
      </xsl:call-template>
    </xsl:copy>
  </xsl:template>

  <xsl:template name="mir" as="element(section)*">
    <xsl:param name="sections" as="element(section)*"/>
    <xsl:if test="not(empty(\$sections))">
      <xsl:call-template name="mir">
        <xsl:with-param name="sections"
                        select="\$sections[position()>1]"/>
      </xsl:call-template>
      \textbf{<xsl:apply-templates select="\$sections[1]"/>}
    </xsl:if>
  </xsl:template>
\end{alltt}
Note how we have to apply templates to the first section in
\texttt{mir} (see code in bold), instead of calling recursively
\texttt{mir} (this call is now done in the template matching chapters
and sections). Since the template applies to only one section,
parallelism is lost, but code sharing is gained nonetheless.

The elements \texttt{xsl:call-template} and
\texttt{xsl:apply-templates} differ also is that the former always
results in a call while the latter may be a non\hyp{}operation if the
\texttt{select} attribute evaluates to an empty sequence. In other
words, \texttt{<xsl:apply-templates select="..."/>} does nothing if the
value of \texttt{"..."} is the empty sequence, whereas
\texttt{<xsl:call-template name="t">} always calls the template named~\texttt{t}, even if the parameters are empty sequences.

It is possible for a matching template to have parameters. Just put
some \texttt{xsl:param} elements just after \texttt{<xsl:template
  match="...">} (this is the definition) and \texttt{xsl:with-param}
just after \texttt{xsl:apply-templates} (this is the
application). This is the same syntax as
\texttt{xsl:call-template}.

Let us then change the call to template \texttt{mir} into a template
application with a parameter and remove the definition of \texttt{mir}
entirely. The shortest transform to achieve the same effects as the
previous ones is
\XSLTinUnchecked{flip4.xsl}

\mypar{Height}

The purpose of this exercise is to write an \XSLT transform which
takes as input a table of contents and outputs its height.
\begin{itemize}

  \item The height of a table of contents is the largest height of its
  chapters.

  \item The height of a chapter (respectively, section) is~\texttt{1}
    plus the largest height of its sections (respectively,
    subsections).

  \item The height of an empty sequence is~\texttt{0}.

\end{itemize}
For instance, a book with no chapters has height~\texttt{0} (it is
empty). A book made only of chapters with no sections at all has
height~\texttt{1} (it is flat). We will use the same \DTD as in the
previous exercise:
\DTDin{book_deep.dtd}
\noindent The same input
\XMLinUnchecked{toc_chap.xml}
\noindent yields the result
\saxon{tmp.txt}{toc_chap}{height1}

The above definition is a parallel algorithm, because the heights of
the chapters and sections can be computed separately. Therefore, let
us write the transform using matching templates only and we reuse the
template named \texttt{max} for finding the maximum of two integers.
\XSLTinUnchecked{height1.xsl}

Same question but this time, instead of computing in parallel de
heights of the children of a given node, let us compute them
sequentially with a named template. The purpose is to avoid computing
a sequence of heights and then taking their maximum. Instead, we would
compute the current height along the traversal.

Two parameters are needed: a parameter \texttt{cur} representing the
height of the sequence so far (the initial value is~\texttt{0}) and a
parameter \texttt{seq} holding the rest of the sequence whose height
we want to know. Then
\begin{enumerate}

  \item we compute the height of the sequence of the children of
    \texttt{\$seq[1]};

  \item we add~\texttt{1} to obtain the height of \texttt{\$seq[1]};

  \item the maximum of this value and \texttt{\$cur} is the value of
    \texttt{cur} in the recursive call with
    \texttt{\$seq[position()>1]}. If \texttt{\$seq} is empty, the
    maximum height of the nodes is \texttt{\$cur}. (This scheme is
    similar to counting the number of sections.)

\end{enumerate}
\noindent This is written in \XSLT as follows:
\XSLTinUnchecked{height2.xsl}

In the previous question, the height is computed
\emph{bottom\hyp{}up}, that is, the increments on the height are
performed just before the recursive calls end and new calls initialise
the height parameter to~\texttt{0}. Instead, we can propose an
alternate design where the height is incremented \emph{top\hyp{}down},
that is, the height parameter is added~\texttt{1} just before the
recursive calls start:
\XSLTinUnchecked{height3.xsl}

\mypar{Numbering}

The purpose of this exercise is to write an \XSLT transform which
takes as input a table of contents and outputs it in \XHTML, first
without numbering chapters and sections, then numbering them. The \DTD
is still the same:
\DTDin{book_deep.dtd}
\noindent The valid input is still
\XMLinUnchecked{toc_chap.xml}
\noindent The expected result (without numbering) is then
\begin{verbatim}
<!DOCTYPE html
  PUBLIC "-//W3C//DTD XHTML 1.0 Strict//EN"
  "http://www.w3.org/TR/xhtml1/DTD/xhtml1-strict.dtd">
<html xmlns:xhtml="http://www.w3.org/1999/xhtml"
      xmlns="http://www.w3.org/1999/xhtml"
      xml:lang="en" lang="en">
  <head>
    <meta http-equiv="Content-Type"
          content="text/html; charset=UTF-8"/>
    <title>Definitive XML Schema</title>
  </head>
  <body>
    <h2>Definitive XML Schema</h2>
    <p>by Priscilla Walmsley</p>
    <h3>Table of contents</h3>
    <ul>
      <li>[I] A quick tour of XML Schema
        <ul>
          <li>[I.1] An example schema</li>
          <li>[I.2] The components of XML Schema
            <ul>
              <li>[I.2.1] Declaration vs. definition</li>
              <li>[I.2.2] Global vs. local components</li>
            </ul>
          </li>
          <li>[I.3] Elements and attributes
            <ul>
              <li>[I.3.1] The tag/type distinction</li>
            </ul>
          </li>
        </ul>
      </li>
      <li>[II] Instances and schemas
        <ul>
          <li>[II.1] Using the instance attributes</li>
          <li>[II.2] Schema processing
            <ul>
              <li>[II.2.1] Validation</li>
              <li>[II.2.2] Augmenting the instance</li>
            </ul>
          </li>
        </ul>
      </li>
    </ul>
  </body>
</html>
\end{verbatim}
which, interpreted by a web browser, would likely render as

\noindent\rule{\linewidth}{0.5pt}
\noindent\textbf{\large Definitive XML Schema}

\bigskip

\noindent by Priscilla Walmsley

\bigskip

\noindent\textbf{Table of contents}
\begin{itemize*}
  \item [I] A quick tour of XML Schema
    \begin{itemize*}
      \item [I.1] An example schema
      \item [I.2] The components of XML Schema
        \begin{itemize*}
          \item [I.2.1] Declaration vs. definition
          \item [I.2.2] Global vs. local components
        \end{itemize*}
      \item [I.3] Elements and attributes
        \begin{itemize*}
          \item [I.3.1] The tag/type distinction
        \end{itemize*}
    \end{itemize*}
  \item [II] Instances and schemas
    \begin{itemize*}
      \item [II.1] Using the instance attributes
      \item [II.2] Schema processing
        \begin{itemize*}
          \item [II.2.1] Validation
          \item [II.2.2] Augmenting the instance
        \end{itemize*}
    \end{itemize*}
\end{itemize*}
\noindent\rule{\linewidth}{0.5pt}

\bigskip

\noindent The following solution should not be difficult by now:
\XSLTinUnchecked{num1.xsl}
\noindent Perhaps it is worth noting \texttt{<xsl:value-of
  select="@title"/>}, since the titles are attribute values, so we
need \texttt{xsl:value-of} to create a text node, just like for
\texttt{"author/@first,author/@last"}, which is the same as
\texttt{"(author/@first,author/@last)"}. Also possible here woud have
been \texttt{"author/@*"}, which means `All attribute values of the
element \texttt{author}, child of the context node.'

Now, let us add a number between square brackets after the \XHTML
\texttt{<li>} tag, which is the position of the item in the list, like
so:
\begin{alltt}
<!DOCTYPE html
  PUBLIC "-//W3C//DTD XHTML 1.0 Strict//EN"
         "http://www.w3.org/TR/xhtml1/DTD/xhtml1-strict.dtd">
<html xmlns:xhtml="http://www.w3.org/1999/xhtml"
      xmlns="http://www.w3.org/1999/xhtml"
      xml:lang="en" lang="en">
  <head>
    <meta http-equiv="Content-Type"
          content="text/html; charset=UTF-8"/>
    <title>Definitive XML Schema</title>
  </head>
  <body>
    <h2>Definitive XML Schema</h2>
    <p>by Priscilla Walmsley</p>
    <h3>Table of contents</h3>
    <ul>
      <li>\textbf{[1]} [I] A quick tour of XML Schema
        <ul>
          <li>\textbf{[1]} [I.1] An example schema</li>
          <li>\textbf{[2]} [I.2] The components of XML Schema
            <ul>
              <li>\textbf{[1]} [I.2.1] Declaration vs. definition</li>
              <li>\textbf{[2]} [I.2.2] Global vs. local components</li>
            </ul>
          </li>
          <li>\textbf{[3]} [I.3] Elements and attributes
            <ul>
              <li>\textbf{[1]} [I.3.1] The tag/type distinction</li>
            </ul>
          </li>
        </ul>
      </li>
      <li>\textbf{[2]} [II] Instances and schemas
        <ul>
          <li>\textbf{[1]} [II.1] Using the instance attributes</li>
          <li>\textbf{[2]} [II.2] Schema processing
            <ul>
              <li>\textbf{[1]} [II.2.1] Validation</li>
              <li>\textbf{[2]} [II.2.2] Augmenting the instance</li>
            </ul>
          </li>
        </ul>
      </li>
    </ul>
  </body>
</html>
\end{alltt}
The added numbers have been set in a bold typeface. The only other
change lies in the template matching chapters and sections:
\begin{alltt}
 <xsl:template match="section|chapter" as="element(xhtml:li)">
    <li xmlns="http://www.w3.org/1999/xhtml">
      \textbf{<xsl:value-of select="('[',position(),'] ',@title)"
                    separator=""/>}
      <xsl:if test="not(empty(section))">
        <ul><xsl:apply-templates select="section"/></ul>
      </xsl:if>
    </li>
  </xsl:template>
\end{alltt}
The \texttt{separator} attribute must be set to the empty string, so
the items in the selection (strings and integer) are converted to one
text node without the default blank separator. For example, the result
of evaluating \texttt{<xsl:value-of select="1,2,3">} is the string
\texttt{'1 2 3'}.

Finally, we can complete the numbering so it becomes what is expected
in a table of contents. Let us resume with an input \emph{without} any
numbers:
\XMLinUnchecked{toc_deep.xml}
\noindent and, for making things a little bit easier, the output will
number the chapters with Arabic numbers, like the sections:
\begin{verbatim}
<!DOCTYPE html
  PUBLIC "-//W3C//DTD XHTML 1.0 Strict//EN"
         "http://www.w3.org/TR/xhtml1/DTD/xhtml1-strict.dtd">
<html xmlns:xhtml="http://www.w3.org/1999/xhtml"
      xmlns="http://www.w3.org/1999/xhtml"
      xml:lang="en" lang="en">
  <head><meta http-equiv="Content-Type"
              content="text/html; charset=UTF-8"/>
        <title>Definitive XML Schema</title>
  </head>
  <body>
    <h2>Definitive XML Schema</h2>
    <p>by Priscilla Walmsley</p>
    <h3>Table of contents</h3>
    <ul>
      <li>[1] A quick tour of XML Schema
        <ul>
          <li>[1.1] An example schema</li>
          <li>[1.2] The components of XML Schema
            <ul>
              <li>[1.2.1] Declarations vs. definitions</li>
              <li>[1.2.2] Global vs. local components</li>
            </ul>
          </li>
          <li>[1.3] Elements and attributes
            <ul>
              <li>[1.3.1] The tag/type distinction</li>
            </ul>
          </li>
        </ul>
      </li>
      <li>[2] Instances and schemas
        <ul>
          <li>[2.1] Using the instance attributes</li>
          <li>[2.2] Schema processing
            <ul>
              <li>[2.2.1] Validation</li>
              <li>[2.2.2] Augmenting the instance</li>
            </ul>
          </li>
        </ul>
      </li>
    </ul>
  </body>
</html>
\end{verbatim}
The idea is to add a parameter \texttt{prefix} to the template
matching chapters and sections, which receives the prefix numbering of
the parent. For instance, when matching the section entitled
`Declarations vs. definitions', the parameter value is the sequence
\texttt{(1,'.',2,'.')}, so we simply concatenate the position of the
section amongst its siblings, that is, \texttt{1}. Then we create a
text node to format \texttt{[1.2.1]}. Here is the change:
\begin{verbatim}
  <xsl:template match="chapter|section" as="element(xhtml:li)">
    <xsl:param name="prefix" as ="item()*"/>
    <xsl:variable name="current" select="($prefix,position())"/>
    <li xmlns="http://www.w3.org/1999/xhtml">
      <xsl:value-of select="('[',$current,'] ',@title)"
                    separator=""/>
      <xsl:if test="not(empty(section))">
        <ul>
          <xsl:apply-templates select="section">
             <xsl:with-param name="prefix"
                             select="($current,'.')"/>
          </xsl:apply-templates>
        </ul>
      </xsl:if>
    </li>
  </xsl:template>
\end{verbatim}
Note that the first application of this template is left unchanged:
\begin{verbatim}
        ...
        <ul><xsl:apply-templates select="chapter"/></ul>
        ...
\end{verbatim}
because, in \XSLT, an empty sequence is implicitly passed, which is
here convenient.


\mypar{Sorting leaves}

The purpose of this exercise is to write an \XSLT transform which
takes as input a document representing a binary tree whose leaves
contain an integer and sort these in nondecreasing order. The integers
in the sorted sequence must be separated by commas and terminated by a
period in the resulting text. For instance, the following \XML
document
\XMLinUnchecked{num.xml}
\noindent yields the following output: \saxon{tmp.txt}{num}{tmerge}
The format of the output should remind us of the comma-separated
values (CSV), \vpageref{par:CSV}, and the ordering of the merging of
ordered sequences, \vpageref{par:merging}. Then, a first idea could be
to traverse the tree and collect the numbers in ordered sequences
which are merged together with the template named \texttt{merge} until
one remains and, finally, we use the template named \texttt{csv}. More
precisely, this traversal can be performed in parallel: the recursive
template applications on the children yield two ordered sequences,
which are merged; if the context node is the root element, then we
call \texttt{csv}. In other words, the mergers are performed purely in
a bottom\hyp{}up fashion (that is, after the end of the recursive
calls). Therefore, we start with
\begin{alltt}
  <xsl:template match="/" as="text()*">
    <xsl:call-template name="csv">
      <!-- The following cast is needed. -->
      <xsl:with-param name="items" \textbf{as="xs:integer*"}>
        <xsl:apply-templates select="num"/>
      </xsl:with-param>
    </xsl:call-template>
    <xsl:text>&#10;</xsl:text>
  </xsl:template>
\end{alltt}
Note how the type annotation \texttt{xs:integer*} is necessary when
invoking the template \texttt{csv}, whose type is
\begin{verbatim}
  <xsl:template name="csv" as="text()*">
    <xsl:param name="items" as="item()*"/>
    ...
  </xsl:template>
\end{verbatim}
The rest is
\begin{verbatim}
  <xsl:template match="num" as="xs:integer*">
    <xsl:choose>
      <xsl:when test="empty(@val)">
        <xsl:call-template name="merge">
          <xsl:with-param name="fst" as="xs:integer*">
            <xsl:apply-templates select="num[1]"/>
          </xsl:with-param>
          <xsl:with-param name="snd" as="xs:integer*">
            <xsl:apply-templates select="num[2]"/>
          </xsl:with-param>
        </xsl:call-template>
      </xsl:when>
      <xsl:otherwise>
        <xsl:value-of select="@val"/>
      </xsl:otherwise>
    </xsl:choose>
  </xsl:template>
\end{verbatim}
The template \texttt{merge} needs to be simplified and generalised
because it was too specialised:
\begin{verbatim}
  <xsl:template name="merge" as="element(item)*">
    <xsl:param name="seq1" as="element(item)*"/>
    <xsl:param name="seq2" as="element(item)*"/>
    ...
  </xsl:template>
\end{verbatim}
We need it to receive integers now:
\begin{verbatim}
  <xsl:template name="merge" as="xs:integer*">
    <xsl:param name="fst" as="xs:integer*"/>
    <xsl:param name="snd" as="xs:integer*"/>
    <xsl:choose>
      <xsl:when test="empty($fst)">
        <xsl:sequence select="$snd"/>
      </xsl:when>
      <xsl:when test="empty($snd)">
        <xsl:sequence select="$fst"/>
      </xsl:when>
      <xsl:when test="$fst[1] lt $snd[1]">
        <xsl:sequence select="$fst[1]"/>
        <xsl:call-template name="merge">
          <xsl:with-param name="fst"
                          select="$fst[position()>1]"/>
          <xsl:with-param name="snd" select="$snd"/>
        </xsl:call-template>
      </xsl:when>
      <xsl:otherwise>
        <xsl:sequence select="$snd[1]"/>
        <xsl:call-template name="merge">
          <xsl:with-param name="fst" select="$fst"/>
          <xsl:with-param name="snd"
                          select="$snd[position()>1]"/>
        </xsl:call-template>
      </xsl:otherwise>
    </xsl:choose>
  </xsl:template>
\end{verbatim}

\paragraph{Exercise}

Is example~2.2 in the book of~\cite{Mangano_2006}, page~39, really in
tail form?
