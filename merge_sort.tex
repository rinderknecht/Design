\chapter{Merge Sort}
\label{chap:merge_sort}
\index{merge sort|(}
\index{sorting|see{merge sort}}

\cite{Knuth_1996} reports that the first computer program, designed
in~\oldstylenums{1945} by the mathematician John von~Neumann, was a
sorting algorithm, nowadays called \emph{merge sort}. It is amongst
the most widely taught sorting algorithms because it illustrates the
important solving strategy known as `\emph{divide and
  conquer}'\index{divide and conquer}: the input is split, each
non\hyp{}trivial part is recursively processed and the partial
solutions are finally combined to form the complete solution.  While
merge sort is not difficult to program, finding its cost requires
advanced mathematical knowledge. Most
textbooks \citep{GrahamKnuthPatashnik_1994,CLRS_2009} show how to find
the order of growth of an upper bound of the cost (expressed by means
of Bachmann's notation \(\mathcal{O}\)) from recurrences it satisfies,
but the general case is often not presented in the main chapters, or
not at all, because a precise asymptotic solution requires skills in
\emph{analytic combinatorics} \citep{FlajoletSedgewick_2001,
  FlajoletSedgewick_2009, FlajoletGolin_1994, Hwang_1998,
  ChenHwangChen_1999}. Moreover, there are several variants of merge
sort \citep{Knuth_1998,GolinSedgewick_1993} and often the only one
introduced, called \emph{top\hyp{}down}, is illustrated on arrays. We
show in this chapter that stacks, as a purely functional data
structure \citep{Okasaki_1998b}, are suitable both for a top\hyp{}down
and a \emph{bottom\hyp{}up} approach of merge
sort \citep{PannyProdinger_1995}.


\section{Merging}
\label{sec:merging}
\index{merge sort!merging|(}

John von~Neumann did not actually described merge sort, but its basic
operation, \emph{merging}, which he named \emph{meshing}. Merging
consists in combining two ordered stacks of keys into one ordered
stack. Without loss of generality, we shall be only interested in
sorting keys in increasing order. For instance, merging \([10,12,17]\)
and \([13,14,16]\) results in \([10,12,13,14,16,17]\). One way to
achieve this consists in comparing the two smallest keys, output the
smallest and repeat the procedure until one of the stacks becomes
empty, in which case the other is wholly appended. We have (compared
keys underlined):
\begin{equation*}
\left\{
\begin{aligned}
&\underline{10}~12~17\\
&\underline{13}~14~16
\end{aligned}
\right.
\rightarrow 10
\left\{
\begin{aligned}
&\underline{12}~17\\
&\underline{13}~14~16
\end{aligned}
\right.
\rightarrow 10~12
\left\{
\begin{aligned}
&\underline{17}\\
&\underline{13}~14~16
\end{aligned}
\right.
\rightarrow 10~12~13
\left\{
\begin{aligned}
&\underline{17}\\
&\underline{14}~16
\end{aligned}
\right.
%\quad\text{etc.}
\end{equation*}
The function \fun{mrg/2}\index{mrg@\fun{mrg/2}} (\emph{merge}) in
\fig~\vref{fig:mrg}
\begin{figure}
\begin{equation*}
\boxed{%
\begin{array}{r@{\;}l@{\;}l}
\fun{mrg}(\el,t)         & \xrightarrow{\smash{\theta}} & t;\\
\fun{mrg}(s,\el)         & \xrightarrow{\smash{\iota}} & s;\\
\fun{mrg}(\cons{x}{s},\cons{y}{t}) & \xrightarrow{\smash{\kappa}}
& \cons{y}{\fun{mrg}(\cons{x}{s},t)},\;\text{if \(x \succ y\)};\\
\fun{mrg}(\cons{x}{s},t) & \xrightarrow{\smash{\lambda}}
                         & \cons{x}{\fun{mrg}(s,t)}.
\end{array}}
\end{equation*}
\caption{Merging two stacks}
\label{fig:mrg}
\index{merge sort!merging!program}
\end{figure}
implements this scheme. Rule~\(\iota\) is not necessary but is
retained because it allows the cost to be symmetric, just as
\fun{mrg/2}\index{mrg@\fun{mrg/2}} is: \(\C{\fun{mrg}}{m,n} =
\C{\fun{mrg}}{n,m}\)\index{mrg@$\C{\fun{mrg}}{m,n}$} and
\(\fun{mrg}(s,t) \equiv \fun{mrg}(t,s)\), where \(m\)~and~\(n\) are
the lengths of~\(s\) and~\(t\). This property enables easier cost
calculations and faster computations. Note that in the definition of
\fun{cat/2}\index{cat@\fun{cat/2}} (equation~\eqref{def:cat} on
page~\pageref{def:cat}), we do not include a similar rule,
\(\fun{cat}(s,\el) \rightarrow s\), because, despite the gain in
speed, the function is asymmetric and cost calculations are simplified
when using \(\C{\fun{cat}}{n}\)\index{cat@$\C{\fun{cat}}{n}$} rather
than \(\C{\fun{cat}}{m,n}\). \Fig~\vref{fig:mrg_247} shows a
\begin{figure}[b]
\begin{equation*}
\boxed{%
\begin{array}{r@{\;}l@{\;}l}
  \fun{mrg}([3,4,7],[1,2,5,6])
& \xrightarrow{\smash{\kappa}}
& \cons{1}{\fun{mrg}([3,4,7],[2,5,6])}\\
& \xrightarrow{\smash{\kappa}}
& \cons{1,2}{\fun{mrg}([3,4,7],[5,6])}\\
& \xrightarrow{\smash{\lambda}}
& \cons{1,2,3}{\fun{mrg}([4,7],[5,6])}\\
& \xrightarrow{\smash{\lambda}}
& \cons{1,2,3,4}{\fun{mrg}([7],[5,6])}\\
& \xrightarrow{\smash{\kappa}}
& \cons{1,2,3,4,5}{\fun{mrg}([7],[6])}\\
& \xrightarrow{\smash{\kappa}}
& \cons{1,2,3,4,5,6}{\fun{mrg}([7],\el)}\\
& \xrightarrow{\smash{\iota}}
& [1,2,3,4,5,6,7].
\end{array}}
\end{equation*}
\caption{\(\fun{mrg}([3,4,7],[1,2,5,6]) \twoheadrightarrow
  [1,2,3,4,5,6,7]\)}
\label{fig:mrg_247}
\index{merge sort!merging!example}
\index{functional language!evaluation!trace}
\end{figure}
trace for \fun{mrg/2}. Rules \(\kappa\)~and~\(\lambda\) involve a
comparison, while \(\theta\)~and~\(\iota\) do not and end the
evaluations; therefore, if
\(\OC{\fun{mrg}}{m,n}\)\index{mrg@$\OC{\fun{mrg}}{m,n}$} is the number
of comparisons to merge with \fun{mrg/2}\index{mrg@\fun{mrg/2}} two
stacks of lengths \(m\)~and~\(n\), we have\index{mrg@$\C{\fun{mrg}}{m,n}$}
\begin{equation}
\C{\fun{mrg}}{m,n} = \OC{\fun{mrg}}{m,n} + 1.\label{def:cost_mrg}
\end{equation}
In order to gain some generality, we shall study
\(\OC{\fun{mrg}}{m,n}\). Graphically, we represent a key from one
stack as a \emph{white node} \((\circ)\) and a key from the other as a
\emph{black node} \((\bullet)\). Nodes of these kinds are printed in a
horizontal line, the leftmost being the smallest. Comparisons are
always performed between black and white nodes and are represented as
\emph{edges} in% \fig~\vref{fig:merged}.
%\begin{figure}
%\centering
\begin{equation}
\includegraphics[bb=73 710 292 718]{merged}%[bb=73 702 292 724]
\label{fig:merged}
\end{equation}
%\caption{Two ordered stacks merged into one\label{fig:merged}}
%\end{figure}
An incoming arrow means that the node is smaller than the other end of
the edge, so all edges point leftwards and the number of comparisons
is the number of nodes with an incoming edge.

%\addcontentsline{toc}{subsection}{Cost}
\paragraph{Minimum cost}
\label{merge:best_case}
\index{merge sort!merging!minimum cost|(}

There are two consecutive white nodes without any edges at the right
end, which suggests that the more keys from one stack we have at the
end of the result, the fewer comparisons we needed for merging: the
minimum number is achieved when \emph{the shorter stack comes first in
  the result}. Consider the following example (the number of
comparisons is the number of black nodes):
\begin{center}
\includegraphics[bb=73 698 292 724]{min_mrg}
\end{center}
The minimum number of comparisons
\(\OB{\fun{mrg}}{m,n}\)\index{mrg@$\OB{\fun{mrg}}{m,n}$} when merging
stacks of size \(m\)~and~\(n\) is
\begin{equation}
\OB{\fun{mrg}}{m,n} = \min\{m,n\}.\label{eq:best_merge}
\end{equation}
\index{merge sort!merging!minimum cost|)}

\paragraph{Maximum cost}
\index{merge sort!merging!maximum cost|(}

We can see that we can increase the number of comparisons with respect
to \(m+n\) by removing, in~\eqref{fig:merged}, those rightmost
nodes in the result \emph{that are not compared}, as can be seen in
\begin{center}
\includegraphics[bb=73 702 258 724]{max_mrg2}
\end{center}
This maximises comparisons because all nodes, but the last, are the
destination of an edge. The maximum number of comparisons
\(\OW{\fun{mrg}}{m,n}\)\index{mrg@$\OW{\fun{mrg}}{m,n}$} is
\begin{equation}
\OW{\fun{mrg}}{m,n} = m + n - 1.\label{eq:worst_merge}
\end{equation}
Interchanging the two rightmost nodes in the previous example leaves
\(m+n-1\) invariant:
\begin{center}
\includegraphics[bb=71 700 260 726]{max_mrg1}
\end{center}
so the maximum number of comparisons occurs when \emph{the last two
  keys of the result come from two stacks.}
\index{merge sort!merging!maximum cost|)}


\paragraph{Average cost}
\index{merge sort!merging!average cost|(}

Let us seek the average number of comparisons in all distinct mergers
of two stacks of lengths \(m\)~and~\(n\). Consider
\fig~\vref{fig:mean_mrg1},
\begin{figure}[b]
\centering
\includegraphics[bb=66 648 350 725]{mean_mrg1}
\caption{All possible mergers with \(m=3\) (\(\circ\)) and \(n=2\)
  (\(\bullet\))}
\label{fig:mean_mrg1}
\end{figure}
with \(m=3\) white nodes and \(n=2\) black nodes which are interleaved
in all possible manners. Note how the figure is structured. The first
column lists the configurations where the rightmost black node is the
last of the result. The second column lists the cases where the
rightmost black node is the penultimate node of the result. The third
column is divided in two groups itself, the first of which lists the
cases where the rightmost black node is the antepenultimate. The total
number of comparisons is~\(35\) and the number of configurations
is~\(10\), thus the average number of comparisons is \(35/10 =
7/2\).\label{seven_two} Let us devise a method to find this ratio for
any \(m\)~and~\(n\). First, the number of configurations: how many
ways are there to combine \(m\)~white nodes and \(n\)~black nodes?
This is the same as asking how many ways there are to paint in black
\(n\)~nodes picked amongst \(m+n\) white nodes. More abstractly, this
is equivalent to wonder how many ways there are to choose
\(n\)~objects amongst \(m+n\). This number is called a \emph{binomial
  coefficient}\index{binomial coefficient} and noted
\(\binom{m+n}{n}\). For example, let us consider the set
\(\{a,b,c,d,e\}\) and the \emph{combinations}\index{combination} of
\(3\)~objects taken from it are
\begin{gather*}
\{a,b,c\},\{a,b,d\},\{a,b,e\},\{a,c,d\},\{a,c,e\},\{a,d,e\},\\
\{b,c,d\},\{b,c,e\},\{b,d,e\},\\
\{c,d,e\}.
\end{gather*}
This enumeration establishes that \(\binom{5}{3} = 10\). Notice that
we use mathematical sets, therefore the order of the elements or their
repetition are not meaningful. It is not difficult to count the
combinations if we recall how we counted the permutations,
\vpageref{par:permutations}. Let us determine \(\binom{r}{k}\). We can
pick the first object amongst~\(r\), the second amongst \(r-1\)
etc. until we pick the \(r\)th object amongst \(r-k+1\), so there are
\(r(r-1)\dots(r-k+1)\) choices. But these arrangements contain
duplicates, for example, we may form \(\{a,b,c\}\) and \(\{b,a,c\}\),
which are to be considered identical combinations because order does
not matter. Therefore, we must divide the number we just obtained by
the number of redundant arrangements, which is the number of
permutations of \(k\)~objects, that is, \(k!\). In the end:
\begin{equation*}
\binom{r}{k} := \frac{r(r-1)\ldots(r-k+1)}{k!} =
\frac{r!}{k!(r-k)!}.
\end{equation*}
We can check now that in \fig~\vref{fig:mean_mrg1}, we must have
\(10\)~cases: \(\binom{5}{2} = 5!/(2!3!) = 10\). The symmetry of the
problem means that merging a stack of \(m\)~keys with a stack of~\(n\)
leads to exactly the same results as merging a stack of \(n\)~keys
with a stack of \(m\)~keys:
\begin{equation*}
\binom{m+n}{n} = \binom{m+n}{m}.
%\label{eq:comb_m_n}
\end{equation*}
This can also be easily proved by means of the definition:
\begin{equation*}
\binom{m+n}{n} := \frac{(m+n)!}{n!(m+n-n)!} =
\frac{(m+n)!}{m!n!} =: \binom{m+n}{m}.
\end{equation*}
The total number \(K(m,n)\) of comparisons needed to merge~\(m\) and
\(n\)~keys in all possible manners with our algorithm is the number of
nodes with incoming edges. Let \(\overline{K}(m,n)\) be the total
number of nodes \emph{without} incoming edges, circled in
\fig~\ref{fig:mean_mrg2}.
\begin{figure}[b]
\centering
\includegraphics[bb=66 625 252 730]{mean_mrg2}
\caption{Counting vertically}
\label{fig:mean_mrg2}
\end{figure}
This figure has been obtained by moving the third column of
\fig~\vref{fig:mean_mrg1} below the second column and by removing the
edges. Since, for each merger, there are \(m+n\) nodes and each has an
incoming edge or not, and because there are \(\binom{m+n}{n}\)
mergers, we have
\begin{equation}
K(m,n) + \overline{K}(m,n) = (m + n) \binom{m+n}{n}.
\label{eq:KoverK}
\end{equation}
It is simple to characterise the circled nodes: they make up the
longest, rightmost contiguous series of nodes of the same
colour. Since there are only two colours, the problem of determining
the total number \(W(m,n)\) of white circled nodes is symmetric to the
determination of the total number \(B(m,n)\) of black circled nodes,
that is,
\begin{equation*}
B(m,n) = W(n,m).
\end{equation*}
Therefore,
\begin{equation}
\overline{K}(m,n) = W(m,n) + B(m,n) = W(m,n) + W(n,m).
\label{eq:K}
\end{equation}
From equations~\eqref{eq:KoverK} and~\eqref{eq:K}, we draw
\begin{equation}
K(m,n) = (m + n) \binom{m+n}{n} - W(m,n) - W(n,m).
\label{eq:K_temp}
\end{equation}
We can decompose \(W(m,n)\) by counting the circled white nodes
\emph{vertically}. In \fig~\ref{fig:mean_mrg2}, \(W(3,2)\) is the sum
of the numbers of mergers with at least one, two and three ending
circled white nodes: \(W(3,2) = 1 + 3 + 6 = 10\). The first column
yields \(B(3,2) = 1 + 4 = 5\). In general, the number of mergers with
one ending circled white node is the number of ways to combine
\(n\)~black nodes with \(m-1\) white nodes: \(\binom{n+m-1}{n}\). The
number of mergers with at least two ending white nodes is
\(\binom{n+m-2}{n}\), etc.
\begin{equation*}
W(m,n)
  = \binom{n+m-1}{n} + \binom{n+m-2}{n} + \dots + \binom{n+0}{n}
  = \!\sum_{j=0}^{m-1}{\!\binom{n+j}{n}}.
\end{equation*}
This sum can actually be simplified, more precisely, it has a closed
form, but in order to understand its underpinnings, we need firstly to
develop our intuition about combinations. By computing combinations
\(\binom{r}{k}\) for small values of~\(r\) and~\(k\) using the
definition, we can fill a table traditionally known as \emph{Pascal's
  triangle}\index{Pascal's triangle} and displayed in
\fig~\vref{fig:pascal_triangle}.
\begin{figure}
\centering
\includegraphics{pascal}
\caption{The corner of Pascal's triangle (in boldface type)}
\label{fig:pascal_triangle}
\end{figure}
Note how we set the convention \(\binom{r}{k} = 0\) if \(k >
r\). Pascal's triangle features many interesting properties relative
to the sum of some of its values. For instance, if we choose a number
in the triangle and look at the one on its right, then the one below
the latter is their sum. For the sake of illustration, let us extract
from \fig~\vref{fig:pascal_triangle} the lines \(r=7\) and \(r=8\):
\begin{center}
\begin{tabular}{||c|c||rrrrrrrrrr||}
\cline{5-6}\cline{8-9}
      & 7 & \textbf{1} & \textbf{7} & \multicolumn{1}{|c}{\textbf{21}} & \multicolumn{1}{c|}{\textbf{35}} &  \textbf{35} &  \multicolumn{1}{|c}{\textbf{21}} & \multicolumn{1}{c|}{\textbf{7}} &  \textbf{1} & 0 & 0\\
\cline{5-5}\cline{8-8}
      & 8 & \textbf{1} & \textbf{8} & \textbf{28} & \multicolumn{1}{|c|}{\textbf{56}} &  \textbf{70} &  \textbf{56} & \multicolumn{1}{|c|}{\textbf{28}} &  \textbf{8} & \textbf{1} & 0\\
\cline{6-6}\cline{9-9}
\end{tabular}
\end{center}
We surrounded two examples of the additive property of combinations we
discussed: \(21 + 35 = 56\) and \(21 + 7 = 28\). We would then bet
that
\begin{equation*}
\binom{r-1}{k-1} + \binom{r-1}{k} = \binom{r}{k}.
\end{equation*}
This is actually not difficult to prove if we go back to the
definition:
\begin{align*}
\binom{r}{k} &:= \frac{r!}{k!(r-k)!}
              = \frac{r}{k} \cdot \frac{(r-1)!}{(k-1)!((r-1)-(k-1))!}
              = \frac{r}{k} \binom{r-1}{k-1}.\\
\binom{r}{k} &:= \frac{r!}{k!(r-k)!}
              = \frac{r}{r-k} \cdot \frac{(r-1)!}{k!((r - 1) - k)!}
              = \frac{r}{r-k} \binom{r-1}{k}.
\end{align*}
The first equality is valid if \(k > 0\) and the second if \(r \neq
k\). We can now replace \(\binom{r-1}{k-1}\) and \(\binom{r-1}{k}\) in
terms of \(\binom{r}{k}\) in the sum
\begin{equation*}
\binom{r-1}{k-1} + \binom{r-1}{k} = \frac{k}{r}\binom{r}{k}
+ \frac{r-k}{r}\binom{r}{k} = \binom{r}{k}.
\end{equation*}
The sum is valid if \(r > 0\). There is direct proof of the formula by
enumerative combinatorics without algebra. Let us suppose that we
\emph{already} have all the subsets of \(k\)~keys chosen
among~\(r\). By definition, there are \(\binom{r}{k}\) of them. We
choose to distinguish an arbitrary key amongst~\(r\) and we want to
group the subsets in two sets: on one side, all the combinations
containing this particular key, on the other side, all the
combinations without it. The former subset has cardinal
\(\binom{r-1}{k-1}\) because its combinations are built from the fixed
key and further completed by choosing \(k-1\) remaining keys amongst
\(r-1\). The latter subset is made of \(\binom{r-1}{k}\) combinations
which are made from \(r-1\) keys, of which \(k\)~have to be selected
because we ignore the distinguished key. This yields the same additive
formula. Now, let us return to our pending sum
\begin{equation*}
W(m,n) = \sum_{j=0}^{m-1}{\binom{n+j}{n}}.
\end{equation*}

% Wrapping figure better declared before a paragraph
%
\begin{wrapfigure}[6]{l}[0pt]{0pt}
% 6 vertical lines
% left placement
% 0pt of margin overhang
\centering
\includegraphics[bb=73 650 110 721]{triangle1}%[... 721]
\end{wrapfigure}
In terms of navigation across Pascal's triangle, we understand that
this sum operates on numbers in the same column. More precisely, it
starts from the diagonal with the number \(\binom{n}{n} = 1\) and goes
down until a total of \(m\)~numbers have been added. So let us choose
a simple example and fetch two adjacent columns were the sum is
small. On the left is an excerpt for \(n=4\) (the left column is the
fifth in Pascal's triangle) and \(m=4\) (height of the left
column). Interestingly, the sum of left column, which is the sum under
study, equals the number at the bottom of the second column: \(1 + 5 +
15 + 35 = 56\). By checking other columns, we may feel justified to
think that this is a general pattern. Before attempting a general
proof, let us see how it may work on our particular example. Let us
start from the bottom of the second column, that is, \(56\), and use
the addition formula in reverse, that is, express \(56\) as the sum of
the numbers in the row above it: \(56 = 35 + 21\).  We would like to
keep~\(35\) because it is part of the equality to prove. So let us
apply the addition formula again to \(21\) and draw \(21 = 15 +
6\). Let us keep \(15\) and resume the same procedure on \(6\) so \(6
= 5 + 1\). Finally, \(1 = 1 + 0\). We just checked \(56 = 35 + (15 +
(5 + (1 + 0)))\), which is exactly what we wanted. Because we want the
number corresponding to \(35\) in our example to be
\(\binom{n+m-1}{n}\), we have the derivation
\begin{align}
\binom{n+m}{n+1}
  &= \binom{n+m-1}{n} + \binom{n+m-1}{n+1}\notag\\
  &= \binom{n+m-1}{n} + \left[\binom{n+m-2}{n} +
     \binom{n+m-2}{n+1}\right]\notag\\
  &= \binom{n+m-1}{n} + \binom{n+m-2}{n} + \dots +
     \left[\binom{n}{n} + \binom{n}{n+1}\right],\notag\\
\binom{n+m}{n+1}
  &= \sum_{j=0}^{m-1}{\binom{n+j}{n}} = W(m,n).
\label{eq:binom_sum}
\end{align}
Now, we can replace this closed form in equation~\eqref{eq:K_temp}
\vpageref{eq:K_temp} so\index{K@$K(m,n)$}
\begin{equation*}
K(m,n) = (m + n)
\binom{m+n}{n} - \binom{m+n}{n+1} - \binom{m+n}{m+1}.
\end{equation*}
By definition, the average number of comparisons
\(\OM{\fun{mrg}}{m,n}\)\index{mrg@$\OM{\fun{mrg}}{m,n}$} is the ratio
of \(K(m,n)\) by \(\binom{m+n}{n}\), therefore
\begin{equation}
\OM{\fun{mrg}}{m,n} = m + n - \frac{m}{n+1} - \frac{n}{m+1}
  = \frac{mn}{m+1} + \frac{mn}{n+1}.
\label{eq:Amrg}
\end{equation}
Since we necessarily expect \(\OB{\fun{mrg}}{m,n} \leqslant
\OM{\fun{mrg}}{m,n} \leqslant \OW{\fun{mrg}}{m,n}\), we may wonder if
and when the bounds are tight. For the upper bound to be tight, we
need \((m,n)\) to satisfy the equation \(m^2 + n^2 - mn = 1\), whose
only natural solutions are \((0,1)\), \((1,0)\) and \((1,1)\). For the
lower bound to be tight, we must have \(mn/(m+1) + mn/(n+1) =
\min\{m,n\}\), whose only natural solutions are \((0,n)\), \((m,0)\)
and \((1,1)\). Furthermore, the cases \((m,1)\) and \((1,n)\) may
suggest that merging one key with others is equivalent to inserting
that key amongst the others, as we did with straight insertion in
section~\ref{sec:straight_ins} \vpageref{sec:straight_ins}. In other
words, we expect the theorem
\begin{equation}
\fun{ins}(x,s) \equiv \fun{mrg}([x],s).
\label{eq:ins_mrg}
\index{ins@\fun{ins/2}}\index{mrg@\fun{mrg/2}}
\end{equation}
Therefore, \emph{insertion is a special case of merging.}
Nevertheless, the average costs are not exactly the same. First, we
have \(\M{\fun{mrg}}{m,n} = \OM{\fun{mrg}}{m,n} + 1\), because we need
to account for using once either rule~\(\theta\) or~\(\iota\) in
\fig~\vref{fig:mrg}, as we already acknowledged by
equation~\eqref{def:cost_mrg}. Then, equations~\eqref{eq:ins}
\vpageref{eq:ins} and~\eqref{eq:Amrg} yield
\begin{equation*}
\M{\fun{mrg}}{1,n} = \frac{1}{2}{n} + 2 - \frac{1}{n+1}
\quad\text{and}\quad
\M{\fun{ins}}{n} = \frac{1}{2}{n} + 1.
\index{ins@$\M{\fun{ins}}{n}$}\index{mrg@$\M{\fun{mrg}}{1,n}$}
\end{equation*}
Asymptotically, they are equivalent:
\begin{equation*}
\M{\fun{mrg}}{1,n} \sim \M{\fun{ins}}{n}.
\end{equation*}
But \fun{mrg/2}\index{mrg@\fun{mrg/2}} is slightly slower in average
than \fun{ins/2}\index{ins@\fun{ins/2}} in this special case:
\begin{equation*}
\M{\fun{mrg}}{1,n} - \M{\fun{ins}}{n} = 1 - \frac{1}{n+1} < 1
\quad\text{and}\quad
\M{\fun{mrg}}{1,n} - \M{\fun{ins}}{n} \sim 1.
\end{equation*}
Also, it may be interesting to see what happens when \(m=n\), that is,
when the two stacks to be merged have the same length:
\begin{equation}
\M{\fun{mrg}}{n,n} = 2n - 1 + \frac{2}{n+1} = \W{\fun{mrg}}{n,n} - 1 +
\frac{2}{n+1} \sim 2n.\index{mrg@$\M{\fun{mrg}}{n,n}$}
\label{eq:Amrg_n_n}
\end{equation}
In other words, the average cost of merging two stacks of identical
length is asymptotically the total number of keys being merged, which
is the worst case.
\index{merge sort!merging!average cost|)}

\paragraph{Termination}
\label{merging_termination}
\index{merge sort!merging!termination|(}

The termination of \fun{mrg/2}\index{mrg@\fun{mrg/2}} in
\fig~\vref{fig:mrg} is easy to prove by considering a lexicographic
order\index{induction!lexicographic order}
(page~\pageref{par:ackermann}) on pairs of stacks which are, in turn,
partially ordered by the immediate subterm
relation\index{induction!immediate subterm order}
(page~\pageref{par:well-founded}), or, more restrictively, the
\emph{immediate substack relation}\index{induction!immediate substack
  order}, that is, \(\cons{x}{s} \succ s\). The dependency
pairs\index{termination!dependency pair} of rules
\(\kappa\)~and~\(\lambda\) are ordered by \((\cons{x}{s},\cons{y}{t})
\succ (\cons{x}{s},t)\) and \((\cons{x}{s},t) \succ
(s,t)\).\index{merge sort!merging!termination|)}\index{merge
  sort!merging|)}\hfill\(\Box\)

\section{Sorting $2^n$ keys}
\label{sec:power_of_two}
\index{merge sort!power@$2^n$ keys|(}

Merging can be used to sort \emph{one} stack of keys as follows. The
initial stack of keys is split in two, then the two pieces are split
again etc. until singletons remain. These are then merged pairwise
etc. until only one stack remains, which is inductively sorted, since
a singleton stack is a sorted stack on its own and
\(\fun{mrg}(s,t)\)\index{mrg@\fun{mrg/2}} is sorted if~\(s\) and~\(t\)
are. The previous scheme leaves open the choice of a splitting
strategy and, perhaps, the most intuitive way is to cut in two halves,
which works well in the case of \(2^p\)~keys. We will see in later
sections how to deal with the general case and with a different
splitting strategy. For now, let us consider in
\fig~\vref{fig:bot_up1}
\begin{figure}
\centering
\includegraphics{bot_up1}
\caption{Sorting \([7,3,5,1,6,8,4,2]\)}
\label{fig:bot_up1}
\end{figure}
all the mergers and their relative order to sort the stack \([7, 3, 5,
  1, 6, 8, 4, 2]\). We name this structure a \emph{merge
  tree}\index{merge sort!merge tree|see{tree}}\index{tree!merge
  $\sim$}, because each node of the tree is a sorted stack, either a
singleton or the merger of its two children. The root logically holds
the result. The merge tree is best understood from a bottom\hyp{}up,
level by level examination. Let us note \(\C{\Join}{p}\) the number of
comparisons to sort \(2^p\)~keys and consider a merge tree with
\(2^{p+1}\)~leaves. It is made of two immediate subtrees with
\(2^p\)~leaves and the root holds \(2^{p+1}\)~keys. Therefore
\begin{equation*}
\C{\Join}{0} = 0,
\quad
\C{\Join}{p+1} = 2 \cdot \C{\Join}{p} + \OC{\fun{mrg}}{2^p,2^p}.
\end{equation*}
Unrolling the recursion, we arrive at
\begin{equation}
\C{\Join}{p+1}
  = 2^p\sum_{k=0}^p{\frac{1}{2^k}\OC{\fun{mrg}}{2^k,2^k}}.
\index{mrg@$\C{\Join}{n}$}
\label{eq:cost_power_2}
\end{equation}

%\addcontentsline{toc}{subsection}{Cost}
\paragraph{Minimum cost}
\index{merge sort!power@$2^n$ keys!minimum cost|(}

When the given stack is already sorted, either in increasing or
decreasing order, the number of comparisons is minimum. In fact, given
a minimum\hyp{}comparison merge tree\index{tree!merge $\sim$}, any
exchange of two subtrees whose roots are merged leaves the number of
comparisons invariant. This happens because the merge tree is built
bottom\hyp{}up and the number of comparisons is a symmetric
function. Let us note \(\B{\Join}{p}\) the minimum number of
comparisons to sort \(2^p\)~keys. From equations
\eqref{eq:cost_power_2} and \eqref{eq:best_merge},
\begin{equation}
%\abovedisplayskip=0pt
\belowdisplayskip=0pt
\B{\Join}{p}
  = 2^{p-1}\!\sum_{k=0}^{p-1}{\frac{1}{2^k}\OB{\fun{mrg}}{2^k,2^k}}
  = p2^{p-1}.\label{eq:best_power}\index{mrg@$\B{\Join}{n}$}
\end{equation}
\index{merge sort!power@$2^n$ keys!minimum cost|)}

\paragraph{Maximum cost}
\index{merge sort!power@$2^n$ keys!maximum cost|(}

Just as with the best case, constructing a maximum\hyp{}comparison
merge sort is achieved by making worst cases for all the subtrees, for
example, \([7,3,5,1,4,8,6,2]\). Let~\(\W{\Join}{p}\) be the maximum
number of comparisons for sorting \(2^p\)~keys. From equations
\eqref{eq:cost_power_2}~and~\eqref{eq:worst_merge},
\begin{equation}
%\abovedisplayskip=2pt
\belowdisplayskip=0pt
\W{\Join}{p}
  = 2^{p-1}\!\sum_{k=0}^{p-1}{\frac{1}{2^k}\OW{\fun{mrg}}{2^k,2^k}}
  = (p-1)2^p + 1.
\label{eq:worst_power}\index{mrg@$\W{\Join}{n}$}
\end{equation}
\index{merge sort!power@$2^n$ keys!maximum cost|)}

\paragraph{Average cost}
\index{merge sort!power@$2^n$ keys!average cost|(}
\label{par:Atms_2p}

For a given stack, all permutations of which are equally likely, the
average cost of sorting it by merging is obtained by considering the
average costs of all the subtrees of the merge tree: all the
permutations of the keys are considered for a given length. Therefore,
equation~\eqref{eq:cost_power_2} is satisfied by
\(\OM{\fun{mrg}}{2^k,2^k}\) and \(\M{\Join}{p}\)
\index{mrg@$\M{\Join}{n}$}, that is, the average number of comparisons
for sorting \(2^p\)~keys. Equations~\eqref{eq:Amrg_n_n}
and~\eqref{def:cost_mrg} yield
\begin{equation*}
%\abovedisplayskip=2pt
%\belowdisplayskip=2pt
\OM{\fun{mrg}}{n,n} = 2n - 2 + \frac{2}{n+1}.
\index{mrg@$\OM{\fun{mrg}}{n,n}$}
\end{equation*}
Together with equation \eqref{eq:cost_power_2}, we further draw, for
\(p > 0\),
\begin{align}
\abovedisplayskip=0pt
\belowdisplayskip=0pt
\M{\Join}{p}
  &= 2^{p-1}\sum_{k=0}^{p-1}{\frac{1}{2^k}\OM{\fun{mrg}}{2^k,2^k}}
  = 2^p\sum_{k=0}^{p-1}{\frac{1}{2^k}\left(2^k - 1 + \frac{1}{2^k +
      1}\right)}\notag\\
  &= 2^{p}\left(p - \sum_{k=0}^{p-1}{\frac{1}{2^k}}
     + \sum_{k=0}^{p-1}{\frac{1}{2^k(2^k+1)}}\right)\notag\\
  &= 2^{p}\left(p - \sum_{k=0}^{p-1}{\frac{1}{2^k}}
     + \sum_{k=0}^{p-1}\left(\frac{1}{2^k}
     - \frac{1}{2^k+1}\right)\!\!\right)
   = p2^p - 2^p \sum_{k=0}^{p-1}\frac{1}{2^k+1}\notag\\
  &= p2^p - 2^p \sum_{k \geqslant 0}\frac{1}{2^k+1}
     + 2^p \sum_{k \geqslant p}\frac{1}{2^k+1}
= (p - \alpha) 2^p + \sum_{k \geqslant 0}\frac{1}{2^{k}+2^{-p}},
\label{eq:Mjoin}
\end{align}
where \(\alpha := \sum_{k \geqslant 0}\frac{1}{2^k+1} \simeq
1.264500\) is irrational \citep{Borwein_1992}. Since \(0 < 2^{-p} <
1\), we have \(1/(2^k + 1) < 1/(2^k+2^{-p}) < 1/2^k\) and we conclude
\begin{equation}
(p - \alpha)2^p + \alpha < \M{\Join}{p} < (p-\alpha)2^p + 2.
\label{ineq:M_join}
\end{equation}
The uniform convergence of the series \(\sum_{k \geqslant
  0}\frac{1}{2^{k}+2^{-p}}\) allows us to interchange the limits
on~\(k\) and~\(p\) and deduce that \(\M{\Join}{p} - (p-\alpha)2^p - 2
\to 0^{-}\), as \(p \to \infty\). In other words,
\(\M{\Join}{p}\) is best approximated by its upper bound, for
sufficiently large values of~\(p\).
\index{merge sort!power@$2^n$ keys!average cost|)}
\index{merge sort!power@$2^n$ keys|)}

\section{Top-down merge sort}
\index{merge sort!top-down $\sim$|(}

When generalising the fifty\hyp{}fifty splitting rule to an arbitrary
number of keys, thus obtaining stacks of lengths \(\floor{n/2}\) and
\(\ceiling{n/2}\), we obtain the variant of merge sort called
\emph{top\hyp{}down}. The corresponding program is shown in
\fig~\vref{fig:tms}\index{merge sort!top-down $\sim$!program}.
\begin{figure}[!b]
\begin{equation*}
\boxed{%
\begin{array}{r@{\;}l@{\;}l}
\fun{tms}(\cons{x,y}{t}) & \rightarrow
                         & \fun{cutr}([x],\cons{y}{t},t);\\
\fun{tms}(t)             & \rightarrow & t.\\
\\
\fun{cutr}(s,\cons{y}{t},\cons{a,b}{u})
                       & \rightarrow & \fun{cutr}(\cons{y}{s},t,u);\\
\fun{cutr}(s,t,u)        & \rightarrow
                         & \fun{mrg}(\fun{tms}(s),\fun{tms}(t)).\\
\\
\fun{mrg}(\el,t)         & \rightarrow & t;\\
\fun{mrg}(s,\el)         & \rightarrow & s;\\
\fun{mrg}(\cons{x}{s},\cons{y}{t}) & \rightarrow
                         & \cons{y}{\fun{mrg}(\cons{x}{s},t)},\;
                           \text{if \(x \succ y\)};\\
\fun{mrg}(\cons{x}{s},t) & \rightarrow & \cons{x}{\fun{mrg}(s,t)}.
\end{array}}
\end{equation*}
\caption{Top-down merge sort with \fun{tms/1}}
\label{fig:tms}
\end{figure}
Note that the call \(\fun{cutr}(s,t,u)\)\index{cutr@\fun{cutr/3}}
reverses the first half of~\(t\) on top of~\(s\) if~\(s=t\). The
technique consists in starting with \(s=\el\) and projecting the keys
of~\(t\) one by one and those of~\(u\) two by two, so when
\(\fun{cutr}(s,t,\el)\) or \(\fun{cutr}(s,t,[y])\) are reached, we
know that \(t\)~is the second half of the original stack and \(s\)~is
the reversed first half (of length \(\floor{n/2}\) if \(n\)~is the
length of the original stack). In the first rule of~\fun{tms/1}, we
saved one recursive call to \fun{cutr/3} and some memory by calling
\(\fun{cutr}([x],\cons{y}{t}, t)\) instead of
\(\fun{cutr}(\el,\cons{x,y}{t},\cons{x,y}{t})\). Moreover, this way,
the second rule implements the two base cases, \(\fun{tms}(\el)\) and
\(\fun{tms}([y])\). Furthermore, notice that, in the second rule of
\fun{cutr/2}, if~\(u = \el\), then the length of the original stack is
even and if \(u = [a]\), then it is odd. One possible drawback of
\fun{tms/1}\index{tms@\fun{tms/1}} is that the sort is
\emph{unstable}, that is, the relative order of equal keys is not
invariant.

%\addcontentsline{toc}{subsection}{Cost}

Since all comparisons are performed by
\fun{mrg/2}\index{mrg@\fun{mrg/2}}, the definition of
\fun{tms/1}\index{tms@\fun{tms/1}} implies that the number of
comparisons satisfies
\begin{equation}
\OC{\fun{tms}}{0} = \OC{\fun{tms}}{1} = 0,
\qquad
\OC{\fun{tms}}{n} = \OC{\fun{tms}}{\floor{n/2}}
+ \OC{\fun{tms}}{\ceiling{n/2}}
+ \OC{\fun{mrg}}{\floor{n/2},\ceiling{n/2}}.
\index{tms@$\OC{\fun{tms}}{n}$}
\label{eq:cost_tms}
\end{equation}

%\addcontentsline{toc}{subsection}{Cost}
\mypar{Minimum cost}
\index{merge sort!top-down $\sim$!minimum cost|(}

The minimum number of comparisons satisfies
\begin{equation*}
%\abovedisplayskip=0pt
\OB{\fun{tms}}{0} = \OB{\fun{tms}}{1} = 0,
\;
\OB{\fun{tms}}{n} = \OB{\fun{tms}}{\floor{n/2}}
+ \OB{\fun{tms}}{\ceiling{n/2}}
+ \OB{\fun{mrg}}{\floor{n/2},\ceiling{n/2}}.
\index{mrg@\fun{mrg/2}}
\index{tms@$\OB{\fun{tms}}{n}$}
\end{equation*}
% Wrapping figure better declared before a paragraph
%
{\setlength{\intextsep}{0pt} % No space before and after a figure
\begin{wrapfigure}[22]{r}[0pt]{0pt}
% 22 vertical lines (3 for each display)
% mandatory right placement (better because of a list)
% 0pt of margin overhang
\centering
\includegraphics[bb=71 435 149 721]{bits}
\caption{}
\label{fig:bits}
%\caption{Binary numbers from \(1\) to \(n\)\label{fig:bits}}
\end{wrapfigure}
We have \(\OB{\fun{tms}}{n} = \OB{\fun{tms}}{\floor{n/2}} +
\OB{\fun{tms}}{\ceiling{n/2}} + \floor{n/2}\), using
equation~\eqref{eq:best_merge}
\vpageref{eq:best_merge}.\index{tms@$\OB{\fun{tms}}{n}$} In particular
\begin{equation*}
\OB{\fun{tms}}{2p} = 2 \cdot \OB{\fun{tms}}{p} + p,\quad
\OB{\fun{tms}}{2p+1} = \OB{\fun{tms}}{p} + \OB{\fun{tms}}{p+1} + p.
\end{equation*}
Let us introduce the difference of two successive terms, \(\Delta_n :=
\OB{\fun{tms}}{n+1} - \OB{\fun{tms}}{n}\), so \(\Delta_0 = 0\), and
find some constraints on it. Because of the floor and ceiling
functions of~\(n/2\), we consider two complementary cases.
\begin{itemize}

 \item \(\Delta_{2p} = \OB{\fun{tms}}{2p+1} - \OB{\fun{tms}}{2p} =
  \OB{\fun{tms}}{p+1} - \OB{\fun{tms}}{p} = \Delta_{p}\).

  \item \(\Delta_{2p+1} = \OB{\fun{tms}}{p+1} - \OB{\fun{tms}}{p} + 1 =
  \Delta_{p} + 1\).

\end{itemize}
We already met~\(\Delta_n\) under the name~\(\nu_n\) in
equation~\eqref{eq:ruler_nu} \vpageref{eq:ruler_nu}. Let us define it
recursively:% as follows:
\begin{equation}
\nu_{0} := 0,\quad \nu_{2n} := \nu_{n},\quad
\nu_{2n+1} := \nu_{n} + 1.\label{def:nu}\index{bit sum}
\end{equation}
This definition becomes obvious when we consider the binary
representations of~\(2n\) and~\(2n+1\). Notice that \(\nu\)~is a
deceptively simple function: it is periodic because \(\nu_{1} =
\nu_{2^p} = 1\), but \(\nu_{2^p-1}=p\). Resuming our argument:
\(\OB{\fun{tms}}{n+1} = \OB{\fun{tms}}{n} + \nu_n\), and summing on
both sides yields
\begin{equation}
\abovedisplayskip=2pt
\abovedisplayshortskip=2pt
\belowdisplayskip=2pt
\OB{\fun{tms}}{n} = \sum_{k=0}^{n-1}{\nu_k}.\label{eq:OB_tms}
\index{tms@$\OB{\fun{tms}}{n}$}
\end{equation}}

\noindent \cite{Trollope_1968} first found a closed form for
\(\sum_{k=0}^{n-1}\nu_k\), whose demonstration was later simplified by
\cite{Delange_1975}, who extended the analysis with Fourier
series. \cite{Stolarsky_1977} provided many references on the
subject. In equation~\eqref{eq:best_power}, we have
\(\OB{\fun{tms}}{2^p} = \frac{1}{2}p2^p\), that is,
\(\OB{\fun{tms}}{n} = \frac{1}{2}n\lg n\) when \(n=2^p\). This should
prompt us to look, like \cite{McIlroy_1974}, for an additional linear
term in the general case, that is, the greatest real constants~\(a\)
and~\(b\) such that, for \(n \geqslant 2\),
\begin{equation}
\pred{Low}{n} \colon \frac{1}{2}n\lg n + an + b \leqslant \OB{\fun{tms}}{n}.
\label{ineq:McIlroy}
\end{equation}\index{Low@\predName{Low}}%
The base case is \(\pred{Low}{2} \colon 2a + b \leqslant 0\). The most
obvious way to structure the inductive argument is to follow the
definition of \(\OB{\fun{tms}}{n}\) when \(n=2p\) and \(n=2p+1\), but
a bound on \(\OB{\fun{tms}}{2p+1}\) would rely on bounds on
\(\OB{\fun{tms}}{p}\) and \(\OB{\fun{tms}}{p+1}\), compounding
imprecision. Instead, if we could have at least one exact value from
which to inductively build the bound, we would gain
accuracy. Therefore, we may expect a better bound if we can find a
decomposition of \(\OB{\fun{tms}}{2^p+i}\), where \(0 < i \leqslant
2^p\), in terms of \(\OB{\fun{tms}}{2^p}\) (exact) and
\(\OB{\fun{tms}}{i}\). This is easy if we count the 1-bits in
\fig~\vref{fig:Btms_table},
\begin{figure}
\centering
\includegraphics[bb=71 595 210 721]{Btms_table}
\caption{$\protect\OB{\fun{tms}}{2^p+i} = \protect\OB{\fun{tms}}{2^p}
  + \protect\OB{\fun{tms}}{i} + i$}
\label{fig:Btms_table}
\end{figure}
which is the same as the table in \fig~\ref{fig:bits}, where
\(n=2^p+i\). (Keep in mind that \(\OB{\fun{tms}}{n}\) is the sum of
the bits up to \(n-1\), as seen in equation~\eqref{eq:OB_tms}.) We
find:
\begin{equation}
\OB{\fun{tms}}{2^p+i} = \OB{\fun{tms}}{2^p} + \OB{\fun{tms}}{i} + i.
\label{eq:OBtms_2m_i}
\end{equation}
(The term~\(i\) is the sum of the leftmost bits.) Therefore, let us
assume \(\pred{Low}{n}\), for all \(1 \leqslant n \leqslant 2^p\), and
prove \(\pred{Low}{2^p+i}\), for all \(0 < i \leqslant 2^p\). The
induction principle entails then that \(\pred{Low}{n}\) holds for
all~\(n \geqslant 2\). The inductive step \(\pred{Low}{2^p+i}\) should
give us the opportunity to maximise the constants~\(a\)
and~\(b\).\index{Low@\predName{Low}} Let \(m=2^p\). Using
\(\OB{\fun{tms}}{2^p} = \tfrac{1}{2}p2^p\) and the inductive
hypothesis \(\pred{Low}{i}\), we have
\begin{equation}
\frac{1}{2}m\lg m + \left(\frac{1}{2}i\lg i + ai + b\right) + i
\leqslant
\OB{\fun{tms}}{m} + \OB{\fun{tms}}{i} + i = \OB{\fun{tms}}{m+i}.
\label{ineq:Btms_n_i}
\end{equation}
We need now to find \(a\)~and~\(b\) such that the inductive step
\(\pred{L}{m+i}\) holds as well, that is,
\(\tfrac{1}{2}(m+i)\lg(m+i) + a(m+i) + b \leqslant \B{}{m+i}\).
Using~\eqref{ineq:Btms_n_i}, this inequality is implied by
\begin{equation*}
\frac{1}{2}(m+i)\lg(m+i) + a(m+i) + b
\leqslant
\frac{1}{2}m\lg m + \left(\frac{1}{2}i\lg i + ai + b\right) + i.
\end{equation*}
We can already notice that this inequality is equivalent to
\begin{equation}
\frac{1}{2}m\lg(m+i) + \frac{1}{2}i\lg(m+i) + am
\leqslant \frac{1}{2}m\lg m + \frac{1}{2}i\lg i + i.
\label{ineq:Btms_n_i_details}
\end{equation}
But \(\tfrac{1}{2}m\lg m < \tfrac{1}{2}m\lg(m+i)\) and
\(\tfrac{1}{2}i\lg i < \tfrac{1}{2}i\lg(m+i)\), therefore the
constant~\(a\) we are seeking must satisfy \(am \leqslant i\) for all
\(0 < i < m\), hence we expect~\(a\) to be strictly negative.

We extend~\(i\) over the real numbers by defining \(i=x2^p=xm\), where
\(x\)~is a real number such that \(0 < x \leqslant 1\). By replacing
\(i\)~by~\(xm\) in inequality~\eqref{ineq:Btms_n_i_details}, we obtain
\begin{equation*}
\frac{1}{2}(1+x)\lg(1+x) + a \leqslant \frac{1}{2}x\lg x + x.
\end{equation*}
Let \(\Phi(x) := \tfrac{1}{2}x\lg x - \tfrac{1}{2}(1+x)\lg(1+x) +
x\). Then, the previous inequality is equivalent to \(a \leqslant
\Phi(x)\). The function~\(\Phi\) can be continuously extended
at~\(0\), as \(\lim_{x \to 0} x\lg x = 0\), and it is differentiable
on the interval \(]0,1]\):
\begin{equation}
\frac{d\Phi}{dx} = \frac{1}{2}\lg\frac{4x}{x+1}.
\label{eq:der_Phi}
\end{equation}
The root of \(d\Phi/dx = 0\) is \myfrac{1}/{3}, and the derivative
is negative before, and positive after. Therefore, \(a_{\max} :=
\min_{0 \leqslant x \leqslant 1}\Phi(x) = \Phi(\tfrac{1}{3}) =
-\tfrac{1}{2}\lg\tfrac{4}{3}\). The base case was \(b \leqslant -2a\),
therefore \(b_{\max} := -2a_{\max} = \lg\tfrac{4}{3}\). Finally,
\begin{equation}
  \frac{1}{2}n\lg n - \left(\frac{1}{2}\lg\frac{4}{3}\right)n + \lg\frac{4}{3}
  \leqslant \OB{\fun{tms}}{n},
\label{ineq:lower_Btms}
\end{equation}
where \(\tfrac{1}{2}\lg\tfrac{4}{3} \simeq 0.2075\) and
\(\lg\tfrac{4}{3} \simeq 0.415\). Importantly, the lower bound is
tight if \(x=\myfrac{1}/{3}\), that is, when
\(2^p+i=2^p+x2^p=(1+1/3)2^p=2^{p+2}\!/3\), or, in general,
\(2^k\!/3\). The nearest integers are \(\floor{2^k\!/3}\) and
\(\ceiling{2^k\!/3}\), so we must find out which one minimises
\(\OB{\fun{tms}}{n} - \tfrac{1}{2}n\lg(\tfrac{3}{4}n)\), because we
have \(\tfrac{1}{2}n\lg n - \left(\tfrac{1}{2}\lg\tfrac{4}{3}\right)n
= \tfrac{1}{2}n\lg(\tfrac{3}{4}n)\). We start with the following
theorems.
\begin{lemma}
\label{lem:div3}
\textsl{Integers of the form \(4^p-1\) are divisible by~\(3\).}
\end{lemma}
\begin{proof}
  Let \(\pred{Div}{p}\)\index{Div@\predName{Div}} be the proposition
  to prove. Trivially, \(\pred{Div}{1}\) is true. Let us assume
  \(\pred{Div}{p}\) and proceed to establish \(\pred{Div}{p+1}\). The
  former means that there exists an integer~\(q\) such that \(4^p - 1
  = 3q\). Therefore, \(4^{p+1} - 1 = 3(4q+1)\), which means that
  \(\pred{Div}{p+1}\) holds. The induction principle then entails that
  the lemma holds for all integers~\(p\).
\end{proof}
\begin{thm}
\label{thm:OB_lambda}
\textsl{We have \(\OB{\mathsf{tms}}{1+\phi_k} -
  \OB{\mathsf{tms}}{\phi_k} = \floor{k/2}\), where \(\phi_k :=
  \floor{2^k\!/3}\).}
\end{thm}
\noindent \emph{Proof.} Let \(\phi_k := \floor{2^k\!/3}\). Either
\(k\)~is even or odd.
\begin{itemize}

  \item If \(k=2m\), then \(2^k\!/3 = (4^m-1)/3 + 1/3\). Since
    \(1/3<1\) and, by lemma~\ref{lem:div3}, \((4^m-1)/3\) is an
    integer, we have \(2^k\!/3 = \floor{2^k\!/3} + 1/3\) and
    \begin{align*}
      \floor{2^k\!/3} &= (4^m-1)/3 = 4^{m-1} + 4^{m-2} + \dots + 1\\
                      &= 2^{2m-2} + 2^{2m-4} + \dots + 1\\
                      &= (1010\dots01)_2.
    \end{align*}
    Hence \(\nu_{\phi_{2m}} = m\). We know that \(\OB{\fun{tms}}{m+1} =
    \OB{\fun{tms}}{m} + \nu_m\), therefore
    \(\OB{\fun{tms}}{1+\phi_{2m}} - \OB{\fun{tms}}{\phi_{2m}} = m\),
    or, equivalently, \(\OB{\fun{tms}}{1+\phi_k} -
    \OB{\fun{tms}}{\phi_k} = \floor{k/2}\).

  \item If \(k=2m+1\), then \(2^k\!/3 = 2(4^m-1)/3 + 2/3\). Since
    \(2/3<1\) and, by lemma~\ref{lem:div3}, \((4^m-1)/3\) is an
    integer, we have
    \begin{equation*}
      2^k\!/3 = \floor{2^k\!/3} + 2/3 = \ceiling{2^k\!/3} - 1/3
    \end{equation*}
    and
    \begin{align*}
      \floor{2^k\!/3} &= 2(4^m-1)/3\\
                      &= 2^{2m-1} + 2^{2m-3} + \dots + 2\\
                      &= (1010\dots10)_2;
    \end{align*}
    so \(\nu_{\phi_{2m+1}} = m\). From \(\OB{\fun{tms}}{m+1} =
    \OB{\fun{tms}}{m} + \nu_m\), we can deduce that:
    \(\OB{\fun{tms}}{1+\phi_{2m+1}} - \OB{\fun{tms}}{\phi_{2m+1}} =
    m\), or, equivalently,\index{tms@$\OB{\fun{tms}}{n}$|(}
    \(\OB{\fun{tms}}{1+\phi_k} - \OB{\fun{tms}}{\phi_k} =
    \floor{k/2}\).\hfill\(\Box\)

\end{itemize}
\noindent Let \(Q(x) := \tfrac{1}{2}x\lg(\tfrac{3}{4}x)\) and let us
proceed to compare \(\OB{\fun{tms}}{\phi_k} - Q(\phi_k)\) with
\(\OB{\fun{tms}}{1+\phi_k} - Q(1+\phi_k)\) by making two cases
depending upon the parity of~\(k\). If the former difference is
smaller, then \(p=\phi_k\)~is the integer which minimises
\(\OB{\fun{tms}}{p} - \tfrac{1}{2}p\lg(\tfrac{3}{4}p)\); otherwise, it
is \(p=1+\phi_k\).
\begin{itemize}

  \item If \(k=2m+2\), then \(\phi_{2m+2} = 2^{2m} + \phi_{2m}\) (see
    proof of theorem~\ref{thm:OB_lambda}). From
    equation~\eqref{eq:OBtms_2m_i}, we draw
    \begin{equation*}
      \OB{\fun{tms}}{\phi_{2m+2}} = \OB{\fun{tms}}{2^{2m}} +
    \OB{\fun{tms}}{\phi_{2m}} + \phi_{2m} = \OB{\fun{tms}}{\phi_{2m}}
    - m4^m + \phi_{2m}.
    \end{equation*}
    Summing both sides from \(m=0\) to \(m=n-1\) yields
    \begin{equation*}
      \OB{\fun{tms}}{\phi_{2n}} = \OB{\fun{tms}}{\phi_0} + S_n
    + \sum_{m=0}^{n-1}{\phi_{2m}},\; \text{where}\; S_n :=
    \sum_{m=0}^{n-1}{m4^m}.
    \end{equation*}
    We need now to make out a closed form for~\(S_n\). We have
    \begin{equation*}
      S_n + n4^n = \sum_{m=1}^{n}{m4^m}
                 = \sum_{m=0}^{n-1}(m+1)4^{m+1}
                 = 4 \cdot S_n + 4\sum_{m=0}^{n-1}4^m.
    \end{equation*}
    Since \(\sum_{m=0}^{n-1}4^m = (4^n-1)/3\), we draw \(9 \cdot S_n =
    (3n -4)4^n + 4\). On the other hand, \(9
    \sum_{m=0}^{n-1}{\phi_{2m}} = 4^n - 3n - 1\). Finally, remarking
    that \(\phi_0 = 0\) and \(\OB{\fun{tms}}{0} = 0\), we gather that
    \begin{equation}
      \OB{\fun{tms}}{\phi_{2n}} = (n-1)\phi_{2n}.
    \label{eq:OBtms_phi_2n}
    \end{equation}
    Let us now work out
    \begin{equation*}
      Q(\phi_{2n}) = \frac{1}{2}\phi_{2n}(2(n-1)
      + \lg(1-1/4^n)) = \OB{\fun{tms}}{\phi_{2n}}
      + \frac{1}{2}\phi_{2n}\lg(1-1/4^n),
    \end{equation*}
    with an application of~\eqref{eq:OBtms_phi_2n}. If we let \(f(x) := (1-x)
    \ln(1-1/x)\), then \(\OB{\fun{tms}}{\phi_{2n}} - Q(\phi_{2n}) =
    f(4^n)/(6\ln 2)\). Elementary analysis shows that
    \(3\ln(3/4) \leqslant f(x) < 1\), for \(x \geqslant 4\),
    that is,
    \begin{equation*}
      1 - \frac{1}{2}\lg 3 \leqslant
    \OB{\fun{tms}}{\phi_{2n}} - Q(\phi_{2n}) < \frac{1}{6\ln 2},\; \text{if \(n
      \geqslant 1\)}.
    \end{equation*}
    This means that
    \begin{equation}
      0.2075 < \smash{\OB{\fun{tms}}{\phi_{2n}}} - Q(\phi_{2n}) <
      0.2405.
      \label{ineq:OBtms_phi_2n}
    \end{equation}
    Now, from theorem~\ref{thm:OB_lambda} and
    equation~\eqref{eq:OBtms_phi_2n}, we have
    \begin{equation}
      \OB{\fun{tms}}{1+\phi_{2n}} = (1+\phi_{2n})(n-1) + 1.
      \label{eq:OBtms_succ_phi_2n}
    \end{equation}
    Furthermore,
    \begin{align*}
      Q(1+\phi_{2n}) &= \frac{1}{2}(1+\phi_{2n}) (2(n-1) + \lg(1+1/2^{2n-1}))\\
                    &= \OB{\fun{tms}}{1+\phi_{2n}} - 1 +
      \frac{1}{6}(4^n + 2)\lg(1+2/4^n),
    \end{align*}
    the last step using~\eqref{eq:OBtms_succ_phi_2n}. If \(g(x) := 1 -
    \tfrac{1}{6}(x+2)\lg(1+2/4^n)\), then
    \(\OB{\fun{tms}}{1+\phi_{2n}} - Q(1+\phi_{2n}) = g(4^n)\). Some
    elementary analysis entails the inequality \(2 - \lg 3 \leqslant
    g(x) < 1 - 1/(3\ln 2)\), when \(x \geqslant 4\), that is, \(2 -
    \lg 3 \leqslant g(4^n) < 1 - 1/(3\ln 2)\), for \(n \geqslant
    1\). Approximately,
    \begin{equation}
      0.4150 < \smash{\OB{\fun{tms}}{1+\phi_{2n}}} - Q(1+\phi_{2n}) <
      0.5192.
      \label{ineq:OBtms_1_phi_2n}
    \end{equation}

    \bigskip \textsl{Comparing~\eqref{ineq:OBtms_phi_2n}
      and~\eqref{ineq:OBtms_1_phi_2n} shows that \(p=\phi_{2n} =
      (1010\dots01)_2\) minimises \(\OB{\fun{tms}}{p} -
      \tfrac{1}{2}p\lg(\tfrac{3}{4}p)\).}
    \bigskip

  \item If \(k=2m+1\), then \(\phi_{2m+1} = 2^{2m-1} + \phi_{2m-1}\)
    (see proof of theorem~\ref{thm:OB_lambda}). From
    equation~\eqref{eq:OBtms_2m_i}, we draw
    \begin{align*}
      \OB{\fun{tms}}{\phi_{2m+1}} &= \OB{\fun{tms}}{2^{2m-1}} +
      \OB{\fun{tms}}{\phi_{2m-1}} + \phi_{2m-1}\\
      &= \OB{\fun{tms}}{\phi_{2m-1}} - (2m-1)4^{m-1} + \phi_{2m-1}.
    \end{align*}
    Summing both sides from \(m=1\) to \(m=n-1\) and multiplying
    by~\(9\) yields \(9\OB{\fun{tms}}{\phi_{2n+1}} = S_{n+1}/2 -
    3(4^n-1) + \sum_{m=0}^{n-1}\phi_{2m+1}\), which simplifies into
    \begin{equation}
      \OB{\fun{tms}}{\phi_{2n+1}} = \frac{1}{2}\phi_{2n+1}(2n - 1).
      \label{eq:OBtms_phi_2n_1}
    \end{equation}
    We have
    \begin{align*}
      Q(\phi_{2n+1}) &= \frac{1}{2}\phi_{2n+1}(2n-1 + \lg(1-1/4^n))\\
      &= \OB{\fun{tms}}{\phi_{2n+1}} + \frac{1}{2}\phi_{2n+1}\lg(1-1/4^n),
    \end{align*}
    where the last equality follows from~\eqref{eq:OBtms_phi_2n_1}.
    If, as we did for \(Q(\phi_{2n})\), we let \(f(x) :=
    (1-x)\ln(1-1/x)\), then we have
    \begin{equation*}
    \OB{\fun{tms}}{\phi_{2n+1}} - Q(\phi_{2n+1}) = f(4^n)/(3\ln 2).
    \end{equation*}
    We know \(3\ln(3/4) \leqslant f(x) < 1\), for \(x \geqslant
    4\), therefore
    \begin{equation*}
      \lg 3 - 2 \leqslant \OB{\fun{tms}}{\phi_{2n+1}} -
      Q(\phi_{2n+1}) < \frac{1}{3\ln 2},\; \text{if \(n \geqslant 1\)}.
    \end{equation*}
    Hence
    \begin{equation}
      0.4150 < \smash{\OB{\fun{tms}}{\phi_{2n+1}}} -
      Q(\phi_{2n+1}) < 0.4809.
      \label{ineq:OBtms_phi_2n_1}
    \end{equation}
    Now, from theorem~\ref{thm:OB_lambda} and
    equation~\eqref{eq:OBtms_phi_2n_1}, we deduce
    \begin{equation}
      \OB{\fun{tms}}{1+\phi_{2n+1}} = \frac{1}{2}(1+\phi_{2n+1})(2n - 1)
      + \frac{1}{2}.
      \label{eq:OBtms_succ_phi_2n_1}
    \end{equation}
    Moreover,
    \begin{align*}
      Q(1+\phi_{2n+1}) &= \frac{1}{2}(1+\phi_{2n+1})(2n-1 + \lg(1+1/2^{2n+1}))\\
                      &= \OB{\fun{tms}}{1+\phi_{2n+1}} - \frac{1}{2}
                         + \frac{1}{6}(1+2^{2n+1})\lg(1+1/2^{2n+1}),
    \end{align*}
    from~\eqref{eq:OBtms_succ_phi_2n_1}. We deduce
    \(\OB{\fun{tms}}{1+\phi_{2n+1}} - Q(1+\phi_{2n+1}) =
    h(2^{2n+1})\), by letting \(h(x) := \tfrac{1}{2} -
    \tfrac{1}{6}(1+x)\lg(1+1/x)\). Elementary analysis shows that \(5
    - 3\lg 3 \leqslant h(x) < 1/2 - 1/(6\ln 2)\), for \(x \geqslant
    8\), that is to say,
    \begin{equation*}
      5 - 3\lg 3 \leqslant \OB{\fun{tms}}{1+\phi_{2n+1}} -
      Q(1+\phi_{2n+1}) < \frac{1}{2} - \frac{1}{6\ln 2},\; \text{if
        \(n \geqslant 1\)}.
    \end{equation*}
    So
    \begin{equation}
      0.2450 < \smash{\OB{\fun{tms}}{1+\phi_{2n+1}}} -
      Q(1+\phi_{2n+1}) < 0.2596.
      \label{ineq:OBtms_1_phi_2n_1}
    \end{equation}

    \bigskip
    \textsl{From~\eqref{ineq:OBtms_phi_2n_1}
      and~\eqref{ineq:OBtms_1_phi_2n_1}, \(p = 1+\phi_{2m+1} =
      (1010\dots1011)_2\) minimises \(\OB{\fun{tms}}{p} -
      \tfrac{1}{2}p\lg p\).}

\end{itemize}
Finally, we conclude that the lower bound in~\eqref{ineq:lower_Btms}
is tight if \(n=2\) (from the base case) and is otherwise the sharpest
when \(n=(1010\dots01)_2\) or \(n=(1010\dots1011)_2\). As a whole,
these values constitute the \emph{Jacobsthal
  sequence},\index{Jacobsthal number} defined as
\begin{equation}
J_0 = 0; \; J_1=1; \; J_{n+2} = J_{n+1} + 2J_{n},\; \text{for \(n
  \geqslant 0\).}
\label{eq:Jacobsthal}
\end{equation}

Let us use now the same inductive approach to find a good upper bound
to \(\OB{\fun{tms}}{n}\). In other words, we want to minimise the real
constants \(a'\)~and~\(b'\) such that, for \(n \geqslant 2\),
\begin{equation*}
\OB{\fun{tms}}{n} \leqslant \frac{1}{2}n\lg n + a'n + b'.
\end{equation*}
The only difference with the search for the lower bound is that
inequalities are reversed, so we want
\begin{equation*}
\Phi(x) \leqslant a', \;\text{where \(\Phi(x) := \frac{1}{2}x\lg x - \frac{1}{2}(1+x)\lg(1+x) + x\)}.
\end{equation*}
Here, we need to find the maximum of~\(\Phi\) on the closed interval
\([0,1]\). The two positive roots of~\(\Phi\) are \(0\)~and~\(1\), and
\(\Phi\)~is negative between them (see
equation~\eqref{eq:der_Phi}). Therefore \(a'_{\min} := \max_{0
  \leqslant x \leqslant 1}\Phi(x) = \Phi(0) = \Phi(1) = 0\). From the
base case, we have \(b'_{\min} = -2a_{\min} = 0\). Therefore, we have
the bounds
\begin{equation}
\frac{1}{2}n\lg n - \left(\frac{1}{2}\lg\frac{4}{3}\right)n + \lg\frac{4}{3}
\leqslant \OB{\fun{tms}}{n} \leqslant \frac{1}{2}n\lg n.
\label{ineq:bounds_Btms}
\index{tms@$\OB{\fun{tms}}{n}$}
\end{equation}
The upper bound is clearly tight when \(n=2^p\) because of
equation~\eqref{eq:best_power}. It is also very obvious now that we
have \(\OB{\fun{tms}}{n} \sim \frac{1}{2}n\lg n\), but if we were only
interested in this asymptotic result, \cite{Bush_1940} gave a very
simple counting argument on the bits in \fig~\vref{fig:bits}.
\cite{Delange_1975} investigated \(\OB{\fun{tms}}{n}\) by means of
advanced real analysis and showed that \(\OB{\fun{tms}}{n} =
\tfrac{1}{2}n\lg n + F_0(\lg n) \cdot n\), where \(F_0\)~is a
continuous, nowhere differentiable function of period~\(1\), and whose
Fourier series shows the mean value to be approximately
\(-0.145599\). \index{tms@$\OB{\fun{tms}}{n}$|)}
\index{merge sort!top-down $\sim$!minimum cost|)}

\mypar{Maximum cost}
\label{tms:maximum}
\index{merge sort!top-down $\sim$!maximum cost|(}

The maximum number of comparisons satisfies
\begin{equation*}
\OW{\fun{tms}}{0} = \OW{\fun{tms}}{1} = 0,
\qquad
\OW{\fun{tms}}{n} = \OW{\fun{tms}}{\floor{n/2}}
+ \OW{\fun{tms}}{\ceiling{n/2}}
+ \OW{\fun{mrg}}{\floor{n/2},\ceiling{n/2}}.
\index{tms@$\OW{\fun{tms}}{n}$|(}
\index{mrg@\fun{mrg/2}}
\end{equation*}
Equation~\eqref{eq:worst_merge} \vpageref{eq:worst_merge} yields
\(\OW{\fun{tms}}{n} = \OW{\fun{tms}}{\floor{n/2}} +
\OW{\fun{tms}}{\ceiling{n/2}} + n - 1\) and
\begin{equation*}
\OW{\fun{tms}}{0} = \OW{\fun{tms}}{1} = 0;\;
\OW{\fun{tms}}{2p} = 2\OW{\fun{tms}}{p} + 2p - 1,\;
\OW{\fun{tms}}{2p+1} = \OW{\fun{tms}}{p} + \OW{\fun{tms}}{p+1} + 2p.
\end{equation*}
Let the difference of two successive terms be \(\Delta_n :=
\smash[t]{\W{}{n+1}} - \W{}{n}\). If we know \(\Delta_n\), we know
\(\W{}{n}\) because \(\sum_{k=1}^{n-1}\Delta_k =
\sum_{k=1}^{n-1}\W{}{k+1} - \sum_{k=1}^{n-1}\W{}{k} = \W{}{n} -
\W{}{1} = \W{}{n}\). We remark that
\begin{itemize}

  \item if \(n=2p\), then \(\Delta_{2p} = \Delta_{p} + 1\),

  \item else \(n=2p+1\) and \(\W{}{2p+2} = 2 \cdot \W{}{p+1} + 2p
    + 1\), so \(\Delta_{2p+1} = \Delta_{p} + 1\).

\end{itemize}
In summary, \(\Delta_0 = 0\) and \(\Delta_n = \Delta_{\floor{n/2}} +
1\). If we start unravelling the recurrence, we get \(\Delta_n =
\Delta_{\floor{\floor{n/2}/2}}+ 2\), so we must simplify
\(\floor{\floor{\floor{\dots}/2}/2}\).
\begin{thm}[Floors and Fractions]
\label{thm:floors}
\textsl{Let \(x\)~be a real number and \(q\)~a natural number. Then
  \(\floor{\floor{x}/q} = \floor{x/q}\).}
\end{thm}
\begin{proof}
  The equality is equivalent to the conjunction of the two
  complementary inequalities \(\floor{\floor{x}/q} \leqslant
  \floor{x/q}\) and \(\floor{x/q} \leqslant \floor{\floor{x}/q}\). The
  former is straightforward because it is a consequence of \(\floor{x}
  \leqslant x\). In the latter, because both sides of the inequality
  are integers, \(\floor{x/q} \leqslant \floor{\floor{x}/q}\) is
  equivalent to stating that \(p \leqslant \floor{x/q} \Rightarrow p
  \leqslant \floor{\floor{x}/q}\), for any integer~\(p\). An obvious
  lemma is that if \(i\)~is an integer and \(y\)~a real number, \(i
  \leqslant \floor{y} \Leftrightarrow i \leqslant y\), so the original
  inequality is equivalent to \(p \leqslant x/q \Rightarrow p
  \leqslant \floor{x}/q\), which is trivially equivalent to \(pq
  \leqslant x \Rightarrow pq \leqslant \floor{x}\). Since \(pq\)~is an
  integer, this implication is true from the same lemma.
\end{proof}
Using theorem~\vref{thm:floors}, we deduce \(\Delta_n = m\), with
\(m\)~being the largest natural number such that \(\floor{n/2^m} =
0\). In other words, \(m\)~is the number of bits in the binary
notation of~\(n\), which is found in equation~\eqref{eq:e_r} to be
\(\Delta_n = \floor{\lg n} + 1\). Since we already know that \(\W{}{n}
= \sum_{k=1}^{n-1}\Delta_k\), we conclude,
with~\eqref{eq:num_of_bits}, that
\begin{equation}
\W{}{n} = \sum_{k=1}^{n-1}(\floor{\lg k}+1).
\label{eq:tms_n_tmp}
\end{equation}
Whilst the minimum cost is the number of \(1\)-bits up to \(n-1\), we
find now that the maximum cost is the total number of bits up to
\(n-1\). Informally, this leads us to bet that \(\OW{\fun{tms}}{n}
\sim 2 \cdot \OB{\fun{tms}}{n} \sim n\lg n\), since we would expect
the number of \(0\)-bits and \(1\)-bits to be the same in
average. Consider again the bit table in \fig~\vref{fig:bits}. The
greatest power of~\(2\) smaller than~\(n\) is~\(2^{\floor{\lg n}}\)
because it is the binary number \((10\dots0)_2\) having the same
number of bits as~\(n\); it thus appears in the same section of the
table as~\(n\). The trick consists in counting the bits in
\emph{columns}, from top to bottom, and leftwards. In the rightmost
column, we find \(n\)~bits. In the second column, from the right, we
find \(n-2^1+1\) bits. The third from the right contains \(n-2^2+1\)
bits etc. until the leftmost column containing \(n-2^{\floor{\lg
    n}}+1\) bits. The total number of bits in the table is
\begin{equation*}
\abovedisplayskip=2pt
\belowdisplayskip=2pt
\sum_{k=1}^{n}{\!(\floor{\lg k}+1)}
   = \sum_{k=0}^{\floor{\lg n}}{\!(n-2^k+1)}
   = (n + 1)(\floor{\lg n} + 1) - 2^{\floor{\lg n}+1} + 1.
\end{equation*}
Let express the binary representation of~\(n\): \(n := (b_{m-1}\dots
b_0)_2\), then \(2^{m-1} \leqslant n \leqslant 2^m - 1\) and \(2^{m-1}
< 2^{m-1} + 1 \leqslant n + 1 \leqslant 2^m\), so we have \(m-1 <
\lg(n+1) \leqslant m\), that is, \(m = \ceiling{\lg(n+1)}\), which,
with equation~\eqref{eq:e_r} \vpageref{eq:e_r}, establishes that
\begin{equation*}
  1 + \floor{\lg n} = \ceiling{\lg(n+1)}.
\end{equation*}
As a consequence, equation~\eqref{eq:tms_n_tmp} can be rewritten as
\begin{equation}
%\abovedisplayskip=4pt
%\belowdisplayskip=4pt
\OW{\fun{tms}}{0} = \OW{\fun{tms}}{1} = 0,
\qquad
\OW{\fun{tms}}{n} = n\ceiling{\lg n} - 2^{\ceiling{\lg n}} + 1.
\label{eq:top}
\end{equation}
This equation is subtler than it seems, due to the periodicity hidden
in \(2^{\ceiling{\lg n}}\). Depending on whether \(n = 2^p\) or not,
two cases arise:
\begin{itemize*}

  \item if~\(n=2^p\), then \(\OW{\fun{tms}}{n} = n\lg n - n +
    1\);

  \item otherwise, we have \(\ceiling{\lg n} = \floor{\lg n} + 1 = \lg
    n - \{\lg n\} + 1\) and then \(\OW{\fun{tms}}{n} = n\lg n +
    \theta(1 - \{\lg n\}) \cdot n + 1\), with \(\theta(x) := x - 2^x\)
    and \(\{x\} := x - \floor{x}\) is the \emph{fractional
    part}\index{fractional part@$\{x\}$|see{fractional
        part}}\index{fractional part} of~\(x\) (\(0 \leqslant \{x\} <
    1\)). The derivative \(\theta'(x) = 1 - 2^x\ln 2\) has one root:
    \(\theta'(x_0) = 0\), that is \(x_0 = -\lg\ln 2\), and it is
    positive before~\(x_0\), and negative after. Concordantly,
    \(\max_{0<x\leqslant 1}\theta(x) = \theta(x_0) =
    -(1+\ln\ln{2})/\!\ln{2} \simeq -0.9139\), and \(\min_{0<x\leqslant
      1}\theta(x) = \theta(1) = -1\). By injectivity, \(\theta(1) =
    \theta(1-\{\lg n\})\) implies \(\{\lg n\} = 0\), that is,
    \(n=2^p\) (first case).
\end{itemize*}
Hence \(\OW{\fun{tms}}{n} = n\lg n + A(\lg n) \cdot n + 1\), where
\(A(x) := 1 - \{x\} - 2^{1 - \{x\}}\) is a periodic function, since
\(A(x) = A(\{x\})\), such that \(-1 \leqslant A(x) < -0.91\). Further
analysis of~\(A(x)\) requires Fourier series or complex analysis; its
mean value is about \(-0.942695\). Read \cite{FlajoletGolin_1994}, as
well as \cite{PannyProdinger_1995}.
\begin{equation}
n\lg n - n + 1 \leqslant \OW{\fun{tms}}{n} <
n\lg n - 0.91 n + 1.\label{ineq:OWtms}
\end{equation}
The lower bound is attained when \(n=2^p\). The upper bound is most
accurate when \(\{\lg n\} = 1 + \lg\ln 2\), that is, when \(n\)~is the
nearest integer to \(2^p\ln 2\) (take the binary expansion of \(\ln
2\), shift the point \(p\)~times to the right and round). Obviously,
\(\OW{\fun{tms}}{n} \sim n\lg n\).\index{tms@$\OW{\fun{tms}}{n}$|)}
\index{merge sort!top-down $\sim$!maximum cost|)}

\mypar{Average cost}
\index{merge sort!top-down $\sim$!average cost|(}

Let \(\OM{\fun{tms}}{n}\)\index{tms@$\OM{\fun{tms}}{n}$|(} be the
average number of comparisons to sort \(n\)~keys top\hyp{}down. All
permutations of the input stack being equally likely,
equation~\eqref{eq:cost_tms} becomes
\begin{equation*}
\OM{\fun{tms}}{0} = \OM{\fun{tms}}{1} = 0,\qquad
\OM{\fun{tms}}{n} = \OM{\fun{tms}}{\floor{n/2}} +
\OM{\fun{tms}}{\ceiling{n/2}} +
\OM{\fun{mrg}}{\floor{n/2},\ceiling{n/2}},\index{mrg@\fun{mrg/2}}
\end{equation*}
which, with equation~\eqref{eq:Amrg}, in turn implies
\begin{equation*}
\OM{\fun{tms}}{n} = \OM{\fun{tms}}{\floor{n/2}} +
\OM{\fun{tms}}{\ceiling{n/2}} + n -
\frac{\floor{n/2}}{\ceiling{n/2}+1}
- \frac{\ceiling{n/2}}{\floor{n/2}+1}.
\end{equation*}
If we proceed as we did for the extremal costs, we get
\begin{equation*}
\OM{\fun{tms}}{2p} = 2\cdot\OM{\fun{tms}}{p} + 2p - 2 +
\frac{2}{p+1},\; \OM{\fun{tms}}{2p+1} = \OM{\fun{tms}}{p} +
\OM{\fun{tms}}{p+1} + 2p - 1 + \frac{2}{p+2}.
\end{equation*}
These recurrences are a bit tricky. Setting \(\Delta_n :=
\OM{\fun{tms}}{n+1} - \OM{\fun{tms}}{n}\) yields
\begin{equation*}
\Delta_{2p} = \Delta_p + 1 + \frac{2}{p+2} - \frac{2}{p+1},
\quad
\Delta_{2p+1} = \Delta_p + 1.
\end{equation*}
Contrary to the difference equations derived for the extremal costs,
these are not helpful, so we should try an inductive approach, as we
did for finding bounds on
\(\OB{\fun{tms}}{n}\). Inequations~\eqref{ineq:M_join}
\vpageref{ineq:M_join} are equivalent to \(n\lg n - \alpha n + \alpha
< \OM{\fun{tms}}{n} < n\lg n - \alpha n + 2\), where \(n = 2^p\), and
this suggests us to also look for bounds of the form \(n\lg n + an +
b\) when \(n \neq 2^p\).

Let us start with the lower bound and set to maximise the
real constants \(a\)~and~\(b\) in
\begin{equation*}
\abovedisplayskip=0pt
\belowdisplayskip=2pt
\pred{H}{n} \colon n\lg n + an + b \leqslant \OM{\fun{tms}}{n},
\; \text{for \(n \geqslant 2\).}
\end{equation*}
Since \(\pred{H}{2p}\) depends on \(\pred{H}{p}\), and
\(\pred{H}{2p\!+\!1}\) depends on \(\pred{H}{p}\) and
\(\pred{H}{p\!+\!1}\), the property \(\pred{H}{n}\), for any \(n>1\),
transitively depends on \(\pred{H}{2}\) alone, because we are
iterating divisions by~\(2\). If we write \(\pred{H}{n} \leadsto
\pred{H}{m}\) to mean `\(\pred{H}{n}\) depends on \(\pred{H}{m}\),' we
have, for example, \(\pred{H}{2^3} \leadsto \pred{H}{2^2} \leadsto
\pred{H}{2^1}\); \(\pred{H}{7} \leadsto \pred{H}{3} \leadsto
\pred{H}{2}\) and \(\pred{H}{7} \leadsto \pred{H}{4} \leadsto
\pred{H}{2}\). \(\pred{H}{2}\) is equivalent to
\begin{equation}
2a + b + 1 \leqslant 0.
\label{ineq:base_lower_Atms}
\end{equation}
Because the definition of \(\OM{\fun{tms}}{n}\) depends on the parity
of~\(n\), the inductive step will be twofold. Let us assume
\(\pred{H}{m}\) for \(m < 2p\), in particular, we suppose
\(\pred{H}{p}\), which, with the expression of \(\OM{\fun{tms}}{2p}\)
above, entails
\begin{equation*}
\abovedisplayskip=2pt
\belowdisplayskip=2pt
 (2p\lg p + 2ap + 2b) + 2p - 2 + \frac{2}{p+1} \leqslant \OM{\fun{tms}}{2p}.
\end{equation*}
We want \(\pred{H}{2p} \colon 2p\lg(2p) + 2ap + b
= 2p\lg p + 2ap + 2p + b \leqslant \OM{\fun{tms}}{2p}\), which holds
if the following condition does:
\begin{equation*}
\abovedisplayskip=0pt
\belowdisplayskip=2pt
2p\lg p + 2ap + 2p + b \leqslant 2p\lg p + 2ap + 2b + 2p - 2 + \frac{2}{p+1},
\end{equation*}
which is equivalent to
\begin{equation*}
\abovedisplayskip=0pt
\belowdisplayskip=2pt
2 - \frac{2}{p+1} = \frac{2p}{p+1} \leqslant b.
\end{equation*}
Let \(\Phi(p) := 2p/(p+1)\). This function is strictly increasing for
\(p > 0\) and \(\Phi(p) \to 2^{-}\), as \(p \to +\infty\).

The other inductive step deals with the odd values of~\(n\). We assume
\(\pred{H}{m}\) for all \(m < 2p+1\), in particular, we suppose
\(\pred{H}{p}\) and \(\pred{H}{p+1}\), which, with the expression of
\(\OM{\fun{tms}}{2p+1}\) above, implies
\begin{equation*}
%\abovedisplayskip=0pt
%\belowdisplayskip=2pt
(p\lg p + ap + b) + ((p+1)\lg(p+1) + a(p+1) + b) + 2p - 1 +
\frac{2}{p+2} \leqslant \OM{\fun{tms}}{2p+1},
\end{equation*}
which may be simplified slightly into
\begin{equation*}
%\abovedisplayskip=0pt
%\belowdisplayskip=2pt
p\lg p + (p+1)\lg(p+1) + a(2p+1) + 2b + 2p - 1 + \frac{2}{p+2}
\leqslant \OM{\fun{tms}}{2p+1}.
\end{equation*}
We want to prove \(\pred{H}{2p+1} \colon (2p+1)\lg(2p+1) +
a(2p+1) + b \leqslant \M{}{2p+1}\), which is thus implied by
\begin{equation}
%\abovedisplayskip=0pt
%\belowdisplayskip=2pt
  (2p+1)\lg(2p+1) \leqslant
  p\lg p + (p+1)\lg(p+1) + b + 2p - 1 + \frac{2}{p+2}.
\label{ineq:Psi_temp}
\end{equation}
Let \(\Psi(p) := (2p+1)\lg(2p+1) - (p+1)\lg(p+1) - p\lg p -
2p + 1 - 2/(p+2)\). Then~\eqref{ineq:Psi_temp} is equivalent to
\(\Psi(p) \leqslant b\). Furthermore,
\begin{equation*}
\frac{d\Psi}{dp}(p) = \frac{2}{(p+2)^2} + \lg\left(1+\frac{1}{4p(p+1)}\right).
\end{equation*}
Clearly, \(d\Psi/dp > 0\), for all \(p > 0\), so \(\Psi(p)\)~is
strictly increasing for \(p > 0\). Let us find \(\lim_{p \to
  +\infty}\Psi(p)\) by rewriting \(\Psi(p)\) as follows:
\begin{align*}
\Psi(p)
  &= 2 - \frac{2}{p+2} + (2p+1)\lg(p+1/2) - (p+1)\lg(p+1)
     - p\lg p\\
  &= 2 - \frac{2}{p+2} + p\left(\lg(p+1/2)^2 - \lg(p+1)
   - \lg p\right) + \lg(p+1/2)\\
  &\phantom{=} \quad - \lg(p+1)\\
  &= 2 - \frac{2}{p+2} + p\lg\left(1 + \frac{1}{4p(p+1)}\right) +
  \lg\frac{p + 1/2}{p+1}.
\end{align*}
The limit of \(x\ln(1+1/x^2)\) as \(x \to +\infty\) can be found by
changing \(x\)~into \(1/y\) and considering the limit as \(y \to
0^{+}\), which is shown by l'H\^{o}pital's rule to be~\(0\). This
result can be extended to apply to the large term in \(\Psi(p)\) and,
since all the other variable terms converge to~\(0\), we can conclude
that \(\Psi(p) \to 2^{-}\), as \(p \to +\infty\).

Because we need to satisfy the conditions \(\Psi(p) \leqslant b\) and
\(\Phi(p) \leqslant b\) for both inductive steps to hold, we have to
compare \(\Psi(p)\)~and~\(\Phi(p)\), when \(p\)~is a natural number:
we have \(\Phi(1) < \Psi(1)\) and \(\Phi(2) < \Psi(2)\), but also
\(\Psi(p) < \Phi(p)\), if \(p \geqslant 3\). Therefore, for~\(b\) not
to depend on~\(p\), we need it to be greater than~\(2\), the smallest
upper bound of~\(\Phi\)
and~\(\Psi\). Inequality~\eqref{ineq:base_lower_Atms} means that we
need to minimise~\(b\) in order to maximise~\(a\) (which is the
priority), so we settle for the limit: \(b_{\min} = 2\), and the same
inequality entails \(a \leqslant -3/2\), hence \(a_{\max} =
-3/2\). The principle of complete induction finally establishes that,
for \(n \geqslant 2\),
\begin{equation}
  n\lg n - \frac{3}{2} n + 2 < \OM{\fun{tms}}{n}.
\label{ineq:lower_Atms}
\end{equation}
This bound is not very good, but it was easy to obtain. We may recall
the lower bound when \(n=2^p\), in~\eqref{ineq:M_join}
\vpageref{ineq:M_join}: \(n\lg n - \alpha n + \alpha <
\OM{\fun{tms}}{n}\), where \(\alpha \simeq 1.264499\). In fact,
\cite{FlajoletGolin_1994} proved
\begin{equation}
n\lg n - \alpha n < \OM{\fun{tms}}{n}.
\label{ineq:best_lower_Atms}
\end{equation}
Asymptotically, that bound is, up to the linear term, the same as for
the case \(n=2^p\). Our inductive method cannot reach this nice result
because it yields sufficient conditions that are too strong, in
particular, we found no obvious way to get the decomposition
\(\OM{\fun{tms}}{2^p+i} = \OM{\fun{tms}}{2^p} + \OM{\fun{tms}}{i} +
\dots\)

Now, let us find the smallest real constants \(a'\)~and~\(b'\) such
that for \(n \geqslant 2\), \(\OM{\fun{tms}}{n} \leqslant n\lg n + a'n
+ b'\). The base case of \(\pred{H}{n}\)
in~\eqref{ineq:base_lower_Atms} is here reversed: \(2a' + b' + 1
\geqslant 0\). Hence, in order to minimise~\(a'\), we need to
maximise~\(b'\). Furthermore, the conditions on~\(b'\) from the
inductive steps are reversed as well with respect to~\(b\): \(b'
\leqslant \Phi(p)\) and \(b' \leqslant \Psi(p)\). The base case is
\(\pred{H}{2}\), that is, \(p=1\), and we saw earlier that \(\Phi(1)
\leqslant \Psi(1)\), thus we must have \(b'\leqslant \Phi(1) =
1\). The maximum value is thus \(b'_{\max} = 1\). Finally, this
implies that \(a'\geqslant -1\), thus \(a'_{\min} = -1\).

Gathering the bounds, we hence established that
\begin{equation*}
n\lg n - \frac{3}{2}n + 2 < \OM{\fun{tms}}{n} < n\lg n - n + 1.
\end{equation*}
Trivially, we have \(\OM{\fun{tms}}{n} \sim n\lg n \sim
\OW{\fun{tms}}{n} \sim 2 \cdot \OB{\fun{tms}}{n}\).
\cite{FlajoletGolin_1994} proved, using complex analysis the following
very strong result:
\begin{equation*}
\OM{\fun{tms}}{n} = n\lg n + B(\lg n) \cdot n + \mathcal{O}(1),
\end{equation*}
where \(B\)~is continuous, non\hyp{}differentiable, periodic with
period~\(1\), of mean value \(-1.2481520\). The notation
\(\mathcal{O}(1)\) is an instance of Bachmann's notation for an
unknown positive constant. The maximum value of~\(B(x)\) is
approximately \(-1.24075\), so
\begin{equation*}
\belowdisplayskip=0pt
\OM{\fun{tms}}{n} = n\lg n - (1.25 \pm 0.01) \cdot n + \mathcal{O}(1).
\end{equation*}
\index{tms@$\OM{\fun{tms}}{n}$|)} \index{merge sort!top-down
  $\sim$!average cost|)} \index{merge sort!top-down $\sim$|)}


\section{Bottom-up merge sort}
\label{sec:general_case}
\index{merge sort!bottom-up $\sim$|(}

Instead of cutting a stack of \(n\)~keys in two halves, we could split
into \(2^{\ceiling{\lg n}-1}\) and \(n-2^{\ceiling{\lg n}-1}\) keys,
where the first number represents the highest power of~\(2\) strictly
smaller than~\(n\). For instance, if \(n=11=2^3+2^1+2^0\), we would
split into \(2^3=8\) and \(2^1+2^0=3\). Of course, if \(n=2^p\), this
strategy, called \emph{bottom\hyp{}up}, coincides with that of
top\hyp{}down merge sort, which, in terms of cost, is expressed as
\(\OC{\fun{bms}}{2^p} = \OC{\fun{tms}}{2^p} = \C{\Join}{p}\), where
\fun{bms/1}\index{bms@\fun{bms/1}} implements \emph{bottom\hyp{}up
  merge sort}. The difference between
top-down and bottom-up merge sort can be easily seen in the
\fig~\ref{fig:top_vs_bot}.
\begin{figure}
\centering
\subfloat[Bottom-up\label{fig:bot_up}]{%
\includegraphics{bot_up}}
\qquad
\subfloat[Top-down\label{fig:top_down}]{%
\includegraphics{top_down}}
\caption{Comparing merge sorts of \([6,3,2,4,1,5]\)}
\label{fig:top_vs_bot}
\end{figure}
In all generality,
\begin{equation}
\OC{\fun{bms}}{0} = \OC{\fun{bms}}{1} = 0,
\quad
\OC{\fun{bms}}{n} = \OC{\fun{bms}}{2^{\ceiling{\lg n}-1}}
+ \OC{\fun{bms}}{n - 2^{\ceiling{\lg n}-1}}
+ \OC{\fun{mrg}}{2^{\ceiling{\lg n}-1},n - 2^{\ceiling{\lg n}-1}}.
\index{bms@$\OC{\fun{bms}}{n}$}
\label{eq:cost_bms}
\end{equation}

\Fig~\vref{fig:bot_up2} shows the merge tree\index{tree!merge $\sim$}
of seven keys being sorted in that fashion. Note how the bottommost
singleton \([4]\) is merged with \([2,6]\), a stack twice as long. The
imbalance in length is further propagated upwards. The general case is
better suggested by retaining at each node only the length of the
associated stack, as shown in \fig~\vref{fig:msort_abs}.
\begin{figure}
\centering
\subfloat[Merge tree of \({[}7,3,5,1,6,2,4{]}\)\label{fig:bot_up2}]{%
\includegraphics[bb=58 632 192 721]{bot_up2}}
\qquad
\subfloat[Lengths only\label{fig:msort_abs}]{%
\includegraphics[bb=71 632 161 721]{msort_abs}}
\caption{Sorting seven keys}
\end{figure}

%\addcontentsline{toc}{subsection}{Cost}
\mypar{Minimum cost}
\index{merge sort!bottom-up $\sim$!minimum cost|(}

Let \(\OB{\fun{bms}}{n}\)\index{bms@$\OB{\fun{bms}}{n}$|(} be the
minimum cost for sorting \(n\)~keys, bottom\hyp{}up. Let \(n=2^p+i\),
with \(0 < i < 2^p\). Then, from equation~\eqref{eq:cost_bms},
\vpageref{eq:cost_bms}, and~\eqref{eq:best_merge}
\vpageref{eq:best_merge}, we deduce
\begin{equation*}
\OB{\fun{bms}}{2^p+i} = \OB{\fun{bms}}{2^p} + \OB{\fun{bms}}{i} + i,
\end{equation*}
which we recognise as an instance of the following functional
equations: \(f(0)=f(1)=0\), \(f(2)=1\) and \(f(2^p+i) = f(2^p) + f(i)
+ i\), where \(f=\OB{\fun{tms}}{}\) as seen in
equation~\eqref{eq:OBtms_2m_i} \vpageref{eq:OBtms_2m_i}. Therefore,
\begin{equation}
\OB{\fun{bms}}{n} = \OB{\fun{tms}}{n} = \sum_{k=0}^{n-1}{\nu_k}.
\label{eq:OBbms}
\end{equation}
We can thus reuse the bounds on \(\OB{\fun{tms}}{n}\):
\begin{equation}
\frac{1}{2}n\lg n - \left(\frac{1}{2}\lg\frac{4}{3}\right)n + \lg\frac{4}{3}
\leqslant \OB{\fun{bms}}{n} \leqslant
\frac{1}{2}n\lg n.
\index{bms@$\OB{\fun{bms}}{n}$}
\end{equation}
The lower bound is tight for \(n=2\) and most accurate when \(n\)~is a
Jacobsthal\index{Jacobsthal number} number (see
equations~\eqref{eq:Jacobsthal} \vpageref{eq:Jacobsthal}). The upper
bound is tight when \(n=2^p\).
\index{bms@$\OB{\fun{bms}}{n}$|)}
\index{merge sort!bottom-up $\sim$!minimum cost|)}

\mypar{Maximum cost}
\index{merge sort!bottom-up $\sim$!maximum cost|(}

Let \(\OW{\fun{bms}}{n}\)\index{bms@$\OW{\fun{bms}}{n}$|(} be the
maximum cost for sorting \(n\)~keys, bottom\hyp{}up. Let \(n=2^p+i\),
with \(0 < i < 2^p\). Then, from equation~\eqref{eq:cost_bms},
\vpageref{eq:cost_bms}, and~\eqref{eq:worst_merge}
\vpageref{eq:worst_merge}, we deduce
\begin{equation}
\OW{\fun{bms}}{2^p+i} = \OW{\fun{bms}}{2^p} + \OW{\fun{bms}}{i} + 2^p
+ i - 1.
\label{eq:Wbms_2p_i}
\end{equation}
Let us search a lower bound of \(\OW{\fun{bms}}{n}\) by induction
based on that equation. Let us find the greatest real constants
\(a\)~and~\(b\) such that, for \(n \geqslant 2\),
\begin{equation*}
n\lg n + an + b \leqslant \OW{\fun{bms}}{n}.
\end{equation*}
The base case is \(n=2\), that is, \(b \leqslant -2a - 1\). Let us
assume the bound holds for \(n=i\) and let us recall
equation~\eqref{eq:worst_power} \vpageref{eq:worst_power}, which here
takes the guise of \(\OW{\fun{bms}}{2^p} = p2^p - 2^p  +
1\). Then~\eqref{eq:Wbms_2p_i} yields
\begin{equation*}
(p2^p - 2^p + 1) + (i\lg i + ai + b) + 2^p + i - 1 \leqslant
\OW{\fun{bms}}{2^p+i},
\end{equation*}
which is equivalent to \(p2^p + i\lg i + i + ai + b \leqslant
\OW{\fun{bms}}{2^p+i}\). We want to prove the bound holds for
\(n=2^p+i\), that is, \((2^p+i)\lg(2^p+i) + a(2^p+i) + b \leqslant
\OW{\fun{bms}}{2^p+i}\). Clearly, this is true if the following
stronger constraint holds:
\begin{equation*}
(2^p+i)\lg(2^p+i) + a(2^p+i) + b \leqslant p2^p + i\lg i + i + ai + b.
\end{equation*}
It is equivalent to \(a2^p \leqslant p2^p - (2^p+i)\lg(2^p+i) + i\lg i
+ i\). Let us extend~\(i\) over the real numbers by defining \(i=x2^p\),
where \(x\)~is a real number such that \(0 < x \leqslant 1\). Then,
the running inequality is equivalent to
\begin{equation*}
a \leqslant \Phi(x),\; \text{where \(\Phi(x) := x\lg x -
  (1+x)\lg(1+x) + x\).}
\end{equation*}
The function~\(\Phi\) can be continuously extended at~\(0\), as
\(\lim_{x \to 0} x\lg x = 0\), and it is differentiable on the
closed interval \([0,1]\):
\begin{equation*}
\frac{d\Phi}{dx} = \lg\frac{2x}{x+1}.
\end{equation*}
The root of \(d\Phi/dx = 0\) is~\(1\), the derivative is negative
before, and positive after; so~\(\Phi\) decreases until \(x=1\):
\(a_{\max} := \min_{0 \leqslant x \leqslant 1}\Phi(x) = \Phi(1)
= -1\). From the base case, \(b_{\max} := -2a_{\max} - 1 =
1\). Therefore, we have
\begin{equation*}
n\lg n - n + 1 \leqslant \OW{\fun{bms}}{n}.
\end{equation*}
The bound is tight when \(x=1\), that is, \(i=2^p\), hence
\(n=2^{p+1}\).

Let us find the smallest real constants \(a'\)~and~\(b'\) such that,
for \(n \geqslant 2\),
\begin{equation*}
\OW{\fun{bms}}{n} \leqslant n\lg n + a'n + b'.
\end{equation*}
The difference with the lower bound is that the inequalities are
reversed and we minimise the unknowns, instead of maximising
them. Thus, the base case here is \(b' \geqslant -2a - 1\) and the
condition for induction is \(a' \geqslant \Phi(x)\). We know the
behaviour of~\(\Phi\), so \(a'_{\min} := \max_{0 \leqslant x \leqslant
  1}\Phi(x) = \Phi(0) = 0\), and \(b'_{\min} := -2a'_{\min} - 1 =
-1\). As a conclusion,
\begin{equation}
n\lg n - n + 1 \leqslant \OW{\fun{bms}}{n} < n\lg n - 1.
\label{ineq:OWbms}
\end{equation}
Because \(\Phi(x)\) was extended at~\(x=0\), the upper bound is best
approched when \(i=1\), the smallest possible integer value, that is,
when \(n=2^p+1\) (the most unbalanced merger: stacks of size \(2^p\)
and~\(1\)). A deeper study by \cite{PannyProdinger_1995}, based on
Fourier analysis, confirms that the linear terms of these bounds
cannot be improved and shows the mean value of the coefficient of the
linear term to be, approximately, \(-0.70057\).

\paragraph{Alternative expression}

While we already bounded \(\OW{\fun{bms}}{n}\) tightly, we may learn
something more about it by expressing it differently from its
definition, in a way more suitable to elementary computations as well.
In all generality, let us set \(n := 2^{e_r} + \dots + 2^{e_1} +
2^{e_0} > 0\), with \(e_r > \dots > e_1 > e_0 \geqslant 0\) and \(r
\geqslant 0\).  We used this decomposition in
\begin{wrapfigure}[9]{r}[0pt]{0pt}
\centering
\includegraphics[bb=71 631 202 707]{msort_gen}
\caption{\(\sum_{j=0}^{r}2^{e_j}\) keys}
\label{fig:msort_gen}
\end{wrapfigure}
equation~\eqref{eq:e_r} on page~\pageref{eq:e_r}. Let us consider in
\fig~\ref{fig:msort_gen} the tree\index{tree!merge $\sim$|(} of all
the mergers when we only retain the stacks lengths. The triangles are
subtrees made of \emph{balanced mergers}, that is, mergers performed
on stacks of same length, for which we already found the number of
comparisons. The lengths of the \emph{unbalanced mergers} are found in
the nodes from the root \(2^{e_r}+ \dots + 2^{e_0}\) down to \(2^{e_1}
+ 2^{e_0}\).\index{tree!merge $\sim$|)} In \fig~\vref{fig:Wn_even}
\begin{figure}
\centering
\subfloat[The sum of the nodes is \(\protect\OW{\protect\fun{bms}}{n}\)
\label{fig:w2p}]%
{\includegraphics[bb=62 602 199 721]{w2p}}
\qquad
\subfloat[The sum of the nodes is \(\protect\OW{\protect\fun{bms}}{n+1}\)
\label{fig:w2p_succ}]%
{\includegraphics[bb=71 604 214 721]{w2p_succ}}
\caption{Maximum-cost trees for \(n\)~even and \(n+1\)}
\label{fig:Wn_even}
\end{figure}
are shown the maximum\hyp{}cost trees for \(n\)~even and \(n+1\). The
boxed expressions are not found in the opposite tree, therefore, the
sum in each tree of the non\hyp{}boxed terms is identical.
\begin{itemize}

  \item \emph{If \(n\)~is even}, in \fig~\vref{fig:w2p}, this sum
  is\index{bms@$\OW{\fun{bms}}{n}$} \(\OW{\fun{bms}}{n} - r\). It
  equals \(\OW{\fun{bms}}{n+1} - 2^{e_0} - \OW{\fun{bms}}{2^0}\) in
  \fig~\vref{fig:w2p_succ}. Equating both counts yields
    \begin{equation}
      \OW{\fun{bms}}{n} - r = \OW{\fun{bms}}{n+1} - 2^{e_0} -
      \OW{\fun{bms}}{2^0}.
      \label{eq:OW1}
    \end{equation}
    Let us explicit that \(e_0\)~is a function of~\(n\) (it is the
    highest power of~\(2\) dividing~\(n\)): \(e_0 :=
    \rho_n\). Furthermore, we already know \(\nu_n = r+1\) and
    \(\OW{\fun{bms}}{1} = 0\). Setting \(n=2k\) in
    equation~\eqref{eq:OW1} is equivalent to
    \begin{equation}
      \OW{\fun{bms}}{2k+1} = \OW{\fun{bms}}{2k} + 2^{\rho_{2k}} +
      \nu_{2k} - 1.
      \label{eq:OWbms_2k_1_tmp}
    \end{equation}
    The function \(\rho_n\) is the \emph{ruler
      function} \citep{GrahamKnuthPatashnik_1994,Knuth_2011},
    \index{ruler function@$\rho_n$|see{ruler function}}\index{ruler
      function} which satisfies, for \(n>0\), the recurrences
    \begin{equation}
      \rho_{1} = 0,\qquad \rho_{2n} = \rho_{n} + 1,\qquad
      \rho_{2n+1} = 0,\label{eq:ruler}
    \end{equation}
    which are easily guessed from the binary notation of~\(n\) as
    \(\rho_n\)~simply counts the number of trailing zeros. This
    enables us to slightly simplify equation~\eqref{eq:OWbms_2k_1_tmp}
    into
    \begin{equation}
     \OW{\fun{bms}}{2k+1} = \OW{\fun{bms}}{2k} + 2 \cdot 2^{\rho_{k}}
     + \nu_{k} - 1.
     \label{eq:OWbms_2k_1}
    \end{equation}

  \item \emph{If \(n\)~is odd}, we make
    \fig~\vref{fig:w2p1},
\begin{figure}[t]
\centering
\subfloat[The sum of the nodes is \(\protect\OW{\protect\fun{bms}}{n}\)
\label{fig:w2p1}]{%
\includegraphics{w2p1}}
\qquad
\subfloat[The sum of the nodes is \(\protect\OW{\protect\fun{bms}}{n+1}\)
\label{fig:w2p1_succ}]{%
\includegraphics[bb=56 602 194 721]{w2p1_succ}}
\caption{Maximum-cost trees for \(n\)~odd and \(n+1\)}
\label{fig:Wn_odd}
\end{figure}
    where the non\hyp{}boxed expressions sum
    \begin{multline*}
      \OW{\fun{bms}}{n} - \sum_{k=0}^{q-1}\OW{\fun{bms}}{2^k} -
      \sum_{k=2}^{q}2^k + 2((q-1)+(r-q+1))\\
      \begin{aligned}
        & = \OW{\fun{bms}}{n} - \sum_{k=0}^{q-1}((k-1)2^{k}+1)
        - \sum_{k=2}^{q}2^k + 2r\\
        & = \OW{\fun{bms}}{n} - (q-1)2^q - q + 2r + 1,
      \end{aligned}
    \end{multline*}
    using equation~\eqref{eq:worst_power} and \(\sum_{k=0}^{q-1}k2^k =
    (q-2)2^{q}+2\). Indeed, let \(S_{q} :=
    \sum_{k=0}^{q-1}{k2^{k-1}}\). Then
    \begin{align*}
      S_{q} + q2^{q-1} & = \smash[t]{\sum_{k=1}^{q}{k2^{k-1}}}
      = \smash[t]{\sum_{k=0}^{q-1}{(k+1)2^{k}}} \\
      & = \smash[t]{\sum_{k=0}^{q-1}{k2^{k}}} + \smash[t]{\sum_{k=0}^{q-1}{2^{k}}}
      = 2 \cdot S_{q-1} + 2^{q} - 1,
    \end{align*}
    hence
    \begin{equation}
      \abovedisplayskip=4pt
      \belowdisplayskip=4pt
      S_{q} = \smash[t]{\textstyle\sum_{k=1}^{q-1}{k2^{k-1}}} =
      (q-2)2^{q-1} + 1.
      \label{eq:Sj}
    \end{equation}
    The same sum in \fig~\ref{fig:w2p1_succ}
    equals
    \begin{equation*}
      \OW{\fun{bms}}{n+1} - \OW{\fun{bms}}{2^q} + (r-q+1) =
      \OW{\fun{bms}}{n+1} - (q-1)2^{q} - q + r.
    \end{equation*}
    Equating the two quantities and simplifying yields
    \begin{equation*}
      \OW{\fun{bms}}{n+1} = \OW{\fun{bms}}{n} + r + 1=
      \OW{\fun{bms}}{n} + \nu_n.
    \end{equation*}
    Recalling the recurrences~\eqref{def:nu} \vpageref{def:nu} and
    setting \(n=2k-1\), this equation is simplified into
    \begin{equation}
      \OW{\fun{bms}}{2k} = \OW{\fun{bms}}{2k-1} + \nu_{k-1} + 1.
      \label{eq:OWbms_2k}
    \end{equation}

\end{itemize}
From equations~\eqref{eq:OWbms_2k} and~\eqref{eq:OWbms_2k_1}, we deduce
\begin{equation*}
\OW{\fun{bms}}{2k+1} = \OW{\fun{bms}}{2k-1} + 2 \cdot 2^{\rho_k} +
\nu_{k-1} + \nu_k,\quad \OW{\fun{bms}}{2k+2} = \OW{\fun{bms}}{2k} + 2
\cdot 2^{\rho_k} + 2\nu_k.
\end{equation*}
These equations allow us to compute the values of
\(\OW{\fun{bms}}{n}\) only with elementary operations. Furthermore,
summing on all sides yields
\begin{align}
\OW{\fun{bms}}{2p+1}
 &= \OW{\fun{bms}}{1} + 2\!\sum_{k=1}^{p} 2^{\rho_k} +
    \!\sum_{k=1}^{p}\nu_{k-1} + \!\sum_{k=1}^p\nu_k
  = 2 \!\sum_{k=1}^{p} 2^{\rho_k}\! + 2\!\sum_{k=1}^{p-1}\nu_k +
    \!\nu_p.\label{eq:OWbms_2p_1}\\
\OW{\fun{bms}}{2p}
  &= \OW{\fun{bms}}{2} + 2\sum_{k=1}^{p-1}2^{\rho_k} +
2\sum_{k=1}^{p-1}\nu_k
  = 1 +  2\sum_{k=1}^{p-1}2^{\rho_k} + 2\sum_{k=1}^{p-1}\nu_k.
\label{eq:OWbms_2p}
\end{align}
These expressions involve two interesting number\hyp{}theoretic
functions, \(\sum_{k=1}^{p-1}2^{\rho_k}\) and
\(\sum_{k=1}^{p-1}\nu_k\), the latter being \(\OB{\fun{bms}}{p}\), as
found in equation~\eqref{eq:OBbms}.\index{bms@$\OW{\fun{bms}}{n}$|)}
\index{merge sort!bottom-up $\sim$!maximum cost|)}


\mypar{Average cost}
\index{merge sort!bottom-up $\sim$!average cost|(}

Let \(\OM{\fun{bms}}{n}\) be the average number of comparisons to sort
\(n\)~keys bottom\hyp{}up. All permutations of the input stack being
equally likely, equation~\eqref{eq:cost_bms} \vpageref{eq:cost_bms}
becomes \(\OM{\fun{bms}}{0} = \OM{\fun{bms}}{1} = 0\)\index{bms@$\OM{\fun{bms}}{n}$} and
\begin{equation*}
\OM{\fun{bms}}{n} = \OM{\fun{bms}}{2^{\ceiling{\lg n}-1}}
+ \OM{\fun{bms}}{n - 2^{\ceiling{\lg n}-1}}
+ \OM{\fun{mrg}}{2^{\ceiling{\lg n}-1},n - 2^{\ceiling{\lg n}-1}},
\end{equation*}
which, with equation~\eqref{eq:Amrg}, in turn implies
\(\OM{\fun{bms}}{0} = \OM{\fun{bms}}{1} = 0\) and
\begin{equation*}
\OM{\fun{bms}}{n} = \OM{\fun{bms}}{2^{\ceiling{\lg n}-1}}
+ \OM{\fun{bms}}{n - 2^{\ceiling{\lg n}-1}}
+ n - \frac{2^{\ceiling{\lg n}-1}}{n - 2^{\ceiling{\lg n}-1} + 1}
- \frac{n - 2^{\ceiling{\lg n}-1}}{2^{\ceiling{\lg n}-1}+1}.
\end{equation*}
This definition is quite daunting, so let us turn to induction to find
bounds, as we did for \(\OB{\fun{tms}}{n}\) in
inequality~\eqref{ineq:McIlroy} \vpageref{ineq:McIlroy}. Let us start
with the lower bound and set to maximise the real constants
\(a\)~and~\(b\) in
\begin{equation*}
\pred{H}{n} \colon n\lg n + an + b \leqslant \OM{\fun{bms}}{n},
\; \text{for \(n \geqslant 2\).}
\end{equation*}
The base case for induction is \(\pred{H}{2}\):
\begin{equation}
2a + b + 1 \leqslant 0.
\label{ineq:base_lower_Btms}
\end{equation}
Let us assume now \(\pred{H}{n}\) for all \(2 \leqslant n \leqslant
2^p\), and let us prove \(\pred{H}{2^p+i}\), for all \(0 < i \leqslant
2^p\). The induction principle entails then that \(\pred{H}{n}\)
holds for any \(n \geqslant 2\). If \(n=2^p+i\), then \(\ceiling{\lg
  n} - 1 = p\), so
\begin{equation}
\OM{\fun{bms}}{2^p+i} = \OM{\fun{bms}}{2^p} + \OM{\fun{bms}}{i}
+ 2^p + i - \frac{2^p}{i+1} - \frac{i}{2^p+1}.
\label{eq:Abms_2p_i}
\end{equation}
By hypothesis, \(\pred{H}{i}\) holds, that is, \(i\lg i + ai + b
\leqslant \OM{\fun{bms}}{i}\), but, instead of using
\(\pred{H}{2^p}\), we will use the exact value in
equation~\eqref{eq:Mjoin} \vpageref{eq:Mjoin}, where \(\alpha :=
\sum_{k \geqslant 0}1/(2^k+1)\). From equation~\eqref{eq:Abms_2p_i},
we derive
\begin{equation*}
(p-\alpha)2^p + \sum_{k \geqslant
    0}\frac{1}{2^{k}+2^{-p}}
+ (i\lg i + ai + b) + 2^p + i -
\frac{2^p}{i+1} - \frac{i}{2^p+1} < \OM{\fun{bms}}{2^p+i}.
\end{equation*}
We want to prove \(\pred{H}{2^p+i} \colon (2^p+i)\lg(2^p+i) +
a(2^p+i) + b \leqslant \OM{\fun{bms}}{2^p+i}\), which is thus implied
by
\begin{equation*}
(2^p+i)\lg(2^p+i) + a2^p \leqslant (p - \alpha + 1)2^p -
\frac{2^p}{i+1} + i\lg i + i - \frac{i}{2^p+1} + c_p,
\end{equation*}
where \(c_p := \sum_{k \geqslant 0}1/(2^{k}+2^{-p})\). Let
\begin{equation*}
  \Psi(p,i) := p - \alpha + 1 - \frac{1}{i+1} + \frac{i}{2^p+1} -
  \frac{1}{2^p}((2^p+i)\lg(2^p+i) - i\lg i - c_p).
\end{equation*}
Then the sufficient condition above is equivalent to \(a \leqslant
\Psi(p,i)\). To study the behaviour of \(\Psi(p,i)\), let us fix~\(p\)
and let~\(i\) range over the real interval \(]0,2^p]\). The partial
derivative of~\(\Psi\) with respect to~\(i\) is
\begin{equation*}
\frac{\partial\Psi}{\partial i}(p,i) = \frac{1}{2^p+1}
+ \frac{1}{(i+1)^2} - \frac{1}{2^p}\lg\left(\frac{2^p}{i}+1\right).
\end{equation*}
Let us also determine the second derivative with respect to~\(i\):
\begin{equation*}
\frac{\partial^2\Psi}{\partial i^2}(p,i) = \frac{1}{(2^p+i)i\ln 2} - \frac{2}{(i+1)^3},
\end{equation*}
where \(\ln x\) is the natural logarithm of~\(x\). Let the cubic
polynomial
\begin{equation*}
K_p(i) := i^3 + (3 - 2\ln 2)i^2 + (3 - 2^{p+1}\ln 2)i + 1.
\end{equation*}
Then \(\partial^2\Psi/\partial i^2 = 0 \Leftrightarrow K_p(i) = 0\)
and the sign of \(\partial^2\Psi/\partial i^2\) is the sign of
\(K_p(i)\). In general, a cubic equation has the form
\begin{equation*}
ax^3 + bx^2 + cx + d = 0, \; \text{with \(a \neq 0\)}.
\end{equation*}
A classic result about the nature of the roots is as follows. Let the
\emph{discriminant}\index{discriminant|(} of the cubic be \(\Delta :=
18abcd - 4b^3d + b^2c^2 - 4ac^3 - 27a^2d^2\).
\begin{enumerate}

  \item If \(\Delta > 0\), the equation has three distinct real roots;

  \item if \(\Delta = 0\), the equation has a multiple root and all
    its roots are real;

  \item if \(\Delta < 0\), the equation has one real root and two
    nonreal complex conjugate roots.

\end{enumerate}
Let us resume now our discussion. Let the cubic polynomial
\begin{equation*}
\Delta(x) \!:= (4\ln 2)x^3 - (9 - 2\ln 2)(3 + 2\ln 2)x^2 + 12(9 - 2\ln
29)x - 4(27 - 8\ln 2).
\end{equation*}
Then the discriminant of \(K_p(i) = 0\) is \(\Delta(2^{p+1}) \cdot
\ln^2 2\). The discriminant of \(\Delta(x) = 0 \) is negative, thus
\(\Delta(x)\) has one real root~\(x_0 \simeq 8.64872\). Because the
coefficient of~\(x^3\) is positive, \(\Delta(x)\) is negative if \(x <
x_0\) and positive if \(x > x_0\).
\begin{enumerate}

  \item Since \(p \geqslant 3\) implies \(2^{p+1} > x_0\), the
    discriminant\index{discriminant|)} of \(K_p(i) = 0\) is positive,
    which means that \(K_p(i)\) has three distinct real roots if \(p
    \geqslant 3\), and so does \(\partial^2\Psi/\partial
    i^2\).

  \item Otherwise, \(K_p(i)\) has one real root if \(0 \leqslant p
    \leqslant 2\).

\end{enumerate}
Before we study these two cases in detail, we need a small reminder
about cubic polynomials. Let \(\rho_0\), \(\rho_1\) and~\(\rho_2\) be
the roots of the polynomial \(P(x) = ax^3 + bx^2 + cx + d\). So \(P(x)
= a(x-\rho_0)(x-\rho_1)(x-\rho_2) = ax^3 - a(\rho_0+\rho_1+\rho_2)x^2
+ a(\rho_0\rho_1 + \rho_0\rho_2 + \rho_1 \rho_2)x -
a(\rho_0\rho_1\rho_2)\), so \(\rho_0\rho_1\rho_2 = -d/a\).
\begin{enumerate}

  \item Let \(p \in \{0,1,2\}\). We just found that \(K_p(i)\) has one
    real root, say~\(\rho_0\), and two nonreal conjugate roots,
    say~\(\rho_1\) and \(\rho_2=\overline{\rho_1}\). Then
    \(\rho_0\rho_1\rho_2 = \rho_0 \len{\rho_1}^2 = -1\), so \(\rho_0 <
    0\). Since the coefficient of~\(x^3\) is positive, this entails
    that \(K_p(i) > 0\) if \(i > 0\), which is true for
    \(\partial^2\Psi/\partial i^2\) as well: \(i > 0 \) implies
    \(\partial^2\Psi/\partial i^2 > 0\), therefore
    \(\partial\Psi/\partial i\) increases. Since
  \begin{equation*}
    \frac{\partial\Psi}{\partial i}(p,i) \xrightarrow[i\to 0^{+}]{}
    -\infty < 0, \;\text{and}\; \left.\frac{\partial\Psi}{\partial
        i}(p,i)\right|_{i=2^p} = -\frac{1}{2^p(2^p+1)^2} < 0,
  \end{equation*}
  we deduce that \(\partial\Psi/\partial i < 0\) if \(i > 0\), which
  means that \(\Psi(p,i)\) decreases when \(i \in\; ]0,2^p]\). Since
  we are looking to minimise \(\Psi(p,i)\), we have \(\min_{0 < i
    \leqslant 2^p}\Psi(p,i) = \Psi(p,2^p)\).

\item If \(p \geqslant 3\), then \(K_p(i)\) has three real
  roots. Here, the product of the roots of \(K_p(i)\) is~\(-1\), so at
  most two of them are positive. Since we have \(K_p(0) = 1 > 0\),
  \(K_p(1) < 0\) and \(\lim_{i\to+\infty}K_p(i) > 0\), we see that
  \(K_p(i)\) has one root in \(]0,1[\) and one in \(]1,+\infty[\), and
  so does \(\partial^2\Psi/\partial i^2\). Furthermore,
  \(\left.\partial\Psi/\partial i\right|_{i=1} > 0\) and
  \(\left.\partial\Psi/\partial i\right|_{i=2^p} < 0\), therefore,
  from the intermediate theorem, there exists a real~\(i_p \in\;
  ]1,2^p[\) such that \(\left.\partial\Psi/\partial i\right|_{i=i_p} =
  0\), and we know that it is unique because \(\partial\Psi^2/\partial
  i^2\) changes sign only once in \(]1,+\infty[\). This also means
  that \(\Psi(p,i)\) increases if~\(i\) increases on \([1,i_p[\),
  reaches its maximum when \(i=i_p\), and then decreases on
  \(]i_p,2^p]\). Since \(\lim_{i \to 0^{+}}\Psi(p,i) = -\infty\) and
  we are searching for a lower bound of \(\Psi(p,i)\), we need to know
  which of \(i=1\) or \(i=2^p\) minimises \(\Psi(p,i)\): actually, we
  have \(\Psi(p,1) \geqslant \Psi(p,2^p)\), so we conclude \(\min_{0 <
    i \leqslant 2^p}\Psi(p,i) = \Psi(p,2^p)\).
\end{enumerate}
In any case, we need to minimise \(\Psi(p,2^p)\). We have:
\begin{equation*}
  \Psi(p,2^p) = - \frac{1}{2^p+1} - \sum_{k=0}^{p}\frac{1}{2^k+1}.
\end{equation*}
We check that \(\Psi(p,2^p) > \Psi(p+1,2^{p+1})\), so the function
decreases for integer points and \(a_{\max} = \min_{p > 0}\Psi(p,2^p)
= \lim_{p \to \infty}\Psi(p,2^p) = -\alpha^{+}\). From
inequation~\eqref{ineq:base_lower_Btms}, we draw \(b_{\max} =
-2a_{\max} - 1 = 2\alpha - 1 \simeq 1.52899\). In total, by the
principle of induction, we have established, for \(n \geqslant 2\),
\begin{equation*}
n\lg n - \alpha n + 2\alpha -1 < \OM{\fun{bms}}{n}.
\end{equation*}
This bound is better than for the average cost of top\hyp{}down merge
sort, inequation~\eqref{ineq:lower_Atms} \vpageref{ineq:lower_Atms},
because there, we had to decompose~\(n\) into even and odd values, not
\(n=2^p+i\) which allowed us here to use the exact value of
\(\OM{\fun{bms}}{2^p}\). It is even slightly better
than~\eqref{ineq:M_join} \vpageref{ineq:M_join}, which is quite a nice
surprise.

We need now to work out an upper bound using the same technique. In
other words, we want to minimise the real constants \(a'\)~and~\(b'\)
in \(\OM{\fun{bms}}{n} \leqslant n\lg n + a'n + b'\), for \(n
\geqslant 2\). The difference with the lower bound is that the
inequations are reversed: \(a'\geqslant \Psi(p,i)\) and \(b' \geqslant
-2a' - 1\). We revisit the two cases above:
\begin{enumerate}

  \item If \(0 \leqslant p \leqslant 2\), then \(\max_{0 < i \leqslant
    2^p}\Psi(p,i) = \Psi(p,1)\). We easily check that \(\max_{0
    \leqslant p \leqslant 2}\Psi(p,1) = \Psi(0,1) = 1 - \alpha\).

  \item If \(p \geqslant 3\), we need to express \(i_p\)~as a function
    of~\(p\), but it is hard to solve the equation
    \(\left.\partial\Psi/\partial i\right|_{i=i_p} = 0\), even
    approximately.

\end{enumerate}
Before giving up, we could try to differentiate~\(\Psi\) with respect
to~\(p\), instead of~\(i\). Indeed, \((p,i,\Psi(p,i))\) defines a
surface in space, and by privileging \(p\)~over~\(i\), we are slicing
the surface along planes perpendicular to the \(i\)~axis. Sometimes,
slicing in one direction instead of another makes the analysis
easier. The problem here is to differentiate~\(c_p\). We can work our
way round with the bound \(c_p < 2\) from~\eqref{ineq:M_join}
\vpageref{ineq:M_join} and define
\begin{equation*}
  \Phi(p,i) := p - \alpha + 1 - \frac{1}{i+1} + \frac{i}{2^p+1} -
  \frac{1}{2^p}((2^p+i)\lg(2^p+i) - i\lg i - 2).
\end{equation*}
Now we have \(\Psi(p,i) < \Phi(p,i)\) and, instead of \(\Psi(p,i)
\leqslant a'\), we can impose the stronger constraint \(\Phi(p,i)
\leqslant a'\) and cross our fingers. In \fig~\vref{fig:phi},
\begin{figure}
\centering
\includegraphics[bb=71 565 400 725]{phi}
\caption{\(\Phi(p,1)\), \(\Phi(p,2)\) and \(\Phi(p,3)\)}
\label{fig:phi}
\end{figure}
are outlined \(\Phi(p,1)\), \(\Phi(p,2)\) and \(\Phi(p,3)\). (The
starting point for each curve is marked by a white disk.)
Differentiating with respect to~\(p\) yields
\begin{equation*}
\frac{\partial\Phi}{\partial p}(p,i) =
\frac{i}{2^p}\ln\left(\frac{2^p}{i}+1\right)
- \frac{\ln 2}{2^{p-1}} - \frac{i2^p\ln 2}{(2^p+1)^2}.
\end{equation*}
To study the sign of \(\partial\Phi(p,i)/\partial p\) when~\(p\)
varies, let us define
\begin{equation*}
\varphi(x,i) := \frac{x}{i\ln 2} \cdot
                \left.\frac{\partial\Phi}{\partial
                    p}(p,i)\right|_{p=\lg x}.
\end{equation*}
Because \(x \geqslant 1\) implies \(x/i\ln 2 > 0\) and \(\lg x
\geqslant 0\), the sign of \(\varphi(x,i)\) when~\(x \geqslant 1\)
varies is the same as the sign of \(\partial\Phi(p,i)/\partial p\)
when~\(p \geqslant 0\) varies, bearing in mind that \(x=2^p\). We
have
\begin{align*}
\varphi(x,i) &= \lg\left(\frac{x}{i}+1\right) -
\left(\!\frac{x}{x+1}\!\right)^2 - \frac{2}{i},\\
\frac{\partial\varphi}{\partial x}(x,i) &=
\frac{1}{(x+i)\ln 2} - \frac{2x}{(x+1)^3}.
\end{align*}
This should remind us of a familiar sight:
\begin{equation*}
\frac{\partial\varphi}{\partial x}(x,i) =
  x \cdot \left.\frac{\partial^2\Psi}{\partial x^2}(p,x)\right|_{p=\lg
  i}.
\end{equation*}
When \(x \geqslant 1\) varies, the sign of
\(\partial\varphi(x,i)/\partial x\) is the same as the sign of
\(\left.\partial^2\Psi(p,x)/\partial x^2\right|_{p=\lg i}\), so we can
reuse the previous discussion on the roots of \(K_p(i)\), while taking
care to replace~\(i\) by~\(x\), and~\(2^p\) by~\(i\):
\begin{enumerate}

  \item If \(i \in \{1,2,3,4\}\), then \(\partial\varphi(x,i)/\partial
    x > 0\) when \(x > 0\).

  \item If \(i \geqslant 5\), then \(\partial\varphi(x,i)/\partial x >
    0\) when \(x \geqslant 1\).

\end{enumerate}
In both cases, \(\varphi(x,i)\) increases when \(x \geqslant 1\),
which, with the facts that \(\lim_{x\to 0^{+}}\varphi(x,i) = -\infty <
0\) and \(\lim_{x\to\infty}\varphi(x,i) = +\infty > 0\), entails that
there exists a unique root~\(\rho > 0\) such that \(\varphi(x,i) < 0\)
if \(x < \rho\), and \(\varphi(x,i) > 0\) if \(x > \rho\), and the
same holds for \(\partial\Phi/\partial p\) (with a different
root). Concordantly, \(\Phi(p,i)\) is decreasing down to its minimum,
and increasing afterwards. (See again \fig~\vref{fig:phi}.)

Moreover \(\overline\lim_{p \to \infty}\Phi(p,i) = i/(i+1) - \alpha <
1 - \alpha = \Phi(0,1)\), so the curves have asymptotes. Since we are
searching for the maximum, we deduce: \(a'_{\min} = \max_{0 < i
  \leqslant 2^p}\Phi(p,i) = 1 - \alpha \simeq -0.2645\), and the
constant is \(b'_{\min} = -2a'_{\min} - 1 = 2\alpha - 3 \simeq
-0.471\). In sum, we found, for \(n \geqslant 2\),
\begin{equation}
n\lg n - \alpha n + (2\alpha - 1) < \OM{\fun{bms}}{n}
< n\lg n - (\alpha - 1)n - (3 - 2\alpha).
\label{ineq:bounds_Mbms}
\end{equation}
The lower bound is most accurate when \(n=2^p\). To interpret the
values of~\(n\) for which the upper bound is most accurate, we need
another glance at \fig~\vref{fig:phi}. We have \(i/(i+1) - \alpha \to
1 - \alpha\), as \(p \to \infty\), but this does not tell us anything
about~\(p\). Unfortunately, as noted earlier, for a given~\(p\), we
cannot characterise explicitly~\(i_p\), which is the value of~\(i\)
maximising \(\Phi(p,i)\) (in the planes perpendicular to this
page). Anyway, the linear terms of these bounds cannot be improved
upon. This means that the additional number of comparisons incurred by
sorting \(n=2^p+i\) keys instead of~\(2^p\) is at most~\(n\). As with
top\hyp{}down merge sort, more advanced mathematics by
\cite{PannyProdinger_1995} show that \(\OM{\fun{bms}}{n} = n\lg n +
B^*(\lg n) \cdot n\)\index{bms@$\OM{\fun{bms}}{n}$}, where \(B^*\)~is
a continuous, non\hyp{}differentiable, periodic function whose average
value is approximately \(-0.965\). Obviously, we have
\(\OM{\fun{bms}}{n} \sim n\lg n \sim \OW{\fun{bms}}{n} \sim 2 \cdot
\OB{\fun{bms}}{n}\).
\index{merge sort!bottom-up $\sim$!average cost|)}

\mypar{Program}
\index{merge sort!bottom-up $\sim$!program}

We managed to analyse the number of comparisons to sort by merging
because the whole process can easily be depicted as a
tree\index{tree!merge $\sim$|(}. It is time to provide a program whose
traces conform to these merge trees. In \fig~\vref{fig:bms} is shown
the definitions of the main sorting function
\fun{bms/1}\index{bms@\fun{bms/1}} and several auxiliaries.
\begin{itemize}

  \item The call \(\fun{solo}(s)\)\index{solo@\fun{solo/1}} is a stack
    containing singletons with all the keys of~\(s\) in the same
    order. In other words, it is the leaves of the merge tree.

  \item The call \(\fun{all}(u)\)\index{all@\fun{all/1}} is a stack
    containing stacks which are the result of merging adjacent stacks
    in~\(u\). In other words, it is the level just above~\(u\) in the
    merge tree.

  \item The call \(\fun{all}(\fun{solo}(s))\) is the sorted stack
    corresponding to the stack of singletons~\(\fun{solo}(s)\). In
    other words, starting with the leaves, it keeps building levels up
    by calling \fun{mrg/2}\index{mrg@\fun{mrg/2}} until the root of
    the merge tree is reached.

\end{itemize}
What is beautiful about this program is that there is no need for two
distinct phases, first building the perfect merge trees and then
performing the unbalanced mergers with the roots of these: it is
possible to achieve the same effect by interleaving rightwards and
upwards constructions.\index{tree!merge $\sim$|)}
\begin{figure}
\begin{equation*}
\boxed{%
\begin{array}{r@{\;}l@{\;}lr@{\;}l@{\;}l}
  \fun{bms}(\el) & \xrightarrow{\smash{\mu}} & \el;
& \fun{solo}(\el) & \xrightarrow{\smash{\xi}} & \el;\\
  \fun{bms}(s) & \xrightarrow{\smash{\nu}}
               & \fun{all}(\fun{solo}(s)).
& \fun{solo}(\cons{x}{s}) & \xrightarrow{\smash{\pi}}
                          & \cons{[x]}{\fun{solo}(s)}.\\
\\
  \fun{all}([s]) & \xrightarrow{\smash{\rho}} & s;
& \fun{nxt}(\cons{s,t}{u}) & \xrightarrow{\smash{\tau}}
                         & \cons{\fun{mrg}(s,t)}{\fun{nxt}(u)};\\
  \fun{all}(s) & \xrightarrow{\smash{\sigma}}
               & \fun{all}(\fun{nxt}(s)).
& \fun{nxt}(u) & \xrightarrow{\smash{\upsilon}} & u.\\
\\
\fun{mrg}(\el,t)         & \xrightarrow{\smash{\theta}} & t;\\
\fun{mrg}(s,\el)         & \xrightarrow{\smash{\iota}} & s;\\
\fun{mrg}(\cons{x}{s},\cons{y}{t}) & \xrightarrow{\smash{\kappa}}
& \multicolumn{4}{@{}l}{\cons{y}{\fun{mrg}(\cons{x}{s},t)},
\;\text{if \(x \succ y\);}}\\
\fun{mrg}(\cons{x}{s},t) & \xrightarrow{\smash{\lambda}}
                         & \cons{x}{\fun{mrg}(s,t)}.
\end{array}}
\end{equation*}
\caption{Sorting by bottom-up mergers with \fun{bms/1}}
\label{fig:bms}
\end{figure}


\paragraph{Additional cost}

In order to determine the cost
\(\C{\fun{bms}}{n}\)\index{bms@$\C{\fun{bms}}{n}$} we need to add to
the number of comparisons the number of rewrite steps that do not
involve comparisons, that is, other than by rules
\(\kappa\)~and~\(\lambda\).
\begin{itemize}

  \item Rules \(\theta\)~and~\(\iota\) are used once to conclude each
  merger. Let \(\OC{\ltimes}{n}\)\index{mrg@$\OC{\ltimes}{n}$} be the
  number of comparisons to perform the unbalanced mergers when there
  are \(n\)~keys to sort. Looking back at \fig~\vref{fig:msort_gen},
  we see that
    \begin{equation}
      \OC{\ltimes}{n} :=
      \sum_{i=1}^{r}{\OC{\fun{mrg}}{2^{e_i},2^{e_{i-1}}+\dots+2^{e_0}}}.
      \label{eq:C_unbal}
    \end{equation}
    The total number of comparisons is the sum of the numbers of
    comparisons of the balanced and unbalanced mergers:
    \begin{equation}
      \OC{\fun{bms}}{n}
      = \sum_{i=0}^{r}{\C{\Join}{e_i}}
      +
      \OC{\ltimes}{n}
      = \sum_{i=0}^{r}{\OC{\fun{bms}}{2^{e_i}}}
      +
      \sum_{i=1}^{r}{\OC{\fun{mrg}}{2^{e_i},2^{e_{i-1}}+\dots+2^{e_0}}}.
      \label{eq:msort_gen}
    \end{equation}
    To find the number of mergers, let us set \(\OC{\fun{mrg}}{m,n} =
    1\) in equation~\eqref{eq:cost_power_2}
    \vpageref{eq:cost_power_2}, yielding \(\OC{\fun{bms}}{2^p} = 2^p -
    1\). By plugging this result in equation~\eqref{eq:msort_gen}, we
    draw \(\OC{\fun{bms}}{n} = n - 1\). In other words, rules
    \(\theta\)~and~\(\iota\) are used \(n-1\)\label{eq:bms_merges}
    times in total.\index{bms@$\OC{\fun{bms}}{n}$}

  \item In rule~\(\tau\), one call
    \(\fun{nxt}(\cons{s,t}{u})\)\index{nxt@\fun{nxt/1}} corresponds to
    one call \(\fun{mrg}(s,t)\)\index{mrg@\fun{mrg/2}}, one for each
    merger. Therefore, \(\tau\)~is used \(n-1\)~times.

  \item Rule~\(\upsilon\) is used once for each level in the merge
    tree\index{tree!merge $\sim$}, except the root, with~\(u\) either
    empty or a singleton.  Let~\(\Lambda(j)\) be the number of nodes
    at level~\(j\), where \(j=0\) represents the level of the
    leaves. Then, the number~\(z\) we are looking for is the greatest
    natural satisfying the equation \(\Lambda(z) = 1\), at the
    root. Function~\fun{nxt/1}\index{nxt@\fun{nxt/1}} implies
    \begin{equation*}
      \Lambda(j+1) = \ceiling{\Lambda(j)/2},\; \text{with \(\Lambda(0)
        = n\)}.
    \end{equation*}
    This recurrence is equivalent to the closed form \(\Lambda(j) =
    \ceiling{n/2^j}\), as a consequence of the following theorem.
\begin{thm}[Ceilings and Fractions]
\label{thm:ceilings}
\textsl{Let \(x\)~be a real number and \(q\)~a natural number. Then
  \(\ceiling{\ceiling{x}/q} = \ceiling{x/q}\).}
\end{thm}
\begin{proof}
  The equality is equivalent to the conjunction of the two
  complementary inequalities \(\ceiling{\ceiling{x}/q} \geqslant
  \ceiling{x/q}\) and \(\ceiling{\ceiling{x}/q} \leqslant
  \ceiling{x/q}\). The former is direct: \(\ceiling{x} \geqslant x
  \Rightarrow \ceiling{x}/q \geqslant x/q \Rightarrow
  \ceiling{\ceiling{x}/q} \geqslant \ceiling{x/q}\). Since both sides
  of the inequality are integers, \(\ceiling{\ceiling{x}/q} \leqslant
  \ceiling{x/q}\) is equivalent to state that \(p \leqslant
  \ceiling{\ceiling{x}/q} \Rightarrow p \leqslant \ceiling{x/q}\), for
  any integer~\(p\). An obvious lemma is that if \(i\)~is an integer
  and~\(y\) a real number, \(i \leqslant \ceiling{y} \Leftrightarrow i
  \leqslant y\), so the original inequality is equivalent to \(p
  \leqslant \ceiling{x}/q \Rightarrow p \leqslant x/q\), for any
  integer \(p\), which is \(pq \leqslant \ceiling{x} \Rightarrow pq
  \leqslant x\). The lemma yields this implication, achieving the
  proof.
\end{proof}
    \noindent To find~\(z\), we express~\(n\) in binary:
    \(n :=\!  \sum_{k=0}^{m-1}{b_k2^k} = (b_{m-1}\ldots
    b_0)_2\), where \(b_k \in \{0,1\}\) and \(b_{m-1} = 1\). It is
    easy to derive a formula for~\(b_i\). We have
    \begin{equation}
      \frac{n}{2^{i+1}}
      = \frac{1}{2^{i+1}}\sum_{k=0}^{m-1}{b_k2^{k}}
      = \frac{1}{2^{i+1}}\sum_{k=0}^{i}{b_k2^k} + (b_{m-1}\dots b_{i+1})_2.
      \label{eq:n_on_power_2}
    \end{equation}
    We prove that \(\floor{n/2^{i+1}} = (b_{m-1}\dots b_{i+1})_2\) as
    follows:
    \begin{equation*}
      \sum_{k=0}^{i}{2^k} < 2^{i+1}
      \Rightarrow
      0 \leqslant \sum_{k=0}^{i}{b_k2^k} < 2^{i+1}
      \Leftrightarrow
      0 \leqslant \frac{1}{2^{i+1}}\sum_{k=0}^{i}{b_k2^k} < 1.
    \end{equation*}
    This and equation~\eqref{eq:n_on_power_2} imply that
    \begin{equation*}
      \left\lceil\frac{n}{2^i}\right\rceil =
      (b_{m-1}\dots b_i)_2
      + \begin{cases}
          0, & \text{if \((b_{i-1}\dots b_0)_2=0\)};\\
          1, & \text{otherwise}.
        \end{cases}
    \end{equation*}
    Therefore, \(\ceiling{n/2^z} = 1\) is equivalent to \(z=m-1\)
    if \(n=2^{m-1}\), and \(z=m\) otherwise. Equation~\eqref{eq:e_r}
    \vpageref{eq:e_r} states \(m = \floor{\lg n} + 1\), thus
    \(z=\floor{\lg n}\) if \(n\)~is a power of~\(2\), and
    \(z=\floor{\lg n} + 1\) otherwise. More simply, this means that
    \(z=\ceiling{\lg n}\).

  \item Rule~\(\rho\) is used once, at the root. Rule~\(\sigma\) is
    used \(z\)~times.

  \item The trace of \(\fun{solo}(s)\)\index{solo@\fun{solo/1}}
    is~\(\pi^n\xi\) if~\(s\) contains \(n\)~keys, so
    \(\C{\fun{solo}}{n} = n + 1\).

  \item The contribution to the total cost of rules
    \(\mu\)~and~\(\nu\) is simply~\(1\).

\end{itemize}
In total, \(\C{\fun{bms}}{n} = \OC{\fun{bms}}{n} + 3n + 2\ceiling{\lg
  n} + 1\) and \(\C{\fun{bms}}{n} \sim
\OC{\fun{bms}}{n}\).\index{bms@$\C{\fun{bms}}{n}$}

\paragraph{Improvement}

It is easy to improve upon \fun{bms/1} by directly building the second
level of the merge tree\index{tree!merge $\sim$} \emph{without using
  \fun{mrg/2}}\index{mrg@\fun{mrg/2}}. Consider the program in
\fig~\vref{fig:bms0},
\begin{figure}
\begin{equation*}
\boxed{%
\begin{array}{r@{\;}l@{\;}l@{\;}l}
\fun{bms}_0(s)   & \rightarrow & \fun{all}(\fun{duo}(s)).\\
\\
\fun{duo}(\cons{x,y}{s}) & \rightarrow & \cons{[y,x]}{\fun{duo}(s)},
                                       & \text{if \(x \succ y\)};\\
\fun{duo}(\cons{x,y}{s}) & \rightarrow & \cons{[x,y]}{\fun{duo}(s)};\\
\fun{duo}(s)             & \rightarrow & [s].\\
\\
\fun{all}([s]) & \rightarrow & s;\\
\fun{all}(s)   & \rightarrow & \fun{all}(\fun{nxt}(s)).\\
\\
\fun{nxt}(\cons{s,t}{u}) & \rightarrow
                         & \cons{\fun{mrg}(s,t)}{\fun{nxt}(u)};\\
\fun{nxt}(u)             & \rightarrow & u.\\
\\
\fun{mrg}(\el,t)         & \rightarrow & t;\\
\fun{mrg}(s,\el)         & \rightarrow & s;\\
\fun{mrg}(\cons{x}{s},\cons{y}{t}) & \rightarrow
                         & \cons{y}{\fun{mrg}(\cons{x}{s},t)},
                         & \text{if \(x \succ y\)};\\
\fun{mrg}(\cons{x}{s},t) & \rightarrow & \cons{x}{\fun{mrg}(s,t)}.
\end{array}}
\end{equation*}
\caption{Faster bottom\hyp{}up merge sort with \fun{bms\(_0\)/1}}
\label{fig:bms0}
\end{figure}
where \fun{solo/1}\index{solo@\fun{solo/1}} has been replaced
by~\fun{duo/1}\index{duo@\fun{duo/1}}. The number of comparisons is
unchanged, but the cost, measured as the number of rewrites, is
slightly smaller. The added cost
of~\(\fun{duo}(s)\)\index{duo@\fun{duo/1}} is \(\floor{n/2}+1\), where
\(n\)~is the length of~\(s\). On the other hand, we save the cost
of~\(\fun{solo}(s)\)\index{solo@\fun{solo/1}}. The first rewrite by
rule~\(\sigma\) is not performed, as well as the subsequent call
\(\fun{nxt}(s)\)\index{nxt@\fun{nxt/1}}, to wit, \(\floor{n/2}\) calls
to \fun{mrg/2}\index{mrg@\fun{mrg/2}} on pairs of singletons
by~\(\kappa\) or~\(\lambda\), plus one rewrite by~\(\theta\)
or~\(\iota\) for the last singleton or the empty stack, totalling
\(\floor{n/2}\C{\fun{mrg}}{1,1} + 1 = 2\floor{n/2} + 1\). In the end,
the total cost is decreased by
\begin{equation*}
  ((n+1) + 1 + (2\floor{n/2}+1)) - (\floor{n/2}+1) = n + \floor{n/2} +
  2.
\end{equation*}
Hence, \(\C{\fun{bms}_0}{n} = \OC{\fun{bms}}{n} + \ceiling{3n/2} +
2\ceiling{\lg n} - 1\), for \(n>0\), and \(\C{\fun{bms}_0}{0} =
3\). Asymptotically, we have \(\C{\fun{bms}_0}{n} \sim
\OC{\fun{bms}}{n}\).\index{bms@$\OC{\fun{bms}}{n}$}


\section{Comparison}

In this section we gather our findings about top\hyp{}down and
bottom\hyp{}up merge sort for an easier comparison, and we also
present new results which relate the costs of both algorithms.

\mypar{Minimum cost}

The minimum cost of both variants of merge sort is the same:
\(\OB{\fun{tms}}{n} = \OB{\fun{bms}}{n}\) and
\begin{equation*}
\frac{1}{2}n\lg n - \left(\frac{1}{2}\lg\frac{4}{3}\right)n + \lg\frac{4}{3}
\leqslant \OB{\fun{tms}}{n} \leqslant \frac{1}{2}n\lg n.
\end{equation*}
The lower bound is tight for \(n=2\) and most accurate when \(n\)~is a
Jacobsthal\index{Jacobsthal number} number (see~\eqref{eq:Jacobsthal}
\vpageref{eq:Jacobsthal}). The upper bound is tight when
\(n=2^p\). These results may not be intuitive \emph{a priori}.

\mypar{Maximum cost}

\hspace*{-8.7pt} In the previous sections, we found the following bounds:
\begin{align*}
n\lg n - n + 1 &\leqslant \OW{\fun{tms}}{n} <
n\lg n - 0.91 n + 1;\\
n\lg n - n + 1 &\leqslant \OW{\fun{bms}}{n} < n\lg n - 1.
\end{align*}
In both cases, the lower bound is tight if, and only if, \(n=2^p\).
The upper bound of top\hyp{}down merge sort is most accurate when
\(n\)~is the nearest integer to \(2^p\ln 2\). The upper bound of
bottom\hyp{}up merge sort is most accurate if \(n=2^p+1\).

It is interesting to bound \(\OW{\fun{bms}}{n}\) in term of
\(\OW{\fun{tms}}{n}\), shedding further light on the relationship
between these two variants of merge sort.

We already noted \(\OC{\fun{bms}}{2^p} = \OC{\fun{tms}}{2^p}\), so
\(\OW{\fun{bms}}{2^p} = \OW{\fun{tms}}{2^p}\). Furthermore, we have
\(\OW{\fun{bms}}{2^p} = \OW{\fun{bms}}{2^p-1} + p\), thus
\(\OW{\fun{bms}}{2^p-1} = \OW{\fun{tms}}{2^p-1}\). Another interesting
value is \(\OW{\fun{tms}}{2^p+1} = (p-1)2^p + p + 2\), so
\(\OW{\fun{bms}}{2^p+1} - \OW{\fun{tms}}{2^p+1} = 2^p - p - 1\). This
leads us to conjecture the following tight bounds in relationship with
top\hyp{}down merge sort:
\begin{equation*}
\OW{\fun{tms}}{n} \leqslant \OW{\fun{bms}}{n} \leqslant
\OW{\fun{tms}}{n} + n - \ceiling{\lg n} - 1.
\index{tms@$\OW{\fun{tms}}{n}$}
\index{bms@$\OW{\fun{bms}}{n}$}
\end{equation*}
We will prove these inequalities by means of mathematical induction
on~\(n\) and, in the process, we will discover when they become
equalities. First, let us deduce from the general recurrence for
the cost of bottom\hyp{}up merge sort the recurrence for the maximum
cost:
\begin{equation}
\OW{\fun{bms}}{0} = \OW{\fun{bms}}{1} = 0;\quad \OW{\fun{bms}}{n} =
\OW{\fun{bms}}{2^{\ceiling{\lg n}-1}} +
\OW{\fun{bms}}{n-2^{\ceiling{\lg n}-1}} + n - 1.
\label{eq:bot}
\end{equation}
Also, we easily check that, for all \(p \geqslant 0\),
\begin{equation}
\OW{\fun{tms}}{2^p} = \OW{\fun{bms}}{2^p}.
\label{eq:2p}
\end{equation}

\paragraph{Lower bound}

Let us prove, for all \(n \geqslant 0\),
\begin{equation}
\pred{W\(_L\)}{n} \colon \OW{\fun{tms}}{n} \leqslant
\OW{\fun{bms}}{n}.
\index{WL@$\predName{W}_L$}
\label{ineq:topbot}
\end{equation}
From~\eqref{eq:2p}, it is clear that \(\pred{W\(_L\)}{2^0}\)
holds. Let the induction hypothesis be \(\forall m \leqslant
2^p.\pred{W\(_L\)}{m}\). The induction principle requires that we
prove \(\pred{W\(_L\)}{2^p+i}\), for all \(0 < i < 2^p\). Note that we
leave aside the case when \(i=2^p\), because
\(\pred{W\(_L\)}{2^{p+1}}\) is already true from~\eqref{eq:2p}.

Equations~\eqref{eq:bot} and~\eqref{eq:2p}
yield\index{bms@$\OW{\fun{bms}}{n}$}
\begin{align*}
\OW{\fun{bms}}{2^p+i} &=
\OW{\fun{bms}}{2^p} + \OW{\fun{bms}}{i} + 2^p + i - 1\\
& = \OW{\fun{tms}}{2^p} + \OW{\fun{bms}}{i} + 2^p + i - 1 \geqslant
\OW{\fun{tms}}{2^p} + \OW{\fun{tms}}{i} + 2^p + i - 1,
\end{align*}
the inequality being the instance \(\pred{W\(_L\)}{i}\) of the
induction hypothesis. Consequently, if the inequality
\begin{equation}
\OW{\fun{tms}}{2^p} + \OW{\fun{tms}}{i} + 2^p + i - 1 \geqslant
\OW{\fun{tms}}{2^p+i}
\label{ineq:suff_cond}
\end{equation}
holds, the result \(\pred{W\(_L\)}{2^p+i}\) ensues. Let us try to
prove it.

Let \(n = 2^p + i\). Then \(p = \floor{\lg n}\) and \(\ceiling{\lg n}
= \floor{\lg n} + 1\). Equation~\eqref{eq:top} \vpageref{eq:top}
entails
\begin{align*}
\OW{\fun{tms}}{2^p+i} &=
(2^p+i)(p+1)-2^{p+1}+1 = ((p-1)2^p+1)+(p+1)i\\
&= \OW{\fun{tms}}{2^p} + (p+1)i.
\end{align*}
Therefore, inequation~\eqref{ineq:suff_cond} is equivalent to \(pi
\leqslant \OW{\fun{tms}}{i} + 2^p - 1\). Using
equation~\eqref{eq:top}, this inequality in turn is equivalent to
\begin{equation}
(p - \ceiling{\lg i})i \leqslant 2^p - 2^{\ceiling{\lg i}}.
\label{conj}
\end{equation}
To prove it, we have two complementary cases to analyse:
\begin{itemize}

  \item \(i=2^q\), with \(0 \leqslant q < p\). Then \(\lg i = q\) and
  equation~\eqref{conj} is equivalent to \((p-q)2^q \leqslant 2^p -
  2^q\), that is
  \begin{equation}
   p-q \leqslant 2^{p-q} - 1.\label{conj0}
  \end{equation}
  Let \(f(x) := 2^x - x - 1\), with \(x > 0\). We have \(f(0) = f(1) =
  0\) and \(f(x) > 0\) for \(x>1\), so the inequality~\eqref{conj0}
  holds and is tight if, and only if, \(x=1\), that is, \(q=p-1\).

  \item \(i = 2^q + j\), with \(0 \leqslant q < p\) and \(0 < j <
    2^q\). Then \(\floor{\lg i} = q = \ceiling{\lg i} - 1\) and
    inequation~\eqref{conj} is then equivalent to the inequality
    \((p-q-1)i \leqslant 2^p - 2^{q+1}\), that is to say,
    \begin{equation}
      (p-q+1)2^q + (p-q-1)j \leqslant 2^p.\label{conj1}
    \end{equation}
    Since \(p-q-1 \geqslant 0\) and \(j < 2^q\), we have \((p-q-1)j
    \leqslant (p-q-1)2^q\) (tight if \(q=p-1\)). Hence
    \begin{equation*}
      (p-q+1)2^q+(p-q-1)j \leqslant (p-q)2^{q+1}.
    \end{equation*}
    Inequation~\eqref{conj1} is entailed if \(2(p-q)
    \leqslant 2^{p-q}\). Let \(g(x) := 2^x - 2x\), with \(x > 0\). We
    have \(g(1) = g(2) = 0\) and \(f(x) > 0\) for \(x > 2\). Thus,
    inequality~\eqref{conj1} holds and is tight if, and only if,
    \(x=1\), that is, \(q=p-1\) (case \(x=2\) implies \(i
    \leqslant 2^{p-1}\), which cannot be tight).\hfill\(\Box\)

\end{itemize}

Let us find now the shape of~\(n\) when \(\pred{W\(_L\)}{n}\) is
tight. We proved above that if \(q=p-1\), that is, the binary notation
of~\(n\) starts with two 1-bits, formally written as
\((11(0+1)^*)_2\), then the following inequality holds:
\begin{equation*}
\OW{\fun{tms}}{2^p+i} =
\OW{\fun{tms}}{2^p} + \OW{\fun{tms}}{i} + 2^p + i - 1 \leqslant
\OW{\fun{tms}}{2^p} + \OW{\fun{bms}}{i} + 2^p + i - 1 =
\OW{\fun{bms}}{2^p+i}.
\end{equation*}
The inequality is tight: \(\OW{\fun{tms}}{2^p+i} =
\OW{\fun{bms}}{2^p+i}\), if, and only if, \(\OW{\fun{bms}}{i} =
\OW{\fun{tms}}{i}\). \index{bms@$\OW{\fun{bms}}{n}$}
\index{tms@$\OW{\fun{tms}}{n}$}

Using the case analysis above, if \(i = 2^q + j\), we have
\(\OW{\fun{tms}}{2^p+i} = \OW{\fun{bms}}{2^p+i}\), if, and only if,
\(\OW{\fun{tms}}{2^{p-1}+j} = \OW{\fun{bms}}{2^{p-1}+j}\). These
equivalences can be repeated, yielding two strictly decreasing
sequences of positive integers, \(2^p+i > 2^{p-1}+j > 2^{p-2}+k >
\dots\) and \(i > j > k > \dots\) The end of the latter recursive
descent is simply~\(0\), which means that the former stops at a power
of~\(2\), for which we know equation~\eqref{eq:2p}. In other words,
the binary representation of~\(n\) is made of a series of one or more
1-bits (from \(2^p\), \(2^{p-1}\), \(2^{p-2}\), \ldots), possibly
followed by successive 0-bits, which we formally write
\(n=(1^+0^*)_2\). This means that \(n\)~is the difference between two
powers of~\(2\):
\begin{equation}
\boxed{\OW{\fun{bms}}{n} = \OW{\fun{tms}}{n} \Leftrightarrow n=2^p - 2^q.}
\label{eq:Wbms_eq_Wtms}
\end{equation}
Note that if \(n=2^p-1\), the number of unbalanced mergers,
bottom\hyp{}up, is maximum, and the maximum costs are the same in both
variants. Also, the case \(n=2^p\) minimises both maximum costs.

\paragraph{Upper bound}

If \(n=2^p+1\), then \(p=\floor{\lg n}=\ceiling{\lg
  n}-1\). Furthermore, definition~\eqref{eq:bot} entails
\(\OW{\fun{bms}}{2^p+1} = p2^p+1\) and definition~\eqref{eq:top}
\vpageref{eq:top} \(\OW{\fun{tms}}{2^p+1} = (p-1)2^p+p+2\), so
\(\OW{\fun{bms}}{2^p+i} - \OW{\fun{tms}}{2^p+i} = 2^p - p - 1\). In
terms of~\(n\), this means that \(\OW{\fun{bms}}{n} -
\OW{\fun{tms}}{n} = n - \ceiling{\lg n} - 1\), if \(n=2^p+1\). We want
to prove that this difference is maximum:
\begin{equation}
  \pred{W\(_U\)}{n} \colon \OW{\fun{bms}}{n} \leqslant \OW{\fun{tms}}{n} + n -
  \ceiling{\lg n} - 1.
\label{ineq:upper_Wbms}
\index{WU@$\predName{W}_U$}
\end{equation}
Notice how equation~\eqref{eq:2p} entails
\(\pred{W\(_U\)}{2^0}\). Consequently, let the induction hypothesis be
\(\forall m \leqslant 2^p.\pred{W\(_U\)}{m}\) and let us prove that
\(\pred{W\(_U\)}{2^p+i}\), for all \(0 < i < 2^p\).

Let \(n=2^p+i\). Equations~\eqref{eq:bot} and~\eqref{eq:2p} yield
\begin{align*}
\OW{\fun{bms}}{2^p+i} &= \OW{\fun{bms}}{2^p} + \OW{\fun{bms}}{i} +
2^p + i - 1
= \OW{\fun{tms}}{2^p} + \OW{\fun{bms}}{i} + 2^p + i - 1\\
&\leqslant \OW{\fun{tms}}{2^p} + \OW{\fun{tms}}{i} + 2^p + 2i -
\ceiling{\lg i} - 2,
\end{align*}
where the inequality is the instance \(\pred{W\(_U\)}{i}\) of the
induction hypothesis. Furthermore, \(n - \ceiling{\lg n} - 1 = 2^p + i
- p - 2\). Therefore, if
\begin{equation*}
\OW{\fun{tms}}{2^p} + \OW{\fun{tms}}{i} +
2^p + 2i - \ceiling{\lg i} - 2 \leqslant \OW{\fun{tms}}{2^p+i} + 2^p +
i - p - 2,
\end{equation*}
then \(\pred{W\(_U\)}{2^p+i}\) would ensue. Using
equation~\eqref{eq:top}, we deduce
\index{bms@$\OW{\fun{bms}}{n}$}\index{tms@$\OW{\fun{tms}}{n}$}
\begin{align*}
\OW{\fun{tms}}{i} &= i\ceiling{\lg i} - 2^{\ceiling{\lg i}} + 1,\\
\OW{\fun{tms}}{2^p} &= (p-1)2^p + 1,\\
\OW{\fun{tms}}{2^p+i} &= \OW{\fun{tms}}{2^p} + (p+1)i.
\end{align*}
The unproven inequality becomes \(\OW{\fun{tms}}{i} + i - \ceiling{\lg
  i} \leqslant (p+1)i - p\), or
\begin{equation}
1 \leqslant (i-1)(p-\ceiling{\lg i}) + 2^{\ceiling{\lg i}}.
\label{eq:conj2}
\end{equation}
We have two complementary cases to consider:
\begin{itemize}

  \item \(i = 2^q\), with \(0 \leqslant q < p\). Then \(\lg i = q\)
    and inequation~\eqref{eq:conj2} is equivalent to \((p-q+1)(2^q-1)
    \geqslant 0\). Since \(0 \leqslant q < p\) implies \(p-q+1>1\) and
    \(2^q \geqslant 1\), the inequality is proved, the bound being
    tight if, and only if, \(q=0\).

  \item \(i = 2^q + j\), with \(0 \leqslant q < p\) and \(0 < j <
    2^q\). Then we have \(\floor{\lg i} = q = \ceiling{\lg i} - 1\)
    and inequation~\eqref{eq:conj2} is then equivalent to \(1
    \leqslant (2^q + j - 1) (p-q) + 2^q\), or
    \begin{equation}
     1 \leqslant (p-q+1)2^q + (p-q-1)(j-1).\label{ineq:2q_j}
    \end{equation}
    From \(q < p\) we deduce \(p-q+1 \geqslant 2\) and \(p-q-1
    \geqslant 0\); we also have \(2^q \geqslant 1\) and \(j \geqslant
    1\). Consequently, \((p-q+1)2^q \geqslant 2\) and \((p-q-1)(j-1)
    \geqslant 0\), hence inequation~\eqref{ineq:2q_j} holds, but the
    bound is never tight.\hfill\(\Box\)

\end{itemize}
As a side\hyp{}effect, we proved that if \(i=1\), that is, \(n=2^p +
1\), then we have the following inequation:
\begin{equation*}
\OW{\fun{bms}}{2^p+1} = \OW{\fun{tms}}{2^p} +
\OW{\fun{bms}}{1} + 2^p \leqslant \OW{\fun{tms}}{2^p} +
\OW{\fun{tms}}{1} + 2^p = \OW{\fun{tms}}{2^p+1} + 2^p - p - 1.
\end{equation*}
But, since \(\OW{\fun{tms}}{1} = \OW{\fun{bms}}{1} = 0\), the
inequality is actually an equality.\index{tms@$\OW{\fun{tms}}{n}$}
\begin{equation*}
\boxed{\OW{\fun{bms}}{n} = \OW{\fun{tms}}{n} + n - \ceiling{\lg n} - 1
  \Leftrightarrow n=1 \;\text{or}\; n=2^p+1.}
\end{equation*}

\paragraph{Program}

Although we will present the programming language \Erlang in
part~\ref{part:implementation}, here is how to compute efficiently the
maximum costs: \ErlangInUnchecked{max} Note how we efficiently
computed the binary exponentiation~\(2^n\) by means of the recurrent
equations
\begin{equation*}
2^0 = 1,\quad 2^{2m} = (2^m)^2,\quad 2^{2m+1} =  2(2^m)^2.
\end{equation*}
The cost \(\C{\fun{exp2}}{n}\) thus satisfies \(\C{\fun{exp2}}{0} =
1\) and \(\C{\fun{exp2}}{n} = 1 + \C{\fun{exp2}}{\floor{n/2}}\), if
\(n > 0\). Therefore, if \(n > 0\), it is \(1\)~plus the number of
bits of~\(n\), that is to say, \(\C{\fun{exp2}}{n} = \floor{\lg n} +
2\), else \(\C{\fun{exp2}}{0} = 1\).

\mypar{Average cost}

In sum, we established, for \(n \geqslant 2\),
\begin{align*}
n\lg n - \frac{3}{2}n + 2 &< \OM{\fun{tms}}{n} < n\lg n - n + 1,\\
n\lg n - \alpha n + (2\alpha - 1) &< \OM{\fun{bms}}{n}
< n\lg n - (\alpha - 1)n - (3 - 2\alpha),
\end{align*}
where \(\alpha \simeq 1.2645\), \(2\alpha - 1 \simeq 1.52899\) and \(3
- 2\alpha \simeq 0.471\). For top\hyp{}down merge sort, the nature
of~\(n\) for the bounds to be most accurate was not conclusively found
by our inductive method. For bottom\hyp{}up merge sort, the lower
bound is most accurate when \(n=2^p\), but we could not determine the
values of~\(n\) that make the upper bound most accurate.

The previous inequalities on \(\OM{\fun{bms}}{n}\) do not allow us to
compare the average costs of the two variants of merge sort we have
studied. Here, we prove that top\hyp{}down merge sort performs fewer
key comparisons than bottom\hyp{}up merge sort in average. Since we
already proved that this is true as well in the worst case
(see~\eqref{ineq:topbot} \vpageref{ineq:topbot}), and that their
minimum costs are equal (see~\eqref{eq:OBbms} \vpageref{eq:OBbms}),
this will be the last nail in the coffin of the bottom\hyp{}up
variant, before its rebirth in section~\vref{sec:online}. We want to
prove by induction
\begin{equation*}
\OM{\fun{tms}}{n} \leqslant \OM{\fun{bms}}{n}.
\end{equation*}
We already now that the bound is tight when \(n=2^p\), so let us check
the inequality for \(n=2\) and let us assume that it holds up
to~\(2^p\) and proceed to establish that it also holds for \(2^p+i\),
with \(0 < i \leqslant 2^p\), thus reaching our goal. Let us recall
equation~\eqref{eq:Abms_2p_i} \vpageref{eq:Abms_2p_i}:
\begin{equation*}
\OM{\fun{bms}}{2^p+i} = \OM{\fun{bms}}{2^p} + \OM{\fun{bms}}{i}
+ 2^p + i - \frac{2^p}{i+1} - \frac{i}{2^p+1}.
\end{equation*}
Since \(\OM{\fun{bms}}{2^p} = \OM{\fun{tms}}{2^p}\) and, by
hypothesis, \(\OM{\fun{tms}}{i} \leqslant \OM{\fun{bms}}{i}\), we have
\begin{equation}
\OM{\fun{bms}}{2^p+i} \geqslant \OM{\fun{tms}}{2^p} + \OM{\fun{tms}}{i}
+ 2^p + i - \frac{2^p}{i+1} - \frac{i}{2^p+1}.
\label{ineq:Atms_Abms}
\end{equation}
If we could show the right\hyp{}hand side to be greater than or equal
to \(\OM{\fun{tms}}{2^p+i}\), we would win. Let us actually generalise
this sufficient condition and express it as the following lemma:
\begin{equation*}
  \pred{T}{m,n} \colon
  \OM{\fun{tms}}{m+n} \leqslant \OM{\fun{tms}}{m} + \OM{\fun{tms}}{n} +
  m + n - \frac{m}{n+1} - \frac{n}{m+1}.
\end{equation*}
Let us use a lexicographic ordering on the pairs \((m,n)\) of natural
numbers \(m\)~and~\(n\) (see definition~\eqref{def:lexico}
\vpageref{def:lexico}). The base case, \((0,0)\), is easily seen to
hold. We observe that the statement to be proved is symmetric,
\(\pred{T}{m,n} \Leftrightarrow \pred{T}{n,m}\), hence we only need to
make three cases: \((2p,2q)\), \((2p,2q+1)\) and \((2p+1,2q+1)\).
\begin{enumerate}

  \item \((m,n) = (2p,2q)\). In this case,
    \begin{itemize}

      \item \(\OM{\fun{tms}}{m+n} = \OM{\fun{tms}}{2(p+q)} =
        2\OM{\fun{tms}}{p+q} + 2(p+q) - 2 + 2/(p+q+1)\);

      \item \(\OM{\fun{tms}}{m} = \OM{\fun{tms}}{2p} =
        2\OM{\fun{tms}}{p} + 2p - 2 + 2/(p+1)\);

      \item \(\OM{\fun{tms}}{n} = \OM{\fun{tms}}{2q} =
        2\OM{\fun{tms}}{q} + 2q - 2 + 2/(q+1)\).

    \end{itemize}
    Then, the right\hyp{}hand side of \(\pred{T}{m,n}\) is
    \begin{equation*}
      r := 2\left(\OM{\fun{tms}}{p} + \OM{\fun{tms}}{q} + 2(p+q) - 2 +
        \tfrac{1}{p+1} + \tfrac{1}{q+1} - \tfrac{p}{2q+1} -
        \tfrac{q}{2p+1}\right).
    \end{equation*}
    The induction hypothesis \(\pred{T}{p,q}\) is
    \begin{equation*}
      \OM{\fun{tms}}{p+q} \leqslant \OM{\fun{tms}}{p} +
      \OM{\fun{tms}}{q} + p + q - \frac{p}{q+1} - \frac{q}{p+1}.
    \end{equation*}
    Therefore, \(\tfrac{1}{2}r \geqslant \OM{\fun{tms}}{p+q} + p + q -
    2 + \tfrac{q+1}{p+1} + \tfrac{p+1}{q+1} - \tfrac{p}{2q+1} -
    \tfrac{q}{2p+1}\). If the right\hyp{}hand side is greater than or
    equal to \(\tfrac{1}{2}\OM{\fun{tms}}{m+n}\), then
    \(\pred{T}{m,n}\) is proved. In other words, we need to prove
    \begin{equation*}
      \frac{p+1}{q+1} + \frac{q+1}{p+1} \geqslant 1 +
      \frac{p}{2q+1} + \frac{q}{2p+1} + \frac{1}{p+q+1}.
    \end{equation*}
    We expand everything in order to get rid of the fractions; we then
    observe that we can factorise~\(pq\) and the remaining bivariate
    polynomial is~\(0\) if \(p=q\) (the inequality is tight), which
    means that we can factorise by \(p-q\) (actually, twice). In the
    end, this inequation is equivalent to \(pq(p-q)^2(2p+2q+3)
    \geqslant 0\), with \(p,q \geqslant 0\), which means that
    \(\pred{T}{m,n}\) holds.

  \item \((m,n) = (2p,2q+1)\). In this case,
    \begin{itemize}

      \item \(\OM{\fun{tms}}{m+n} = \OM{\fun{tms}}{2(p+q)+1} =
        \OM{\fun{tms}}{p+q} + \OM{\fun{tms}}{p+q+1} + 2(p+q) - 1 +
        \tfrac{2}{p+q+2}\);

      \item \(\OM{\fun{tms}}{m} = \OM{\fun{tms}}{2p} =
        2\OM{\fun{tms}}{p} + 2p - 2 + 2/(p+1)\);

      \item \(\OM{\fun{tms}}{n} = \OM{\fun{tms}}{2q+1} =
        \OM{\fun{tms}}{q} + \OM{\fun{tms}}{q+1} + 2q - 1 + 2/(q+2)\).

    \end{itemize}
    Then, the right\hyp{}hand side of \(\pred{T}{m,n}\) is
    \begin{equation*}
      r := 2\OM{\fun{tms}}{p} + \OM{\fun{tms}}{q} +
      \OM{\fun{tms}}{q+1} + 4(p+q) - 2 + \tfrac{2}{p+1} +
      \tfrac{2}{q+2} - \tfrac{p}{q+1} - \tfrac{2q+1}{2p+1}.
    \end{equation*}
    The induction hypotheses \(\pred{T}{p,q}\) and \(\pred{T}{p,q+1}\)
    are
    \begin{itemize}

      \item \(\OM{\fun{tms}}{p+q} \leqslant \OM{\fun{tms}}{p} +
      \OM{\fun{tms}}{q} + p + q - \frac{p}{q+1} - \frac{q}{p+1}\),

      \item \(\OM{\fun{tms}}{p+(q+1)} \leqslant \OM{\fun{tms}}{p} +
      \OM{\fun{tms}}{q+1} + p + (q + 1) - \frac{p}{q+2} -
      \frac{q+1}{p+1}\).

    \end{itemize}
    Thus, \(r \geqslant \OM{\fun{tms}}{p+q} + \OM{\fun{tms}}{p+q+1} +
    2(p+q) - 3 + \tfrac{2q+3}{p+1} + \tfrac{p+2}{q+2} -
    \tfrac{2q+1}{2p+1}\). If the right\hyp{}hand side is greater than
    or equal to \(\OM{\fun{tms}}{m+n}\), then \(\pred{T}{m,n}\) is
    proved. In other words, we need to prove
    \begin{equation*}
      \frac{2q+3}{p+1} + \frac{p+2}{q+2} \geqslant 2 +
      \frac{2q+1}{2p+1} + \frac{2}{p+q+2}.
    \end{equation*}
    By expanding and getting rid of the fractions, we obtain a
    bivariate polynomial with the trivial factors~\(p\) and \(p-q\)
    (because if \(p=q\), the inequality is tight). After that, a
    computer algebra system can finish the factorisation and the
    inequality is found to be equivalent to \(p(p-q)(p-q-1)(2p+2q+5)
    \geqslant 0\), therefore \(\pred{T}{m,n}\) holds.

  \item \((m,n) = (2p+1,2q+1)\). In this case,
    \begin{itemize}

      \item \(\OM{\fun{tms}}{m+n} = \OM{\fun{tms}}{2(p+q+1)} =
        2\OM{\fun{tms}}{p+q+1} + 2(p+q) + 2/(p+q+2)\);

      \item \(\OM{\fun{tms}}{n} = \OM{\fun{tms}}{2p+1} =
        \OM{\fun{tms}}{p} + \OM{\fun{tms}}{p+1} + 2p - 1 + 2/(p+2)\);

      \item \(\OM{\fun{tms}}{n} = \OM{\fun{tms}}{2q+1} =
        \OM{\fun{tms}}{q} + \OM{\fun{tms}}{q+1} + 2q - 1 + 2/(q+2)\).

    \end{itemize}
    Then, the right\hyp{}hand side of \(\pred{T}{m,n}\) is
    \begin{equation*}
      r := \OM{\fun{tms}}{p} + \OM{\fun{tms}}{q} + \OM{\fun{tms}}{p+1}
      + \OM{\fun{tms}}{q+1} + 4(p+q) + \tfrac{2}{p+2} + \tfrac{2}{q+2}
      - \tfrac{2p+1}{2q+2} - \tfrac{2q+1}{2p+2}.
    \end{equation*}
    The (symmetric) induction hypotheses \(\pred{T}{p,q+1}\) and
    \(\pred{T}{p+1,q}\):
    \begin{itemize}

      \item \(\OM{\fun{tms}}{p+(q+1)} \leqslant \OM{\fun{tms}}{p} +
        \OM{\fun{tms}}{q+1} + p + q + 1 - \tfrac{p}{q+2}
        - \tfrac{q+1}{p+1}\);

      \item \(\OM{\fun{tms}}{(p+1)+q} \leqslant \OM{\fun{tms}}{p+1} +
        \OM{\fun{tms}}{q} + p + q + 1 - \tfrac{p+1}{q+1} -
        \tfrac{q}{p+2}\).

    \end{itemize}
    Thus, \(r \geqslant 2\OM{\fun{tms}}{p+q+1} + 2(p+q) - 2 +
    \tfrac{q+1}{p+1} + \tfrac{q}{p+2} + \tfrac{p+1}{q+1} +
    \tfrac{p}{q+2} + \tfrac{2}{p+2} + \tfrac{2}{q+2} -
    \tfrac{2p+1}{2q+2} - \tfrac{2q+1}{2p+2}\). If the right\hyp{}hand
    side is greater than or equal to \(\OM{\fun{tms}}{m+n}\), then
    \(\pred{T}{m,n}\) is proved. In other words, we need to prove
    \begin{equation*}
      \frac{q+1}{p+1} + \frac{q+2}{p+2} + \frac{p+2}{q+2} +
      \frac{p+1}{q+1} \geqslant 2 + \frac{2p+1}{2q+2} +
      \frac{2q+1}{2p+2} + \frac{2}{p+q+2}.
    \end{equation*}
    After expansion to form a positive polynomial, we note that the
    inequality is tight if \(p=q\), so the polynomial has a factor
    \(p-q\). After division, another factor \(p-q\) is clear. The
    inequality is thus equivalent to \((p-q)^2(2p^2(q+1) + p(2q^2 + 9q
    + 8) + 2(q+2)^2) \geqslant 0\), so \(\pred{T}{m,n}\) holds in this
    case as well.

\end{enumerate}
In total, \(\pred{T}{m,n}\) holds in each case, therefore the lemma is
true for all \(m\)~and~\(n\). By applying the lemma
to~\eqref{ineq:Atms_Abms}, we prove the theorem \(\OM{\fun{tms}}{n}
\leqslant \OM{\fun{bms}}{n}\), for all~\(n\). Collecting all the cases
where the bound is tight shows what we would expect: \(m=n\),
\(m=n+1\) or \(n=m+1\). For~\eqref{ineq:Atms_Abms}, this means
\(i=2^p\) or \(i=2^p-1\). In other words,
\begin{equation*}
\boxed{\OM{\fun{tms}}{n} = \OM{\fun{bms}}{n} \Leftrightarrow n=2^p
  \;\text{or}\; n=2^p-1, \text{with \(p \geqslant
    0\)}.}
\end{equation*}

\paragraph{Program}

In \Erlang, we would implement as follows the computation of the
average costs of top\hyp{}down and bottom\hyp{}up merge sort:
\ErlangInUnchecked{mean}

\mypar{Merging vs.\@ inserting}
\index{merge sort!vs. insertion sort|(}

Let us compare insertion sort and bottom\hyp{}up merge sort in their
fastest variant. We found in equation~\eqref{eq:ave_i2wb}
\vpageref{eq:ave_i2wb} the average cost of balanced 2-way insertion
sort:
\begin{equation*}
\M{\fun{i2wb}}{n}\index{12wb@$\M{\fun{i2wb}}{n}$}
  = \frac{1}{8}(n^2 + 13n - \ln 2n + 10) + \epsilon_n,\;
\text{with \(0 < \epsilon_n < \frac{7}{8}\)}.
\end{equation*}
We also just found that the cost in addition to comparisons is
\(\ceiling{3n/2} + 2\ceiling{\lg n} - 1\) for
\fun{bms\(_0\)/1}\index{bms0@\fun{bms\(_0\)/1}}, and
\(\OC{\fun{bms\(_0\)}}{n} = \OC{\fun{bms}}{n}\). Moreover, we found
bounds on \(\OM{\fun{bms}}{n}\) in~\eqref{ineq:bounds_Mbms}
\vpageref{ineq:bounds_Mbms}, the upper one being excellent. Therefore
\begin{align*}
\M{\fun{bms\(_0\)}}{n} &< (n\lg n - (\alpha - 1)n - (3-2\alpha))
+ (\ceiling{3n/2} + 2\ceiling{\lg n} - 1)\\
& < (n+2)\lg n + 1.236n + 1.529;\\
\M{\fun{bms\(_0\)}}{n} &> (n\lg n - 1.35n + 1.69) + (\ceiling{3n/2} +
2\ceiling{\lg n} - 1)\\
&> (n+2)\lg n + 0.152n + 0.69;
\end{align*}
\begin{equation*}
(n^2 + 13n - \ln 2n + 10)/8 < \M{\fun{i2wb}}{n} < (n^2 + 13n - \ln 2n + 17)/8.
\end{equation*}
where \(\alpha \simeq 1.2645\) and \(\ceiling{x} < x + 1\).

Hence, \((n+2)\lg n + 1.236n + 1.529 < (n^2 + 13n - \ln 2n + 10)/8\)
implies \(\M{\fun{bms\(_0\)}}{n} < \M{\fun{i2wb}}{n}\), and also
\((n^2 + 13n - \ln 2n + 17)/8 < (n+2)\lg n + 0.152n + 0.69\) implies
\(\M{\fun{i2wb}}{n} < \M{\fun{bms\(_0\)}}{n}\). With the help of a
computer algebra system, we find that
\begin{enumerate}

  \item \(\M{\fun{i2wb}}{n} < \M{\fun{bms\(_0\)}}{n}\) if \(3
    \leqslant n \leqslant 29\),

  \item \(\M{\fun{bms\(_0\)}}{n} < \M{\fun{i2wb}}{n}\) if \(43
    \leqslant n\).

\end{enumerate}
For the case \(n=2\), we find: \(\M{\fun{i2wb}}{2} = 11/2 > 5 =
\M{\fun{bms\(_0\)}}{2}\). If we set aside this peculiar case, we may
conclude that insertion sort is faster, in average, for stacks of less
than \(30\)~keys, and the opposite is true for stacks of at
least~\(43\) keys.

In\hyp{}between, we do not know, but we can compute efficiently the
average costs and use dichotomy on the interval from
\(30\)~to~\(43\). By using the \Erlang program above, we quickly find
that insertion sort is first beaten by bottom\hyp{}up merge sort at
\(n=36\). This suggests to drop \fun{duo/1} in favour of a function
that constructs chunks of \(35\)~keys from the original stack, then
sorts them using balanced 2-way insertions and, finally, if there are
more than \(35\)~keys, starts merging those sorted stacks. This
improvement amounts to not constructing the first \(35\)~levels in the
merge tree\index{tree!merge $\sim$} but, instead, build the \(35\)th
level by insertions.

Despite the previous analysis, we should be aware that it relies on a
measure based on the number of function calls, which assumes that each
function call is indeed performed by the run\hyp{}time system (no
inlining), that all context switchings have the same duration, that
other operations take a negligible time in comparison, that cache,
jump predictions and instruction pipelining have no effect etc. Even
using the same compiler on the same machine does not exempt from
careful benchmarking.

\index{merge sort!vs. insertion sort|)}
\index{merge sort!bottom-up $\sim$|)}

\section{Online merge sort}
\label{sec:online}
\index{merge sort!online $\sim$|(}

Sorting algorithms can be distinguished depending on whether they
operate on the whole stack of keys, or key by key. The former are said
\emph{off\hyp{}line}, as keys are not sorted while they are coming in,
and the latter are called \emph{online}, as the sorting process
can be temporally interleaved with the input process. Bottom\hyp{}up
merge sort is an off\hyp{}line algorithm, but it can be easily
modified to become online by remarking that balanced mergers
can be repeated whenever a new key arrives, and the unbalanced
mergers are performed only when the sorted stack is required.

More precisely, consider again \fig~\ref{fig:msort_gen}
\vpageref{fig:msort_gen} without the unbalanced mergers. The addition
of another key (at the right) yields two cases: if~\(n\) is even, that
is, \(e_0>0\), then nothing is done as the key becomes a singleton,
sorted stack of length \(2^0\); otherwise, a cascade of mergers
between stacks of identical lengths \(2^{e_i}\), with \(e_i=i\), is
triggered until \(e_j > j\). This is exactly the binary addition
of~\(1\) to~\(n\), except that mergers, instead of bitwise additions,
are performed as long as a carry is issued and propagated.

To our knowledge, only \cite{Okasaki_1998a} mentions this variant; he
shows that it can be efficiently implemented with purely functional
data structures, just as the off\hyp{}line version. (Notice that his
context is nevertheless different from ours as he relies on lazy
evaluation and amortised analysis.) Online merge sort is used in the
standard library of the proof assistant \index{Coq@\textsf{Coq}}
\textsf{Coq} \citep{BertotCasteran_2004}.

Our code is shown in \fig~\ref{fig:oms}\index{merge sort!online
  $\sim$!program}.
\begin{figure}
\begin{equation*}
\boxed{%
\begin{array}{r@{\;}l@{\;}l}
\fun{oms}(s)   & \xrightarrow{\smash{\phi}}
               & \fun{unb}(\fun{sum}(s,\el),\el).\\
\\
\fun{sum}(\el,t)         & \xrightarrow{\smash{\chi}} & t;\\
\fun{sum}(\cons{x}{s},t) & \xrightarrow{\smash{\psi}}
                         & \fun{sum}(s,\fun{add}([x],t)).\\
\\
\fun{add}(s,\el) & \xrightarrow{\smash{\omega}} & [\fun{one}(s)];\\
\fun{add}(s,\cons{\fun{zero}()}{t})
                    & \xrightarrow{\smash{\gamma}}
                    & \cons{\fun{one}(s)}{t};\\
\fun{add}(s,\cons{\fun{one}(u)}{t}) & \xrightarrow{\smash{\delta}}
                    & \cons{\fun{zero}()}{\fun{add}(\fun{mrg}(s,u),t)}.\\
\\
\fun{unb}(\el,u) & \xrightarrow{\smash{\mu}} & u;\\
\fun{unb}(\cons{\fun{zero}()}{s},u)
                 & \xrightarrow{\smash{\nu}} & \fun{unb}(s,u);\\
\fun{unb}(\cons{\fun{one}(t)}{s},u)
                 & \xrightarrow{\smash{\xi}}
                 & \fun{unb}(s,\fun{mrg}(t,u)).\\
\\
\fun{mrg}(\el,t)         & \xrightarrow{\smash{\theta}} & t;\\
\fun{mrg}(s,\el)         & \xrightarrow{\smash{\iota}} & s;\\
\fun{mrg}(\cons{x}{s},\cons{y}{t}) & \xrightarrow{\smash{\kappa}}
                         & \cons{y}{\fun{mrg}(\cons{x}{s},t)},\;
                           \text{if \(x \succ y\)};\\
\fun{mrg}(\cons{x}{s},t) & \xrightarrow{\smash{\lambda}}
                         & \cons{x}{\fun{mrg}(s,t)}.
\end{array}}
\end{equation*}
\caption{Online merge sort with \fun{oms/1}}
\label{fig:oms}
\end{figure}
We use \(\fun{zero}()\)\index{zero@\fun{zero/0}} to represent a
\(0\)-bit in the binary notation of the number of currently sorted
keys. Dually, the call \(\fun{one}(s)\)\index{one@\fun{one/1}} denotes
a \(1\)-bit, where the stack~\(s\) holds a number of sorted keys equal
to the associated power of two in the binary notation. Each call to
\fun{one/1} corresponds to a subtree\index{tree!merge $\sim$} in
\fig~\vref{fig:msort_gen}. For instance, \([\fun{one}([4]),
\fun{zero}(), \fun{one}([3,6,7,9])]\) corresponds to the binary number
\((101)_2\), hence the stack holds \(1 \cdot 2^2 + 0 \cdot 2^1 + 1
\cdot 2^0 = 5\) keys in total. Keep in mind that the bits are reversed
in the stack, as the subsequent processing of key~\(5\) would yield
\([\fun{zero}(),\fun{one}([4,5]),\fun{one}([3,6,7,9])]\).

Note that the program in \fig~\vref{fig:oms} does not capture the
normal use case of online merge sort, as, in practice, the
argument~\(s\) of the call \(\fun{oms}(s)\)\index{oms@\fun{oms/1}}
would not be known in its entirety, so
\fun{add/2}\index{add@\fun{add/2}} would only be called whenever a key
becomes available. In the following analysis, however, we are
interested in the number of comparisons of a sequence of updates
by~\fun{sum/2}\index{sum@\fun{sum/2}} (a framework we used in
section~\ref{sec:queueing}), followed by a series of unbalanced
mergers by~\fun{unb/2}\index{unb@\fun{unb/2}} (\emph{unbalanced}) in
order to obtain a sorted stack; therefore, our program is suitable
because we do want to assess~\(\OC{\fun{oms}}{n}\)\index{oms@$\OC{\fun{oms}}{n}$}.

Let us note \(\OC{\fun{add}}{n}\)\index{add@$\OC{\fun{add}}{n}$} the
number of comparisons to add a new key to a current stack of
length~\(n\) and recall that
\(\OC{\fun{mrg}}{m,n}\)\index{mrg@$\OC{\fun{mrg}}{m,n}$} is the number
of comparisons to merge two stacks of lengths \(m\)~and~\(n\) by
calling \fun{mrg/2}\index{mrg@\fun{mrg/2}}. If \(n\)~is even, then
there are no comparisons, as this is similar to adding~\(1\) to a
binary sequence \((\Xi{0})_2\), where \(\Xi\)~is an arbitrary bit
string. Otherwise, a series of balanced mergers of size~\(2^i\) are
performed, as this is dual to adding~\(1\) to \((\Xi{011}\ldots
1)_2\), where \(\Xi\)~is arbitrary. Therefore
\begin{equation*}
%\abovedisplayskip=2pt
%\belowdisplayskip=2pt
\OC{\fun{add}}{2j} = 0,\qquad
\OC{\fun{add}}{2j-1} = \sum_{i=0}^{\rho_{2j}}{\OC{\fun{mrg}}{2^i,2^i}},
\end{equation*}
where \(\rho_n\)~is the highest power of~\(2\) dividing~\(n\) (ruler
function)\index{ruler function}.  Let
\(\OC{\fun{sum}}{n}\)\index{sum@$\OC{\fun{sum}}{n}$} be the number of
comparisons to add \(n\)~keys to the empty stack~\(\el\). That is to
say, we have the following:
\begin{gather}
\abovedisplayskip=0pt
\belowdisplayskip=0pt
\OC{\fun{sum}}{n} = \sum_{k=0}^{n-1}{\OC{\fun{add}}{k}}.\notag\\[0mm]
\OC{\fun{sum}}{2p} = \OC{\fun{sum}}{2p+1}
= \sum_{k=1}^{2p-1}{\OC{\fun{add}}{k}}
= \sum_{j=1}^{p}{\OC{\fun{add}}{2j-1}}
= \sum_{j=1}^{p}\sum_{i=0}^{1+\rho_{j}}{\OC{\fun{mrg}}{2^i,2^i}}.
\label{eq:sum_2p}
\end{gather}
From~\eqref{eq:C_unbal}, the number of comparisons of the
unbalanced mergers is
\begin{equation}
\abovedisplayskip=0pt
\belowdisplayskip=0pt
\OC{\fun{unb}}{n}
= \OC{\ltimes}{n}
= \sum_{i=1}^{r}{\OC{\fun{mrg}}{2^{e_i},2^{e_{i-1}}+\dots+2^{e_0}}}.
\index{unb@$\OC{\fun{unb}}{n}$}
\label{eq:cost_unb}
\end{equation}
Let \(\OC{\fun{oms}}{n}\)\index{oms@$\OC{\fun{oms}}{n}$} the number of
comparisons to sort \(n\)~keys online. We have
\begin{equation}
\OC{\fun{oms}}{n} = \OC{\fun{sum}}{n} + \OC{\fun{unb}}{n}.
\index{sum@$\OC{\fun{sum}}{n}$}
\label{eq:ocost_online}
\end{equation}

%\addcontentsline{toc}{subsection}{Cost}
\paragraph{Minimum cost}
\index{merge sort!online $\sim$!minimum cost|(}

Replacing~\(\Cost\) by~\(\Best\) in equation~\eqref{eq:sum_2p}, we
obtain the equations for the minimum number of comparisons, allowing
us to simplify~\(\OB{\fun{sum}}{n}\)\index{sum@$\OB{\fun{sum}}{n}$}
with the help of equation~\eqref{eq:best_merge}
\vpageref{eq:best_merge}:
\begin{equation}
\OB{\fun{sum}}{2p}
  = \OB{\fun{sum}}{2p+1}
  = \sum_{j=1}^{p}\sum_{i=0}^{1+\rho_{j}}{\OB{\fun{mrg}}{2^i,2^i}}
  = \sum_{j=1}^{p}\sum_{i=0}^{1+\rho_{j}}{2^{i}}
  = 4\sum_{j=1}^{p}{2^{\rho_{j}}} - p.
\label{eq:B_oplus}
\end{equation}
Let \(T_p := \sum_{j=1}^{p}{2^{\rho_{j}}}\). The recurrences on the
ruler function\index{ruler function} \eqref{eq:ruler}
\vpageref{eq:ruler} help us in finding a recurrence for~\(T_p\) as
follows:
\begin{align*}
T_{2q} &= \sum_{k=0}^{\smash[t]{q-1}}{2^{\rho_{2k+1}}} +
\sum_{k=1}^{q}{2^{\rho_{2k}}} = q + 2 \cdot T_{q},\\[2mm]
T_{2q+1}
&= \sum_{j=1}^{\smash[t]{2q+1}}{2^{\rho_{j}}} = 1 + T_{2q} = (q + 1) +
2 \cdot T_{q}.
\end{align*}
Equivalently, \(T_{p} = 2 \cdot T_{\floor{p/2}} + \ceiling{p/2} = 2
\cdot T_{\floor{p/2}} + p - \floor{p/2}\). Therefore, unravelling a
few terms of the recurrence quickly reveals the equation
\begin{equation*}
2 \cdot T_p = 2p + \sum_{j=1}^{\floor{\lg p}}
            {\left\lfloor{\frac{p}{2^j}}\right\rfloor 2^j},
\end{equation*}
using Theorem~\vref{thm:floors}. By definition, \(\{x\} := x -
\floor{x}\), thus
\begin{equation*}
2 \cdot T_p = p\floor{\lg p} + 2p - \sum_{j=1}^{\floor{\lg
    p}}\left\lbrace\frac{p}{2^j}\right\rbrace 2^j.
\end{equation*}
Using \(0 \leqslant \{x\} < 1\), we obtain the bounds
\begin{equation*}
p\floor{\lg p} + 2p - 2^{\floor{\lg p}+1} + 2 < 2 \cdot T_p \leqslant
p\floor{\lg p} + 2p.
\end{equation*}
Furthermore, \(x - 1 < \floor{x} \leqslant x\) and \(\floor{x} = x -
\{x\}\), therefore
\begin{align*}
p(\lg p - \{\lg p\}) + 2p - 2^{\lg p - \{\lg p\} +
  1} + 2 < 2 \cdot T_p &\leqslant p\lg p + 2p,\\
p\lg p + 2p + 2 - p \cdot \theta_L(\{\lg p\}) < 2 \cdot T_p
&\leqslant p\lg p + 2p,
\end{align*}
with \(\theta_L(x) := x + 2^{1 - x}\). Since \(\max_{0 \leqslant x <
  1}\theta_L(x) = \theta_L(0) = 2\), we conclude:
\begin{equation*}
p\lg p + 2 < 2 \cdot T_p \leqslant p\lg p + 2p.
\end{equation*}
The upper bound is tight if \(p=2^q\). Applying these bounds to the
definition of~\(\OB{\fun{sum}}{2p}\) in~\eqref{eq:B_oplus} yields
\begin{equation}
2p\lg p - p + 4 < \OB{\fun{sum}}{2p} \leqslant 2p\lg p + 3p.
\label{ineq:B_oplus_2p}
\end{equation}
Consequently, \(\OB{\fun{sum}}{2p} = \OB{\fun{sum}}{2p+1} \sim 2p\lg
p\), hence \(\OB{\fun{sum}}{n} \sim n\lg n\).\index{sum@$\OB{\fun{sum}}{n}$}

\bigskip

\noindent Equation~\eqref{eq:best_merge} and~\eqref{eq:cost_unb} imply
\(\OB{\fun{unb}}{n} = \sum_{i=1}^{r}{\min\{2^{e_i},2^{e_{i-1}} + \dots
  + 2^{e_0}\}}\)\index{unb@$\OB{\fun{unb}}{n}$|(}. Let us commence by
noting that
\begin{equation*}
  \sum_{j=0}^{i}{2^{e_j}} \leqslant \sum_{j=0}^{e_i}{2^j} = 2 \cdot
  2^{e_i} - 1.
\end{equation*}
This is equivalent to a given binary number being always lower than or
equal to the number with the same number of bits all set to~\(1\), for
example, \((10110111)_2 \leqslant (11111111)_2\). By definition
of~\(e_i\), we have \(e_{i-1} + 1 \leqslant e_i\), so
\begin{equation*}
  \sum_{j=0}^{i-1}{2^{e_j}} \leqslant 2^{e_{i-1}+1} - 1 \leqslant 2^{e_i} - 1 < 2^{e_i}
\end{equation*}
and \(\min\{2^{e_i},2^{e_{i-1}} + \dots + 2^{e_0}\} = 2^{e_{i-1}} +
\dots + 2^{e_0}\). We have now
\begin{equation}
\OB{\fun{unb}}{n} = \sum_{i=1}^{r}\sum_{j=0}^{i-1}{2^{e_j}} < n.
\label{ineq:OBunb}
\end{equation}
Trivially, \(0 < \OB{\fun{unb}}{n}\), so
equation~\eqref{eq:ocost_online} entails \(\OB{\fun{oms}}{n}
\sim n\lg n \sim 2 \cdot \OB{\fun{bms}}{n}\).
\index{merge sort!online $\sim$!minimum cost|)}
\index{unb@$\OB{\fun{unb}}{n}$|)}

\paragraph{Maximum cost}
\index{merge sort!online $\sim$!maximum cost|(}

Replacing~\(\Cost\) by~\(\Worst\) in equation~\eqref{eq:sum_2p}
\vpageref{eq:sum_2p}, we obtain equations for the maximum number of
comparisons, which we can simplify with the help of
equation~\eqref{eq:worst_merge} \vpageref{eq:worst_merge} into
\begin{equation}
\OW{\fun{sum}}{2p}
  = \OW{\fun{sum}}{2p+1}
  = \sum_{j=1}^{p}\sum_{i=0}^{1+\rho_{j}}{\OW{\fun{mrg}}{2^i,2^i}}
%  = \sum_{j=1}^{p}\sum_{i=0}^{1+\rho_{j}}{(2^{i+1}-1)}\\
  = 8\sum_{j=1}^{p}{2^{\rho_j}} - \sum_{j=1}^{p}\rho_j - 4p.
\label{eq:OW_sum_2p}
\index{sum@$\OW{\fun{sum}}{n}$}
\end{equation}
\index{bit sum|(} \index{ruler function|(} We can reach a closed form
for \(\sum_{j=1}^{p}\rho_j\) if we think of the carry propagation and
the number of~\(1\)-bits when adding~\(1\) to a binary number (since
\(j\)~ranges over successive integers). This amounts to finding a
relationship between \(\rho_j\), \(\rho_{j+1}\), \(\nu_j\) and
\(\nu_{j+1}\). Let us assume that \(2n+1 = (\Xi 01^a)_2\), where
\(\Xi\)~is an arbitrary bit string and \((1^a)_2\)~is a 1-bit string
of length~\(a\). Then \(\nu_{2n+1} = \nu_{\Xi} + a\) and \(\rho_{2n+1}
= 0\). The next integer is \(2n+2 = (\Xi 10^a)_2\), so \(\nu_{2n+2} =
\nu_{\Xi} + 1\) and \(\rho_{2n+2} = a\). Now, we can relate
\(\rho\)~and~\(\nu\) by means of~\(a\):
\begin{equation*}
  \rho_{2n+2} = \nu_{2n+1} - \nu_{\Xi} = \nu_{2n+1} - (\nu_{2n+2} - 1)
  = 1 + \nu_{2n+1} - \nu_{2n+2}.
\end{equation*}
We can check now that the same pattern also works
for~\(\rho_{2n+1}\) by simply using the definitions of
\(\rho\)~and~\(\nu\): \(\rho_{2n+1} = 1 + \nu_{2n} -
\nu_{2n+1}\). This achieves to establish, for any integer \(n>0\),
that \(\rho_n = 1 + \nu_{n-1} - \nu_{n}\). Summing on both sides
yields
\begin{equation*}
\sum_{j=1}^{p}{\rho_j} = p - \nu_p.
\end{equation*}
Interestingly, we already met \(p - \nu_p\) \index{bit
  sum|)}\index{ruler function|)} in equation~\eqref{eq:ruler_nu},
\vpageref{eq:ruler_nu}. We can now further
simplify~\eqref{eq:OW_sum_2p} as follows:
\begin{equation*}
\OW{\fun{sum}}{2p}
 = \OW{\fun{sum}}{2p+1}
 = 8\sum_{j=1}^{p}{2^{\rho_j}} - 5p - \nu_p
 = 2 \cdot \OB{\fun{sum}}{2p} - 3p - \nu_p.
\index{sum@$\OW{\fun{sum}}{n}$}
\end{equation*}
Reusing the bounds on~\(\OB{\fun{sum}}{2p}\)
in~\eqref{ineq:B_oplus_2p} leads to \(\OW{\fun{sum}}{2p} =
\OW{\fun{sum}}{2p+1} \sim 4p\lg p\). Equations~\eqref{eq:worst_merge}
and~\eqref{eq:cost_unb} and inequation~\eqref{ineq:OBunb} imply
\begin{equation*}
\OW{\fun{unb}}{n} = \sum_{i=1}^{r}\sum_{j=0}^{i}{2^{e_j}} - \nu_n + 1
                  = \OB{\fun{unb}}{n} + n - \rho_n - \nu_n + 1
                  < 2n + 1.
\end{equation*}
Therefore, \(\OW{\fun{oms}}{n} \sim 2n\lg n \sim 2
\cdot \OW{\fun{bms}}{n}\).
\index{merge sort!online $\sim$!maximum cost|)}

\paragraph{Additional cost}

Let us account now for all the rewrites in the evaluation of a call
\(\fun{oms}(s)\). Let
\(\C{\fun{oms}}{n}\)\index{oms@$\C{\fun{oms}}{n}$} be this number. We
already know the contribution due to the comparisons,
\(\OC{\fun{oms}}{n}\)\index{oms@$\OC{\fun{oms}}{n}$}, either in
rule~\(\kappa\) or~\(\lambda\), so let us assess \(\C{\fun{oms}}{n} -
\OC{\fun{oms}}{n}\):
\begin{itemize}

  \item Rule~\(\phi\) is used once.

  \item Rules \(\chi\)~and~\(\psi\) are involved in the subtrace
    \(\psi^n\chi\), hence are used \(n+1\)~times.

  \item Rules \(\omega\), \(\gamma\) and~\(\delta\) are used \(F(n) =
    2n - \nu_n\) times, as seen in equation~\eqref{eq:ruler_nu}. We
    also must account for the rules \(\theta\)~and~\(\iota\) requested
    by the calls \(\fun{mrg}(s,u)\) in rule~\(\delta\). Each \(1\)-bit
    in the binary notations of the numbers from \(1\)~to~\(n-1\)
    triggers such a call, that is, \(\sum_{k=1}^{n-1}\nu_k\).

  \item Rules \(\nu\)~and~\(\xi\) are used for each bit in the binary
    notation of~\(n\) and rule~\(\mu\) is used once, making up
    \(\floor{\lg n} + 2\) calls. We also need to add the number of
    calls \(\fun{mrg}(t,u)\)\index{mrg@\fun{mrg/2}} in rule~\(\xi\),
    witnessing the application of rules \(\theta\)~and~\(\iota\). This
    is the number of \(1\)-bits in~\(n\), totalling~\(\nu_n\).

\end{itemize}
In total, we have \(\C{\fun{oms}}{n} - \OC{\fun{oms}}{n} = 3n +
\floor{\lg n} + \sum_{k=1}^{n-1}\nu_k + 2\). Equation~\eqref{eq:OBbms}
\vpageref{eq:OBbms} entails \(\C{\fun{oms}}{n} = \OC{\fun{oms}}{n} +
3n + \floor{\lg n} + \OB{\fun{bms}}{n} +
2\). Bounds~\eqref{ineq:bounds_Btms} \vpageref{ineq:bounds_Btms} imply
\(\OB{\fun{bms}}{n} \sim \tfrac{1}{2}n\lg n\), thus \(\C{\fun{oms}}{n}
\sim \OC{\fun{oms}}{n}\).  \index{merge sort|)} \index{merge
  sort!online $\sim$|)}

\section*{Exercises}

\begin{enumerate}

  \item Prove \(\fun{mrg}(s,t) \equiv \fun{mrg}(t,s)\).

  \item Prove that \(\fun{mrg}(s,t)\)\index{mrg@\fun{mrg/2}} is a sorted
  stack if \(s\)~and~\(t\) are sorted.

  \item Prove that all the keys of~\(s\) and~\(t\) are in
  \(\fun{mrg}(s,t)\).

  \item Prove the termination of \(\fun{bms/1}\), \(\fun{oms/1}\) and
    \(\fun{tms/1}\).\index{bms@\fun{bms/1}}\index{oms@\fun{oms/1}}
    \index{tms@\fun{tms/1}}

  \item Is \fun{bms/1} stable? What about \fun{tms/1}?

  \item Find \(\C{\fun{tms}}{n} - \OC{\fun{tms}}{n}\). \emph{Hint:}
    mind equation~\eqref{eq:ruler_nu} \vpageref{eq:ruler_nu}.

  \item Page~\pageref{eq:bms_merges}, we found that the number of
    mergers of \(\fun{bms}(s)\) is~\(n-1\) if~\(n\) is the number of
    keys in~\(s\). Show that \(\fun{tms}(s)\) performs the same number
    of mergers. (\emph{Hint}: Consider equation~\eqref{eq:cost_tms}
    \vpageref{eq:cost_tms}.)

  \item Find a counting argument on the table of \fig~\vref{fig:bits}
    showing that
    \begin{equation*}
      \sum_{k=1}^{p-1}{2^{\rho_k}}
      = \sum_{i=0}^{\ceiling{\lg p}-1}%
      {\left\lceil\frac{p-2^i}{2^{i+1}}\right\rceil 2^i}.
    \end{equation*}

  \item Compare the number of (\texttt{|})-nodes created by
    \fun{bms/1} and \fun{tms/1}.

\end{enumerate}
