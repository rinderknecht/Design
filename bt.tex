\chapter{Binary Trees}
\label{binary tree}

In this chapter, we focus on the binary tree as a data structure in
itself, redefining it on the way and introducing related classical
algorithms and measures.

\Fig~\vref{fig:bt_shape} displays a binary tree as an example.
\begin{figure}
\centering
\includegraphics{bt_shape}
\caption{A binary tree}
\label{fig:bt_shape}
\end{figure}
Nodes are of two kinds: internal\index{binary tree!internal node|(}
(\(\circ\) and \(\bullet\)) or external\index{binary tree!external
  node|(} (\(\scriptstyle \Box\)). The characteristic feature of a
binary tree is that internal nodes are downwardly connected to two
nodes, called \emph{children}\index{tree!node!child}, whilst external
nodes have no such links. The root\index{tree!node!root} is the
topmost internal node, represented with a circle and a diameter.
\emph{Leaves}\index{binary tree!leaf} are internal nodes whose
children are two external nodes; they are depicted as black discs.

% Wrapping figure better declared before a paragraph
%
\begin{wrapfigure}[8]{r}[0pt]{0pt}
% [8] vertical lines
% {r} mandatory right placement
% [0pt] of margin overhang
\centering
\includegraphics[bb=71 648 177 710]{leaf_tree}%723]{leaf_tree}
\caption{A leaf tree}
\label{fig:leaf_tree}
\end{wrapfigure}
Internal nodes are usually associated with some kind of information,
whilst external nodes are not, like the one shown in
\fig~\vref{fig:bt_ex1}:
\begin{figure}[b]
\centering
\subfloat[Extended binary tree\label{fig:bt_ex1}]{
\includegraphics{bt_ex1}}
\qquad
\subfloat[Pruned binary tree\label{fig:bt_ex2}]{
\includegraphics[bb=67 645 169 721]{bt_ex2}}
\caption{Two representations of a binary tree}
\label{fig:bt_ex}
\end{figure}
this is the default representation we use in the present
book. Sometimes, in order to draw more attention to the internal
nodes, the external nodes may be omitted, as in
\fig~\vref{fig:bt_ex2}. Moreover, only the leaves might then carry
information, in which case the tree is a \emph{leaf tree}\index{binary
  tree!leaf tree}, as in \fig~\vref{fig:leaf_tree}. Another variant
sees all nodes carrying some piece of data, as was the case with
comparison trees (refer for example to \fig~\vref{fig:cmp_tree}, where
external nodes contain permutations and internal nodes hold
comparisons of keys).

As seen in the study of sorting algorithms optimal in average,
\vpageref{par:minimax}, an \emph{external path}\index{tree!external
  path} is a path from the root to an external
node\index{tree!node!external $\sim$} and the
\emph{height}\index{tree!height} of a tree is the length of the
maximum external paths. For example, the height of the binary tree in
\fig~\vref{fig:bt_shape} is~\(5\) and there are two external paths of
maximum length. A path from the root to an internal node is an
\emph{internal path}\index{tree!internal path}. The \emph{internal
  path length of a tree}\index{tree!internal path length} is the sum
of the lengths of all its internal paths. We already have met the
\emph{external path length of a tree}\index{tree!external path
  length}, \vpageref{external_path_length}, which is the sum of the
lengths of all its external paths. (The length of a path is the number
of its edges.)

\paragraph{Warning}

Some authors use a different nomenclature for the definition of the
leaves and height. It is also common to stumble upon the concept of
\emph{depth}\index{binary tree!depth} of a tree, which can be mistaken
for its height, the former counting the number of nodes on a maximum
path.
\begin{thm}%[Internal and external nodes]
\label{thm:int_ext}
\textsl{A binary tree with \(n\)~internal nodes has \(n+1\)~external
  nodes.}
\end{thm}
\begin{proof}
Let \(e\)~be the number of external nodes to be determined. We can
count the edges in two complementary ways. Top\hyp{}down, we see that
each internal node has exactly two children, so \(l=2n\), where
\(l\)~is the number of edges. Bottom\hyp{}up, we see that each node
has exactly one parent, except the root, which as none. Therefore,
\(l=(n+e)-1\). Identifying the two values of~\(l\) yields \(e=n+1\).
\end{proof}

\paragraph{Data structure}

There are many ways to represent a binary tree as a data
structure. First, we can remark that, just as a stack can be empty or
not, there are two kinds of nodes, internal or external, and the empty
tree can be identified to an external node. Therefore, we only need
two data constructors, say \fun{ext/0}\index{ext@\fun{ext/0}} for
external nodes, and \fun{int/3}\index{int@\fun{int/3}} for internal
nodes. The latter applies to three arguments because two children are
expected, as well as some information. The order of these arguments
may differ. Because we may see an internal node to stand,
horizontally, between its subtrees \(t_1\)~and~\(t_2\), we may prefer
to write \(\fun{int}(t_1,x,t_2)\). (Semitic readers might wish to swap
\(t_1\)~and~\(t_2\).) Alternatively, we may consider that an internal
node lies, vertically, before its subtrees, in which case we might
prefer to write \(\fun{int}(x,t_1,t_2)\). The latter makes typing or
handwriting small trees easier, in particular for
testing\index{testing} purposes. For example, the binary tree of
\fig~\vref{fig:bt_ex1} formally corresponds to \(\fun{int}(8, t_1,
\fun{int}(3, \fun{ext}(), \fun{ext}(2, \fun{ext}(), \fun{ext}())))\),
where \(t_1=\fun{int}(1, \fun{ext}(3, \fun{ext}(), \fun{int}(5,
\fun{ext}(), \fun{ext}())), \fun{ext}(9, \fun{ext}(),
\fun{ext}()))\). We will use henceforth the convention
\(\fun{int}(x,t_1,t_2)\), sometimes called \emph{prefix
  notation}.\index{binary tree!prefix notation}\index{binary
  tree!internal node|)}\index{binary tree!external node|)}

The \emph{size}\index{binary tree!size} of a binary tree is the number
of its internal nodes. It is the most common measure used on trees
when expressing costs of functions. As a warm\hyp{}up exercise, let us
write a program computing the size of a given binary tree:
\begin{equation}
\fun{size}(\fun{ext}()) \rightarrow 0;\quad
\fun{size}(\fun{int}(x,t_1,t_2))
  \rightarrow 1 + \fun{size}(t_1) + \fun{size}(t_2).
\index{size@\fun{size/1}}
\label{eq:size}
\end{equation}
Notice the similarity with computing the length of a stack:
\begin{equation*}
\fun{len}(\fun{nil}()) \rightarrow 0;\quad
\fun{len}(\fun{cons}(x,s)) \rightarrow 1 + \fun{len}(s).
\index{len@\fun{len/1}}
\index{nil@\fun{nil/0}}
\index{cons@\fun{cons/2}}
\end{equation*}
The difference lies in that two recursive calls are needed to visit
all the nodes of a binary tree, instead of one for a stack. This
bidimensional topology gives rise to many kinds of visits, called
\emph{walks} or \emph{traversals}.

\section{Traversals}
\index{tree!traversal}
\index{tree!walk|see{traversal}}
\label{sec:traversals}

In this section, we present the classic traversals of binary trees,
which are distinguished by the order in which the data stored in the
internal nodes is pushed onto a stack originally empty.

\mypar{Preorder}
\index{binary tree!preorder|(}
\label{preorder}

A \emph{preorder}\index{binary tree!preorder} traversal of a
non\hyp{}empty binary tree consists in having in a stack the root
(recursively, it is the current internal node), followed by the nodes
in preorder of the left subtree and, finally, the nodes in preorder of
the right subtree. (For the sake of brevity, we will identify the
contents of the nodes with the nodes themselves, when there is no
ambiguity.) For example, the (contents of the) nodes in preorder of
the tree in \fig~\vref{fig:bt_ex1} are \([8,1,3,5,9,3,2]\). Because
this method first visits the children of a node before its
sibling\index{tree!node!sibling} (two internal nodes are siblings if
they have the same parent), it is a
\emph{depth\hyp{}first}\index{binary tree!depth-first traversal}
traversal. A simple program is
\begin{align}
\fun{pre}_0(\fun{ext}()) &\xrightarrow{\smash{\gamma}} \el; \nonumber\\
\fun{pre}_0(\fun{int}(x,t_1,t_2)) &\xrightarrow{\smash{\delta}}
\cons{x}{\fun{cat}(\fun{pre}_0(t_1),\fun{pre}_0(t_2))}.
\label{eq:pre0}
\index{pre0@\fun{pre\(_0\)/1}}
\end{align}
We used the concatenation\index{stack!concatenation} on stacks provided by
\fun{cat/2}\index{cat@\fun{cat/2}}, defined in~\eqref{def:cat}
\vpageref{def:cat}, to order the values of the subtrees. We know that
the cost of \fun{cat/2} is linear in the size of its first argument:
\(\C{\fun{cat}}{p} := \Call{\fun{cat}(s,t)} = p+1\), where \(p\)~is
the length of~\(s\). Let
\(\C{\fun{pre}_0}{n}\)\index{pre0@$\C{\fun{pre}_0}{n}$} be the cost of
\(\fun{pre}_0(t)\), where \(n\)~is the number of internal nodes
of~\(t\). From the definition of \fun{pre\(_0\)/1}, we deduce
\begin{equation*}
\C{\fun{pre}_0}{0} = 1;\quad
\C{\fun{pre}_0}{n+1} =
  1 + \C{\fun{pre}_0}{p} + \C{\fun{pre}_0}{n-p} + \C{\fun{cat}}{p},
\index{pre0@$\C{\fun{pre}_0}{n}$}
\end{equation*}
where \(p\)~is the size of~\(t_1\). So \(\C{\fun{pre}_0}{n+1} =
\C{\fun{pre}_0}{p} + \C{\fun{pre}_0}{n-p} + p + 2\). This recurrence
belongs to a class called \emph{divide and conquer}
\index{design!big-step $\sim$} \index{divide and conquer|see{design,
    big-step $\sim$}} because it springs from strategies which
consists in splitting the input (here, of size \(n+1\)), recursively
applying the relevant solving strategy to the smaller parts (here, of
sizes~\(p\) and \(n-p\)) and finally combining the solutions of the
parts into a solution of the partition. The extra cost incurred by
combining smaller solutions (here, \(p+2\)) is called the
\emph{toll}\index{toll} and the closed form and asymptotic behaviour
of the solution to the recurrence crucially depends upon its kind.

In another context (see page~\pageref{big-step}), we,
idiosyncratically, called this strategy \emph{big\hyp{}step design}
\index{design!big-step $\sim$} because we wanted a convenient way to
contrast it with another sort of modelling which we called
\emph{small\hyp{}step design}.\index{design!small-step $\sim$} As a
consequence, we already have seen instances of `divide and conquer,'
for example, in the case of merge sort\index{merge sort} in
chapter~\vref{chap:merge_sort}, which often epitomises the concept
itself.

The maximum cost
\(\W{\fun{pre}_0}{k}\)\index{pre0@$\W{\fun{pre}_0}{n}$} satisfies the
extremal recurrence
\begin{equation}
\W{\fun{pre}_0}{0} = 1;\quad
\W{\fun{pre}_0}{k+1} =
  2 + \max_{0 \leqslant p \leqslant k}\{\W{\fun{pre}_0}{p}
                                  + \W{\fun{pre}_0}{k-p} + p\}.
\index{pre0@$\W{\fun{pre}_0}{n}$}
\label{eq:pre0_max}
\end{equation}
Instead of attacking frontally these equations, we can guess a
possible solution and check it. Here, we could try to consistently
choose \(p=k\), prompted by the idea that maximising the toll at each
node of the tree will perhaps lead to a total maximum (\emph{eager
  solving}). Thus, we envisage
\begin{equation}
\W{\fun{pre}_0}{0} = 1;\quad
\W{\fun{pre}_0}{k+1} = \W{\fun{pre}_0}{k} + k + 3.
\label{eq:max_pre0_rec}
\end{equation}
Summing both sides from \(k=0\) to \(k=n-1\) and simplifying yields
\begin{equation*}
\W{\fun{pre}_0}{n} = \frac{1}{2}(n^2 + 5n + 2) \sim \frac{1}{2}n^2.
\end{equation*}
At this point, we check whether this closed form satisfies
equation~\eqref{eq:pre0_max}. We have \(2(\W{\fun{pre}_0}{p} +
\W{\fun{pre}_0}{n-p} + p + 2) = 2p^2 + 2(1-n)p + n^2 + 5n + 8\). This
is the equation of a parabola whose minimum occurs at \(p = (n-1)/2\)
and maximum at \(p = n\), over the interval \([0,n]\). The maximum,
whose value is \(n^2 + 7n + 8\), equals \(2 \cdot
\W{\fun{pre}_0}{n+1}\), so the closed form satisfies the extremal
recurrence.

What does a binary tree maximising the cost of
\fun{pre\(_0\)/1}\index{pre0@\fun{pre\(_0\)/1}} look like? The toll
\(k+3\) in~\eqref{eq:max_pre0_rec} is a consequence of taking the
maximum cost of \fun{cat/2}\index{cat@\fun{cat/2}} at each node, which
means that all the internal nodes being concatenated come from the left
subtrees, the left subtree of the left subtree etc. so these nodes are
concatenated again and again while going up (that is, returning from the
recursive calls), leading to a quadratic cost. The shape of such a
tree is shown in \fig~\vref{fig:max_pre0}.
\begin{figure}[b]
\centering
\subfloat[Maximum\label{fig:max_pre0}]{
  \includegraphics{max_pre0}}
\qquad
\subfloat[Minimum\label{fig:min_pre0}]{
  \includegraphics{min_pre0}}
\caption{Extremal trees for \(\C{\fun{pre}_0}{n}\)}
\label{fig:tree_stack}
\end{figure}

Dually, the minimum cost
\(\B{\fun{pre}_0}{k}\)\index{pre0@$\B{\fun{pre}_0}{n}$} satisfies the
extremal recurrence
\begin{equation*}
\B{\fun{pre}_0}{0} = 1;\quad
\B{\fun{pre}_0}{k+1} =
  2 + \min_{0 \leqslant p \leqslant k}\{\B{\fun{pre}_0}{p}
                                  + \B{\fun{pre}_0}{k-p} + p\}.
\end{equation*}
Along the same line as before, but on the opposite direction, we may
try to minimise the toll by choosing \(p=0\), which means that all
external nodes, but one, are left subtrees. Consequently, we have
\begin{equation}
\B{\fun{pre}_0}{0} = 1;\quad
\B{\fun{pre}_0}{k+1} = \B{\fun{pre}_0}{k} + 3.
\label{eq:min_pre0_rec}
\end{equation}
Summing both sides from \(k=0\) to \(k=n-1\) and simplifying yields
\begin{equation*}
\B{\fun{pre}_0}{n} = 3n + 1 \sim 3n.
\end{equation*}
It is easy to check that this is indeed a solution
to~\eqref{eq:min_pre0_rec}. The shape of the corresponding tree is
shown in \fig~\vref{fig:min_pre0}. Note that both extremal trees are
isomorphic to a stack (that is, the abstract syntax
tree\index{tree!abstract syntax $\sim$} of a stack) and, as such, are
instances of \emph{degenerate trees}\index{binary tree!degenerate
  $\sim$}. Also, the maximum cost of
\fun{pre\(_0\)/1}\index{pre0@\fun{pre\(_0\)/1}} is quadratic, which
calls for some improvement.

Another big\hyp{}step design\index{design!big-step $\sim$} we may come
up with consists in not using \fun{cat/2}\index{cat@\fun{cat/2}} and
calling instead
\fun{flat/1}\index{stack!flattening}\index{flat@\fun{flat/1}}, defined
in \fig~\vref{fig:flat}, \emph{once at the end}. \Fig~\vref{fig:pre1}
shows that new version of the preorder traversal, named
\fun{pre\(_1\)/1}\index{pre1@\fun{pre\(_1\)/1}}.
\begin{figure}
\begin{equation*}
\boxed{
\begin{array}{r@{\;}c@{\;}l}
\fun{pre}_1(t) & \rightarrow & \fun{flat}(\fun{pre}_2(t)).\\
\\
\fun{pre}_2(\fun{ext}()) & \rightarrow & \el;\\
\fun{pre}_2(\fun{int}(x,t_1,t_2)) & \rightarrow &
  \cons{x,\fun{pre}_2(t_1)}{\fun{pre}_2(t_2)}.
\end{array}}
\end{equation*}
\caption{Preorder traversal using \fun{flat/1}}
\label{fig:pre1}
\end{figure}
The cost of \(\fun{pre}_2(t)\)\index{pre2@\fun{pre\(_2\)/1}} is now
reduced to \(2n+1\) (see theorem~\ref{thm:int_ext}
\vpageref{thm:int_ext}). On page~\pageref{cost:flat}, the cost of
\(\fun{flat}(s)\)\index{flat@\fun{flat/1}} was \(1 + n + \Omega +
\Gamma + L\), where \(n\)~is the length of
\(\fun{flat}(s)\)\index{flat@\fun{flat/1}}, \(\Omega\)~is the number
of empty stacks in~\(s\), \(\Gamma\)~is the number of non\hyp{}empty
stacks and \(L\)~is the sum of the lengths of the embedded stacks. The
value of~\(\Omega\) is \(n+1\) because this is the number of external
nodes. The value of~\(\Gamma\) is \(n-1\), because each internal node
yields a non\hyp{}empty stack by the second rule of
\fun{pre\(_2\)/1}\index{pre2@\fun{pre\(_2\)/1}} and the root is
excluded because we only count embedded stacks. The value of~\(L\) is
\(3(n-1)\) because those stacks have length~\(3\) by the same rule. In
the end, \(\Call{\fun{flat}(s)} = 6n - 2\), where \(\fun{pre}_2(t)
\twoheadrightarrow s\) and \(n\)~is the size of~\(t\). Finally, we
must account for the rule defining
\fun{pre\(_1\)/1}\index{pre1@\fun{pre\(_1\)/1}} and assess afresh the
cost incurred by the empty tree:
\begin{equation*}
\C{\fun{pre}_1}{0} = 3;\index{pre1@$\C{\fun{pre}_1}{n}$}
\quad
\C{\fun{pre}_1}{n} = 1 + (2n+1) + (6n - 2) = 8n,\;
\text{when \(n > 0\).}
\end{equation*}

Despite a significant improvement in the cost and the lack of extreme
cases, we should try a small\hyp{}step design\index{design!small-step
  $\sim$} before giving up. The underlying principle in this kind of
approach is to do as little as possible in each rule. Looking back at
\fun{pre\(_0\)/1}\index{pre0@\fun{pre\(_0\)/1}}, it is clear that the
root is correctly placed, but, without resorting to
\(\fun{pre}_3(t_1)\)\index{pre3@\fun{pre\(_3\)/1}} and
\(\fun{pre}_3(t_2)\) in the following canvas, what can be done
further?
\begin{equation*}
\fun{pre}_3(\fun{ext}()) \rightarrow \el;
\quad
\fun{pre}_3(\fun{int}(x,t_1,t_2)) \rightarrow
  \cons{x}{\fbcode{CCCCCCC}\,}.
\end{equation*}
The way forth is to think in terms of forests\index{tree!forest},
instead of single trees, because a forest is a stack of trees and, as
such, can also be used to accumulate trees\index{functional
  language!accumulator}. This is a common technique when processing
trees. See \fig~\vref{fig:pre3}.
\begin{figure}
\begin{equation*}
\boxed{
\begin{array}{r@{\;}c@{\;}l}
\fun{pre}_3(t) & \xrightarrow{\smash{\alpha}} & \fun{pre}_4([t]).\\
\\
\fun{pre}_4(\el) & \xrightarrow{\smash{\beta}} & \el;\\
\fun{pre}_4(\cons{\fun{ext}()}{f})
  & \xrightarrow{\smash{\gamma}} & \fun{pre}_4(f);\\
\fun{pre}_4(\cons{\fun{int}(x,t_1,t_2)}{f})
  & \xrightarrow{\smash{\delta}} &
  \cons{x}{\fun{pre}_4(\cons{t_1,t_2}{f})}.
\end{array}}
\end{equation*}
\caption{Efficient preorder traversal with a forest}
\label{fig:pre3}
\end{figure}
Empty trees in the forest are skipped in rule~\clause{\gamma}. In
rule~\clause{\delta}, the subtrees \(t_1\) and~\(t_2\) are now simply
pushed back onto the forest~\(f\), for later processing. This way,
there is no need to compute
\(\fun{pre}_4(t_1)\)\index{pre4@\fun{pre\(_4\)/1}} or
\(\fun{pre}_4(t_2)\) immediately. This method is slightly different
from using an accumulator which contains, at every moment, a partial
result or a reversed partial result. Here, no parameter is added but,
instead, a stack replaces the original parameter, and it does not
contain partial results, but pieces of the original tree from which to
pick internal nodes easily (the root of the first tree) in the
expected order. An example is given in \fig~\vref{fig:pre3_abcde},
\begin{figure}[!b]
\centering
\includegraphics[bb=71 650 374 710]{pre3_abcde_0} %[... ... ... 723]
\includegraphics[bb=71 668 306 721]{pre3_abcde_1}
\includegraphics{pre3_abcde_2}
\caption{A preorder traversal with \fun{pre\(_4\)/1}}
\label{fig:pre3_abcde}
\end{figure}
where the forest is the argument of \fun{pre\(_4\)/1} and the circle
nodes are the current value of~\(x\) in rule~\(\delta\) of
\fig~\ref{fig:pre3}.

\par\vskip\baselineskip

\noindent The cost of
\(\fun{pre}_3(t)\)\index{pre3@\fun{pre\(_3\)/1}}, where the size
of~\(t\) is~\(n\), is simple:
\begin{itemize*}

  \item rule \clause{\alpha} is used once;

  \item rule \clause{\beta} is used once;

  \item rule \clause{\gamma} is used once for each external node, that
    is, \(n+1\) times;% (see theorem~\ref{thm:int_ext}   \vpageref{thm:int_ext});

  \item rule \clause{\delta} is used once for each internal node, so
    \(n\)~times, by definition.

\end{itemize*}
In total, \(\C{\fun{pre}_3}{n} = 2n + 3 \sim
2n\)\index{pre3@$\C{\fun{pre}}{n}$}, which is a notable
improvement. The keen reader may remark that we could further reduce
the cost by not visiting the external nodes, as shown in
\fig~\ref{fig:pre5}.
\begin{figure}[t]
\begin{equation*}
\boxed{
\begin{array}{r@{\;}c@{\;}l}
\fun{pre}_5(t) & \rightarrow & \fun{pre}_6([t]).\\
\\
\fun{pre}_6(\el) & \rightarrow & \el;\\
\fun{pre}_6(\cons{\fun{ext}()}{f})
  & \rightarrow & \fun{pre}_6(f);\\
\fun{pre}_6(\cons{\fun{int}(x,\fun{ext}(),\fun{ext}())}{f})
  & \rightarrow & \cons{x}{\fun{pre}_6(f)};\\
\fun{pre}_6(\cons{\fun{int}(x,\fun{ext}(),t_2)}{f})
  & \rightarrow & \cons{x}{\fun{pre}_6(\cons{t_2}{f})};\\
\fun{pre}_6(\cons{\fun{int}(x, t_1, \fun{ext}())}{f})
  & \rightarrow & \cons{x}{\fun{pre}_6(\cons{t_1}{f})};\\
\fun{pre}_6(\cons{\fun{int}(x,t_1,t_2)}{f})
  & \rightarrow & \cons{x}{\fun{pre}_6(\cons{t_1,t_2}{f})}.
\end{array}}
\end{equation*}
\caption{Lengthy definition of a preorder traversal}
\label{fig:pre5}
\end{figure}
We then have
\begin{equation*}
\C{\fun{pre}_5}{n} = \C{\fun{pre}_3}{n} - (n+1) = n +
2.
\end{equation*}
\index{pre6@\fun{pre\(_6\)/1}}\index{pre5@\fun{pre\(_5\)/1}}\index{pre5@$\C{\fun{pre}_5}{n}$}
Despite the gain, the optimised program is significantly longer and
the right\hyp{}hand sides of the new rules are partial
evaluations\index{functional language!evaluation!partial $\sim$} of
rule~\clause{\delta}. The measure of the input we use for calculating
the costs does not include the abstract time needed to select the rule
to apply but it is likely, though, that the more patterns, the higher
this hidden penalty. In this book, we prefer to visit the external
nodes unless there is a logical reason not to do so, if only for the
sake of conciseness.

The total number of cons\hyp{}nodes\index{cons-node} created by the
rules~\(\alpha\) and~\(\delta\) is the total number of nodes,
\(2n+1\), but if we want to know how many there can be at any time, we
need to consider how the shape of the original tree influences the
rules~\(\gamma\) and~\(\delta\). In the best case,
\(t_1\)~in~\(\delta\) is \(\fun{ext}()\) and will be eliminated next
by rule~\clause{\gamma} without additional creation of nodes. In the
worst case, \(t_1\)~maximises the number of internal nodes on its left
branch. Therefore, these two configurations correspond to the extremal
cases for the cost of \fun{pre\(_0\)/1}\index{pre0@\fun{pre\(_0\)/1}}
in \fig~\vref{fig:tree_stack}. In the worst case, all the \(2n+1\)
nodes of the tree will be in the stack at one point, whilst, in the
best case, only two will be. The question of the average stack size
will be considered later in this text in relation with the average
height.

The distinction between big\hyp{}step design\index{design!big-step
  $\sim$} (or `divide and conquer'\index{divide and conquer}) and
small\hyp{}step design\index{design!small-step $\sim$} is not always a
clear cut and is mainly intended to be a didactical means. In
particular, we should not assume that there are only two possible
kinds of design for every given task. To bring further clarity to the
subject, let us use an heterogeneous approach to design another
version, \fun{pre/1}\index{pre@\fun{pre/1}}, which computes
efficiently the preorder traversal of a given binary tree. Looking
back at \fun{pre\(_0\)/1}\index{pre0@\fun{pre\(_0\)/1}}, we can
identify the source of the inefficiency in the fact that, in the worst
case,
\begin{equation*}
\fun{pre}_0(t) \twoheadrightarrow
\cons{x_1}{\fun{cat}(\cons{x_2}{\fun{cat}(\dots
    \fun{cat}(\cons{x_n}{\fun{cat}(\el, \el)},\el) \dots)})}
\index{cat@\fun{cat/2}}
\end{equation*}
where \(t = \fun{int}(x_1,\fun{int}(x_2,\dots, \fun{int}(x_n,
\fun{ext}(), \fun{ext}()), \dots, \fun{ext}()),\fun{ext}())\) is the
tree in \fig~\vref{fig:max_pre0}. We met this kind of partial rewrite
in formula~\eqref{eq:rev0}\index{rev0@\fun{rev\(_0\)/1}}
\vpageref{eq:rev0} and~\eqref{inv_isrt}
\vpageref{inv_isrt}\index{isrt@\fun{isrt/1}}\index{ins@\fun{ins/2}},
and we found that it leads to a quadratic cost.  Whilst the use of
\fun{cat/2}\index{cat@\fun{cat/2}} in itself is not the issue, but
rather the accumulation of calls to \fun{cat/2}\index{cat@\fun{cat/2}}
as their first argument, let us nevertheless seek out a definition not
relying on concatenation at all. This means that we want to build the
preorder stack by using exclusively pushes. Therefore, we must add an
auxiliary parameter, originally set to the empty stack, on which the
contents of the nodes are pushed in the proper order: \(\fun{pre}(t)
\rightarrow \fun{pre}(t,\el)\).\index{pre@\fun{pre/2}} Now, we should
wonder what the interpretation of this accumulator\index{functional
  language!accumulator} is when considering the pattern
\(\fun{pre}(t,s)\). Let us have a look at the internal node
\(t=\fun{int}(x,t_1,t_2)\) in \fig~\vref{fig:pre_tree}.
\begin{figure}
\centering
\subfloat[Preorder with \fun{pre/1}\label{fig:pre_tree}]{
  \includegraphics[bb=71 672 166 721]{pre_tree}}
\subfloat[Inorder with \fun{in/1}\label{fig:in_tree}]{
  \includegraphics[bb=71 672 183 721]{in_tree}}
\subfloat[Postorder with \fun{post/1}\label{fig:post_tree}]{
  \includegraphics[bb=71 672 183 721]{post_tree}}
\caption{Efficient classic traversals}
\label{fig:classic_walks}
\end{figure}
The arrows evince the traversal in the tree and connect different
stages of the preorder stack: a downwards arrow points to the argument
of a recursive call on the corresponding child; an upward arrow points
to the result of the call on the parent. For instance, the
subtree~\(t_2\) corresponds to the recursive call
\(\fun{pre}(t_2,s)\)\index{pre@\fun{pre/2}} whose value is
named~\(s_1\). Likewise, we have \(\fun{pre}(t_1,s_1)
\twoheadrightarrow s_2\), which is therefore equivalent to
\(\fun{pre}(t_1,\fun{pre}(t_2,s)) \twoheadrightarrow s_2\). Finally,
the root is associated with the evaluation \(\fun{pre}(t,s)
\twoheadrightarrow \cons{x}{s_2}\), that is, \(\fun{pre}(t,s) \equiv
\cons{x}{\fun{pre}(t_1,\fun{pre}(t_2,s))}\).\index{pre@\fun{pre/2}}
The rule for external nodes is not shown and simply consists in
letting the stack invariant. We can finally write the functional
program in \fig~\vref{fig:pre}.
\begin{figure}[b]
\begin{equation*}
\boxed{
\begin{array}{r@{\;}c@{\;}l}
\fun{pre}(t) & \xrightarrow{\smash{\theta}} & \fun{pre}(t,\el).\\
\\
\fun{pre}(\fun{ext}(),s) & \xrightarrow{\smash{\iota}} & s;\\
\fun{pre}(\fun{int}(x,t_1,t_2),s)
  & \xrightarrow{\smash{\kappa}}
  & \cons{x}{\fun{pre}(t_1,\fun{pre}(t_2,s))}.
\end{array}}
\end{equation*}
\caption{Cost and memory efficient preorder}
\label{fig:pre}
\end{figure}
We now understand that, given \(\fun{pre}(t,s)\), the nodes in the
stack~\(s\) are the nodes that follow, in preorder, the nodes in the
subtree~\(t\). The cost is extremely simple:
\begin{equation}
\C{\fun{pre}}{n} = 1 + (n+1) + n = 2n+2.\label{eq:pre}
\index{pre@$\C{\fun{pre}}{n}$}
\end{equation}
(Keep in mind that there are \(n+1\) external nodes when there are
\(n\) internal nodes.)

This variant is to be preferred over
\fun{pre\(_3\)/1}\index{pre3@\fun{pre\(_3\)/1}} because its
memory\index{memory} usage is lower: rule~\clause{\delta} in
\fig~\vref{fig:pre3} pushes \(t_1\) and~\(t_2\), thus allocates two
cons\hyp{}nodes\index{cons-node} per internal node, totalling \(2n\)
supplementary nodes. By contrast, \fun{pre/1}\index{pre@\fun{pre/1}}
creates none, but allocates \(n\)~call\hyp{}nodes
\fun{pre}\index{pre@\fun{pre/2}} (one per internal node), so the
advantage stands, albeit moot. Note that
\fun{pre\(_3\)/1}\index{pre3@\fun{pre\(_3\)/1}} is in tail
form\index{functional language!tail form}, but not \fun{pre/2}.

\paragraph{Preorder numbering}

\Fig~\vref{fig:preorder}
\begin{figure}
\centering
\subfloat[Preorder\label{fig:preorder}]{\includegraphics{preorder}}
\;
\subfloat[Postorder\label{fig:postorder}]{\includegraphics{postorder}}
\;
\subfloat[Inorder\label{fig:inorder}]{\includegraphics{inorder}}
\;
\subfloat[Level order\label{fig:lorder}]{\includegraphics{lorder}}
\caption{Classic numberings of a pruned binary tree}
\label{fig:orders}
\end{figure}
shows a binary tree whose internal nodes have been replaced by their
rank in preorder, with the smallest number, \(0\), at the root. In
particular, the preorder traversal of that tree yields
\([0,1,2,3,4,5,6]\). Producing such a tree from some initial tree is a
\emph{preorder numbering}\index{binary tree!preorder numbering}. A
complete example is shown in \fig~\vref{fig:bt_ex3},
\begin{figure}[b]
\centering
\includegraphics{bt_ex3}
\caption{Preorder numbers as exponents}
\label{fig:bt_ex3}
\end{figure}
where the preorder numbers are exponents to the internal nodes. Note
how these numbers increase along downwards paths.

Their generation can be tackled in two phases: first, we need to
understand how to produce the right number for a given node; second,
we need to use these numbers to build a tree. The scheme for the
former is shown on internal nodes in \fig~\vref{fig:prenum1}.
\begin{figure}[t]
\centering
\subfloat[Numbers only\label{fig:prenum1}]{%
\includegraphics[bb=71 668 170 721]{prenum1}}
\qquad
\subfloat[Numbers and tree\label{fig:prenum0}]{%
\includegraphics[bb=71 668 217 721]{prenum0}}
\caption{Preorder numbering in two phases}
\label{fig:prenum}
\end{figure}
A number on the left side of a node indexes it, for example node~\(x\)
is indexed with~\(i\): these numbers descend in the tree. A number on
the right side of a node is the smallest number not used in the
numbering of the subtrees attached to that node: these numbers ascend
and may be used for the numbering of another subtree. For instance,
\(j\)~is the smallest integer not numbering the nodes of the subtree
\(t_1\). External nodes do not change their incoming number and are
not depicted.

The second and final design phase consists in the construction of the
tree made of these numbers and it is pictured in
\fig~\vref{fig:prenum0}. Conceptually, it is the completion of the
first phase in the sense that upwards arrows, which denote values of
recursive calls on subtrees, now point at pairs whose first component
is the number we found earlier and the second component is a numbered
tree. As usual with recursive definitions, we assume that the
recursive calls on substructures are correct (that is, they yield the
expected values) and we infer the value of the call on the whole
structure at hand, here \(\pair{k}{\fun{int}(i,t'_1,t'_2)}\).

The function
\fun{npre/1}\index{npre@\fun{npre/1}} (\emph{number in preorder}) in
\fig~\vref{fig:npre}
\begin{figure}[t]
\abovedisplayskip=0pt
\belowdisplayskip=0pt
\centering
\framebox[0.75\columnwidth]{\vbox{%
\begin{gather*}
\inferrule
  {\fun{npre}(0,t) \twoheadrightarrow \pair{i}{t'}}
  {\fun{npre}(t) \twoheadrightarrow t'}\,.
\qquad
\fun{npre}(i,\fun{ext}()) \rightarrow \pair{i}{\fun{ext}()};
\\
\inferrule
  {\fun{npre}(i+1,t_1) \twoheadrightarrow \pair{j}{t'_1}
   \and
   \fun{npre}(j,t_2) \twoheadrightarrow \pair{k}{t'_2}}
  {\fun{npre}(i,\fun{int}(x,t_1,t_2)) \twoheadrightarrow
   \pair{k}{\fun{int}(i,t'_1,t'_2)}}\,.
\end{gather*}
}}
\caption{Preorder numbering}
\label{fig:npre}
\end{figure}
implements this algorithm, using a function
\fun{npre/2}\index{npre@\fun{npre/2}} such that \(\fun{npre}(i,t)
\twoheadrightarrow \pair{j}{t'}\), where \(t'\)~is the preorder
numbering of~\(t\), with root~\(i\), and \(j\)~is the smallest integer
not found in~\(t'\) (in other words, \(j-i\) is the size of~\(t\)
and~\(t'\)). This function is the one illustrated in
\fig~\vref{fig:prenum0}.

By the way, we should perhaps recall that inference
systems\index{induction!inference system}, first seen
\vpageref{par:infsys}, can be eliminated by the introduction of
auxiliary functions (one for each premise). In this instance, we could
equivalently write\index{snd@\fun{snd/1}} the program in
\fig~\vref{fig:npre_bis}.
\begin{figure}
\begin{equation*}
%\belowdisplayskip=0pt
\boxed{
\begin{array}{r@{\;}c@{\;}l}
\fun{npre}(t) & \rightarrow & \fun{snd}(\fun{npre}(0,t)).\\
\\
\fun{snd}(\pair{x}{y}) & \rightarrow & y.\\
\\
\fun{npre}(i,\fun{ext}()) & \rightarrow & \pair{i}{\fun{ext}()};\\
\fun{npre}(i,\fun{int}(x,t_1,t_2)) & \rightarrow &
\fun{t}_1(\fun{npre}(i+1,t_1),i,t_2).\\
\\
\fun{t}_1(\pair{j}{t'_1},i,t_2) & \rightarrow &
\fun{t}_2(\fun{npre}(j,t_2),i,t'_1).\\
\\
\fun{t}_2(\pair{k}{t'_2},i,t'_1) & \rightarrow & \pair{k}{\fun{int}(i,t'_1,t'_2)}.
\end{array}
}
\end{equation*}
\caption{Version of \fun{npre/1} without inference rules}
\label{fig:npre_bis}
\end{figure}
\index{binary tree!preorder|)}

\paragraph{Termination}
\index{termination!preorder traversal|(}
\index{binary tree!preorder!termination|(}

It is easy to prove the termination of
\fun{pre/2}\index{pre@\fun{pre/2}} because the technique used for
proving the termination of Ackermann's function on
page~\pageref{par:ackermann} is pertinent in the current context as
well. We define a lexicographic order\index{induction!lexicographic
  order} on the calls to \fun{pre/2}\index{pre@\fun{pre/2}}
(dependency pairs\index{termination!dependency pair}) as follows:
\begin{equation}
\fun{pre}(t,s) \succ \fun{pre}(t',s') :\Leftrightarrow \text{\(t
  \succ_{B} t'\) or (\(t = t'\) and \(s \succ_{S} s'\))},
\label{eq:BS_order}
\end{equation}
where \(B\)~is the set of all binary trees, \(S\)~is the set of all
stacks, \(t \succ_{B} t'\) means that the tree~\(t'\) is an immediate
subtree\index{tree!subtree!immediate $\sim$} of~\(t\), and \(s
\succ_{S} s'\) means that the stack \(s'\)~is an immediate
substack\index{induction!immediate substack order} of~\(s\)
(page~\pageref{par:well-founded}).  From the definition in
\fig~\vref{fig:pre}, we see that rule~\(\theta\) maintains termination
if \fun{pre/2}\index{pre@\fun{pre/2}} terminates; rule~\(\iota\)
terminates; finally, rule~\(\kappa\) rewrites a call into smaller
calls:
\begin{align*}
  \fun{pre}(\fun{int}(x,t_1,t_2),s) &\succ \fun{pre}(t_2,s),\\
  \fun{pre}(\fun{int}(x,t_1,t_2),s) &\succ \fun{pre}(t_1,u),\; \text{for
    all~\(u\),}
\end{align*}
in particular if \(\fun{pre}(t_2,s) \twoheadrightarrow u\). As a
consequence, \fun{pre/1}\index{pre@\fun{pre/1}} terminates on all
inputs.\index{binary
  tree!preorder!termination|)}\index{termination!preorder
  traversal|)}\hfill\(\Box\)

\paragraph{Equivalence}

To see how structural induction can be used to prove properties on
binary trees, we will consider a simple statement we made earlier,
formally expressed as \(\pred{Pre}{t} \colon \fun{pre}_0(t) \equiv
\fun{pre}(t)\).\index{Pre@\predName{Pre}}\index{pre0@\fun{pre\(_0\)/1}}\index{pre@\fun{pre/1}}
We need to prove
\begin{itemize*}

  \item the basis \(\pred{Pre}{\fun{ext}()}\);

  \item the inductive step \(\forall t_1.\pred{Pre}{t_1} \Rightarrow
    \forall t_2.\pred{Pre}{t_2} \Rightarrow \! \forall
    x.\pred{Pre}{\fun{int}(x,t_1,t_2)}\).

\end{itemize*}
The basis is easy because
\begin{equation*}
  \fun{pre}_0(\fun{ext}())
  \xrightarrow{\smash{\gamma}} \el \xleftarrow{\smash{\iota}}
  \fun{pre}(\fun{ext}(),\el) \xleftarrow{\smash{\theta}}
  \fun{pre}(\fun{ext}()).
\end{equation*}
See definition of \fun{pre\(_0\)/1} at
equation~\eqref{eq:pre0} \vpageref{eq:pre0}. It is useful here to
recall the definition of \fun{cat/2}\index{cat@\fun{cat/2}}:
\begin{equation*}
\fun{cat}(\el,t) \xrightarrow{\smash{\alpha}} t;
\qquad
\fun{cat}(\cons{x}{s},t) \xrightarrow{\smash{\beta}}
\cons{x}{\fun{cat}(s,t)}.
\end{equation*}
In order to discover how to use the two induction hypotheses
\(\pred{Pre}{t_1}\) and \(\pred{Pre}{t_2}\)\index{Pre@\predName{Pre}},
let us start from one side of the equivalence we wish to establish,
for example, the left\hyp{}hand side, and rewrite it until we reach
the other side or get stuck. Let \(t := \fun{int}(x,t_1,t_2)\), then
\begin{equation*}
\begin{array}{@{}r@{\;}c@{\;}l@{\qquad}r@{}}
\fun{pre}_0(t)
& = & \fun{pre}_0(\fun{int}(x,t_1,t_2)) \\
& \xrightarrow{\smash{\delta}}
& \cons{x}{\fun{cat}(\fun{pre}_0(t_1), \fun{pre}_0(t_2))}
& (\fun{pre}_0)\\
& \equiv & \cons{x}{\fun{cat}(\fun{pre}(t_1), \fun{pre}_0(t_2))}
& (\pred{Pre}{t_1})\\
& \equiv & \cons{x}{\fun{cat}(\fun{pre}(t_1), \fun{pre}(t_2))}
& (\pred{Pre}{t_2})\\
& \xrightarrow{\smash{\theta}}
& \cons{x}{\fun{cat}(\fun{pre}(t_1,\el), \fun{pre}(t_2))}
& (\fun{pre})\\
& \xrightarrow{\smash{\theta}}
& \cons{x}{\fun{cat}(\fun{pre}(t_1,\el), \fun{pre}(t_2,\el))}
& (\fun{pre}).
\end{array}
\end{equation*}
At this point, we start rewriting the other side, until we get stuck
as well:
\begin{equation*}
\begin{array}{@{}r@{\;}l@{\;}l@{\;\;}r@{}}
  \fun{pre}(t) = \fun{pre}(\fun{int}(x,t_1,t_2))
\xrightarrow{\smash{\theta}} \fun{pre}(\fun{int}(x,t_1,t_2),\el)
\xrightarrow{\smash{\kappa}}
\cons{x}{\fun{pre}(t_1,\fun{pre}(t_2,\el))}.
\index{pre@\fun{pre/1}}
\index{pre@\fun{pre/2}}
\end{array}
\end{equation*}
Comparing the two stuck expressions\index{rewrite system!stuck
  expression} suggests a subgoal to reach.

Let \(\pred{CatPre}{t,s} \colon \fun{cat}(\fun{pre}(t,\el),s) \equiv
\fun{pre}(t,s)\)\index{CatPre@\predName{CatPre}}\index{cat@\fun{cat/2}}\index{pre@\fun{pre/2}}. When
a predicate depends upon two parameters, we have different options to
ponder: either we need lexicographic induction, or simple induction on
one of the variables. It is best to use a lexicographic
ordering\index{induction!lexicographic order} on pairs and, if we
realise afterwards that only one component was needed, we can rewrite
the proof with a simple induction on that component. Let us then
define
\begin{equation*}
(t,s) \succ_{B \times S} (t',s') :\Leftrightarrow \text{\(t \succ_{B}
    t'\) or (\(t = t'\) and \(s \succ_{S} s'\))}.
\end{equation*}
This is conceptually the same order as the one on the calls to
\fun{pre/1}\index{pre@\fun{pre/1}}, in
definition~\eqref{eq:BS_order}. If we find out later that immediate
subterm relations are too restrictive, we would choose here general
subterm relations, which means, in the case of binary trees, that a
tree is a subtree of another. The minimum element for the
lexicographic order we just defined is \((\fun{ext}(),\el)\). The
well\hyp{}founded induction principle then requires that we establish
\begin{itemize*}

  \item the basis \(\pred{CatPre}{\fun{ext}(),\el}\);

  \item \(\forall t,s. (\forall t',s'.(t,s) \succ_{B \times S} (t',s')
    \Rightarrow \pred{CatPre}{t',s'}) \Rightarrow
    \pred{CatPre}{t,s}\).

\end{itemize*}
The basis is easy:
\begin{equation*}
  \fun{cat}(\fun{pre}(\fun{ext}(),\el),\el)
  \xrightarrow{\smash{\iota}} \fun{cat}(\el,\el)
  \xrightarrow{\smash{\alpha}} \el \xleftarrow{\smash{\iota}}
  \fun{pre}(\fun{ext}(),\el).
\index{cat@\fun{cat/2}}
\end{equation*}
We then assume \(\forall t',s'.(t,s) \succ_{B \times S} (t',s')
\Rightarrow \pred{CatPre}{t',s'}\), which is the induction hypothesis,
and proceed to rewrite the left\hyp{}hand side after letting \(t :=
\fun{int}(x,t_1,t_2)\). The result is shown in \fig~\vref{fig:CatPre},
\index{CatPre@\predName{CatPre}}
\begin{figure}[t]
\begin{equation*}
\boxed{%
\begin{array}{@{}r@{\;}c@{\;}l@{\qquad}r@{}}
%\begin{array}{r@{\;}c@{\;}l}
  \fun{cat}(\fun{pre}(t,\el),s)
& =
& \fun{cat}(\fun{pre}(\fun{int}(x,t_1,t_2),\el),s)\\
& \xrightarrow{\smash{\kappa}}
& \fun{cat}(\cons{x}{\fun{pre}(t_1,\fun{pre}(t_2,\el))},s)
& (\fun{pre/2})\\
& \xrightarrow{\smash{\beta}}
& \cons{x}{\fun{cat}(\fun{pre}(t_1,\fun{pre}(t_2,\el)),s)}
& (\fun{cat})\\
& \equiv_0
& \cons{x}{\fun{cat}(\fun{cat}(\fun{pre}(t_1,\el),\fun{pre}(t_2,\el)),s)}\\
& \equiv_1
& \cons{x}{\fun{cat}(\fun{pre}(t_1,\el),\fun{cat}(\fun{pre}(t_2,\el),s))}\\
& \equiv_2
& \cons{x}{\fun{cat}(\fun{pre}(t_1,\el),\fun{pre}(t_2,s))}\\
& \equiv_3
& \cons{x}{\fun{pre}(t_1,\fun{pre}(t_2,s))}\\
& \xleftarrow{\smash{\kappa}}
& \fun{pre}(\fun{int}(x,t_1,t_2),s)
& (\fun{pre/2})\\
& =
& \fun{pre}(t,s)
& \hfill\Box
\end{array}}
\end{equation*}
\caption{Proof of \(\pred{CatPre}{t} \colon
  \fun{cat}(\fun{pre}(t,\el),s) \equiv
  \fun{pre}(t,s)\)}
\label{fig:CatPre}
\end{figure}
where
\begin{itemize*}

  \item (\(\equiv_0\)) is the instance
  \(\pred{CatPre}{t_1,\fun{pre}(t_2,\el)}\) of the induction
  hypothesis because \((t,s) \succ_{B \times S} (t_1,s')\), for all
  stacks~\(s'\), in particular when \(\fun{pre}(t_2,\el)
  \twoheadrightarrow s'\);\index{pre@\fun{pre/2}}

  \item (\(\equiv_1\)) is an application of the lemma on the
  associativity of stack concatenation\index{stack!concatenation}
  (page~\pageref{proof:assoc_cat}), namely
  \index{pre@\fun{pre/2}}\index{CatAssoc@\predName{CatAssoc}}
  \(\pred{CatAssoc}{\fun{pre}(t_1,\el),\fun{pre}(t_2,\el),s}\);

  \item (\(\equiv_2\)) is the instance
  \(\pred{CatPre}{t_2,s}\)\index{CatPre@\predName{CatPre}} of the
  induction hypothesis because \((t,s) \succ_{B \times S} (t_2,s)\);

  \item (\(\equiv_3\)) is the instance \(\pred{CatPre}{t_1,
    \fun{pre}(t_2,s)}\) of the induction hypothesis because \((t,s)
    \succ_{B \times S} (t_1,s')\), for all stacks~\(s'\), in
    particular when \(s' = \fun{pre}(t_2,s)\).\index{pre@\fun{pre/2}}
\end{itemize*}
We can resume conclusively in \fig~\vref{fig:pre0_pre}.\hfill\(\Box\)
\begin{figure}
\begin{equation*}
\boxed{%
\begin{array}{r@{\;}c@{\;}l@{\qquad}r}
\fun{pre}_0(t)
& = & \fun{pre}_0(\fun{int}(x,t_1,t_2))\\
& \xrightarrow{\smash{\delta}}
& \cons{x}{\fun{cat}(\fun{pre}_0(t_1), \fun{pre}_0(t_2))}
& (\fun{pre}_0)\\
& \equiv & \cons{x}{\fun{cat}(\fun{pre}(t_1), \fun{pre}_0(t_2))}
& (\pred{Pre}{t_1})\\
& \equiv & \cons{x}{\fun{cat}(\fun{pre}(t_1), \fun{pre}(t_2))}
& (\pred{Pre}{t_2})\\
& \xrightarrow{\smash{\theta}}
& \cons{x}{\fun{cat}(\fun{pre}(t_1,\el), \fun{pre}(t_2))}
& (\fun{pre})\\
& \xrightarrow{\smash{\theta}}
& \cons{x}{\fun{cat}(\fun{pre}(t_1,\el), \fun{pre}(t_2,\el))}
& (\fun{pre})\\
& \equiv
& \cons{x}{\fun{pre}(t_1,\fun{pre}(t_2,\el))}
& (\pred{CatPre}{t_1,\fun{pre}(t_2,\el)})\\
& \xleftarrow{\smash{\kappa}}
& \fun{pre}(\fun{int}(x,t_1,t_2),\el)
& (\fun{pre/2})\\
& \xleftarrow{\smash{\theta}}
& \fun{pre}(\fun{int}(x,t_1,t_2))
& (\fun{pre})\\
& = & \fun{pre}(t). & \hfill\Box
\end{array}}
\end{equation*}
\caption{Proof of \(\pred{Pre}{t} \colon \fun{pre}_0(t) \equiv
  \fun{pre}(t)\)}
\label{fig:pre0_pre}
\end{figure}

\paragraph{Flattening revisited}
\index{stack!flattening|(}

In section~\vref{sec:flattening}, we defined two functions for
flattening\index{stack!flattening} stacks (see \fig~\ref{fig:flat0} on
page~\pageref{fig:flat0} and \fig~\ref{fig:flat} on
page~\pageref{fig:flat}). With the understanding of preorder
traversals, it may occur to us that the flattening of a stack is
equivalent to the preorder traversal of a binary leaf
tree\index{binary tree!leaf tree} (see \fig~\vref{fig:leaf_tree} for
an example) which ignores empty stacks. The key is to see a stack,
possibly containing other stacks, as a leaf tree, as shown for example
in \fig~\vref{fig:stacks}, where the internal nodes (\texttt{|}) are
cons\hyp{}nodes.
\begin{figure}
\centering
\includegraphics[bb=71 643 186 721]{stacks}
\caption{Embedded stacks as a leaf tree}
\label{fig:stacks}
\end{figure}

The first step consists in defining inductively the set of the binary
leaf trees as the smallest set~\(L\) generated by the deductive
(downwards) reading of the inference system
\begin{mathpar}
\inferrule*{}{\fun{leaf}(x) \in L}
\qquad
\inferrule{t_1 \in L \and t_2 \in L}{\fun{fork}(t_1,t_2) \in L}.
\end{mathpar}
In other words, a leaf containing the piece of data~\(x\) is noted
\(\fun{leaf}(x)\)\index{leaf@\fun{leaf/1}} and the other internal
nodes are \emph{forks}\index{binary tree!fork}, written
\(\fun{fork}(t_1,t_2)\)\index{fork@\fun{fork/2}}, with \(t_1\)
and~\(t_2\) being binary leaf trees themselves. The second step
requires a modification of \fun{pre/1}, defined in
\fig~\vref{fig:pre}, so it processes binary leaf trees. The new
function, \fun{lpre/1}\index{lpre@\fun{lpre/1}}, is shown in
\fig~\vref{fig:lpre}.
\begin{figure}[b]
\begin{equation*}
\boxed{
\begin{array}{r@{\;}c@{\;}l}
\fun{lpre}(t) & \rightarrow & \fun{lpre}(t,\el).\\
\\
\fun{lpre}(\fun{leaf}(x),s) & \rightarrow & \cons{x}{s};\\
\fun{lpre}(\fun{fork}(t_1,t_2),s)
  & \rightarrow & \fun{lpre}(t_1,\fun{lpre}(t_2,s)).
\end{array}}
\end{equation*}
\caption{Preorder on binary leaf trees}
\label{fig:lpre}
\end{figure}
The final step is the translation of \fun{lpre/1} and \fun{lpre/2}
into \fun{flat\(_2\)/1} and \fun{flat\(_2\)/2}, respectively, in
\fig~\vref{fig:flat2}.
The key is to see that
\(\fun{fork}(t_1,t_2)\)\index{fork@\fun{fork/2}} becomes
\(\cons{t_1}{t_2}\), and \(\fun{leaf}(x)\), when \(x\)~is not an empty
stack, becomes~\(x\) as the \emph{last} pattern. The case
\(\fun{leaf}(\el)\)\index{leaf@\fun{leaf/1}} becomes~\(\el\) as the
first pattern.

The cost of \fun{pre/1}\index{pre@\fun{pre/1}} was found to be
\(\C{\fun{pre}}{n} = 2n+2\)\index{pre@$\C{\fun{pre}}{n}$} in
equation~\eqref{eq:pre} on page~\pageref{eq:pre}, where \(n\)~was the
number of internal nodes. Here, \(n\)~is the length of
\(\fun{flat}_2(t)\)\index{flat2@\fun{flat\(_0\)/1}}, namely, the number
of non\hyp{}stack leaves in the leaf tree. With this definition, the
number of cons\hyp{}nodes is \(n + \Omega + \Gamma\), where
\(\Omega\)~is the number of embedded empty stacks and \(\Gamma\)~is
the number of embedded non\hyp{}empty stacks, so \(S := 1 + \Omega +
\Gamma\) is the total number of stacks. Consequently,
\(\C{\fun{flat}_2}{n} = 2(n + S)\).\index{flat2@$\C{\fun{flat}_2}{n}$}
For example,
\begin{equation*}
\fun{flat}_2([1,[\el,[2,3]],[[4]],5]) \xrightarrow{22} [1,2,3,4,5],
\end{equation*}
because \(n=5\) and \(S=6\). (The latter is the number of opening
square brackets.) When \(S=1\), that is, the stack is flat, then
\(\C{\fun{pre}}{n} = \C{\fun{flat}_2}{n}\), otherwise
\(\C{\fun{pre}}{n} < \C{\fun{flat}_2}{n}\).

%\bigskip

\begin{figure}[t]
\begin{equation*}
\boxed{
\begin{array}{r@{\;}c@{\;}l}
\fun{flat}_2(t) & \rightarrow & \fun{flat}_2(t,\el).\\
\\
\fun{flat}_2(\el,s) & \rightarrow & s;\\
\fun{flat}_2(\cons{t_1}{t_2},s)
  & \rightarrow & \fun{flat}_2(t_1,\fun{flat}_2(t_2,s));\\
\fun{flat}_2(x,s) & \rightarrow & \cons{x}{s}.
\end{array}}
\end{equation*}
\caption{Flattening like \fun{lpre/1}}
\label{fig:flat2}
\end{figure}
\index{stack!flattening|)}

\mypar{Inorder}
\index{binary tree!inorder|(}
\label{inorder}

The \emph{inorder} (or \emph{symmetric}) traversal of a non\hyp{}empty
binary tree consists in having in a stack the nodes of the left
subtree in inorder, followed by the root and then the nodes of the
right subtree in inorder. For example, the nodes in inorder of the
tree in \fig~\vref{fig:bt_ex1} are \([3,5,1,9,8,3,2]\). Clearly, it is
a depth\hyp{}first traversal\index{binary tree!depth-first traversal},
like a preorder, because children are visited before siblings.
According to our findings about \fun{pre/1}\index{pre@\fun{pre/1}} in
\fig~\vref{fig:pre}, we understand that we should structure our
program to follow the strategy depicted in \fig~\vref{fig:in_tree},
where the only difference with \fig~\ref{fig:pre_tree} is the moment
when the root~\(x\) is pushed on the accumulator\index{functional
  language!accumulator}: between the inorder traversals of the
subtrees \(t_1\)~and~\(t_2\). The implicit rewrites in
\fig~\ref{fig:in_tree} are
\begin{align*}
  \fun{in}(t_2,s) &\twoheadrightarrow s_1,\index{in@\fun{in/2}}\\
  \fun{in}(t_1,\cons{x}{s_1}) &\twoheadrightarrow s_2,\\
  \fun{in}(t,s) &\twoheadrightarrow s_2,
\end{align*}
where \(t=\fun{int}(x,t_1,t_2)\). By eliminating the intermediary
variables \(s_1\)~and~\(s_2\) we obtain the equivalence
\begin{equation*}
  \fun{in}(t_1,\cons{x}{\fun{in}(t_2,s)}) \equiv \fun{in}(t,s).
\end{equation*}
The case of the external node is the same as for preorder. This
reasoning yields the function defined in \fig~\vref{fig:in},
\begin{figure}[t]
\begin{equation*}
\boxed{
\begin{array}{r@{\;}c@{\;}l}
\fun{in}(t) & \xrightarrow{\xi} & \fun{in}(t,\el).\\
\\
\fun{in}(\fun{ext}(),s) & \xrightarrow{\smash{\pi}} & s;\\
\fun{in}(\fun{int}(x,t_1,t_2),s)
  & \xrightarrow{\smash{\rho}}
  & \fun{in}(t_1,\cons{x}{\fun{in}(t_2,s)}).
\end{array}}
\end{equation*}
\caption{Inorder traversal}
\label{fig:in}
\end{figure}
whose cost is the same as for preorder: \(\C{\fun{in}}{n} =
\C{\fun{pre}}{n} = 2n + 2\).\index{in@$\C{\fun{in}}{n}$}

\Fig~\vref{fig:inorder} gives the example of a binary tree which is
the result of an \emph{inorder numbering}\index{binary tree!inorder
  numbering}. The inorder traversal of that tree yields
\([0,1,2,3,4,5,6]\). Inorder numberings have an interesting property:
given any internal node, all the nodes in its left subtree have
smaller numbers, and all nodes in its right subtree have greater
numbers. Let \fun{nin/1}\index{nin@\fun{nin/1}}\index{nin@\fun{nin/2}}
be a function computing the inorder numbering of a given tree in
\fig~\vref{fig:nin},
\begin{figure}
\centering
\framebox[0.75\columnwidth]{\vbox{%
\begin{gather*}
\inferrule
  {\fun{nin}(0,t) \twoheadrightarrow \pair{i}{t'}}
  {\fun{nin}(t) \twoheadrightarrow t'}\,.
\qquad
\fun{nin}(i,\fun{ext}()) \rightarrow \pair{i}{\fun{ext}()};
\\
\inferrule
  {\fun{nin}(i,t_1) \twoheadrightarrow \pair{j}{t'_1}
   \and
   \fun{nin}(j+1,t_2) \twoheadrightarrow \pair{k}{t'_2}}
  {\fun{nin}(i,\fun{int}(x,t_1,t_2)) \twoheadrightarrow
   \pair{k}{\fun{int}(j,t'_1,t'_2)}}\,.
\end{gather*}
}}
\caption{Inorder numbering}
\label{fig:nin}
\end{figure}
where~\(j\), at the root, is the smallest number greater than any
number in~\(t_1\).

\paragraph{Flattening revisited}
\label{par:rotation}

The design of
\fun{flat/1}\index{stack!flattening}\index{flat@\fun{flat/1}} in
\fig~\vref{fig:flat} may suggest a new approach to inorder
traversals. By composing right rotations\index{binary tree!rotation|(}
as defined in \figs~\vrefrange{fig:lift1}{fig:lift2} (the converse is,
of course, a \emph{left rotation}), the node to be visited first in
inorder can be brought to be the root of a tree whose left subtree is
empty. Recursively, the right subtree is then processed, in a
top\hyp{}down fashion. This algorithm is sound because \emph{inorder
  traversals are invariant through rotations}, which is formally
expressed as follows.
\begin{equation*}
\pred{Rot}{x,y,t_1,t_2,t_3} \colon
\fun{in}(\fun{int}(y,\fun{int}(x,t_1,t_2),t_3))
\equiv
\fun{in}(\fun{int}(x,t_1,\fun{int}(y,t_2,t_3))).\index{in@\fun{in/1}}
\label{def:Rot}
\end{equation*}
In \fig~\vref{fig:rot},
\begin{figure}[t]
\centering
\includegraphics[bb=71 627 374 715]{rot}
\caption{Composing right\hyp{}rotations, top\hyp{}down}
\label{fig:rot}
\end{figure}
we show how a binary tree becomes a right\hyp{}leaning,
degenerate\index{binary tree!degenerate $\sim$} tree, isomorphic to a
stack, by repeatedly applying right rotations, top\hyp{}down.  Dually,
we could compose left rotations and obtain a left\hyp{}leaning,
degenerate tree, whose inorder traversal is also equal to the inorder
traversal of the original tree. The function
\fun{in\(_1\)/1}\index{in1@\fun{in\(_1\)/1}} based on right rotations
is given in \fig~\vref{fig:in1}.
\begin{figure}
\begin{equation*}
\boxed{
\begin{array}{r@{\;}c@{\;}l}
\fun{in}_1(\fun{ext}()) & \xrightarrow{\smash{\alpha}} & \el;\\
\fun{in}_1(\fun{int}(y,\fun{int}(x,t_1,t_2),t_3))
  & \xrightarrow{\smash{\beta}} & \fun{in}_1(\fun{int}(x,t_1,\fun{int}(y,t_2,t_3)));\\
\fun{in}_1(\fun{int}(y,\fun{ext}(),t_3))
  & \xrightarrow{\smash{\gamma}} & \cons{y}{\fun{in}_1(t_3)}.
\end{array}}
\end{equation*}
\caption{Inorder traversal by right rotations}
\label{fig:in1}
\end{figure}
Note how, in rule~\(\gamma\), we push the root~\(y\) in the result as
soon as we can, which would not be possible had we used left rotations
instead, and thus we do \emph{not} build the whole rotated tree, as in
\fig~\vref{fig:rot}.

The cost \(\C{\fun{in}_1}{n}\)\index{in1@$\C{\fun{in}_1}{n}$} depends
on the topology of the tree at hand. Firstly, let us note that
rule~\(\alpha\) is used only once, on the rightmost external
node. Secondly, if the tree to be traversed is already a
right\hyp{}leaning degenerate \index{binary tree!degenerate $\sim$}
tree, rule~\(\beta\) is not used and rule~\(\gamma\) is used
\(n\)~times. Clearly, this is the best case and \(\B{\fun{in}_1}{n} =
n + 1\).\index{in1@$\B{\fun{in}_1}{n}$} Thirdly, we should remark that
a right rotation brings exactly one more node (named~\(x\) in
\(\pred{Rot}{x,y,t_1,t_2,t_3}\))\index{Rot@\predName{Rot}} into the
\emph{rightmost branch}\index{tree!branch}, that is, the series of
nodes starting with the root and reached using repeatedly right edges
(for instance, in the initial tree in \fig~\vref{fig:rot}, the
rightmost branch\index{tree!branch} is \([\fun{e}, \fun{g},
\fun{h}]\)). Therefore, if we want to maximise the use of
rule~\(\beta\), we must have an initial tree whose right subtree is
empty, so the left subtree contains \(n-1\) nodes (the root belongs,
by definition, to the rightmost branch): this yields
\begin{equation*}
  \W{\fun{in}_1}{n} = (n-1) + (n+1) = 2n.
  \index{in1@$\W{\fun{in}_1}{n}$}
  \index{binary tree!rotation|)}
\end{equation*}

\paragraph{Exercise}

Prove \(\forall x,y,t_1,t_2,t_3.\pred{Rot}{x,y,t_1,t_2,t_3}\).
\index{Rot@\predName{Rot}}

\paragraph{Mirroring}

Let us define a function \fun{mir/1}\index{mir@\fun{mir/1}}
(\emph{mirror}) such that \(\fun{mir}(t)\) is the symmetric of the
binary tree~\(t\) with respect to an exterior vertical line. An
example is given in \fig~\vref{fig:mirror}.
\begin{figure}[t]
\centering
\subfloat[\(t\)]{\includegraphics[bb=77 660 163 723]{bt_ex2}}
\qquad
\subfloat[\(\protect\fun{mir}(t)\)]{\includegraphics[bb=74 660 160 723]{bt_ex4}}
\caption{Mirroring a binary tree}
\label{fig:mirror}
\end{figure}
It is easy to define this function:
\begin{equation*}
\fun{mir}(\fun{ext}()) \xrightarrow{\smash{\sigma}} \fun{ext}();
\quad
\fun{mir}(\fun{int}(x,t_1,t_2)) \xrightarrow{\smash{\tau}}
\fun{int}(x,\fun{mir}(t_2),\fun{mir}(t_1)).
\end{equation*}
From the previous example, it is quite simple to postulate the
property
\begin{equation*}
\pred{InMir}{t} \colon \fun{in}(\fun{mir}(t)) \equiv
\fun{rev}(\fun{in}(t)),\index{InMir@\predName{InMir}}
\end{equation*}
where \fun{rev/1}\index{rev@\fun{rev/1}}
reverses\index{stack!reversal} its stack argument (see
definition~\eqref{def:rev} \vpageref{def:rev}). This property is useful
because the left\hyp{}hand side of the equivalence is more costly than
the right\hyp{}hand side:
\begin{equation*}
  \Call{\fun{in}(\fun{mir}(t))} =
  \C{\fun{mir}}{n} + \C{\fun{in}}{n} = (2n+1) + (2n+2) = 4n + 3,
\end{equation*}
to be compared with \(\Call{\fun{rev}(\fun{in}(t))} = \C{\fun{in}}{n}
+ \C{\fun{rev}}{n} = (2n+2) + (n+2) = 3n +
4\).\index{mir@$\C{\fun{mir}}{n}$}

\medskip

\noindent Structural induction on the immediate subtrees requires that
we establish
\begin{itemize}

  \item the basis
  \(\pred{InMir}{\fun{ext}()}\);\index{InMir@\predName{InMir}|(}

  \item the step \(\forall t_1.\pred{InMir}{t_1} \Rightarrow \forall
    t_2.\pred{InMir}{t_2} \Rightarrow \forall
    x.\pred{InMir}{\fun{int}(x,t_1,t_2)}.\)

\end{itemize}
The former results from
\begin{equation*}
\begin{array}{@{}r@{\;}c@{\;}l@{\qquad}r@{}}
  \fun{in}(\fun{mir}(\fun{ext}()))
  & \xrightarrow{\smash{\sigma}}
  & \fun{in}(\fun{ext}())
  & (\fun{mir})\\
  & \xrightarrow{\smash{\xi}}
  & \fun{in}(\fun{ext}(),\el)
  & (\fun{in})\\
  & \xrightarrow{\smash{\pi}}
  & \el
  & (\fun{in/2})\\
  & \xleftarrow{\smash{\zeta}}
  & \fun{rcat}(\el,\el)
  & (\fun{rcat})\\
  & \xleftarrow{\smash{\epsilon}}
  & \fun{rev}(\el)
  & (\fun{rev})\\
  & \xleftarrow{\smash{\pi}}
  & \fun{rev}(\fun{in}(\fun{ext}(),\el))
  & (\fun{in/2})\\
  & \xleftarrow{\smash{\xi}}
  & \fun{rev}(\fun{in}(\fun{ext}()))
  & (\fun{in}).
  \index{rcat@\fun{rcat/2}}\index{rev@\fun{rev/1}}
\end{array}
\end{equation*}
Let us assume \(\pred{InMir}{t_1}\) and \(\pred{InMir}{t_2}\) and let
\(t := \fun{int}(x,t_1,t_2)\), for any~\(x\). We rewrite the
left\hyp{}hand side of the equivalence to be proved until we reach the
right\hyp{}hand side or we get stuck:
\begin{equation*}
\begin{array}{@{}r@{\;}l@{\;}l@{\qquad}r@{}}
\fun{in}(\fun{mir}(t)) & = &
\fun{in}(\fun{mir}(\fun{int}(x,t_1,t_2)))\\
& \xrightarrow{\smash{\tau}}
& \fun{in}(\fun{int}(x,\fun{mir}(t_2),\fun{mir}(t_1)))
& (\fun{mir})\\
& \xrightarrow{\smash{\xi}}
& \fun{in}(\fun{int}(x,\fun{mir}(t_2),\fun{mir}(t_1)),\el)
& (\fun{in})\\
& \xrightarrow{\smash{\rho}}
& \fun{in}(\fun{mir}(t_2),\cons{x}{\fun{in}(\fun{mir}(t_1),\el)})
& (\fun{in/2})\\
& \xleftarrow{\smash{\zeta}}
& \fun{in}(\fun{mir}(t_2),\cons{x}{\fun{in}(\fun{mir}(t_1))})
& (\fun{in})\\
& \equiv
& \fun{in}(\fun{mir}(t_2),\cons{x}{\fun{rev}(\fun{in}(t_1))})
& (\pred{InMir}{t_1})\\
& \xrightarrow{\smash{\xi}}
& \fun{in}(\fun{mir}(t_2),\cons{x}{\fun{rev}(\fun{in}(t_1,\el))})
& (\fun{in})\\
& \xrightarrow{\smash{\epsilon}}
& \fun{in}(\fun{mir}(t_2),\cons{x}{\fun{rcat}(\fun{in}(t_1,\el),\el)})
& (\fun{rev}).
\end{array}
\end{equation*}
We cannot use the induction hypothesis \(\pred{InMir}{t_2}\) to get
rid of \(\fun{mir}(t_2)\). Close examination of the terms suggests to
\emph{weaken} the property and overload \predName{InMir} with a new
definition:
\begin{equation*}
\pred{InMir}{t,s} \colon \fun{in}(\fun{mir}(t),s) \equiv
\fun{rcat}(\fun{in}(t,\el),s).
\end{equation*}
We have \(\pred{InMir}{t,\el} \Leftrightarrow \pred{InMir}{t}\). Now,
we rewrite as follows:
\begin{equation*}
\begin{array}{@{}r@{\;}l@{\;}l@{\qquad}r@{}}
\fun{in}(\fun{mir}(t),s)
& =
& \fun{in}(\fun{mir}(\fun{int}(x,t_1,t_2)),s)\\
& \xrightarrow{\smash{\tau}}
& \fun{in}(\fun{int}(x,\fun{mir}(t_2),\fun{mir}(t_1)),s)
& (\fun{mir})\\
& \xrightarrow{\smash{\rho}}
& \fun{in}(\fun{mir}(t_2),\cons{x}{\fun{in}(\fun{mir}(t_1),s)})
& (\fun{in/2})\\
& \equiv
& \fun{in}(\fun{mir}(t_2),\cons{x}{\fun{rcat}(\fun{in}(t_1,\el),s)})
& (\pred{InMir}{t_1,s})\\
& \equiv_0
& \fun{rcat}(\fun{in}(t_2,\el),\cons{x}{\fun{rcat}(\fun{in}(t_1,\el),s)})\\
& \xleftarrow{\smash{\eta}}
& \fun{rcat}(\cons{x}{\fun{in}(t_2,\el)},
\fun{rcat}(\fun{in}(t_1,\el),s))
& (\fun{rcat}),
\end{array}
\end{equation*}
where (\(\equiv_0\)) is
\(\pred{InMir}{t_2,\cons{x}{\fun{rcat}(\fun{in}(t_1,\el),s)}}\).\index{rcat@\fun{rcat/2}}\index{InMir@\predName{InMir}|)}
The right\hyp{}hand side now:
\begin{equation*}
%\begin{array}{@{}r@{\;}c@{\;}l@{\qquad}r@{}}
\fun{rcat}(\fun{in}(t,\el),s)
\! = \!
 \fun{rcat}(\fun{in}(\fun{int}(x,t_1,t_2),\el),s)
\! \xrightarrow{\smash{\rho}}\!
 \fun{rcat}(\fun{in}(t_1,\cons{x}{\fun{in}(t_2,\el)}),s).
%\end{array}
\end{equation*}
The two stuck expressions share the subterm
\(\cons{x}{\fun{in}(t_2,\el)}\)\index{in@\fun{in/2}}. The main
difference is that the former expression contains two calls to
\fun{rcat/2}\index{rcat@\fun{rcat/2}|(}, instead of one in the
latter. Can we find a way to have only one call in the former too?  We
need to find an expression equivalent to
\(\fun{rcat}(s,\fun{rcat}(t,u))\), whose shape is \(\fun{rcat}(v,w)\),
where \(v\) and~\(w\) contain no call to \fun{rcat/2}. Some examples
quickly suggest\index{Rcat@\predName{Rcat}}
\begin{equation*}
  \pred{Rcat}{s,t,u} \colon \fun{rcat}(s,\fun{rcat}(t,u)) \equiv \fun{rcat}(\fun{cat}(t,s),u).
\end{equation*}
We actually do not need induction to prove this theorem if we recall
what we already proved:
\begin{itemize*}

  \item \(\pred{CatRev}{s,t} \colon \fun{cat}(\fun{rev}_0(t),
    \fun{rev}_0(s)) \equiv \fun{rev}_0(\fun{cat}(s,t))\);

  \item \(\pred{EqRev}{s} \colon \fun{rev}_0(s) \equiv \fun{rev}(s)\);

  \item \(\pred{CatAssoc}{s,t,u} \colon \fun{cat}(s,\fun{cat}(t,u))
    \equiv \fun{cat}(\fun{cat}(s,t),u)\);

  \item \(\pred{RevCat}{s,t} \colon \fun{rcat}(s,t) \equiv
    \fun{cat}(\fun{rev}(s),t)\).

\end{itemize*}
Then, we have the equivalences
\begin{equation*}
\begin{array}{@{}r@{\,}c@{\,}l@{\,}r@{}}
\fun{rcat}(s,\fun{rcat}(t,u))
& \equiv & \fun{rcat}(s,\fun{cat}(\fun{rev}(t),u))
& (\pred{RevCat}{t,u}\!)\\
& \equiv & \fun{rcat}(s,\fun{cat}(\fun{rev}_0(t),u))
& (\pred{EqRev}{t}\!)\\
& \equiv & \fun{cat}(\fun{rev}(s),\fun{cat}(\fun{rev}_0(t),u))
& (\pred{RevCat}{s,\fun{cat}(\fun{rev}_0(t),u)}\!)\\
& \equiv & \fun{cat}(\fun{rev}_0(s),\fun{cat}(\fun{rev}_0(t),u)\!)
& (\pred{EqRev}{s}\!)\\
& \equiv & \fun{cat}(\fun{cat}(\fun{rev}_0(s),\fun{rev}_0(t)\!),u)\!\!
& (\!\pred{CatAssoc}{\fun{rev}(s),\fun{rev}(t),u}\!)\\
& \equiv & \fun{cat}(\fun{rev}_0(\fun{cat}(t,s)),u)
& (\pred{CatRev}{s,t}\!)\\
& \equiv & \fun{cat}(\fun{rev}(\fun{cat}(t,s)),u)
& (\pred{EqRev}{\fun{cat}(t,s)}\!)\\
& \equiv & \fun{rcat}(\fun{cat}(t,s),u)
& (\pred{RevCat}{\fun{cat}(t,s),u}\!)
\end{array}
\end{equation*}
Let us resume rewriting the first stuck expression:
\begin{equation*}
\begin{array}{@{}r@{\;}c@{\;}l@{\qquad}r@{}}
\fun{in}(\fun{mir}(t),s)
& \equiv_0
& \fun{rcat}(\cons{x}{\fun{in}(t_2,\el)},
  \fun{rcat}(\fun{in}(t_1,\el),s))\\
& \equiv_1
& \fun{rcat}(\fun{cat}(\fun{in}(t_1,\el),
  \cons{x}{\fun{in}(t_2,\el)}), s),\index{rcat@\fun{rcat/2}|)}
\end{array}
\end{equation*}
where (\(\equiv_0\)) is a short\hyp{}hand for the previous derivation
and (\(\equiv_1\)) is the instance
\(\pred{Rcat}{\cons{x}{\fun{in}(t_2,\el)}, \fun{in}(t_1,\el),
  s}\). Another comparison of the stuck expressions reveals that we
need to prove \(\fun{cat}(\fun{in}(t,\el),s) \equiv
\fun{in}(t,s)\). This equivalence is likely to be true, as it is
similar to \(\pred{CatPre}{t,s}\).\index{CatPre@\predName{CatPre}}
Assuming this lemma, we achieve the proof.\hfill\(\Box\)

\paragraph{Exercises}
\begin{enumerate*}

  \item Prove the missing lemma \(\pred{CatIn}{t,s} \colon
  \fun{cat}(\fun{in}(t,\el),s) \equiv
  \fun{in}(t,s)\).\index{CatIn@\predName{CatIn}}\index{cat@\fun{cat/2}}

  \item Define a function which builds a binary tree from its preorder
    and inorder traversals, assuming that its internal nodes are all
    distinct. Make sure its cost is linear in the number of internal
    nodes. Compare your solution with that of \cite{MuBird_2003}.

\end{enumerate*}
\index{binary tree!inorder|)}

\mypar{Postorder}
\index{binary tree!postorder|(}

A \emph{postorder}\index{binary tree!postorder} traversal of a
non\hyp{}empty binary tree consists in storing in a stack the nodes of
the right subtree in postorder, followed by the nodes of the left
subtree in postorder, and, in turn, by the root. For example, the
nodes in postorder of the tree in \fig~\vref{fig:bt_ex1} are
\([5,3,9,1,2,3,8]\). Clearly, it is a depth\hyp{}first
traversal\index{binary tree!depth-first traversal}, like a preorder,
but, unlike a preorder, it saves the root last in the resulting
stack. This approach is summarised for internal nodes in
\fig~\vref{fig:post_tree}. The difference with
\fun{pre/1}\index{pre@\fun{pre/1}} and \fun{in/1}\index{in@\fun{in/1}}
lies in the moment when the root is pushed in the accumulator. The
function definition is given in \fig~\vref{fig:post}.
\begin{figure}
%\abovedisplayskip=0pt
%\belowdisplayskip=0pt
\begin{equation*}
\boxed{
\begin{array}{r@{\;}c@{\;}l}
\fun{post}(t) & \xrightarrow{\lambda} & \fun{post}(t,\el).\\
\\
\fun{post}(\fun{ext}(),s) & \xrightarrow{\smash{\mu}} & s;\\
\fun{post}(\fun{int}(x,t_1,t_2),s)
  & \xrightarrow{\smash{\nu}}
  & \fun{post}(t_1,\fun{post}(t_2,\cons{x}{s})).
\end{array}}
\end{equation*}
\caption{Postorder traversal}
\label{fig:post}
\end{figure}
The meaning of the stack~\(s\) in the call
\(\fun{post}(t,s)\)\index{post@\fun{post/2}}\index{post@\fun{post/1}}
is the same as in \(\fun{pre}(t,s)\), modulo ordering: \(s\)~is made
of the contents, in postorder, of the nodes that follow, in postorder,
the nodes of the subtree~\(t\). The cost is familiar as well:
\(\C{\fun{post}}{n} = \C{\fun{in}}{n} = \C{\fun{pre}}{n} = 2n +
2\).\index{post@$\C{\fun{post}}{n}$}

An example of \emph{postorder numbering}\index{binary tree!postorder
  numbering} is shown in \fig~\vref{fig:postorder}, so the postorder
traversal of that tree yields \([0,1,2,3,4,5,6]\). Notice how the
numbers increase along upwards paths. \Fig~\vref{fig:npost}\index{npost@\fun{npost/1}}\index{npost@\fun{npost/2}}
\begin{figure}
%\abovedisplayskip=0pt
%\belowdisplayskip=0pt
\centering
\framebox[0.75\columnwidth]{\vbox{%
\begin{gather*}
\inferrule
  {\fun{npost}(0,t) \twoheadrightarrow \pair{i}{t'}}
  {\fun{npost}(t) \rightarrow t'}\,.
\qquad
\fun{npost}(i,\fun{ext}()) \rightarrow \pair{i}{\fun{ext}()};
\\
\inferrule
  {\fun{npost}(i,t_1) \twoheadrightarrow \pair{j}{t'_1}
   \and
   \fun{npost}(j,t_2) \twoheadrightarrow \pair{k}{t'_2}}
  {\fun{npost}(i,\fun{int}(x,t_1,t_2)) \rightarrow
   \pair{k+1}{\fun{int}(k,t'_1,t'_2)}}\,.
\end{gather*}
}}
\caption{Postorder numbering}
\label{fig:npost}
\end{figure}
shows the program to number a binary tree in postorder. The root is
numbered with the number coming up from the right subtree, following
the pattern of a postorder traversal.

\paragraph{A proof}\label{proof:PreMir}

Let \(\pred{PreMir}{t} \colon \fun{pre}(\fun{mir}(t)) \equiv
\fun{rev}(\fun{post}(t))\).\index{PreMir@\predName{PreMir}|(} Previous
experience with proving \(\pred{InMir}{t}\) leads us to weaken
(generalise) the property in order to facilitate the proof: Let
\begin{equation*}
  \pred{PreMir}{t,s} \colon \fun{pre}(\fun{mir}(t),s) \equiv
  \fun{rcat}(\fun{post}(t,\el),s).
  \index{rcat@\fun{rcat/2}}
\end{equation*}
Clearly, \(\pred{PreMir}{t,\el} \Leftrightarrow
\pred{PreMir}{t}\). Let us then define
\begin{equation*}
  (t,s) \succ_{B \times S} (t',s') :\Leftrightarrow \text{\(t \succ_{B}
    t'\) or (\(t = t'\) and \(s \succ_{S} s'\))}.
\end{equation*}
This is conceptually the same order as the one on the calls to
\fun{pre/1}\index{pre@\fun{pre/1}}, in definition~\eqref{eq:BS_order}
\vpageref{eq:BS_order}. The minimum element for this lexicographic
order is \((\fun{ext}(),\el)\). Well\hyp{}founded induction then
requires that we prove
\begin{itemize*}

  \item the basis \(\pred{PreMir}{\fun{ext}(),\el}\);

  \item \(\forall t,s. (\forall t',s'.(t,s) \succ_{B \times S} (t',s')
    \Rightarrow \pred{PreMir}{t',s'}) \Rightarrow
    \pred{PreMir}{t,s}\).

\end{itemize*}
The basis is proved as follows:\index{mir@\fun{mir/1}|(}
\begin{align*}
  \fun{pre}(\fun{mir}(\fun{ext}()),\el)
  &\xrightarrow{\smash{\sigma}} \fun{pre}(\fun{ext}(),\el)\\
  &\xrightarrow{\smash{\iota}} \el\\
  &\xleftarrow{\smash{\zeta}} \fun{rcat}(\el,\el)\\
  &\xleftarrow{\smash{\mu}}
  \fun{rcat}(\fun{post}(\fun{ext}(),\el),\el).
\end{align*}
Let \(t := \fun{int}(x,t_1,t_2)\). In \fig~\vref{fig:premir}, we have
the rewrites of the left\hyp{}hand side,
\begin{figure}
\begin{equation*}
\boxed{
\begin{array}{@{}r@{\;}l@{\;}l@{\qquad}r@{}}
\fun{pre}(\fun{mir}(t),s)
& = & \fun{pre}(\fun{mir}(\fun{int}(x,t_1,t_2)),s)\\
& \xrightarrow{\smash{\tau}}
& \fun{pre}(\fun{int}(x,\fun{mir}(t_2),\fun{mir}(t_1)),s)
& (\fun{mir})\\
& \xrightarrow{\smash{\kappa}}
& \cons{x}{\fun{pre}(\fun{mir}(t_2),\fun{pre}(\fun{mir}(t_1),s))}
& (\fun{pre})\\
& \equiv_0
& \cons{x}{\fun{pre}(\fun{mir}(t_2),
           \fun{rcat}(\fun{post}(t_1,\el),s))}\\
& \equiv_1
& \cons{x}{\fun{rcat}(\fun{post}(t_2,\el),
           \fun{rcat}(\fun{post}(t_1,\el),s))}\\
& \equiv_2
& \cons{x}{\fun{rcat}(\fun{cat}(\fun{post}(t_1,\el), \fun{post}(t_2,\el)), s)}\\
& \xleftarrow{\smash{\zeta}}
& \fun{rcat}(\el,\cons{x}{\fun{rcat}(\fun{cat}(\fun{post}(t_1,\el),
                                       \fun{post}(t_2,\el)), s)})
& (\fun{rcat})\\
& \xleftarrow{\smash{\eta}}
& \fun{rcat}([x],\fun{rcat}(\fun{cat}(\fun{post}(t_1,\el), \fun{post}(t_2,\el)), s))
& (\fun{rcat})\\
& \equiv_3
& \fun{rcat}(\fun{cat}(\fun{cat}(\fun{post}(t_1,\el), \fun{post}(t_2,\el)), [x]), s)\\
& \equiv_4
& \fun{rcat}(\fun{cat}(\fun{post}(t_1,\el),
                       \fun{cat}(\fun{post}(t_2,\el),
                                 [x])),
             s)\\
& \equiv_5
& \fun{rcat}(\fun{cat}(\fun{post}(t_1,\el),
                       \fun{post}(t_2,[x])),
             s)\\
& \equiv_6
& \fun{rcat}(\fun{post}(t_1,\fun{post}(t_2,[x])),s)\\
& \xleftarrow{\smash{\nu}}
& \fun{rcat}(\fun{post}(\fun{int}(x,t_1,t_2),\el),s)
& (\fun{post/2})\\
& =
& \fun{rcat}(\fun{post}(t,\el),s).
& \hfill\Box
\end{array}
}
\end{equation*}
\caption{Proof of \(\fun{pre}(\fun{mir}(t),s) \equiv
  \fun{rcat}(\fun{post}(t,\el),s)\)}
\label{fig:premir}
\end{figure}
where
\begin{itemize*}

  \item (\(\equiv_0\)) is \(\pred{PreMir}{t_1,s}\), an instance of the
  induction hypothesis;\index{PreMir@\predName{PreMir}}

  \item (\(\equiv_1\)) is
  \(\pred{PreMir}{t_2,\fun{rcat}(\fun{post}(t_1,\el),s)}\), as
  inductive hypothesis; \index{PreMir@\predName{PreMir}}

  \item (\(\equiv_2\)) is
    \(\pred{Rcat}{\fun{post}(t_2,\el),\fun{post}(t_1,\el),s}\);
    \index{Rcat@\predName{Rcat}}

  \item (\(\equiv_3\)) is
    \(\pred{Rcat}{[x],\fun{cat}(\fun{post}(t_1,\el),
      \fun{post}(t_2,\el)),s}\);
    \index{Rcat@\predName{Rcat}}

  \item (\(\equiv_4\)) is \(\pred{CatAssoc}{\fun{post}(t_1,\el),
    \fun{post}(t_2,\el),[x]}\);\index{CatAssoc@\predName{CatAssoc}}

  \item (\(\equiv_5\)) is
  \(\pred{CatPost}{t_2,[x]}\)\index{CatPost@\predName{CatPost}} if
  \(\pred{CatPost}{t,s} \colon \fun{cat}(\fun{post}(t),s) \equiv
  \fun{post}(t,s)\);\index{cat@\fun{cat/2}}

  \item (\(\equiv_6\)) is \(\pred{CatPost}{t_1,\fun{post}(t_2,[x])}\).
  \index{post@\fun{post/2}}\index{CatPost@\predName{CatPost}}

\end{itemize*}
Then \index{mir@\fun{mir/1}|)} \(\pred{CatPost}{t,s} \Rightarrow
\pred{PreMir}{t,s} \Rightarrow \fun{pre}(\fun{mir}(t)) \equiv
\fun{rev}(\fun{post}(t))\).\hfill\(\Box\)

\paragraph{Duality}
\label{thm:duality}

The dual theorem\index{PostMir@\predName{PostMir}} \(\pred{PostMir}{t}
\colon \fun{post}(\fun{mir}(t)) \equiv
\fun{rev}(\fun{pre}(t))\)\index{mir@\fun{mir/1}}
\index{pre@\fun{pre/1}} \index{post@\fun{post/1}}
\index{rev@\fun{rev/1}} can be proved in at least two ways: either we
design a new proof in the spirit of the proof of \(\pred{PreMir}{t}\),
or we take advantage of the fact that the theorem is an equivalence
and we produce equivalent but simpler theorems. Let us do the latter
and start by considering \(\pred{PreMir}{\fun{mir}(t)}\)
\index{PreMir@\predName{PreMir}|)} and proceed by finding equivalent
expressions on both sides of the equivalence, until we reach
\(\pred{PostMir}{t}\):
\begin{equation*}
\begin{array}{@{}r@{\;}c@{\;}l@{\qquad}r@{}}
          \fun{pre}(\fun{mir}(\fun{mir}(t)))
& \equiv
& \fun{rev}(\fun{post}(\fun{mir}(t)))
& (\pred{PreMir}{\fun{mir}(t)})\\
  \fun{pre}(t)
& \equiv
& \fun{rev}(\fun{post}(\fun{mir}(t)))
& (\pred{InvMir}{t})\\
  \fun{rev}(\fun{pre}(t))
& \equiv
& \fun{rev}(\fun{rev}(\fun{post}(\fun{mir}(t))))\\
  \fun{rev}(\fun{pre}(t))
& \equiv
& \fun{post}(\fun{mir}(t))
& (\pred{InvRev}{\fun{post}(\fun{mir}(t))}),
\end{array}
\end{equation*}
where\index{InvMir@\predName{InvMir}} \(\pred{InvMir}{t} \colon
\fun{mir}(\fun{mir}(t)) \equiv t\) and\index{InvRev@\predName{InvRev}}
\(\pred{InvRev}{s} :\Leftrightarrow \pred{Inv}{s} \wedge
\pred{EqRev}{s}\).\hfill\(\Box\)

\paragraph{Exercises}
\begin{enumerate*}

  \item Prove the lemma\index{CatPost@\predName{CatPost}}
  \(\pred{CatPost}{t,s} \colon \fun{cat}(\fun{post}(t,\el),s) \equiv
  \fun{post}(t,s)\).

  \item Use \fun{rev\(_0\)/1} instead of \fun{rev/1} in
  \(\pred{InMir}{t}\) and \(\pred{PreMir}{t}\). Are the proofs easier?

  \item Prove the missing lemma\index{InvMir@\predName{InvMir}}
  \(\pred{InvMir}{t} \colon \fun{mir}(\fun{mir}(t)) \equiv
  t\).\label{ex:mir_mir}

  \item Can you build a binary tree from its postorder and inorder
    traversals, assuming that its internal nodes are all distinct?

\end{enumerate*}
\index{binary tree!postorder|)}

\mypar{Level order}
\index{binary tree!level order|(}

The \emph{level}~\(l\)\index{tree!level} in a tree is a stack of nodes
in preorder\index{binary tree!preorder} whose internal path lengths
are~\(l\).\index{tree!internal path length} In particular, the root is
the only node at level~\(0\). In the tree of \fig~\vref{fig:bt_ex1},
\([3,9,2]\) is level~\(2\). To understand the preorder condition, we
need to consider the preorder numbering of the tree, shown in
\fig~\vref{fig:bt_ex3} with preorder numbers as left exponents to the
contents. This way, there is no more ambiguity when referring to
nodes. For instance, \([3,9,2]\) was in fact ambiguous because there
are two nodes whose associated data is~\(3\). We meant that
\([{}^{2}{3}, {}^{4}{9}, {}^{6}{2}]\) is the level~\(2\) because these
nodes all have internal path lengths~\(2\) \emph{and} have increasing
preorders (\(2,4,6\)).

A \emph{level-order}\index{binary tree!level order} traversal consists
in making a stack with the nodes of all the levels by increasing path
lengths. For instance, the level order of the tree in
\fig~\vref{fig:bt_ex1} is \([8,1,3,3,9,2,5]\). Because this method
visits the sibling of a node before its children, it is said
\emph{breadth\hyp{}first}\index{binary tree!breadth-first traversal}.

In \fig~\vref{fig:lorder} is shown the \emph{level\hyp{}order
  numbering} of the tree in \fig~\vref{fig:bt_ex3}, more often called
\emph{breadth numbering}.  \index{binary tree!breadth numbering} (Mind
the common misspellings `bread numbering' and `breath numbering.')
Notice how the numbers increase along downwards path between nodes, as
in a preorder numbering.

We may now realise that the notion of level in a tree is not
straightforward. The reason is simple: the nodes in a level are not
siblings\index{tree!node!sibling}, except in level~\(1\), so, in
general, we cannot expect to build a level of a tree
\(\fun{int}(x,t_1,t_2)\) by means of levels in \(t_1\) and~\(t_2\)
alone, that is, with a big\hyp{}step design\index{design!big-step
  $\sim$}. As a consequence, a small\hyp{}step approach is called for,
standing in contrast, for example, with
\fun{size/1}\index{size@\fun{size/1}} in~\eqref{eq:size}
\vpageref{eq:size}.

Let \fun{bf\(_0\)/1}\index{bf0@\fun{bf\(_0\)/1}}
(\emph{breadth\hyp{}first}) be the function such that
\(\fun{bf}_0(t)\) is the stack of nodes of~\(t\) in level order. It is
partially defined in \fig~\vref{fig:lo0}.
\begin{figure}[t]
\centering
\includegraphics[bb=108 682 395 721]{lo0}
\caption{Level order \fun{bf\(_0\)/1}}
\label{fig:lo0}
\end{figure}
If we imagine that we cut off the root of a binary tree, we obtain the
immediate subtrees. If, in turn, we cut down these trees, we obtain
more subtrees. This suggests that we should better work on general
forests\index{tree!forest} instead of trees, one or two at a time.

The cutting function is \fun{def/1}\index{def@\fun{def/1}}
(\emph{deforest}), defined in \fig~\vref{fig:def},
\begin{figure}[b]
\abovedisplayskip=0pt
\belowdisplayskip=0pt
\centering
\includegraphics[bb=71 660 324 721]{def}
\caption{Deforestation}
\label{fig:def}
\end{figure}
such that \(\fun{def}(f)\), where \(f\)~represents a forest, evaluates
in a pair \(\pair{r}{f'}\), where \(r\)~are the roots in preorder of
the trees in~\(f\), and \(f'\)~is the immediate subforest
of~\(f\). (Beware, the word deforestation is used by scholars of
functional languages with a different meaning, but it will do for us,
as we already encountered a function
\fun{cut/2}\index{cut@\fun{cut/2}}.) Note how, in \fig~\vref{fig:def},
the inference rule augments the partial level~\(r\) with the
root~\(x\), and how \(t_2\)~is pushed before~\(t_1\) onto the rest of
the immediate forest of~\(f\), to be processed later by
\fun{bf\(_0\)/1}\index{bf0@\fun{bf\(_0\)/1}}. Instead of building the
stack of levels \([[8], [1,3], [3,9,2], [5]]\), we actually flatten
step by step simply by calling \fun{cat/2}\index{cat@\fun{cat/2}} in
rule~\clause{\mu}. If we really want the levels, we would write
\(\cons{r}{\fun{bf}_1(f')}\)\index{bf1@\fun{bf\(_1\)/1}} instead of
\(\fun{cat}(r,\fun{bf}_1(f'))\)\index{cat@\fun{cat/2}}, which, by the
way, reduces the cost.

The underlying concept here is that of the \emph{traversal of a
  forest}. Except in inorder, all the traversals we have discussed
naturally carry over to binary forests: the preorder traversal of a
forest consists in the preorder traversal of the first tree in the
forest, followed by the preorder traversal of the rest of the
forest. Same logic for postorder and level order. This uniformity
stems from the fact that all these traversals are performed
rightwards, to wit, a left child is visited just before its sibling.
The notion of height\index{tree!height} of a tree also extends
naturally to a forest: the height of a forest is the maximum height of
each individual tree. The reason why this is simple is because height
is a purely vertical view of a tree, thus independent of the siblings'
order.

To assess now the cost
\(\C{\fun{bf}_0}{n,h}\)\index{bf0@$\C{\fun{bf}_0}{n,h}$} of the call
\(\fun{bf}_0(t)\), where \(n\)~is the size of the binary tree~\(t\)
and \(h\)~is its height\index{binary tree!height}, it is convenient to
work with \emph{extended levels}\index{binary tree!extended level}. An
extended level is a level where external nodes are included (they are
not implicitly numbered in preorder because external nodes are
indistinguishable). For example, the extended level~\(2\) of the tree
in \fig~\vref{fig:bt_ex3} is \([{}^{2}{3}, {}^{4}{9}, \Box,
{}^{6}{2}]\). If it is needed to draw a contrast with the other kind
of level, we may call the latter \emph{pruned levels}, which is
consistent with our terminology in \fig~\ref{fig:bt_ex} on
page~\pageref{fig:bt_ex}. Note that there is always one more extended
level than pruned levels, made entirely of external nodes. (We avoided
writing that these are the highest nodes, as the trouble with the term
`height'\index{tree!height} is that it really makes sense if the trees
are laid out with the root at the bottom of the page. That is perhaps
why some authors prefer the less confusing concept of
\emph{depth}\index{binary tree!depth}, in use in graph theory. For a
survey of the different manners of drawing trees, see
\cite{Knuth_1997} in its section~2.3.) In other words, \(l_h=0\),
where \(l_i\)~is the number of internal nodes on the extended
level~\(i\).
\begin{itemize*}

  \item Rule~\clause{\iota} is used once;

  \item rule~\clause{\kappa} is used once;

  \item rules~\clause{\lambda} and~\clause{\mu} are used once for each
    extended level of the original tree; these amount to \(2(h+1)\)
    calls;

  \item the cost of \(\fun{cat}(r,\fun{loc}_1(f'))\) is the length of
    the level~\(r\), plus one, thus the cumulative cost of concatenation
    is \(\sum_{i=0}^{h}\C{\fun{cat}}{l_i} = n + h + 1\);

  \item rule~\clause{\epsilon} is used once per extended level, that
    is, \(h+1\) times;

  \item rule~\clause{\zeta} is used once per external node, that is,
    \(n+1\) times;

  \item rules \clause{\eta} and~\clause{\theta} are used once per
    internal node, that is \(2n\)~times.

\end{itemize*}
Gathering these enumerations, we find that
\begin{equation*}
\C{\fun{bf}_0}{n,h} = 4n + 4h + 7. \index{bf0@$\C{\fun{bf}_0}{n,h}$}
\end{equation*}
By definition, the minimum cost is \(\B{\fun{bf}_0}{n} =
\min_h\C{\fun{bf}_0}{n,h}\). The height\index{binary tree!height} is
minimum when the binary tree is \emph{perfect}\index{binary
  tree!perfect $\sim$}, that is, when all levels are full (see
\fig~\vref{fig:comp_tree}). In this case, \(l_i=2^i\), for \(0
\leqslant i \leqslant h-1\), and, by extension, there are
\(2^h\)~external nodes. Theorem~\vref{thm:int_ext} yields the equation
\(n+1 = 2^h\), so \(h=\lg(n+1)\), and \(\B{\fun{bf}_0}{n} = 4n +
4\lg(n+1) + 7 \sim 4n\). \index{bf0@$\B{\fun{bf}_0}{n}$}

% Wrapping figure better declared before a paragraph
%
\begin{wrapfigure}[7]{r}[0pt]{0pt}
% [7] vertical lines
% {r} mandatory right placement
% [0pt] of margin overhang
\centering
\includegraphics[bb=62 653 115 721]{zigzag}
\caption{}
\label{fig:zigzag}
\end{wrapfigure}
The maximum cost is obtained by maximising the height\index{binary
  tree!height}, while keeping the size constant. This happens for
degenerate \index{binary tree!degenerate $\sim$} trees, as the ones
shown in \fig~\vref{fig:tree_stack} and \fig~\vref{fig:zigzag}. Here,
\(h=n\) and the cost is \(\W{\fun{bf}_0}{n} = 8n + 7 \sim
8n\). \index{bf0@$\W{\fun{bf}_0}{n}$}

By contrast, we found programs that perform preorder, postorder and
inorder traversals in \(2n+2\) function calls. It is possible to
reduce the cost by using a different design, based on
\fun{pre\(_3\)/1}\index{pre3@\fun{pre\(_3\)/1}} in
\fig~\vref{fig:pre3}. The difference is that, instead of using a stack
to store subtrees to be traversed later, we use a queue\index{queue},
a linear data structure introduced in
section~\ref{sec:queueing}. Consider in \fig~\vref{fig:lo_abcde} the
algorithm at work on the same example found in
\fig~\vref{fig:pre3_abcde}.
\begin{figure}[b]
\centering
\includegraphics[bb=70 655 385 723]{lo_abcde_0}
\bigskip
\includegraphics[bb=70 690 388 723]{lo_abcde_1}
\caption{A level-order traversal with a queue}
\label{fig:lo_abcde}
\end{figure}
Keep in mind that trees are dequeued on the right side of the forest
and enqueued on the left. (Some authors prefer the other way.) The
root of the next tree to be dequeued is circled.

In order to compare with
\fun{pre\(_4\)/1}\index{pre4@\fun{pre\(_4\)/1}} in
\fig~\vref{fig:pre3}, we write
\(\enq{x}{q}\)\index{queue}\index{enq@$\enq{x}{q}$} instead of
\(\fun{enq}(x,q)\)\index{enq@\fun{enq/2}}, and
\(\deq{x}{q}\)\index{deq@$\deq{x}{q}$} instead of \(\pair{q}{x}\). The
empty queue is noted~\(\emq\)\index{queue@$\emq$}. More importantly,
we will allow these expressions in the patterns of
\fun{bf\(_2\)/1}\index{bf2@\fun{bf\(_2\)/1}}, which performs a
level\hyp{}order traversal in \fig~\ref{fig:bf1}.
\begin{figure}
\begin{equation*}
\boxed{
\begin{array}{r@{\;}c@{\;}l}
\fun{bf}_1(t) & \rightarrow & \fun{bf}_2(\enq{t}{\emq}).\\
\\
\fun{bf}_2(\emq) & \rightarrow & \el;\\
\fun{bf}_2(\deq{\fun{ext}()}{q}) & \rightarrow & \fun{bf}_2(q);\\
\fun{bf}_2(\deq{\fun{int}(x,t_1,t_2)}{q})
  & \rightarrow & \cons{x}{\fun{bf}_2(\enq{t_2}{\enq{t_1}{q}})}.
\end{array}}
\end{equation*}
\caption{Abstract level-order traversal with a queue}
\label{fig:bf1}
\end{figure}
The difference in data structure (accumulator) has already been
mentioned: \fun{pre\(_4\)/1}\index{pre4@\fun{pre\(_4\)/1}} uses a
stack and \fun{bf\(_2\)/1}\index{bf2@\fun{bf\(_2\)/1}} a queue, but,
as far as algorithms are concerned, they differ only in the relative
order in which \(t_1\) and~\(t_2\) are added to the accumulator.

In section~\ref{sec:queueing}, we saw how to implement a queue with
two stacks as \(\fun{q}(r,f)\):\index{q@\fun{q/2}} the rear
stack~\(r\), where items are pushed (logically enqueued), and the
front stack~\(f\), from whence items are popped (logically
dequeued). Also, we defined enqueueing by
\fun{enq/2}\index{enq@\fun{enq/2}} in~\eqref{def:enq},
\vpageref{def:enq}, and dequeueing with
\fun{deq/1}\index{deq@\fun{deq/1}} in~\eqref{def:deq}. This allows us
to refine the definition of
\fun{bf\(_2\)/1}\index{bf2@\fun{bf\(_2\)/1}} into
\fun{bf\(_3\)/1}\index{bf3@\fun{bf\(_3\)/1}} in \fig~\vref{fig:bf3}.
\begin{figure}
\abovedisplayskip=0pt
\belowdisplayskip=0pt
\centering
\framebox[0.9\columnwidth]{\vbox{%
\begin{gather*}
\fun{bf}_3(t) \rightarrow \fun{bf}_4(\fun{enq}(t,\fun{q}(\el,\el))).
\qquad
\fun{bf}_4(\fun{q}(\el,\el)) \rightarrow \el;\\
%
\inferrule
  {\fun{deq}(q) \twoheadrightarrow \pair{q'}{\fun{ext}()}}
  {\fun{bf}_4(q) \twoheadrightarrow \fun{bf}_4(q')};
\quad
\inferrule
  {\fun{deq}(q) \twoheadrightarrow \pair{q'}{\fun{int}(x,t_1,t_2)}}
  {\fun{bf}_4(q) \twoheadrightarrow
   \cons{x}{\fun{bf}_4(\fun{enq}(t_2,\fun{enq}(t_1,q')))}}\,.
\end{gather*}}}
\caption{Refinement of \fig~\vref{fig:bf1}}
\label{fig:bf3}
\end{figure}

We can specialise the program further, so we save some memory by not
using the constructor \fun{q/2}\index{q@\fun{q/2}} and remembering
that its first argument is the rear and the second is the
front. Moreover, instead of calling \fun{deq/1}\index{deq@\fun{deq/1}}
and \fun{enq/2}\index{enq@\fun{enq/1}}, we can expand their
definitions and merge them with the definition at hand. The result is
shown in \fig~\vref{fig:bf}.
\begin{figure}[b]
\begin{equation*}
\boxed{
\begin{array}{r@{\;}c@{\;}l}
\fun{bf}(t) & \xrightarrow{\smash{\nu}} & \fun{bf}(\el,[t]).\\
\\
\fun{bf}(\el,\el) & \xrightarrow{\smash{\xi}} & \el;\\
\fun{bf}(\cons{t}{r},\el) & \xrightarrow{\smash{\pi}} &
  \fun{bf}(\el,\fun{rcat}(r,[t]));\\
\fun{bf}(r,\cons{\fun{ext}()}{f})
  & \xrightarrow{\smash{\rho}} & \fun{bf}(r,f);\\
\fun{bf}(r,\cons{\fun{int}(x,t_1,t_2)}{f})
  & \xrightarrow{\smash{\sigma}}
  & \cons{x}{\fun{bf}(\cons{t_2,t_1}{r},f)}.
\end{array}}
\end{equation*}
\caption{Refinement of \fig~\vref{fig:bf3}}
\label{fig:bf}
\end{figure}
Recall that \fun{rcat/2}\index{rcat@\fun{rcat/2}} (\emph{reverse and
  concatenate}) is defined in equation~\eqref{def:rev}
\vpageref{def:rev}. Note that
\(\fun{bf}_4(\fun{enq}(t,\fun{q}(\el,\el)))\)\index{bf4@\fun{bf\(_4\)/1}}
has been optimised in \(\fun{bf}(\el,[t])\)\index{bf@\fun{bf/2}} in
order to save a stack reversal. The definition of
\fun{bf/1}\index{bf@\fun{bf/1}} can be considered the most
\emph{concrete} of the refinements, the more \emph{abstract} program
being the original definition of
\fun{bf\(_1\)/1}\index{bf1@\fun{bf\(_1\)/1}}. The former is shorter
than \fun{bf\(_0\)/1}\index{bf0@\fun{bf\(_0\)/1}}, but the real gain
is the cost. Let \(n\)~be the size of the binary tree at hand and
\(h\)~its height\index{binary tree!height}. Rules are applied as
follows:
\begin{itemize*}

  \item rule \clause{\nu} is used once;

  \item rule \clause{\xi} is used once;

  \item rule \clause{\pi} is used once per level, except the first one
    (the root), hence, in total \(h\)~times;

  \item all levels but the first (the root) are reversed by
    \fun{rev/1}, accounting \(\sum_{i=1}^{h}\C{\fun{rev}}{e_i} =
    \sum_{i=1}^{h}(e_i+2) = (n-1)+(n+1)+ 2h = 2n +
    2h\),\index{rev@$\C{\fun{rev}}{n}$} where \(e_i\)~is the number of
    nodes on the extended level~\(i\);

  \item rule \clause{\rho} is used once per external node, that is,
    \(n+1\);

  \item rule \clause{\sigma} is used once per internal node, so
    \(n\)~times.

\end{itemize*}
Gathering all these enumerations yields the formula
\begin{equation*}
\C{\fun{bf}}{n,h} = 4n + 3h + 3.\index{bf@$\C{\fun{bf}}{n,h}$}
\end{equation*}
As with \fun{bf\(_0\)/1}\index{bf0@\fun{bf\(_0\)/1}}, the minimum cost
happens here when \(h = \lg(n+1)\), so \(\B{\fun{bf}}{n} = 4n +
3\lg(n+1) + 3 \sim 4n\).\index{bf@$\B{\fun{bf}}{n,h}$} The maximum
cost occurs when \(h=n\), so \(\W{\fun{bf}}{n} = 7n + 3 \sim 7n\). We
can now compare \fun{bf\(_0\)/1}\index{bf0@\fun{bf\(_0\)/1}} and
\fun{bf/1}\index{bf@\fun{bf/1}}: \(\C{\fun{bf}}{n,h} <
\C{\fun{bf}_0}{n,h}\)\index{bf@$\C{\fun{bf}}{n,h}$}
\index{bf0@$\C{\fun{bf}_0}{n,h}$} and the difference in cost is most
observable in their worst cases, which are both degenerate
trees. Therefore \fun{bf/1} is preferable in any case.

\paragraph{Termination}
\index{termination!level order|(} \index{binary tree!level
  order!termination|(} With the aim to prove the termination of
\fun{bf/2}\index{bf@\fun{bf/2}}, we reuse the lexicographic order on
pairs of stacks, based on the immediate substack order
(\(\succ_{S}\))\index{induction!immediate substack order} that proved
the termination of \fun{mrg/2}\index{mrg@\fun{mrg/2}} on page
\pageref{merging_termination}:
\begin{equation*}
(s,t) \succ_{S^2} (s',t') :\Leftrightarrow \text{\(s \succ_{S} s'\) or
    (\(s = s'\) and \(t \succ_{S} t'\))}.
\end{equation*}
Unfortunately, (\(\succ_{S^2}\)) fails to monotonically order (with
respect to the rewrite relation) the left\hyp{}hand side and
right\hyp{}hand side of rule~\(\sigma\), because of
\((r,\cons{\fun{int}(x,t_1,t_2)}{f}) \nsucc_{S^2}
(\cons{t_2,t_1}{r},f)\). Another approach consists in defining a
well\hyp{}founded order on the number of nodes in a pair of forests:
\begin{equation*}
(r,f) \succ_{S^2} (r',f') :\Leftrightarrow \fun{dim}(r) +
  \fun{dim}(f) > \fun{dim}(r') + \fun{dim}(f'),
\end{equation*}
with
\begin{equation*}
\fun{dim}(\el) \rightarrow \el;\qquad
\fun{dim}(\cons{t}{f}) \rightarrow \fun{size}(t) + \fun{dim}(f).
\end{equation*}
where \fun{size/1}\index{size@\fun{size/1}}\index{binary tree!size} is
defined in~\eqref{eq:size} \vpageref{eq:size}. This is a kind of
polynomial measure\index{termination!polynomial measure} on dependency
pairs\index{termination!dependency pair}, as exemplified with
\fun{flat/1}\index{flat@\fun{flat/1}} on
page~\pageref{flattening:termination}. Here, \(\measure{\fun{bf}(s,t)}
:= \fun{dim}(s) + \fun{dim}(t)\). Unfortunately, this order
monotonically fails on rule~\(\pi\), because \((r,\el) \nsucc_{S^2}
(\el,\fun{rev}(r))\).

The conundrum can be lifted if we visualise the complete set of
traces\index{functional language!evaluation!trace} of
\fun{bf/2}\index{bf@\fun{bf/2}} in a compact manner. If we assume that
\fun{rev/1}\index{rev@\fun{rev/1}} is a constructor, the
right\hyp{}hand sides either contain no call or exactly one recursive
call. The traces of calls to such definitions are nicely represented
as finite automata.\index{finite automaton} An example of a
deterministic finite automaton\index{finite automaton!deterministic
  $\sim$|(} (DFA) was given in \fig~\vref{fig:abacabac}. Here, a
transition is a rewrite rule and a state corresponds to an abstraction
of the input. In the case of \fun{bf/2}, the input is a pair of
stacks. Let us decide for the moment that we will only take into
account whether a stack is empty or not, yielding four states. Let
`\texttt{|}' denote an arbitrary non\hyp{}empty stack. Examining the
definitions of \fun{bf/1}\index{bf@\fun{bf/1}} and
\fun{bf/2}\index{bf@\fun{bf/2}} in \fig~\vref{fig:bf}, we see that
\begin{itemize*}

  \item rule~\(\xi\) applies to the state \((\el,\el)\) only;

  \item rule~\(\pi\) applies to the state \((\mid,\el)\), and leads to
    a state \((\el,\mid)\);

  \item rule~\(\rho\) applies to the states \((\el,\mid)\) and
    \((\mid,\mid)\), and leads to any state;

  \item rule~\(\sigma\) applies to the states \((\el,\mid)\) and
    \((\mid,\mid)\), and leads to the states \((\mid,\el)\) and
    \((\mid,\mid)\).

\end{itemize*}
In \fig~\vref{fig:lo_nfa},
\begin{figure}
\centering
\subfloat[Non-deterministic (NFA)\label{fig:lo_nfa}]{
\includegraphics[bb=71 580 209 721]{lo_nfa}}
\quad
\subfloat[Deterministic (DFA)\label{fig:lo_dfa}]{
\includegraphics[bb=71 580 209 721]{lo_dfa}}
\caption{Traces of \fun{bf/2} as finite automata}
\end{figure}
we gather all this connectivity into a finite automaton. Note that, by
definition, the initial state has an incoming edge~\(\nu\) without
source and the final state has an outgoing edge~\(\xi\) without
destination. A trace is any sequence of transitions from the initial
state \((\el,\mid)\) to the final state \((\el,\el)\), for example,
\(\nu\rho^p\sigma^q\pi\rho\xi\), with \(p \geqslant 0\) and \(q
\geqslant 2\). This automaton is called a \emph{non\hyp{}deterministic
  finite automaton}\index{finite automaton!non\hyp{}deterministic
  $\sim$|(} (NFA) because a state may have more than one outgoing
transition with the same label (consider the initial state and the two
transitions labelled~\(\sigma\), for example).

It is always possible to construct a \emph{deterministic finite
  automaton} (DFA) equivalent to a given non\hyp{}deterministic finite
automaton \citep{VanLeeuwen_1990c,HopcroftMotwaniUllman_2003}. The
outgoing transitions of each state of the former have a unique
label. Equivalence means that the sets of traces of each automaton are
the same. If `\texttt{?}' denotes a stack, empty or not,
\fig~\vref{fig:lo_dfa} shows an equivalent DFA for the traces of
\fun{bf/1}\index{bf@\fun{bf/1}} and \fun{bf/2}\index{bf@\fun{bf/2}}.

As we observed earlier, using the well\hyp{}founded order
\((\succ_{S^2})\) based on the sizes of the stacks, all transitions
\(x \rightarrow y\) in the DFA satisfy \(x \succ_{S^2} y\),
except~\(\pi\), for which \(x =_{S^2} y\) holds (the total number of
nodes is invariant). We can nevertheless conclude that \fun{bf/2}
terminates because the only way to have non\hyp{}termination would be
the existence of a \(\pi\)-\emph{circuit}, that is, a series of
successive transitions from a state to itself all labelled
with~\(\pi\), along which the number of nodes remains identical. In
fact, all traces have~\(\pi\) followed by~\(\rho\) or~\(\sigma\).

Yet another spin on this matter would be to prove, by examining all
rules in isolation and all compositions of two rules, that
\begin{equation*}
x \rightarrow y \Rightarrow x \succcurlyeq_{S^2} y
\quad \text{and} \quad
x \xrightarrow{\smash{2}} y \Rightarrow x \succ_{S^2} y.
\end{equation*}
Consequently, if \(n > 0\), then \(x \xrightarrow{\smash{2n}} y\)
entails \(x \succ_{S^2} y\), because \((\succ_{S^2})\) is
transitive. Furthermore, if \(x \xrightarrow{\smash{2n}} y \rightarrow
z\), then \(x \succ_{S^2} y \succcurlyeq_{S^2} z\), hence \(x
\succ_{S^2} z\). In the end, \(x \xrightarrow{\smash{n}} y\) implies
\(x \succ_{S^2} y\), for all \(n > 1\).\index{finite
  automaton!non\hyp{}deterministic $\sim$|)}\index{finite
  automaton!deterministic $\sim$|)}\index{binary tree!level
  order!termination|)}\index{termination!level order|)}\hfill\(\Box\)

\paragraph{Breadth-first numbering}
\index{binary tree!breadth numbering|(}

As mentioned earlier, \fig~\vref{fig:lorder} shows an example of
breadth\hyp{}first numbering. This problem has received notable
attention \citep{JonesGibbons_1993,GibbonsJones_1998,Okasaki_2000}
because functional programmers usually feel a bit challenged by this
problem. A good approach consists in modifying the function
\fun{bf\(_1\)/2}\index{bf1@\fun{bf\(_1\)/2}} in \fig~\vref{fig:bf1} so
that it builds a tree instead of a stack. We do so by enqueueing
immediate subtrees, so they are recursively numbered when they are
dequeued, and by incrementing a counter, initialised with~\(0\), every
time a non\hyp{}empty tree has been dequeued.

Consider \fun{bfn\(_1\)/1}\index{bfn1@\fun{bfn\(_1\)/1}} and
\fun{bfn\(_2\)/2}\index{bfn2@\fun{bfn\(_2\)/2}}
(\emph{breadth\hyp{}first numbering}) in \fig~\vref{fig:bfn1},
\begin{figure}
\abovedisplayskip=0pt
\belowdisplayskip=0pt
\centering
\framebox[0.87\columnwidth]{\vbox{%
\begin{gather*}
\inferrule
  {\fun{bfn}_2(0,\enq{t}{\emq}) \twoheadrightarrow \deq{t'}{\emq}}
  {\fun{bfn}_1(t) \twoheadrightarrow t'}.
\\
\fun{bfn}_2(i,\emq) \rightarrow \emq;
\qquad
\fun{bfn}_2(i,\deq{\fun{ext}()}{q}) \rightarrow
  \enq{\fun{ext}()}{\fun{bfn}_2(i,q)};
\\
\inferrule
  {\fun{bfn}_2(i+1,\enq{t_2}{\enq{t_1}{q}}) \twoheadrightarrow
   \deq{t'_2}{\deq{t'_1}{q'}}}
  {\fun{bfn}_2(i,\deq{\fun{int}(x,t_1,t_2)}{q}) \twoheadrightarrow
   \enq{\fun{int}(i,t'_1,t'_2)}{q'}}.
\end{gather*}}}
\caption{Abstract breadth-first numbering}
\label{fig:bfn1}
\end{figure}
and compare them with the definitions in \fig~\vref{fig:bf1}. In
particular, notice how, in contrast with
\fun{bf\(_1\)/2}\index{bf1@\fun{bf\(_1\)/2}}, external nodes are
enqueued instead of being discarded, because they are later needed to
make the numbered tree.

In \fig~\vref{fig:bf2_ex}
\begin{figure}[t]
\centering
\includegraphics{bf2_abcd_0}
\includegraphics{bf2_abcd_1}
\includegraphics{bf2_abcd_2}
\includegraphics{bf2_abcd_3}
\includegraphics[bb=70 674 275 721]{bf2_abcd_4}
\includegraphics[bb=69 691 276 721]{bf2_abcd_5}
\includegraphics[bb=71 690 276 721]{bf2_abcd_6}
\caption{Example of breadth-first numbering}
\label{fig:bf2_ex}
\end{figure}
is shown an example, where the numbers on the left represent the
values of~\(i\) (the first argument of
\fun{bf\(_2\)/2}\index{bf2@\fun{bf\(_2\)/2}}), the downward rewrites
define the successive states of the working queue (the second argument
of \fun{bf\(_2\)/2}\index{bf2@\fun{bf\(_2\)/2}}), and the upward
rewrites show the successive states of the resulting queue
(right\hyp{}hand side of
\fun{bf\(_2\)/2}\index{bf2@\fun{bf\(_2\)/2}}). Recall that trees are
enqueued on the left and dequeued on the right (other authors may use
the opposite convention, as \cite{Okasaki_2000}) and pay attention to
the fact that, in the vertical rewrites on the left, \(t_1\)~is
enqueued first whilst, on the right, \(t'_2\)~is dequeued first, which
appears when contrasting \(\enq{t_2}{\enq{t_1}{q}} =
\enq{t_2}{(\enq{t_1}{q})}\)\index{enq@$\enq{x}{q}$} and
\(\deq{t'_2}{\deq{t'_1}{q'}} =
\deq{t'_2}{(\deq{t'_1}{q'})}\)\index{deq@$\deq{x}{q}$}.

We can refine \fun{bfn\(_1\)/1}\index{bfn1@\fun{bfn\(_1\)/1}} and
\fun{bfn\(_2\)/2}\index{bfn2@\fun{bfn\(_2\)/2}} by introducing
explicitly the function calls for enqueueing and dequeueing, as shown
in \fig~\vref{fig:bfn3},
\begin{figure}[t]
\centering
\includegraphics{bfn3}
\caption{Refinement of \fig~\vref{fig:bfn1}}
\label{fig:bfn3}
\end{figure}
which could be contrasted with \fig~\vref{fig:bf3}.
\index{binary tree!breadth numbering|)}
\index{binary tree!level order|)}

\paragraph{Exercises}
\begin{enumerate*}

  \item How would you proceed to prove the correctness of
  \fun{bfn/1}?\index{bfn@\fun{bfn/1}}

  \item Find the cost \(\C{\fun{bfn}}{n}\) of \(\fun{bfn}(t)\), where
  \(n\)~is the size
  of~\(t\).\index{bfn@$\C{\fun{bfn}}{n}$}\index{bfn@\fun{bfn/1}}

\end{enumerate*}



\section{Classic shapes}

In this section, we briefly review some particular binary trees which
are useful in assessing the extremal costs of many algorithms.

\paragraph{Perfection}
\label{par:perfection}

\index{binary tree!perfect $\sim$|(} We already mentioned what a
\emph{perfect binary tree} is in the context of optimal sorting (see
\fig~\vref{fig:comp_tree}). One way to define such trees is to say
that all their external nodes belong to the same level or,
equivalently, the immediate subtrees of any node have same
height\index{tree!height}. (The height\index{tree!height} of an
external node is~\(0\).) In \fig~\vref{fig:per} is shown the
definition of \fun{per/1}\index{per@\fun{per/1}} (\emph{perfection}).
If the tree~\(t\) is perfect, we also know its height~\(h\):
\(\fun{per}(t) \twoheadrightarrow \fun{true}(h)\).
\begin{figure}[b]
\centering
\includegraphics{per}
\caption{Abstract checking of perfection}
\label{fig:per}
\end{figure}
Note that the rules are ordered, so the last one may only apply when
the previous ones have not been matched. A refinement without
inference rules is shown in \fig~\vref{fig:per0},
\begin{figure}
\centering
\includegraphics[bb=71 604 251 721]{per0}
\caption{Refinement of \fig~\ref{fig:per}}
\label{fig:per0}
\end{figure}
where \(\fun{per}_0(t_1)\) is evaluated before
\(\fun{per}_0(t_2)\)\index{per0@\fun{per\(_0\)/1}} since
\(\fun{t}(\fun{per}_0(t_1), \fun{per}_0(t_2))\) is inefficient if
\(\fun{per}_0(t_1) \twoheadrightarrow \fun{false}()\).\index{binary
  tree!perfect $\sim$|)}


\paragraph{Completeness}

A binary tree is \emph{complete}\index{binary tree!complete $\sim$} if
the children of every internal node are either two external nodes or
two internal nodes themselves. An example is provided in
\fig~\vref{fig:complete}.
% Wrapping figure better declared before a paragraph
%
\begin{wrapfigure}[8]{r}[0pt]{0pt}
% [8] vertical lines
% {r} mandatory right placement
% [0pt] of margin overhang
%\begin{figure}[b]
\centering
\includegraphics{complete}
\caption{A complete tree}
\label{fig:complete}
\end{wrapfigure}
Recursively, a given tree is complete if,
and only if, its immediate subtrees are complete. This is the same
rule we used for perfection. In other words, perfection and
completeness are propagated bottom\hyp{}up. Therefore, we need to
decide what to say about the external nodes, in particular, the empty
tree. If we decide that the latter is complete, then
\(\fun{int}(x,\fun{ext}(),\fun{int}(y,\fun{ext}(),\fun{ext}()))\)
would, incorrectly, be deemed complete. If not, leaves\index{binary
  tree!leaf} \(\fun{int}(x,\fun{ext}(),\fun{ext}())\) would,
incorrectly, be found to be incomplete. Thus, we can choose either
option and handle the problematic case separately; for example, we may
choose that external nodes are incomplete trees, but leaves are
complete trees. The program is shown in \fig~\vref{fig:comp}. The last
rule applies either if \(t=\fun{ext}()\) or
\(t=\fun{int}(x,t_1,t_2)\), with \(t_1\) or~\(t_2\) incomplete.

\bigskip

\begin{figure}[H]
\centering
\includegraphics{comp}
\caption{Checking completeness}
\label{fig:comp}
\end{figure}

\paragraph{Balance}

\begin{wrapfigure}[7]{r}[0pt]{0pt}
\centering
\includegraphics[bb=71 653 153 709]{bal}
\caption{}
\label{fig:bal}
\end{wrapfigure}
The last interesting kind of binary trees is the \emph{balanced
  trees}.\index{binary tree!balanced $\sim$} There are two sorts of
criteria to define balance: either height or size
\citep{NievergeltReingold_1972,HiraiYamamoto_2011}. In the latter
case, siblings are roots of trees with similar sizes; in the former
case, they have similar height. The usual criterion being height, we
will use it in the following. Depending on the algorithm, what
`similar' means may vary. For example, we may decide that two trees
whose heights differ at most by~\(1\) have similar heights. See
\fig~\vref{fig:bal} for an instance. Let us start with a definition of
the height\index{binary tree!height}\index{height@\fun{height/1}} of a
binary tree and then modify it to obtain a function checking the
balance:
\begin{equation*}
\fun{height}(\fun{ext}()) \rightarrow 0;
\;
\fun{height}(\fun{int}(x,t_1,t_2)) \rightarrow 1 +
\max\{\fun{height}(t_1), \fun{height}(t_2)\}.
\end{equation*}

The modification is shown in \fig~\vref{fig:balance}, where the
inference rule of \fun{bal\(_0\)/1}\index{bal0@\fun{bal\(_0\)/1}} is
needed to check the condition \(|h_1 - h_2| \leqslant 1\). Note that a
perfect tree is balanced (\(h_1 = h_2\)).

\bigskip

\begin{figure}[h]
\centering
\includegraphics{balance}
\caption{Checking balance}
\label{fig:balance}
\end{figure}

\section{Tree encodings}

In general, many binary trees yield the same preorder, postorder or
inorder traversal, so it is not possible to rebuild the original tree
from one traversal alone. The problem of uniquely representing a
binary tree by a linear structure is called \emph{tree
  encoding} \citep{Makinen_1991} and is related to the problem of
generating all binary trees of a given
size \citep[7.2.1.6]{Knuth_2011}. One simple approach consists in
extending a traversal with the external nodes; this way, enough
information from the binary tree is retained in the encoding, allowing
us to unambiguously go back to the original tree.

The encoding function
\fun{epost/1}\index{epost@\fun{epost/1}}\index{epost@\fun{epost/2}}
(\emph{extended postorder})\index{binary tree!postorder!encoding} in
\fig~\vref{fig:epost},
\begin{figure}
\begin{equation*}
\boxed{
\begin{array}{r@{\;}c@{\;}l}
\fun{epost}(t) & \rightarrow & \fun{epost}(t,\el).\\
\\
\fun{epost}(\fun{ext}(),s) & \rightarrow
  & \cons{\fun{ext}()}{s};\\
\fun{epost}(\fun{int}(x,t_1,t_2),s) & \rightarrow
  & \fun{epost}(t_1,\fun{epost}(t_2,\cons{x}{s})).
\end{array}}
\end{equation*}
\caption{Postorder encoding}
\label{fig:epost}
\end{figure}
is a simple modification of \fun{post/1}\index{post@\fun{post/1}} in
\fig~\vref{fig:post}. For example, the tree in
\fig~\vref{fig:postorder} yields \([\Box, \Box, \Box, 0, 1, \Box,
\Box, 2, 3, \Box, \Box, \Box, 4, 5, 6]\), where \(\Box\) stands for
\(\fun{ext}()\). Since a binary tree with \(n\)~internal nodes has
\(n+1\) external nodes (see theorem~\ref{thm:int_ext}
\vpageref{thm:int_ext}), the cost is straightforward to find:
\(\C{\fun{epost}}{n} = 2n + 2\).\index{epost@$\C{\fun{epost}}{n}$}

We already noticed that the postorder of the nodes increases along
upwards paths, which corresponds to the order in which a tree is
built: from the external nodes up to the root. Therefore, all we have
to do is to identify the growing subtrees by putting the unused
numbers and subtrees in an auxiliary stack: when the contents of a
root (anything different from \(\fun{ext}()\)) appears in the original
stack, we can make an internal node with the two first subtrees in the
auxiliary stack.

The definition of \fun{post2b/1}\index{post2b@\fun{post2b/1}}
(\emph{extended postorder to binary tree}) is given in
\fig~\vref{fig:post2b}.\index{binary tree!postorder!decoding}
\begin{figure}[b]
\begin{equation*}
\boxed{
\begin{array}{r@{\;}c@{\;}l}
\fun{post2b}(s) & \rightarrow & \fun{post2b}(\el,s).\\
\\
\fun{post2b}([t],\el) & \rightarrow & t;\\
\fun{post2b}(f,\cons{\fun{ext}()}{s}) & \rightarrow & \fun{post2b}(\cons{\fun{ext}()}{f},s);\\
\fun{post2b}(\cons{t_2,t_1}{f},\cons{x}{s}) & \rightarrow & \fun{post2b}(\cons{\fun{int}(x,t_1,t_2)}{f},s).
\end{array}}
\end{equation*}
\caption{Postorder decoding}
\label{fig:post2b}
\end{figure}
Variable~\(f\) stands for \emph{forest}\index{tree!forest}, which is
how stacks of trees are usually called in computing science. Note that
a problem would arise if the original tree contains trees because, in
that case, an external node contained in an internal node would
confuse \fun{post2b/1}\index{post2b@\fun{post2b/1}}. The cost is easy
to assess because a postorder encoding must have length \(2n+1\),
which is the total number of nodes of a binary with \(n\) internal
nodes. Therefore, \(\C{\fun{post2b}}{n} = 2n +
3\).\index{post2b@$\C{\fun{post2b}}{n}$} The expected theorem is, of
course,
\begin{equation}
\fun{post2b}(\fun{epost}(t)) \equiv t.
\label{thm:post2b_epost}
\end{equation}

Considering preorder now, the encoding function
\fun{epre/1}\index{epre@\fun{epre/1}}\index{epre@\fun{epre/2}}
(\emph{extended preorder}) in \fig~\vref{fig:epre}
\begin{figure}
\begin{equation*}
\boxed{
\begin{array}{r@{\;}c@{\;}l}
\fun{epre}(t) & \rightarrow & \fun{epre}(t,\el).\\
\\
\fun{epre}(\fun{ext}(),s) & \rightarrow
  & \cons{\fun{ext}()}{s};\\
\fun{epre}(\fun{int}(x,t_1,t_2),s) & \rightarrow
  & \cons{x}{\fun{epre}(t_1,\fun{epre}(t_2,s))}.
\end{array}}
\end{equation*}
\caption{Preorder encoding}
\label{fig:epre}
\end{figure}
is a simple modification of \fun{pre/1}\index{pre@\fun{pre/1}} in
\fig~\vref{fig:pre}. The cost is as simple as for a postorder
encoding: \(\C{\fun{epre}}{n} =
2n+2\).\index{epre@$\C{\fun{epre}}{n}$}

Working out the inverse function, from preorder encodings back to
binary trees, is a little trickier than for postorder traversals,
because the preorder numbers increase downwards in the trees, which is
the opposite direction in which trees grow (programmers have trees
grow from the leaves to the root). One solution consists in recalling
the relationship \(\pred{PreMir}{t}\)\index{PreMir@\predName{PreMir}}
between preorder and postorder we proved earlier
\vpageref{proof:PreMir}:
\begin{equation*}
\abovedisplayskip=7pt
\belowdisplayskip=7pt
\fun{pre}(\fun{mir}(t)) \equiv \fun{rev}(\fun{post}(t)).
\end{equation*}
We should extend the proof of this theorem so we have
\begin{equation}
\abovedisplayskip=7pt
\belowdisplayskip=7pt
\fun{epre}(\fun{mir}(t)) \equiv \fun{rev}(\fun{epost}(t)).
\label{thm:epre_epost}
\index{rev@\fun{rev/1}}
\end{equation}
In section~\ref{sec:reversal}, we proved
\(\pred{Inv}{s}\)\index{Inv@\predName{Inv}} and
\(\pred{EqRev}{s}\)\index{EqRev@\predName{EqRev}}, that is to say, the
involution of
\fun{rev/1}\index{rev@\fun{rev/1}}\index{stack!reversal!involution}:
\begin{equation}
\abovedisplayskip=7pt
\belowdisplayskip=7pt
\fun{rev}(\fun{rev}(s)) \equiv t.
\label{thm:rev_inv}
\end{equation}
Property~\eqref{thm:rev_inv} and~\eqref{thm:epre_epost} yield
\begin{equation*}
\abovedisplayskip=7pt
\belowdisplayskip=7pt
\fun{rev}(\fun{epre}(\fun{mir}(t))) \equiv
\fun{rev}(\fun{rev}(\fun{epost}(t))) \equiv \fun{epost}(t).
\end{equation*}
Applying~\eqref{thm:post2b_epost}, we obtain
\begin{equation*}
\abovedisplayskip=7pt
\belowdisplayskip=7pt
\fun{post2b}(\fun{rev}(\fun{epre}(\fun{mir}(t)))) \equiv
\fun{post2b}(\fun{epost}(t)) \equiv t.
\end{equation*}
From exercise~\ref{ex:mir_mir} \vpageref{ex:mir_mir}, we have
\(\fun{mir}(\fun{mir}(t)) \equiv t\), therefore
\begin{equation*}
\abovedisplayskip=7pt
\belowdisplayskip=7pt
\fun{post2b}(\fun{rev}(\fun{epre}(t))) \equiv \fun{mir}(t)
\quad\text{hence}\quad
\fun{mir}(\fun{post2b}(\fun{rev}(\fun{epre}(t)))) \equiv t.
\end{equation*}
Because we want the encoding followed by the decoding to be the
identity, \(\fun{pre2b}(\fun{epre}(t)) \equiv t\), we have
\(\fun{pre2b}(\fun{epre}(t)) \equiv
\fun{mir}(\fun{post2b}(\fun{rev}(\fun{epre}(t))))\), that is, setting
the stack \(s := \fun{epre}(t)\),
\begin{equation*}
\abovedisplayskip=7pt
\belowdisplayskip=7pt
\fun{pre2b}(s) \equiv \fun{mir}(\fun{post2b}(\fun{rev}(s))).
\end{equation*}
We obtain \fun{pre2b/1}\index{pre2b@\fun{pre2b/1}}\index{binary
  tree!preorder!decoding} by modifying
\fun{post2b/1}\index{post2b@\fun{post2b/1}} in \fig~\vref{fig:pre2b}.
\begin{figure}
\begin{equation*}
\boxed{
\begin{array}{r@{\;}c@{\;}l}
\fun{pre2b}(s) & \rightarrow & \fun{pre2b}(\el,\fun{rev}(s)).\\
\\
\fun{pre2b}([t],\el) & \rightarrow & t;\\
\fun{pre2b}(f,\cons{\fun{ext}()}{s}) & \rightarrow & \fun{pre2b}(\cons{\fun{ext}()}{f},s);\\
\fun{pre2b}(\cons{t_1,t_2}{f},\cons{x}{s}) & \rightarrow &
\fun{pre2b}(\cons{\fun{int}(x,t_1,t_2)}{f},s).
\end{array}}
\end{equation*}
\caption{Preorder decoding}
\label{fig:pre2b}
\end{figure}
The difference between \fun{pre2b/2}\index{pre2b@\fun{pre2b/2}} and
\fun{post2b/2}\index{post2b@\fun{post2b/2}} lies in their last
pattern, that is, \(\fun{post2b}(\cons{t_2,t_1}{f},\cons{x}{s})\)
versus \(\fun{pre2b}(\cons{t_1,t_2}{f},\cons{x}{s})\), which
implements the fusion of \fun{mir/1} and
\fun{post2b/1}. Unfortunately, the cost of \(\fun{pre2b}(t)\)
\index{pre2b@\fun{pre2b/1}} is greater than the cost of
\(\fun{post2b}(t)\) because of the stack reversal \(\fun{rev}(s)\) at
the start: \(\C{\fun{pre2b}}{n} = 2n + 3 + \C{\fun{rev}}{n} = 3n +
5\).\index{pre2b@$\C{\fun{pre2b}}{n}$}\index{rev@$\C{\fun{rev}}{n}$}

The design of \fun{pre2b/1}\index{pre2b@\fun{pre2b/1}} is based on
small steps with an accumulator. A more direct approach would extract
the left subtree and then the right subtree from the rest of the
encoding. In other words, the new version
\(\fun{pre2b}_1(s)\)\index{pre2b1@\fun{pre2b\(_1\)/1}} would return a
tree build from a prefix of the encoding~\(s\), paired with the rest
of the encoding. The definition is displayed in
\fig~\vref{fig:pre2b0}.
\begin{figure}[b]
\abovedisplayskip=0pt
\belowdisplayskip=0pt
\centering
\framebox[0.87\columnwidth]{\vbox{%
\begin{gather*}
\inferrule
   {\fun{pre2b}_1(s) \twoheadrightarrow \pair{t}{\el}}
   {\fun{pre2b}_0(s) \twoheadrightarrow t}.
\qquad
\fun{pre2b}_1(\cons{\fun{ext}()}{s}) \rightarrow
\pair{\fun{ext}()}{s};
\\
\inferrule
  {\fun{pre2b}_1(s) \twoheadrightarrow \pair{t_1}{s_1}
   \and
   \fun{pre2b}_1(s_1) \twoheadrightarrow \pair{t_2}{s_2}}
  {\fun{pre2b}_1(\cons{x}{s}) \twoheadrightarrow \pair{\fun{int}(x,t_1,t_2)}{s_2}}.
\end{gather*}}}
\caption{Another preorder decoding}
\label{fig:pre2b0}
\end{figure}
Notice the absence of any adventitious concept, contrary to
\fun{pre2b/1}\index{pre2b@\fun{pre2b/1}}, which relies on the
reversal\index{stack!reversal} of a stack and a theorem about mirror
trees and postorders. To wit,
\fun{pre2b\(_0\)/1}\index{pre2b0@\fun{pre2b\(_0\)/1}} is conceptually
simpler, although its cost is greater than that of
\fun{pre2b/1}\index{pre2b@\fun{pre2b/1}} because we count the number
of function calls after the inference rules\index{inference system}
are translated into the core functional language (so two more calls
matching \(\pair{t_1}{s_1}\) and \(\pair{t_2}{s_2}\) are implicit).

Tree encodings show that it is possible to compactly represent binary
trees, as long as we do not care for the contents of the internal
nodes. For instance, we mentioned that the tree in
\fig~\vref{fig:postorder} yields the extended postorder\index{binary
  tree!postorder!encoding} traversal \([\Box, \Box, \Box, 0, 1, \Box,
\Box, 2, 3, \Box, \Box, \Box, 4, 5, 6]\). If we only want to retain
the shape of the tree, we could replace the contents of the internal
nodes by~\(0\) and the external nodes by~\(1\), yielding the encoding
\([1,1,1,0,0,1,1,0,0,1,1,1,0,0,0]\). A binary tree of size~\(n\) can
be uniquely represented by a binary number of \(2n+1\) bits. In fact,
we can discard the first bit because the first two bits are
always~\(1\), so \(2n\)~bits are actually enough. For an extended
preorder traversal, we choose to map external nodes to~\(0\) and
internal nodes to~\(1\), so, the tree in \fig~\vref{fig:preorder}
yields \([0, 1, 2, \Box, 3, \Box, \Box, 4, \Box, \Box, 5, \Box, 6,
\Box, \Box]\) and \((111010010010100)_2\). We can also discard the
rightmost bit, since the last two bits are always~\(0\).

\section{Random traversals}

Some applications require a tree traversal to depend on the
interaction with a user or another piece of software, that is, the
tree is supplemented with the notion of a current node so the next
node to be visited can be chosen amongst any of the children, the
parent or even the siblings. This interactivity stands in contrast
with preorder, inorder and postorder, where the visit order is
predetermined and cannot be changed during the traversal.

Normally, the visit of a functional data structure starts always at
the same location, for example, in the case of a stack, it is the top
item and, in the case of a tree, the access point is the
root. Sometimes, updating a data structure with an online algorithm
(see page~\pageref{par:online_vs_offline} and section~\ref{sec:online}
\vpageref{sec:online}) requires to keep a direct access `inside' the
data structure, usually where the last update was performed, or
nearby, in view of a better amortised (see
page~\pageref{par:amortised_cost}) or average cost (see 2-way
insertion in section~\ref{sec:2-way} \vpageref{sec:2-way}).

\begin{wrapfigure}[8]{r}[0pt]{0pt}
\centering
\includegraphics[bb=64 663 118 721]{binzip}
\caption{}
\label{fig:binzip}
\end{wrapfigure}
Let us call the current node the \emph{slider}\index{binary
  tree!slider}, also called the \emph{focus}. A
\emph{zipper}\index{binary tree!zipper} on a binary tree is made of a
subtree, whose root is the slider, and a \emph{path} from it up to the
root. That path is the reification, in reverse order, of the recursive
calls that led to the subtree (the \emph{call stack}\index{functional
  language!call stack}), together with the subtrees left unvisited on
the way down. Put in more abstract terms, a zipper is made of a linear
context (a rooted path) and a substructure (at the end of that path),
whose handle is the focus. In one move, it is possible to visit the
children in any order, the parent or the sibling. Consider
\fig~\ref{fig:binzip} where the slider is the node~\fun{d}. The
substructure is
\begin{equation*}
  s := \fun{int}(\fun{d}(),\fun{int}(\fun{e}(),\fun{ext}(),\fun{ext}()),
       \fun{int}(\fun{f}(),\fun{ext}(),\fun{ext}())).
\end{equation*}
To define rooted paths for the zipper, we need three data
constructors: one denoting the empty path, \(\fun{top}()\), one
denoting a turn to the left, \(\fun{left}(x,t,p)\), where \(x\)~is a
node on the path, \(t\)~is the right subtree of~\(x\), and~\(p\) is
the rest of the path up to the root, and one constructor denoting a
turn to the right, \(\fun{right}(x,t,p)\), where \(t\)~is the left
subtree of~\(x\).

Resuming our example above, the zipper is then the pair
\(\pair{p}{s}\), with the path~\(p\) being \(\fun{right}(\fun{b}(),
\fun{int}(\fun{c}(), \fun{ext}(), \fun{ext}()),p_1)\), meaning that
the node \fun{b}~has an unvisited left child~\fun{c} (or,
equivalently, we turned right when going down), and where \(p_1 :=
\fun{left}(\fun{a}(), \fun{int}(\fun{g}(), \fun{ext}(), \fun{ext}()),
\fun{top}())\), meaning that the node~\fun{a} has an unvisited right
child~\fun{g}, and that it is the root of the whole tree, due to
\(\fun{top}()\). Note that since \fun{b}~is the first in the path up,
it is the parent of the slider~\fun{d}.

At the beginning, the original tree~\(t\) is injected into a zipper
\(\pair{\fun{top}()}{t}\). Then, the operations we want for traversing
a binary tree on demand are \fun{up/1}\index{up@\fun{up/1}} (go to the
parent), \fun{left/1}\index{left@\fun{left/1}} (go to the left child),
\fun{right/1}\index{right@\fun{right/1}} (go to the right child) and
\fun{sibling/1}\index{sibling@\fun{sibling/1}} (go to the
sibling). All take a zipper as input and all calls evaluate into a
zipper. After any of these steps is performed, a new
zipper\index{binary tree!zipper} is assembled as the value of the
call. See \fig~\vref{fig:zip_move}
\begin{figure}
\begin{equation*}
\boxed{
\begin{array}{r@{\;}c@{\;}l}
\fun{up}(\pair{\fun{left}(x,t_2,p)}{t_1}) & \rightarrow &
  \pair{p}{\fun{int}(x,t_1,t_2)};\\
\fun{up}(\pair{\fun{right}(x,t_1,p)}{t_2}) & \rightarrow &
\pair{p}{\fun{int}(x,t_1,t_2)}.\\
\\
\fun{left}(\pair{p}{\fun{int}(x,t_1,t_2)}) & \rightarrow &
  \pair{\fun{left}(x,t_2,p)}{t_1}.\\
\\
\fun{right}(\pair{p}{\fun{int}(x,t_1,t_2)}) & \rightarrow &
  \pair{\fun{right}(x,t_1,p)}{t_2}.\\
\\
\fun{sibling}(\pair{\fun{left}(x,t_2,p)}{t_1}) & \rightarrow &
  \pair{\fun{right}(x,t_1,p)}{t_2};\\
\fun{sibling}(\pair{\fun{right}(x,t_1,p)}{t_2}) & \rightarrow &
  \pair{\fun{left}(x,t_2,p)}{t_1}.
\end{array}}
\end{equation*}
\caption{Basic steps in a binary tree}
\label{fig:zip_move}
\end{figure}
for the program. Beyond random traversals of a binary tree, this
technique, which is an instance of \emph{Huet's zipper}
\citep{Huet_1997,Huet_2003}, also allows local editing. This simply
translates as the replacement of the current tree by another:
\begin{equation*}
\fun{graft}(t',\pair{p}{t}) \rightarrow \pair{p}{t'}.
\end{equation*}
If we only want to change the slider, we would use
\begin{equation*}
\fun{slider}(x',\pair{p}{\fun{int}(x,t_1,t_2)}) \rightarrow
\pair{p}{\fun{int}(x',t_1,t_2)}.
\end{equation*}
If we want to go up to the root and extract the new tree:
\begin{equation*}
\fun{zip}(\pair{\fun{top}()}{t}) \rightarrow t;\quad
\fun{zip}(z) \rightarrow \fun{zip}(\fun{up}(z)).
\end{equation*}
We do not need a zipper\index{binary tree!zipper} to perform a
preorder, inorder or postorder traversal, because it is primarily
designed to open down and close up paths from the root of a tree, in
the manner of a zipper\index{binary tree!zipper} in a
cloth. Nevertheless, if we retain one aspect of its design, namely,
the accumulation of unvisited nodes and subtrees, we can define the
classic traversals in tail form\index{functional language!tail form},
that is, by means of a definition where the right\hyp{}hand sides
either are a value or a function call whose arguments are not function
calls themselves. Such definitions are equivalent to loops in
imperative languages and may be a target for some compilers
\citep{Appel_1992}.

We show a preorder traversal following this design in
\fig~\vref{fig:pre_tail},
\begin{figure}[t]
\begin{equation*}
\boxed{
\begin{array}{r@{\;}c@{\;}l}
\fun{pre}_7(t) & \rightarrow & \fun{pre}_8(\el,\el,t).\\
\\
\fun{pre}_8(s,\el,\fun{ext}()) & \rightarrow & s;\\
\fun{pre}_8(s,\cons{\fun{int}(x,t_1,\fun{ext}())}{f},\fun{ext}()) &
\rightarrow & \fun{pre}_8(s,\cons{x}{f},t_1);\\
\fun{pre}_8(s,\cons{x}{f},\fun{ext}()) & \rightarrow &
\fun{pre}_8(\cons{x}{s},f,\fun{ext}());\\
\fun{pre}_8(s,f,\fun{int}(x,t_1,t_2)) & \rightarrow &
\fun{pre}_8(s,\cons{\fun{int}(x,t_1,\fun{ext}())}{f},t_2).
\end{array}}
\end{equation*}
\caption{Preorder in tail form}
\label{fig:pre_tail}
\end{figure}
where, in \(\fun{pre}_8(s,f,t)\), the stack~\(s\) is expected to
collect the visited nodes in preorder, the stack~\(f\) (forest) is the
accumulator of unvisited parts of the original tree and \(t\)~is the
current subtree to be traversed. The cost is simple:
\(\C{\fun{pre}_7}{n} = 3n + 2\).


\section{Enumeration}

Many publications \cite[\S~2.3.4.4]{Knuth_1997}
\cite[\S~5.1]{SedgewickFlajolet_1996} show how to find the number of
binary trees of size~\(n\) using an advanced mathematical tool called
\emph{generating functions}\index{enumerative combinatorics!generating
  function} \cite[chap.~7]{GrahamKnuthPatashnik_1994}. Instead, for
didactical purposes, we opt for a more intuitive technique in
enumerative combinatorics\index{enumerative combinatorics} which
consists in constructing a one\hyp{}to\hyp{}one correspondence between
two finite sets, so the cardinal of one is the cardinal of the
other. In other words, we are going to relate bijectively, on the one
hand, binary trees, and, on the other hand, other combinatorial
objects which are relatively easy to count, for a given size.

We actually know the appropriate objects in the instance of \emph{Dyck
  paths}\index{Dyck path}, introduced in section~\ref{sec:queueing}
about queueing\index{queue}. A Dyck path is a broken line in a grid
from the point \((0,0)\) to \((2n,0)\), made up of the two kinds of
segments shown in \fig~\vref{fig:rise_fall},
\begin{figure}
\centering
\subfloat[Rise\label{fig:rise}]{%
  \includegraphics[bb=69 662 132 721,scale=0.75]{enqueue}
}
\qquad
\subfloat[Fall\label{fig:fall}]{%
  \includegraphics[bb=69 662 132 721,scale=0.75]{dequeue}
}
\caption{Basic steps in a grid}
\label{fig:rise_fall}
\end{figure}
such that it remains above the abscissa axis or reaches it. Consider
again the example given in \fig~\vref{fig:dyck_path}, without taking
into account the individual costs associated to each step. Following
the same convention as in chapter~\ref{chap:factoring}, we would say
here that a \emph{Dyck word}\index{Dyck word|see{Dyck path}} is a
finite word over the alphabet made of the letters~\fun{r}
(rise\index{Dyck path!rise}) and~\fun{f} (fall\index{Dyck path!fall}),
such that all its prefixes\index{word factoring!prefix} contain more
letters~\fun{r} than~\fun{f}, or an equal number. This condition is
equivalent to the geometrical characterisation `above the abscissa
axis or reaches it.' For instance, \fun{rff} is not a Dyck word
because the prefix \fun{rff} (actually, the whole word) contains more
falls than rises, so the associated path ends below the axis. The Dyck
word corresponding to the Dyck path in \fig~\vref{fig:dyck_path} is
\fun{rrrfrfrrfffrff}. Conceptually, there is no difference between a
Dyck path and a Dyck word, we use the former when a geometrical
framework is more intuitive and the latter when symbolic reasoning and
programming are expected.

First, let us map injectively binary trees to Dyck words, in other
words, we want to traverse any tree and produce a Dyck word which is
not the mapping of another tree. Since, by definition, non\hyp{}empty
binary trees are made of an internal node connected to two binary
subtrees, we may wonder how to split a Dyck word into three parts: one
corresponding to the root of the tree and two corresponding to the
immediate subtrees. Since any Dyck word starts with a rise and ends
with a fall, we may ask what is the word in\hyp{}between. In general,
it is not a Dyck word; for example, chopping off the ends of
\fun{rfrrff} yields \fun{frrf}. Instead, we seek a decomposition of
Dyck words into Dyck words. If the Dyck word has exactly one
\emph{return}, that is, one fall leading to the abscissa axis, then
cutting out the first rise and that unique return (which must be the
last fall) yields another Dyck word. For instance, \(\fun{rrfrrfff} =
\fun{r} \cdot \fun{rfrrff} \cdot \fun{f}\). Such words are called
\emph{prime}, because any Dyck word can be uniquely decomposed as the
concatenation of such words (whence the reference to prime factorisation
in elementary number theory): for all non\hyp{}empty Dyck words~\(d\),
there exists \({n > 0}\) unique prime Dyck words~\(p_i\) such that \(d
= p_1 \cdot p_2 \cdots p_n\). This naturally yields the \emph{arch
  decomposition}\index{Dyck path!decomposition!arch $\sim$}, whose
name stems from an architectural analogy: for all Dyck words~\(d\),
there exists \({n > 0}\) Dyck words \(d_i\) and returns~\(\fun{f}_i\)
such that
\begin{equation*}
d = (\fun{r} \cdot d_1 \cdot \fun{f}_1) \cdots (\fun{r} \cdot d_n
\cdot \fun{f}_n).
\end{equation*}
See \cite{PanayotopoulosSapounakis_1995, Lothaire_2005,
  FlajoletSedgewick_2009}. Unfortunately, this analysis is not
suitable as it stands, because \(n\)~may be greater than~\(2\),
precluding any analogy with binary trees. The solution is simple
enough: let us keep the first prime factor \(\fun{r} \cdot d_1 \cdot
\fun{f}_1\) and \emph{not} factorise the suffix, which is a Dyck
word. To wit, for all non\hyp{}empty Dyck words~\(d\), there exists
one return~\(\fun{f}_1\) and two Dyck subwords \(d_1\) and~\(d_2\)
(possibly empty) such that
\begin{equation*}
d = (\fun{r} \cdot d_1 \cdot \fun{f}_1) \cdot d_2.
\end{equation*}
This is the \emph{first return decomposition}\index{Dyck
  path!decomposition!first return $\sim$}, also known as
\emph{quadratic decomposition}\index{Dyck path!decomposition!quadratic
  $\sim$} --~also possible is \(d = d_1 \cdot (\fun{r} \cdot d_2 \cdot
\fun{f}_1)\). For example, the Dyck word \(\fun{rrfrffrrrffrff}\),
shown in \fig~\ref{fig:dyck_quad}
\begin{figure}
\centering
\includegraphics[bb=86 616 502 725,scale=0.8]{dyck_quad}
\caption{Quadratic decomposition of a Dyck path}
\label{fig:dyck_quad}
\end{figure}
admits the quadratic decomposition \(\fun{r} \cdot \fun{rfrf} \cdot
\fun{f} \cdot \fun{rrrffrff}\). This decomposition is unique because
the prime factorisation is unique.

Given a tree \(\fun{int}(x,t_1,t_2)\), the rise and fall explicitly
distinguished in the quadratic decomposition are to be conceived as a
pair which is the mapping of~\(x\), while \(d_1\)~is the mapping
of~\(t_1\) and \(d_2\)~is the mapping of~\(t_2\). More precisely, the
value of~\(x\) is not relevant here, only the existence of an internal
node, and a fork\index{binary tree!fork} in a leaf tree\index{binary
  tree!leaf tree} would be mapped just as well. Formally,
if~\(\delta(t)\) is the Dyck word mapped from the binary tree~\(t\),
then we expect the following equations to hold:
\begin{equation}
\delta(\fun{ext}()) = \varepsilon; \quad \delta(\fun{int}(x,t_1,t_2))
= \fun{r}_{x} \cdot \delta(t_1) \cdot \fun{f} \cdot \delta(t_2).
\label{eq:delta}
\end{equation}
Note that we attached the node contents~\(x\) to the rise, so we do
not lose information. For example, the tree in \fig~\vref{fig:dyck2bt}
\begin{figure}
\centering
\subfloat[Binary tree\label{fig:dyck2bt}]{%
\includegraphics[bb=71 655 142 721]{dyck2bt}}
\qquad\quad
\subfloat[Dyck path corresponding to preorder encoding $[0,1,\Box,2,\Box,\Box,3,\Box,\Box\char93$\label{fig:bt2dyck}]{%
\includegraphics[scale=0.9,bb=93 650 330 724]{bt2dyck}}
\caption{Bijection between a binary tree and a Dyck path}
\end{figure}
would formally be
\begin{equation*}
  t := \fun{int}(0, \fun{int}(1, \fun{ext}(),
\fun{int}(2, \fun{ext}(), \fun{ext}())), \fun{int}(3, \fun{ext}(),
\fun{ext}()),
\end{equation*}
and mapped into the Dyck path in
\fig~\ref{fig:bt2dyck}\index{binary tree!preorder!encoding} as
follows:
\begin{align*}
\delta(t)
&= \fun{r}_0 \cdot \delta(\fun{int}(1, \fun{ext}(),
\fun{int}(2, \fun{ext}(), \fun{ext}()))) \cdot \fun{f} \cdot
\delta(\fun{int}(3, \fun{ext}(), \fun{ext}()))\\ &= \fun{r}_0
\cdot (\fun{r}_1 \cdot \delta(\fun{ext}()) \cdot \fun{f} \cdot
\delta(\fun{int}(2, \fun{ext}(), \fun{ext}()))) \cdot
\fun{f} \cdot \delta(\fun{int}(3, \fun{ext}(), \fun{ext}()))\\
&= \fun{r}_0 \fun{r}_1 \varepsilon \cdot \fun{f} \cdot (\fun{r}_2
\cdot \delta(\fun{ext}()) \cdot \fun{f} \cdot \delta(\fun{ext}()))
\cdot \fun{f} \cdot (\fun{r}_3 \cdot \delta(\fun{ext}()) \cdot \fun{f}
\cdot \delta(\fun{ext}()))\\
&= \fun{r}_0 \fun{r}_1 \fun{f} \cdot (\fun{r}_2 \cdot
\varepsilon \cdot \fun{f} \cdot \varepsilon) \cdot
\fun{f} \cdot (\fun{r}_3 \cdot \varepsilon \cdot \fun{f} \cdot
\varepsilon)
= \fun{r}_0\fun{r}_1\fun{f}\fun{r}_2\fun{f}\fun{f}\fun{r}_3\fun{f}.
\end{align*}
Notice that if we replace the rises by their associated contents (in
subscript) and the falls by~\(\Box\), we obtain \([0, 1, \Box, 2,
\Box, \Box, 3, \Box]\), which is the preorder encoding\index{binary
  tree!preorder!encoding} of the tree without its last \(\Box\). We
could then modify \fun{epre/2}\index{epre@\fun{epre/2}} in
\fig~\vref{fig:epre} to map a binary tree to a Dyck\index{Dyck path}
path, but we would have to remove the last item from the resulting
stack, so it is more efficient to directly implement \(\delta\) as
function \fun{dpre/1}\index{dpre@\fun{dpre/1}} (\emph{Dyck path as
  preorder}) in \fig~\vref{fig:dpre}.
\begin{figure}[b]
\begin{equation*}
\boxed{
\begin{array}{r@{\;}c@{\;}l}
\fun{dpre}(t) & \rightarrow & \fun{dpre}(t,\el).\\
\\
\fun{dpre}(\fun{ext}(),s) & \rightarrow & s;\\
\fun{dpre}(\fun{int}(x,t_1,t_2),s)
  & \rightarrow
  & \cons{\fun{r}(x)}{\fun{dpre}(t_1,\cons{\fun{f}()}{\fun{dpre}(t_2,s)})}.
\end{array}}
\end{equation*}
\caption{Mapping in preorder a binary tree to a Dyck path}
\label{fig:dpre}
\end{figure}
If the size of the binary tree is~\(n\), then \(\C{\fun{dpre}}{n} =
2n+2\)\index{dpre@$\C{\fun{dpre}}{n}$} and the length of the Dyck path
is~\(2n\).

This mapping is clearly reversible, as we already solved the problem
of decoding a tree in preorder in
\figs~\ref{fig:pre2b}\index{pre2b@\fun{pre2b/1}}
and~\vref{fig:pre2b0}\index{pre2b0@\fun{pre2b\(_0\)/1}}, and we
understand now that the reversed mapping is based on the quadratic
(`first return') decomposition of the path. \index{Dyck
  path!decomposition!first return $\sim$}

If we are concerned about efficiency, though, we may recall that using
a postorder encoding\index{binary tree!preorder!encoding} yields a
more efficient decoding, as we saw in
\figs~\ref{fig:epost}\index{epost@\fun{epost/1}}
and~\ref{fig:post2b}\index{post2b@\fun{post2b/1}} on
page~\pageref{fig:epost}, therefore a faster reverse mapping. To
create a Dyck\index{Dyck path} path based on a postorder, we map
external nodes to rises and internal nodes to falls (with associated
contents), and then remove the first rise. See
\fig~\vref{fig:dyck_post}
\begin{figure}
\centering
\subfloat[Binary tree]{\includegraphics[bb=71 655 142 721]{dyck2bt}}
\qquad\quad
\subfloat[Dyck path corresponding to postorder encoding $[\Box,\Box,\Box,2,1,\Box,\Box,3,0\char93$\label{fig:dyck_post}]{%
\includegraphics[scale=0.9,bb=93 650 330 724]{dyck_post}}
\caption{Bijection between a binary tree and a Dyck path}
\end{figure}
for the Dyck path obtained from the postorder traversal of the same
previous tree. Of course, just as we did with the preorder mapping, we
are not going to make the postorder encoding\index{binary
  tree!postorder!encoding}, but instead go directly from the binary
tree to the Dyck\index{Dyck path} path, as shown in
\fig~\vref{fig:dpost}.\index{dpost@\fun{dpost/1}}\index{dpost@\fun{dpost/2}}
\begin{figure}[t]
\begin{equation*}
\boxed{
\begin{array}{r@{\;}c@{\;}l}
\fun{dpost}(t) & \rightarrow & \fun{dpost}(t,\el).\\
\\
\fun{dpost}(\fun{ext}(),s) & \rightarrow & s;\\
\fun{dpost}(\fun{int}(x,t_1,t_2),s)
  & \rightarrow
  & \fun{dpost}(t_1,\cons{\fun{r}()}{\fun{dpost}(t_2,\cons{\fun{f}(x)}{s})}).
\end{array}}
\end{equation*}
\caption{Mapping in postorder a binary tree to a Dyck path}
\label{fig:dpost}
\end{figure}
Note that we push \(\fun{r}()\) and \(\fun{f}(x)\) in the same rule,
so we do not have to remove the first rise at the end. (We use a
similar optimisation with \fun{dpre/2}\index{dpre@\fun{dpre/2}} in
\fig~\vref{fig:dpre}.) In structural terms, the inverse of this
postorder mapping corresponds to a decomposition \(d = d_1 \cdot
(\fun{r} \cdot d_2 \cdot \fun{f}_1)\), which we mentioned earlier in
passing as an alternative to the `first return'
decomposition.\index{Dyck path!decomposition!first return $\sim$}

The mapping from Dyck\index{Dyck path} paths encoded in postorder to
binary trees is a simple variation on
\fun{post2b/1}\index{post2b@\fun{post2b/1}} and
\fun{post2b/2}\index{post2b@\fun{post2b/2}} in \fig~\vref{fig:post2b}:
simply make the auxiliary stack be \([\fun{ext}()]\) at the
beginning. The definition is named \fun{d2b/1}\index{d2b@\fun{d2b/1}}
and shown in \fig~\vref{fig:d2b}.
\begin{figure}
\begin{equation*}
\boxed{
\begin{array}{r@{\;}c@{\;}l}
\fun{d2b}(s) & \rightarrow & \fun{d2b}([\fun{ext}()],s).\\
\\
\fun{d2b}([t],\el) & \rightarrow & t;\\
\fun{d2b}(f,\cons{\fun{r}()}{s}) & \rightarrow & \fun{d2b}(\cons{\fun{ext}()}{f},s);\\
\fun{d2b}(\cons{t_2,t_1}{f},\cons{\fun{f}(x)}{s}) & \rightarrow & \fun{d2b}(\cons{\fun{int}(x,t_1,t_2)}{f},s).
\end{array}}
\end{equation*}
\caption{Mapping a Dyck path to a binary tree in postorder}
\label{fig:d2b}
\end{figure}
The cost is
\begin{equation*}
  \C{\fun{d2b}}{n} = 2n + 2.
  \index{d2b@$\C{\fun{d2b}}{n}$}
\end{equation*}

Whether we choose a preorder or a postorder mapping, as a consequence
of the established bijections, we know that there are as many binary
trees of size~\(n\) as Dyck paths of length~\(2n\). We already know,
from chapter~\ref{chap:Catalan}, and equation~\eqref{eq:Cn}
\vpageref{eq:Cn}, that there are \(C_n\) paths:
\begin{equation*}
  C_n = \frac{1}{n+1}\binom{2n}{n} \sim \frac{4^n}{n\sqrt{\pi n}}.
\end{equation*}
Other encodings of binary trees can be found
in \cite[2.3.3]{Knuth_1997} and \cite[5.11]{SedgewickFlajolet_1996}.

\mypar{Average path length}

Most of the usual average parameters of binary trees, like
average internal path length, average height and width, are quite
difficult to derive and require mathematical tools which are beyond
the scope of this book.

The \emph{internal path length}~\(I(t)\)\index{binary tree!internal
  path length} of a binary tree~\(t\) is the sum of the path lengths
from the root to every internal node. We already saw the concept of
external path length~\(E(t)\)\index{binary tree!external path length},
that is, the sum of the path lengths from the root to every internal
node, in section~\ref{sec:opt_sort} on page~\pageref{sec:opt_sort}
about optimal sorting, where we showed that the binary tree with
minimum average external path length has all its external nodes on two
successive levels. The relation between these two path lengths is
quite simple because the binary structure yields an equation depending
only on the size~\(n\):
\begin{equation}
E_n = I_n + 2n.\label{eq:EI}
\end{equation}
Indeed, let \(\fun{int}(x,t_1,t_2)\) be a tree with \(n\)~internal
nodes. Then we have
\begin{equation}
  I(\fun{ext}()) = 0,\quad
  I(\fun{int}(x,t_1,t_2)) = I(t_1) + I(t_2) + n - 1,
\label{eq:I}
\end{equation}
because each path in~\(t_1\) and~\(t_2\) is extended with one more
edge back to the root~\(x\), and there are \(n-1\) such paths by
definition. On the other hand,
\begin{equation}
  E(\fun{ext}()) = 0,\quad
  E(\fun{int}(x,t_1,t_2)) = E(t_1) + E(t_2) + n + 1,
\label{eq:E}
\end{equation}
because the paths in~\(t_1\) and~\(t_2\) are extended by one more step
to the root~\(x\) and there are \(n+1\) such paths, from
theorem~\vref{thm:int_ext}. Subtracting equation~\eqref{eq:I}
from~\eqref{eq:E} yields
\begin{align*}
  E(\fun{ext}()) - I(\fun{ext}()) &= 0,\\
  E(\fun{int}(x,t_1,t_2)) - I(\fun{int}(x,t_1,t_2))
  &= (E(t_1) - I(t_1)) + (E(t_2) - I(t_2)) + 2.\!\!
\end{align*}
In other words, each internal node adds~\(2\) to the difference
between the external and internal path lengths from it. Since the
difference is~\(0\) at the external nodes, we get
equation~\eqref{eq:EI} for the tree of size~\(n\). Unfortunately,
almost anything else is quite hard to prove. For instance, the
\emph{average internal path length}~\(\Expected{I_n}\)\index{binary
  tree!average internal path length} has been shown to be
\begin{equation*}
\Expected{I_n}  = \frac{4^n}{C_n} - 3n - 1 \sim n \sqrt{\pi n}
\end{equation*}
by \cite{Knuth_1997}, in exercise~5 of section~2.3.4.5,
and \cite{SedgewickFlajolet_1996}, in Theorem~5.3 of section~5.6.
Using equation~\eqref{eq:EI}, we deduce \(\Expected{E_n} =
\Expected{I_n} + 2n\), implying that the cost for traversing a random
binary tree of size~\(n\) from the root to a random external node is
\(\Expected{E_n}/(n+1) \sim \sqrt{\pi n}\). Moreover, the value
\(\Expected{I_n}/n\) can be understood as the average level of a
random internal node.

\mypar{Average height}

The average height~\(h_n\)\index{binary tree!average height} of a
binary tree of size~\(n\) is even more difficult to obtain and was
studied by
\cite{FlajoletOdlyzko_1981,BrownShubert_1984,FlajoletOdlyzko_1984,Odlyzko_1984}:
\begin{equation*}
h_n \sim 2 \sqrt{\pi n}.
\end{equation*}
In the case for Catalan trees, to wit, trees whose internal nodes may
have any number of children, the analysis of the average height has
been carried out by \cite{DasarathyYang_1980,DershowitzZaks_1981},
\cite{Kemp_1984} in section~5.1.1,
\cite{DershowitzZaks_1990,KnuthdeBruijnRice_2000b} and
\cite{SedgewickFlajolet_1996}, in
section~5.9.\index{tree!height!average $\sim$}

\mypar{Average width}

The \emph{width}\index{binary tree!width} of a binary tree is the
length of its largest extended level\index{binary tree!extended
  level}. It can be shown that the \emph{average
  width}~\(w_n\)\index{binary tree!width} of a binary tree of
size~\(n\) satisfies
\begin{equation*}
w_n \sim \sqrt{\pi n} \sim \tfrac{1}{2} h_n.
\end{equation*}
In particular, this result implies that the average size of the stack
needed to perform the preorder traversal with
\fun{pre\(_4\)/2}\index{pre4@\fun{pre\(_4\)/2}} in
\fig~\vref{fig:pre3} is twice the average size of the queue needed to
perform a level\hyp{}order traversal with
\fun{bf/1}\index{bf@\fun{bf/1}} in \fig~\vref{fig:bf}. This is not
obvious, as the two stacks used to simulate the queue do not always
hold a complete level.

For general trees, a bijective correspondence with binary trees and
the transfer of some average parameters has been nicely presented by
\cite{DasarathyYang_1980}.

\section*{Exercises}

\begin{enumerate}

  \item Prove \(\fun{post2b}(\fun{epost}(t)) \equiv t\).
    \index{post2b@\fun{post2b/1}}

  \item Prove \(\fun{pre2b}(\fun{epre}(t)) \equiv t\).
    \index{pre2b@\fun{pre2b/1}}

  \item Prove \(\fun{epre}(\fun{mir}(t)) \equiv
    \fun{rev}(\fun{epost}(t))\).\index{epost@\fun{epost/1}}
    \index{rev@\fun{rev/1}}\index{epre@\fun{epre/1}}
    \index{mir@\fun{mir/1}}

  \item Define the encoding of a tree based on its inorder traversal.
    \index{binary tree!inorder!encoding}

\end{enumerate}
