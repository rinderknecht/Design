\section{Concatenating}

\begin{quote}
\small[\textsl{This slow restart may be skipped by impatient readers.}]
\end{quote}
In the introduction, we did not explain the design of \fun{cat/2}, the
function which concatenates its two stack arguments. Let us start from
scratch and proceed slowly.

Consider writing a function \fun{join/2} doing the same thing as
\fun{cat/2}. First, the requirement should be expressed in English as
a function which takes two stacks and computes a stack containing all
the items of the first stack followed by all the items of the second
one, while retaining the relative order of the items. A requirement
would be incomplete without some examples. For instance,
\begin{equation*}
\fun{join}([3,5],[2]) \twoheadrightarrow [3,5,2].
\end{equation*}
We should try some extreme cases, that is, special configurations of
the arguments, in order to understand more precisely what is expected
from this function. Both arguments being stacks, a hint comes
naturally to mind because it is well known that stacks must either be
empty or non\hyp{}empty. This simple observation leads to consider
four distinct cases: both stacks are empty; the first is empty and the
second is not; the first is not empty and the second is; both are not
empty:
\begin{equation*}
\begin{array}{r@{\;}l@{\;}l}
\fun{join}(\el,\el) & \phantom{\rightarrow} & \phantom{\el;}\\
\fun{join}(\el,\cons{y}{t}) & \phantom{\rightarrow} & \phantom{\cons{y}{t};}\\
\fun{join}(\cons{x}{s},\el) & \phantom{\rightarrow} & \phantom{\cons{x}{s};}\\
\fun{join}(\cons{x}{s},\cons{y}{t}) & \phantom{\rightarrow} &
  \phantom{\cons{x}{\fun{join}(s,\cons{y}{t})}.}
\end{array}
\end{equation*}
It is important not to rush to write the right\hyp{}hand sides and
wonder: Are some cases left out? No, because there are exactly two
arguments which can each be either empty or not, leading to exactly
\(2 \cdot 2 = 4\) cases. Then, we write the corresponding canvas:
\begin{equation*}
\begin{array}{r@{\;}l@{\;}l}
\fun{join}(\el,\el) & \rightarrow & \fbcode{CCCCCCCCCC};\\
\fun{join}(\el,\cons{y}{t}) & \rightarrow & \fbcode{CCCCCCCCCC};\\
\fun{join}(\cons{x}{s},\el) & \rightarrow & \fbcode{CCCCCCCCCC};\\
\fun{join}(\cons{x}{s},\cons{y}{t}) & \rightarrow & \fbcode{CCCCCCCCCC}.
\end{array}
\end{equation*}
Let us ponder which rule looks easier, because there is no reason to
complete them in order if we do not want to. When reading the
patterns, a clear mental representation of the situation should
arise. Here, it seems that the first rule is the easiest because it
contains no variable at all: what is the stack made of all the items
of the (first) empty stack followed by all the items of the (second)
empty stack? Since the empty stack, by definition, contains no item,
the answer is the empty stack:
\begin{equation*}
\begin{array}{r@{\;}l@{\;}l}
\fun{join}(\el,\el) & \rightarrow & \el;\\
\fun{join}(\el,\cons{y}{t}) & \rightarrow & \fbcode{CCCCCCCCCC};\\
\fun{join}(\cons{x}{s},\el) & \rightarrow & \fbcode{CCCCCCCCCC};\\
\fun{join}(\cons{x}{s},\cons{y}{t}) & \rightarrow & \fbcode{CCCCCCCCCC}.
\end{array}
\end{equation*}
Perhaps we might wonder whether this case should be dropped. In other
words, is it a meaningful case? Yes, because the English description
of the behaviour of \fun{join/2} mandates the result be always a stack
made of items of other stacks (the arguments), so, in the absence of
any items to ``put'' in the result, the final stack ``remains'' empty.

It should be evident enough that the second and third rules are
symmetric, because appending a non\hyp{}empty stack to the right of an
empty stack is the same as appending it to the left: the result is
always the non\hyp{}empty stack in question.
\begin{equation*}
\begin{array}{r@{\;}l@{\;}l}
\fun{join}(\el,\el) & \rightarrow & \el;\\
\fun{join}(\el,\cons{y}{t}) & \rightarrow & \cons{y}{t};\\
\fun{join}(\cons{x}{s},\el) & \rightarrow & \cons{x}{s};\\
\fun{join}(\cons{x}{s},\cons{y}{t}) & \rightarrow & \fbcode{CCCCCCCCCC}.
\end{array}
\end{equation*}
The last rule is the trickiest. What does the pattern in the
left\hyp{}hand side of the rule reveal about the situation? That both
stacks are not empty, more precisely, the top of the first stack is
denoted by the variable~\(x\), the corresponding rest (which can
either be empty or not) is~\(s\), and similarly for the second stack
and \(y\)~and~\(t\). Are these bricks enough for building the
right\hyp{}hand side, that is, the next step towards the result? We
understand that appending two stacks \(p\)~and~\(q\) preserves in the
result the total order of the items present in both, but also the
relative order of the items of~\(p\) with respect to the items
of~\(q\). In other words, the items of \(p=\cons{x}{s}\) must be
present in the result before the items of \(q=\cons{y}{t}\) and the
items from~\(p\) must be in the same order as in~\(p\) (same
for~\(q\)). With this ordering in mind, it is perhaps natural
sketching the following:
\begin{equation*}
\fun{join}(\cons{x}{s},\cons{y}{t}) \rightarrow \fbcode{II} \; x \;
\fbcode{II} \; s \; \fbcode{II} \; y \; \fbcode{II} \; t \; \fbcode{II}.
\end{equation*}
In one rewrite step, can the item~\(x\) be placed in its final place,
that is, in a position from where it does not need to be moved again?
The answer is yes: it must be the top of the result.
\begin{equation*}
\fun{join}(\cons{x}{s},\cons{y}{t}) \rightarrow \cons{x}{\fbcode{II}
  \; s \; \fbcode{II} \; y \; \fbcode{II} \; t \; \fbcode{II}}.
\end{equation*}
What about the other top, variable~\(y\)? It should remain on
top of~\(t\):
\begin{equation*}
\fun{join}(\cons{x}{s},\cons{y}{t}) \rightarrow \cons{x}{\fbcode{II}
  \; s \; \fbcode{II} \; \cons{y}{t}}.
\end{equation*}
What is the relationship between \(s\)~and~\(\cons{y}{t}\)? We
may be tempted by
\begin{equation*}
\fun{join}(\cons{x}{s},\cons{y}{t}) \rightarrow
  \cons{x}{\cons{s}{\cons{y}{t}}}.
\end{equation*}
which is flawed. The reason is that, while we may allow stacks to be
used as items for other stacks (in other words, stacks can be
arbitrarily embedded in other stacks), \(s\)~should not be used as an
item here, as it is when it is used as a top in
\(\cons{s}{\cons{y}{t}}\). Let us consider the running example
\(\fun{join}([3,5],[2])\): here, the left\hyp{}hand side of the rule
in question bounds \(x\)~to~\(3\), \(s\)~to~\([5]\), \(y\)~to~\(2\)
and \(t\)~to~\(\el\), therefore the corresponding putative
right\hyp{}hand side \(\cons{x}{\cons{s}{\cons{y}{t}}}\) is actually
\erlcode{[3|[\textbf{[5]}|[2|[]]]]}, differing from the result
\erlcode{[3|[\textbf{5}|[2|[]]]]} in that \erlcode{[5]} is
not~\erlcode{5}.

How can \(s\)~and~\(\cons{y}{t}\) be concatenated? Of course, this is
exactly the purpose of the function \fun{join/2} currently being
defined. Therefore, what we need here is a \emph{recursive call}
(underlined below):
\begin{equation*}
\begin{array}{r@{\;}l@{\;}l}
\fun{join}(\el,\el) & \rightarrow & \el;\\
\fun{join}(\el,\cons{y}{t}) & \rightarrow & \cons{y}{t};\\
\fun{join}(\cons{x}{s},\el) & \rightarrow & \cons{x}{s};\\
\fun{join}(\cons{x}{s},\cons{y}{t}) & \rightarrow & \cons{x}{\underline{\fun{join}(s,\cons{y}{t})}}.
\end{array}
\end{equation*}

\paragraph{Correctness and completeness}
\index{soundness|(}
\index{completeness|(}

It is extremely common that, in the course of designing a function, we
focus on some aspect, some rule, and then on some other part of the
code, and we may miss the forest for the trees. When we settle for a
function definition, the next step is to check whether it is correct
and complete with respect to the conception we had of its expected
behaviour.

We say that a definition is \emph{correct} if all function calls that
can be rewritten in one step can be further rewritten in the expected
result, \emph{and} if every failing call was expected to fail. By
failure, we mean that a \emph{stuck expression} is reached, that is,
an expression containing a function call which can not be further
rewritten.

We say that a definition is \emph{complete} if all function calls that
we expect a priori to be computable are indeed computable. In other
words, we must also check that the definition enables rewriting into a
value any input we deem acceptable.

How do we check that the last definition of \fun{join/2} is correct
and complete? If the concept of ``expected result'' is not formally
defined, typically by means of mathematics, we resort to \emph{code
  review} and \emph{testing}. One important aspect of the reviewing
process consists in verifying again the left\hyp{}hand sides of the
definition and see if all possible inputs are accepted or not. In case
some inputs are not matched by the patterns, we must justify that fact
and record the reason in a comment. The left\hyp{}hand sides of
\fun{join/2} match all the combinations of two stacks, whether they
are empty or not, and this is exactly what was expected: no more, no
less. The next step is to inspect the right\hyp{}hand sides and wonder
twofold:
\begin{enumerate}

  \item Are the right\hyp{}hand sides rewritten into the expected type
    of value, for all function calls?

  \item Are the function calls being provided the expected type of
    arguments?

\end{enumerate}
These checks stem from the fact that some functional languages, like
\Erlang, do not include \emph{type inference} at
compile\hyp{}time. Other functional languages, like \OCaml and
\Haskell, would have their compilers automatically establish these
properties. The examination of the right\hyp{}hand sides in the
definition of \fun{join/2} confirms that
\begin{itemize}

\item the right\hyp{}hand sides of the first three rules are stacks
  containing the same kind of items as the arguments;

\item the arguments of the unique recursive call in the last rule
  are stacks made of items from the parameters;

\item assuming that the recursive call has the expected type, we
  deduce that the right\hyp{}hand side of the last rule is a stack
  made of items from the arguments.

\end{itemize}
As a conclusion, the two questions above have been positively
answered. Notice how we had to assume that the recursive call already
had the type we were trying to establish for the current
definition. There is nothing wrong with this reasoning, called
\emph{inductive}, and it is rife in mathematics. We shall revisit it
in different contexts.

The following stage consists in testing the definition. This means to
define a set of inputs which lead to a set of outputs and failures
that are all expected. For example, it is expected that
\(\fun{join}(\el,\el) \twoheadrightarrow \el\), so we could assess
the validity of this statement by running the code, and the function
call indeed passes the test. How should we choose the inputs
meaningfully? There are no general rules, but some guidelines are
useful. One is to consider the empty case or the smaller case,
whatever that means in the context of the function. For example, if
some argument is a stack, then let us try the empty stack. If some
argument is a nonnegative integer, then let us try zero. Another
advice is to have at least \emph{test cases}, that is, some function
calls whose values are known, which exert each rule. In the case of
\fun{join/2}, there are four rules to be covered by the test cases.

\paragraph{Improvement}

Once we are convinced that the function we just defined is correct and
complete,\index{soundness|)} \index{completeness|)}
 it is often worth considering again the code for
improvement. There are several directions in which improvements, often
called \emph{optimisations} although the result may not be optimal,
can be achieved:
\begin{itemize}

  \item Can we rewrite the definition so that in all or some cases it
    is faster?

  \item Is there an equivalent definition which uses less memory in
    all or some cases?

  \item Can we shorten the definition by using fewer rules (perhaps
    some are useless or redundant) or shorter right\hyp{}hand sides?

  \item Can we use less parameters in the definition? (This is related
    to memory usage.)

\end{itemize}
Let us reconsider \fun{join/2}:
\begin{equation*}
\begin{array}{r@{\;}l@{\;}l}
\fun{join}(\el,\el) & \rightarrow & \el;\\
\fun{join}(\el,\cons{y}{t}) & \rightarrow & \cons{y}{t};\\
\fun{join}(\cons{x}{s},\el) & \rightarrow & \cons{x}{s};\\
\fun{join}(\cons{x}{s},\cons{y}{t}) & \rightarrow & \cons{x}{\fun{join}(s,\cons{y}{t})}.
\end{array}
\end{equation*}
and focus our attention on the two first rules, whose common point is
to have the first stack being empty. It is clear now that the
right\hyp{}hand sides are, in both rules, the second stack, whether it
is empty (first rule) or not (second rule). Therefore, there is no
need to discriminate on the structure of the second stack when the
first one is empty and we can equivalently write
\begin{equation*}
\begin{array}{r@{\;}l@{\;}l}
\fun{join}(\el,t) & \rightarrow & t,\;\text{where \(t\) is a stack};\\
\fun{join}(\cons{x}{s},\el) & \rightarrow & \cons{x}{s};\\
\fun{join}(\cons{x}{s},\cons{y}{t}) & \rightarrow &
  \cons{x}{\fun{join}(s,\cons{y}{t})}.
\end{array}
\end{equation*}
Note how the new definition does not formally ascertain that \(t\)~is
a stack --~hence the comment~-- so it is not strictly equivalent to
the original definition: now \(\fun{join}(\el,5) \rightarrow 5\). Let
us compromise by favouring the conciseness of the latter definition or
assume type inference.

Let us consider next the two last rules and look for common
patterns. It turns out that, in the penultimate rule, the first stack
is matched as \(\cons{x}{s}\) but nothing is done with \(x\)~and~\(s\)
except \emph{rebuilding} \(\cons{x}{s}\) in the right\hyp{}hand
side. This suggests that we could simplify the rule as follows:
\begin{equation*}
\begin{array}{r@{\;}l@{\;}l}
\fun{join}(\el,t) & \rightarrow & t;\\
\fun{join}(s,\el) & \rightarrow & s,\;
  \text{where \(s\) is a non-empty stack};\\
\fun{join}(\cons{x}{s},\cons{y}{t}) & \rightarrow &
  \cons{x}{\fun{join}(s,\cons{y}{t})}.
\end{array}
\end{equation*}
It is important to check that changing \(\cons{x}{s}\) into~\(s\) does
not affect the pattern matching, that is to say, exactly the same
inputs which used to match the pattern are still matching it. Indeed,
it is possible in theory that the new~\(s\) matches an empty
stack. Can we prove that \(s\)~is never empty? The left\hyp{}hand side
of the penultimate rule matches only if the previous rule did not
match, in other words, rules are tried in the order of the writing,
that is, top\hyp{}down. Therefore, we know that \(s\)~can not be bound
to the empty stack, because \(\el\)~is used in the previous rule
\emph{and} the second parameter can be any stack. Nevertheless, as
happened before, \(s\)~is not necessarily a stack anymore, for
instance, \(\fun{join}(5,\el) \rightarrow 5\). Again, we will ignore
this side\hyp{}effect and choose the conciseness of the latter
definition.

In the last rule, we observe that the second argument, matched by
\(\cons{y}{t}\), is simply passed over to the recursive call, thus it
is useless to distinguish \(y\)~and~\(t\) and we can try
\begin{equation*}
\begin{array}{r@{\;}l@{\;}l}
\fun{join}(\el,t) & \rightarrow & t;\\
\fun{join}(s,\el) & \rightarrow & s;\\
\fun{join}(\cons{x}{s},\underline{t}) & \rightarrow &
\cons{x}{\fun{join}(s,\underline{t})}.
\end{array}
\end{equation*}
(Can~\(t\) be empty?) Again, we must make sure that \(t\)~can not
match an empty stack: it can not be empty because, otherwise, the
previous pattern would have matched the call. As it is, the
penultimate pattern is included in the last, that is, all input
matched by the penultimate could be matched by the last, which leads
us to consider whether the definition would still be correct if
\(t\)~could be empty after all. Let us label the rules as follows:
\begin{equation*}
\begin{array}{r@{\;}l@{\;}l}
\fun{join}(\el,t) & \smashedrightarrow{\alpha} & t;\\
\fun{join}(s,\el) & \smashedrightarrow{\beta} & s;\\
\fun{join}(\cons{x}{s},t) & \smashedrightarrow{\gamma} &
\cons{x}{\fun{join}(s,t)}.
\end{array}
\end{equation*}
Let \(s\)~be some stack containing \(n\)~items, which we write
informally as \(s = [x_0, x_1, \dots, x_{n-1}]\). (Informally because
of the ellipsis.) The subscript~\(i\) in~\(x_i\) is the
\emph{position} of the item in the stack, the top being at
position~\(0\). Then we would rewrite in one step
\begin{equation*}
\fun{join}(s,\el) \smashedrightarrow{\beta} s.
\end{equation*}
Had rule~\clause{\beta} been erased, we would have had instead the
series
\begin{equation*}
\begin{array}{r@{\;}c@{\;}l}
\fun{join}(s,\el)
& \smashedrightarrow{\gamma} &
  [x\sb{0}|\fun{join}([x\sb{1},\ldots,x\sb{n-1}],\el)]\\
& \smashedrightarrow{\gamma} &
  [x\sb{0}|[x\sb{1}|\fun{join}([x\sb{2},\ldots,x\sb{n-1}],\el)]]\\
& = & [x\sb{0},x\sb{1}|\fun{join}([x\sb{2},\ldots,x\sb{n-1}],\el)]\\
& \smashedrightarrow{\gamma} &
  [x\sb{0},x\sb{1}|[x\sb{2}|\fun{join}([x\sb{3},\ldots,x\sb{n-1}],\el)]]\\
& = & [x\sb{0},x\sb{1},x\sb{2}|\fun{join}([x\sb{3},\ldots,x\sb{n-1}],\el)]\\
& \smash{\vdots} &\\
& \smashedrightarrow{\gamma} &
  [x\sb{0},x\sb{1},\ldots,x\sb{n-1}|\fun{join}(\el,\el)]\\
& \smashedrightarrow{\alpha} & [x\sb{0},x\sb{1},\ldots,x\sb{n-1}|\el]\\
& = & [x\sb{0},x\sb{1},\ldots,x\sb{n-1}]\\
& = & s.
\end{array}
\end{equation*}
In short, we found \(\fun{join}(s,\el) \twoheadrightarrow s\).
This means that rule~\clause{\beta} is useless, since its removal
allows us to reach the same result~\(s\), although more slowly:
\(n\)~steps by rule~\clause{\gamma} plus~\(1\) by
rule~\clause{\alpha}, instead of one step by
rule~\clause{\beta}. We are hence in a situation where we discover
that the original definition was already specialised for speed when
the second stack is empty. If we remove rule~\clause{\beta}, the
program is shorter but becomes slower in that special case. This kind
of dilemma is quite common in programming and there is sometimes no
clear\hyp{}cut answer as to what is the best design. Perhaps another
argument can here tip the scale slightly in favour of the
removal. Indeed, whilst the removal slows down some calls, it makes
the number of rewrites easy to remember: it is the number of items of
the first stack plus~\(1\); in particular, the length of the second
argument is irrelevant. So let us settle for a new definition with a
new name:
\begin{equation*}
\begin{array}{r@{\;}l@{\;}l}
\fun{cat}(\el,t) & \xrightarrow{\alpha} & t;\\
\fun{cat}(\cons{x}{s},t) & \xrightarrow{\beta} &
\cons{x}{\fun{cat}(s,t)}.
\end{array}
\end{equation*}
Let us note that \(\fun{cat}(5,\el)\) fails again, as it does in the
original version.

When programming medium or large applications, it is recommended to
use evocative variables, like~\erlcode{StackOfProc}, instead of
enigmatic ones, like~\(s\). But in this presentation we are mostly
concerned with short programs, not software engineering, so short
variables will do. Nevertheless, we need to opt for a naming
convention so we can easily recognise the type of the variables across
function definitions.

\mypar{Tail form}

As the rewrite of \(\fun{cat}(s,\el)\) shows, the right\hyp{}hand side
of rule~\clause{\beta} of \fun{cat/2} features a call with the
\emph{context} \(\cons{x}{\text{\textvisiblespace}}\), the
mark~\textvisiblespace{} standing for the location of the call
\(\fun{cat}(s,t)\). When all the right\hyp{}hand sides of a definition
are either values, or arithmetic expressions, or expressions made only
of data constructors, or one function call whose arguments are values
or arithmetic expressions or data constructors, it is said to be in
\emph{tail form}.

We may wonder whether a tail form variant is necessary or not and we
will discuss this issue later. At the moment, let us take this as a
stylistic exercise and, instead of presenting a systematic
transformation, let us envisage a pragmatic approach. In the case of
\fun{cat/2}, as stated above, the only context is
\(\cons{x}{\text{\textvisiblespace}}\) and the operator~(\erlcode{|})
is not associative, that is, \(\cons{x}{\cons{y}{z}} \not\equiv
\cons{\cons{x}{y}}{z}\). The idea is to add another parameter in which
we will store the values from the context, and when the input is
exhausted, we will rebuild the context from that parameter, called an
\emph{accumulator}. Here, we want a new function \fun{cat/3} whose
definition has the shape
\begin{equation*}
\begin{array}{r@{\;}c@{\;}l}
\fun{cat}(\el,t,\underline{u}) & \rightarrow & \fbcode{CCCCCCC};\\
\fun{cat}(\cons{x}{s},t,\underline{u}) & \rightarrow & \fbcode{CCCCCCC}.
\end{array}
\end{equation*}
The new parameter~\(u\) is the accumulator in question. Since we want
to store many~\(x\) in it, it must be a stack. Furthermore, its initial
value should be the empty stack, otherwise extraneous items would be
found in the result. Therefore, the definition in tail form,
equivalent to \fun{cat/2} and named \fun{cat\(_0\)/2}, calls
\fun{cat/3} with the extra argument~\(\el\):
\begin{equation*}
\fun{cat\(_0\)}(s,t) \rightarrow \fun{cat}(s,t,\el).
\end{equation*}
Let us go back to \fun{cat/3} and push~\(x\) on~\(u\):
\begin{equation*}
\begin{array}{r@{\;}c@{\;}l}
\fun{cat}(\el,t,u) & \smashedrightarrow{\alpha} & \fbcode{CCCCCCC};\\
\fun{cat}(\cons{x}{s},t,u) & \smashedrightarrow{\beta} &
  \fun{cat}(s,t,\cons{x}{u}).
\end{array}
\end{equation*}
What is the accumulator with respect to the expected result? We
already know that it can not be a partial result, because
(\erlcode{|})~is not associative. So some more work has to be done
with \(u\)~\emph{and}~\(t\), but, first, we should understand what
\(u\)~contains at this point and unfolding a call, with a piece of
paper and a pencil, is quite enlightening. Let \(s\)~be a stack of
\(n\)~items \([x_0,x_1,\dots,x_{n-1}]\). We have the following:
\begin{equation*}
\begin{array}{r@{\;}c@{\;}l}
\fun{cat}(s,t,\el) & \smashedrightarrow{\beta} &
  \fun{cat}([x\sb{1},\ldots,x\sb{n-1}],t,[x\sb{0}])\\
              & \smashedrightarrow{\beta} &
  \fun{cat}([x\sb{2},\ldots,x\sb{n-1}],t,[x\sb{1},x\sb{0}])\\
              & \smash{\vdots}\\
              & \xrightarrow{\beta} & \fun{cat}(\el,t,[x\sb{n-1},x\sb{n-2},\ldots,x\sb{0}])\\
              & \smashedrightarrow{\alpha} & \fbcode{CCCCCCC}.
\end{array}
\end{equation*}
Therefore, \(u\)~in the left\hyp{}hand side of rule~\clause{\alpha} is
bound to a stack which contains the same items as the original first
argument~\(s\), but in \emph{reverse order}. In other words, given the
call \(\fun{cat}(s,t,\el)\), the parameter~\(u\) in the first pattern
of \fun{cat/3} holds~\(s\) reversed. What can we do with
\(u\)~and~\(t\) in order to reach the result?  The key is to realise
that the answer depends on the contents of~\(u\), which, therefore,
needs to be matched more accurately: is~\(u\) empty or not? This leads
to split rule~\clause{\alpha} into \(\alpha_0\) and \(\alpha_1\):
\newlength\Split\settowidth\Split{\(_{\alpha_0}\)}
\begin{equation*}
\begin{array}{r@{\;}c@{\;}l}
\fun{cat}(\el,t,\el) & \MyArrow{\Split}{\alpha\sb{0}} &
  \fbcode{CCCCCCC};\\
\fun{cat}(\el,t,\cons{x}{u}) & \MyArrow{\Split}{\alpha\sb{1}} &
  \fbcode{CCCCCCC};\\
\fun{cat}(\cons{x}{s},t,u) & \MyArrow{\Split}{\beta} &
 \fun{cat}(s,t,\cons{x}{u}).
\end{array}
\end{equation*}
Notice that rules \clause{\alpha_0}~and~\clause{\alpha_1} could be
swapped, as they filter completely distinct cases. The right\hyp{}hand
side of rule~\clause{\alpha_0} is easy to guess: it must be~\(t\),
since it corresponds to the case when we want to append the empty
stack to~\(t\):
\begin{equation*}
\begin{array}{r@{\;}c@{\;}l}
\fun{cat}(\el,t,\el) & \MyArrow{\Split}{\alpha\sb{0}} & t;\\
\fun{cat}(\el,t,\cons{x}{u}) & \MyArrow{\Split}{\alpha\sb{1}} & \fbcode{CCCCCCC};\\
\fun{cat}(\cons{x}{s},t,u) & \MyArrow{\Split}{\beta} &
  \fun{cat}(s,t,\cons{x}{u}).
\end{array}
\end{equation*}
How do we relate \(t\), \(x\)~and~\(u\) in rule \clause{\alpha_1} with
the result we are looking for? Given the rewrite \(\fun{cat}(s,t,\el)
\twoheadrightarrow \fun{cat}(\el,t,\cons{x}{u})\), we know that
\(\cons{x}{u}\)~is~\(s\) reversed, so item \(x\)~is last in~\(s\) and
it should be on top of~\(t\) in the result. What should we do
with~\(u\)? The key is to realise that we need to start the same
process again, that is, we need another recursive call:
\begin{equation*}
\begin{array}{r@{\;}c@{\;}l}
\fun{cat}(\el,t,\el) & \MyArrow{\Split}{\alpha\sb{0}} & t;\\
\fun{cat}(\el,t,\cons{x}{u}) & \MyArrow{\Split}{\alpha\sb{1}} &
  \fun{cat}(\el,\cons{x}{t},u);\\
\fun{cat}(\cons{x}{s},t,u) & \MyArrow{\Split}{\beta} &
  \fun{cat}(s,t,\cons{x}{u}).
\end{array}
\end{equation*}
To test the correctness of this definition, we can try a small
example:
\begin{equation*}
\begin{array}{r@{\;}c@{\;}l}
\fun{cat}([1,2,3],[4,5],\el)
  & \MyArrow{\Split}{\beta} & \fun{cat}([2,3],[4,5],[1])\\
  & \MyArrow{\Split}{\beta} & \fun{cat}([3],[4,5],[2,1])\\
  & \MyArrow{\Split}{\beta} & \fun{cat}(\el,[4,5],[3,2,1])\\
  & \MyArrow{\Split}{\alpha\sb{1}} &
    \fun{cat}(\el,[3,4,5],[2,1])\\
  & \MyArrow{\Split}{\alpha\sb{1}} &
    \fun{cat}(\el,[2,3,4,5],[1])\\
  & \MyArrow{\Split}{\alpha\sb{1}} &
    \fun{cat}(\el,[1,2,3,4,5],\el)\\
  & \MyArrow{\Split}{\alpha\sb{0}} & [1,2,3,4,5].
\end{array}
\end{equation*}
As a conclusion, the tail form version of \fun{cat/2}, called
\fun{cat\(_0\)/2}, requires an auxiliary function \fun{cat/3} with an
accumulator whose purpose is to reverse the first argument:
\begin{equation*}
\begin{array}{r@{\;}c@{\;}l}
\fun{cat\(_0\)}(s,t) & \smashedrightarrow{\alpha} & \fun{cat}(s,t,\el).\\
\fun{cat}(\el,t,\el) & \smashedrightarrow{\beta} & t;\\
\fun{cat}(\el,t,\cons{x}{u}) & \smashedrightarrow{\gamma} &
  \fun{cat}(\el,\cons{x}{t},u);\\
\fun{cat}(\cons{x}{s},t,u) & \smashedrightarrow{\delta} &
  \fun{cat}(s,t,\cons{x}{u}).
\end{array}
\end{equation*}
We also know what to do when the context is not made of a call to some
associative operator: push the values of the variables it contains and
when the input stack is empty, pop these and use them to fill the
context, which is then evaluated. We will revisit this method.

\paragraph{Efficiency}

The number of steps to rewrite \(\fun{cat\(_0\)}(s,t)\) into a value is
greater than with \(\fun{cat}(s,t)\), as we guessed while writing the
previous example. Indeed, assuming that \(s\)~contains \(n\)~items, we
have
\begin{itemize*}

  \item one step to obtain \(\fun{cat}(s,t,\el)\), by
  rule~\clause{\alpha};

  \item \(n\)~steps to reverse \(s\) in the accumulator, by
  rule~\clause{\delta};

  \item \(n\)~steps to reverse the accumulator on top of~\(t\), by
    rule~\clause{\gamma};

  \item one step when the accumulator is finally empty, by
    rule~\clause{\beta}.

\end{itemize*}
Thus, the total number of steps is \(2n+2\), which is twice the cost
of the previous version. Why the difference between \fun{cat/2} and
\fun{cat\(_0\)/2}? The operation applied to the accumulator consists
in pushing an item onto a stack and has to be undone later: the
accumulator is not a partial result but a \emph{temporary stack} used
to hold the items of the first stack in reverse order. We shall find
many occurrences of this situation. Meanwhile, it is important to
remember that a tail form variant of a function operating on stacks
may lead to a slower program. Also, the derived definition in tail
form may be longer, as illustrated by \fun{cat\(_0\)/2}: four rules
instead of two.

The rewrites \(\fun{cat}_0([1,2,3],[4,5]) \twoheadrightarrow
[1,2,3,4,5]\) seen above can be abstractly conceived as a product
(composition) of rules: \(\alpha \cdot \delta^n \cdot \gamma^n \cdot
\beta\), or, simply, \(\alpha\delta^n\gamma^n\beta\). This expression
is called the \emph{execution trace} and its length is the number of
rules \(\C{\fun{cat\(_0\)}}{n}\) of \(\fun{cat\(_0\)}(s,t)\), given
that the length of a rule is~\(1\), hence \(\len{\alpha} = \len{\beta}
= \len{\gamma} = \len{\delta} = 1\) and the length of the composition
of two rules is the sum of their lengths: \(\len{\alpha \cdot \delta}
= \len{\alpha} + \len{\delta}\). Therefore,
\begin{align*}
\C{\fun{cat\(_0\)}}{n}
  &= \len{\alpha\delta^n\gamma^n\beta}
   = \len{\alpha} + \len{\delta^n} + \len{\gamma^n} + \len{\beta}
   = 1 + \len{\delta} \cdot n + \len{\gamma} \cdot n + 1\\
  &= 2n + 2.
\end{align*}

\paragraph{Digression}

Let us reconsider the definition~\eqref{def:fact} of
\fun{fact/1}\index{fact@\textsf{fact/1}} \vpageref{def:fact}:
\begin{equation*}
\begin{array}{r@{\;}l@{\;}l}
\fun{fact}(0) & \smashedrightarrow{\alpha} & 1;\\
\fun{fact}(n) & \smashedrightarrow{\beta} & n \cdot \fun{fact}(n-1).
\end{array}
\end{equation*}
For instance, we have
\begin{equation*}
\begin{array}{r@{\;}c@{\;}lc@{\;}l}
\fun{fact}(3) & \xrightarrow{\beta} & 3 \cdot \fun{fact}(3-1)
              & = & 3 \cdot \fun{fact}(2)\\
              & \xrightarrow{\smash[t]{\beta}} &
              3 \cdot (2 \cdot \fun{fact}(2-1))
              & = & 3 \cdot (2 \cdot \fun{fact}(1))\\
              & \xrightarrow{\smash[t]{\alpha}} &
                3 \cdot (2 \cdot (1)) & = & 6 = 3!
\end{array}
\end{equation*}
It is often clearer to implicitly compose intermediary arithmetic
operations~(\(=\)) with the current rewrite and write in short
\begin{equation*}
\fun{fact}(3) \xrightarrow{\beta} 3 \cdot \fun{fact}(2)
\xrightarrow{\smash[t]{\beta}} 3 \cdot (2 \cdot \fun{fact}(1))
\xrightarrow{\smash[t]{\alpha}} 3 \cdot (2 \cdot (1)) = 6.
\end{equation*}
Note how the last rewrite (\(\xrightarrow{\smash[t]{\alpha}}\)) must
be followed by a series of multiplications~\(3 \cdot (2 \cdot 1)\)
because each individual multiplication had to be delayed until
\(\fun{fact}(1)\) be computed. This could have been anticipated
because the call to \fun{fact/1} in the right\hyp{}hand side of the
rewrite rule (\(\xrightarrow{\smash[t]{\beta}}\)), that is, the
underlined text in
\begin{equation*}
\fun{fact}(n) \xrightarrow{\smash[t]{\beta}} n \cdot \underline{\fun{fact}(n-1)}
\end{equation*}
has the non\hyp{}empty context
``\erlcode{\(n\)~*~\textvisiblespace}''. To understand why this is
important, let us consider a slightly longer series of
rewrites:\label{trace:fact_5}
\begin{equation*}
\begin{array}{r@{\;}c@{\;}l}
\fun{fact}(5)
& \smashedrightarrow{\beta} & 5 \cdot \fun{fact}(4)\\
& \smashedrightarrow{\beta} & 5 \cdot (4 \cdot \fun{fact}(3))\\
& \smashedrightarrow{\beta} & 5 \cdot (4 \cdot (3 \cdot \fun{fact}(2)))\\
& \smashedrightarrow{\beta} & 5 \cdot (4 \cdot (3 \cdot (2 \cdot
                              \fun{fact}(1))))\\
& \smashedrightarrow{\alpha} & 5 \cdot (4 \cdot (3 \cdot (2 \cdot
(1)))).
\end{array}
\end{equation*}
It is clear that each rewrite by (\(\xrightarrow{\smash[t]{\beta}}\))
yields a longer expression. Let us focus now only on the shapes of
these expressions:
\begin{equation*}
\setlength\fboxsep{0mm}
\begin{array}{r@{\;}c@{\;}l}
\fun{fact}(5)
& \smashedrightarrow{\beta} & \fbox{\phantom{$5 \cdot \fun{fact}(4)$}}\\
& \smashedrightarrow{\beta} & \fbox{\phantom{$5 \cdot (4 \cdot \fun{fact}(3))$}}\\
& \smashedrightarrow{\beta} & \fbox{\phantom{$5 \cdot (4 \cdot (3 \cdot \fun{fact}(2)))$}}\\
& \smashedrightarrow{\beta} & \fbox{\phantom{$5 \cdot (4 \cdot (3 \cdot (2 \cdot \fun{fact}(1))))$}}\\
& \smashedrightarrow{\alpha} & \fbox{\phantom{$5 \cdot (4 \cdot (3
    \cdot (2 \cdot (1))))$}}.
\end{array}
\end{equation*}
This phenomenon suggests that a great deal of space, that is, computer
\emph{memory}, is needed to keep the expressions before the final,
long arithmetic computations. The example leads to induce that the
larger term occurring in the computing of \(\fun{fact}(n)\) is the one
just before (\(\xrightarrow{\smash[t]{\alpha}}\)) and its size is
likely to be proportional to~\(n\), since all the integers from
\(n\)~to~\(1\) had to be kept until the end.

A tail form version
\fun{fact\(_0\)/1}\index{fact0@\textsf{fact\(_0\)/1}} would be
\begin{equation*}
\begin{array}{r@{\;}c@{\;}l}
\fun{fact}_0(n) & \rightarrow & \fun{fact}_0(n,1),
\;\text{if \(n \geqslant 1\)}.\\
\fun{fact}_0(1,a) & \rightarrow & a;\\
\fun{fact}_0(n,a) & \rightarrow & \fun{fact}_0(n-1,a \cdot n).
\end{array}
\end{equation*}
Here, in contrast with \fun{cat/3}, the operation applied to the
accumulator is associative (a multiplication) and the accumulator is,
at all times, a partial result. Instead of delaying the
multiplications, exactly one multiplication is going to be computed at
each rewrite, thus, in the end, nothing remains to be done: there is
no instance of the context to resume. This kind of definition is thus
in tail form.\index{functional language!tail form}

Notice that the cost of \fun{fact\(_0\)/1} is~\(n+1\), whilst the cost
of \fun{fact/1} is~\(n\), so, contrary to \fun{cat\(_0\)/2} and
\fun{cat/2}, the tail form here does not significantly increases the
cost. This is due to the nature of the operations on the accumulator,
which do not require to be reversed or undone.

The previously considered function call \(\fun{fact}_0(5)\) is
evaluated thusly:
\begin{equation*}
\begin{array}{r@{\;}c@{\;}lc@{\;}l}
\fun{fact}_0(5)
& \xrightarrow{\smash[t]{\alpha}} & \fun{fact}_0(5,1),
&& \text{since \(5 > 1\)},\\
& \xrightarrow{\smash[t]{\gamma}} & \fun{fact}_0(5-1,1 \cdot 5)
& = & \fun{fact}_0(4,5)\\
& \xrightarrow{\smash[t]{\gamma}} & \fun{fact}_0(4-1,5 \cdot 4)
& = & \fun{fact}_0(3,20)\\
& \xrightarrow{\smash[t]{\gamma}} & \fun{fact}_0(3-1,20 \cdot 3)
& = & \fun{fact}_0(2,60)\\
& \xrightarrow{\smash[t]{\gamma}} & \fun{fact}_0(2-1,60 \cdot 2)
& = & \fun{fact}_0(1,120)\\
& \xrightarrow{\smash[t]{\beta}} & 120.
\end{array}
\end{equation*}
The reason why \(\fun{fact}_0(5) \equiv \fun{fact}(5)\) is that
\begin{equation}
  (((1 \cdot 5) \cdot 4) \cdot 3) \cdot 2
= 5 \cdot (4 \cdot (3 \cdot (2 \cdot 1))).\label{eq:fact5}
\end{equation}
This equality holds because, in general, for all numbers~\(x\),
\(y\)~and~\(z\),
\begin{enumerate*}

\item \label{mult_assoc} the multiplication is associative: \(x \cdot
  (y \cdot z) = (x \cdot y) \cdot z\);

\item \label{mult_one} the number \(1\) is neutral with respect to
  (\(\cdot\)): \(x \cdot 1 = 1 \cdot x = x\).

\end{enumerate*}
To show exactly why, let us write~(\(\eqn{\ref{mult_assoc}}\))
and~(\(\eqn{\ref{mult_one}}\)) to denote, respectively, the use of
associativity and neutrality, then lay out the following equalities
leading from the left\hyp{}hand side to the right\hyp{}hand side of
the purported equality~\eqref{eq:fact5}:
\begin{align*}
  (((1 \cdot 5) \cdot 4) \cdot 3) \cdot 2
  &\eqn{\smash{\ref{mult_one}}} ((((1 \cdot 5) \cdot 4) \cdot 3) \cdot
  2) \cdot 1\\
  &\eqn{\smash{\ref{mult_assoc}}} (((1 \cdot 5) \cdot 4) \cdot 3)
  \cdot (2 \cdot 1)\\
  &\eqn{\smash{\ref{mult_assoc}}} ((1 \cdot 5) \cdot 4) \cdot (3 \cdot (2
    \cdot 1))\\
  &\eqn{\smash{\ref{mult_assoc}}} (1 \cdot 5) \cdot (4 \cdot (3 \cdot
  (2 \cdot 1))\\
  &\eqn{\smash{\ref{mult_assoc}}} 1 \cdot (5 \cdot (4 \cdot (3 \cdot
  (2 \cdot 1))))\\
  &\eqn{\smash{\ref{mult_one}}} 5 \cdot (4 \cdot (3 \cdot (2 \cdot
  1))).\quad \Box
\end{align*}
Furthermore, if we do not want to rely upon the neutrality of~\(1\),
we could define another equivalent function \fun{fact\(_1\)/1} which
sets the initial call to \(\fun{fact}_1(n-1,n)\), instead of
\(\fun{fact}_0(n,1)\), and stops when the number is~\(0\), instead
of~\(1\):
\begin{equation*}
\begin{array}{r@{\;}c@{\;}l}
\fun{fact}_1(n) & \smashedrightarrow{\alpha} &
\fun{fact}_1(n-1,n),\;\text{if \(n > 0\)}.\\
\fun{fact}_1(0,a) & \smashedrightarrow{\beta} & a;\\
\fun{fact}_1(n,a) & \smashedrightarrow{\gamma} &
\fun{fact}_1(n-1,a \cdot n).
\end{array}
\end{equation*}
The same example now runs as
\begin{equation*}
\begin{array}{r@{\;}c@{\;}lc@{\;}l}
\fun{fact}_1(5)
& \xrightarrow{\smash[t]{\alpha}} & \fun{fact}_1(5-1,5)
& = & \fun{fact}_1(4,5)\\
& \xrightarrow{\smash[t]{\gamma}} & \fun{fact}_1(4-1,5 \cdot 4)
& = & \fun{fact}_1(3,20)\\
& \xrightarrow{\smash[t]{\gamma}} & \fun{fact}_1(3-1,20 \cdot 3)
& = & \fun{fact}_1(2,60)\\
& \xrightarrow{\smash[t]{\gamma}} & \fun{fact}_1(2-1,60 \cdot 2)
& = & \fun{fact}_1(1,120)\\
& \xrightarrow{\smash[t]{\gamma}} & \fun{fact}_1(1-1,120 \cdot 1)
& = & \fun{fact}_1(0,120)\\
& \xrightarrow{\smash[t]{\beta}} & 120.
\end{array}
\end{equation*}
This new version relies on the following equality which can be proved
only by means of associativity: \((((5 \cdot 4) \cdot 3) \cdot 2)
\cdot 1 = 5 \cdot (4 \cdot (3 \cdot (2 \cdot 1)))\).

The number of rewrites of \fun{fact\(_0\)/1} is almost the same as
with \fun{fact/1}, precisely one more step due to the
rule~\clause{\alpha}. But the former presents an advantage in terms of
memory usage, as long as it is assumed that all integers within a
certain range occupy the same space. This means that, for instance,
that the memory needed to store the number~\(120\) is the same as for
the number~\(5\). Then the shape of the previous rewrites:
\begin{equation*}
\setlength\fboxsep{0mm}
\begin{array}{r@{\;}c@{\;}l}
\fun{fact}_0(5)
& \xrightarrow{\smash[t]{\alpha}} &
\fbox{\phantom{$\fun{fact}_0(0,10)$}}\\
& \xrightarrow{\smash[t]{\gamma}} &
\fbox{\phantom{$\fun{fact}_0(0,10)$}}\\
& \xrightarrow{\smash[t]{\gamma}} &
\fbox{\phantom{$\fun{fact}_0(0,10)$}}\\
& \xrightarrow{\smash[t]{\gamma}} &
\fbox{\phantom{$\fun{fact}_0(0,10)$}}\\
& \xrightarrow{\smash[t]{\gamma}} &
\fbox{\phantom{$\fun{fact}_0(0,10)$}}\\
& \xrightarrow{\smash[t]{\beta}} & \fbox{\phantom{CCC}}.
\end{array}
\end{equation*}
It seems probable that this version uses a constant chunk of memory,
while \fun{fact/1} uses an increasing amount of memory, more precisely
a space proportional to~\(n\) when computing~\(n!\). (In the following
sections, we shall see that a more precise model of memory allocation
is provided by abstract syntax trees.) This phenomenon has been
anticipated by the keen reader who noticed that there is no context
for the calls in the rules defining \fun{fact\(_0\)/2}, so there are
no delayed computations that accumulate until the last step. As a
conclusion, \fun{fact\(_0\)/1} is always preferable to \fun{fact/1}.

The previous discussions on obtaining equivalent definitions which are
in tail form suppose to consider programs as some kind of data. At
this point, it is a methodological standpoint only and we do not mean
that functions can be processed as stacks (we shall come back on this
later when discussing higher\hyp{}order functions and
continuation\hyp{}passing style), but, more informally, we mean that
definitions can be transformed into other definitions and that this is
often an excellent method, as opposed to trying to figure out from
scratch the final definition. It would have been probably more
difficult to write the tail form variant of \fun{cat/2} without having
first designed the version not in tail form.

It is in general not a good idea to start head\hyp{}on by defining a
function in tail form because it may either be unnecessary or lead to
a mistake since these kinds of definitions are usually more
involved. In the following sections and chapters, we will explain when
tail form definitions are useful and how to obtain them using a
systematic method.

Let us consider a simple case by defining a function \fun{last/1} such
that \(\fun{last}(s)\)~computes the last item of the non\hyp{}empty
stack~\(s\). The correct approach is to forget about tail forms and
aim straight at the heart of the problem. We know that \(s\)~can not
be empty, so let us start with the following left\hyp{}hand side:
\begin{equation*}
\fun{last}(\cons{x}{s}) \rightarrow \fbcode{CCCCCCC}.
\end{equation*}
Can we reach the result in one step? No, because we do not know
whether \(x\)~is the sought item: we need to know more
about~\(s\). This additional information about the structure of~\(s\)
is given by more precise patterns: \(s\)~can be empty or not, that is:
\(s = \el\) or \(s=\cons{y}{t}\). We then have:
\begin{equation*}
\begin{array}{r@{\;}c@{\;}l}
\fun{last}(\cons{x}{\el}) & \rightarrow & \fbcode{CCCCCCC};\\
\fun{last}(\cons{x}{\cons{y}{t}}) & \rightarrow & \fbcode{CCCCCCC}.
\end{array}
\end{equation*}
The first pattern can be simplified as follows:
\begin{equation*}
\begin{array}{r@{\;}c@{\;}l}
\fun{last}([x]) & \rightarrow & \fbcode{CCCCCCC};\\
\fun{last}(\cons{x}{\cons{y}{t}}) & \rightarrow & \fbcode{CCCCCCC}.
\end{array}
\end{equation*}
The first right\hyp{}hand side is easy to guess:
\begin{equation*}
\begin{array}{r@{\;}c@{\;}l}
\fun{last}([x]) & \rightarrow & x;\\
\fun{last}(\cons{x}{\cons{y}{t}}) & \rightarrow & \fbcode{CCCCCCC}.
\end{array}
\end{equation*}
In the last rule, how do \(x\), \(y\)~and~\(t\) relate to the result?
Can we reach it in one step? No, despite we know that \(x\)~is
\emph{not} the result, we still don't know whether \(y\)~is, so we
have to start over again, which means a recursive call is required:
\begin{equation*}
\begin{array}{r@{\;}c@{\;}l}
\fun{last}([x]) & \rightarrow & x;\\
\fun{last}(\cons{x}{\cons{y}{t}}) & \rightarrow &
\fun{last}(\fbcode{CCCC}).
\end{array}
\end{equation*}
Note how knowing that some specific part of the input is not useful to
build the output is useful knowledge. We can not call recursively
\(\fun{last}(t)\) because \(t\)~may be empty and the call would then
fail, meaning that the answer was actually~\(y\). Therefore, we must
call with \(\cons{y}{t}\) to give~\(y\) the chance to be the last:
\begin{equation*}
\begin{array}{r@{\;}c@{\;}l}
\fun{last}([x]) & \rightarrow & x;\\
\fun{last}(\cons{x}{\cons{y}{t}}) & \rightarrow &
\fun{last}(\cons{y}{t}).
\end{array}
\end{equation*}
As we advocated previously, the next phase consists in testing this
definition for correctness and completeness, using meaningful examples
(covering extreme cases and all the rules at least once). For the sake
of the argument, let us assume that \fun{last/1} is correct and
complete. The next step is then to try and improve upon it. Let us
look for patterns occurring in both sides of the same rule and ponder
whether they can be avoided. For instance, we observe that
\(\cons{y}{t}\)~is used as a whole, in other words, \(y\)~and~\(t\)
are not used separately in the second right\hyp{}hand side. Therefore,
it is worth trying to fold back and replace the pattern by a more
general one, in this case: \(s = \cons{y}{t}\).
\begin{equation*}
\begin{array}{r@{\;}c@{\;}l}
\fun{last}([x]) & \rightarrow & x;\\
\fun{last}(\cons{x}{s}) & \rightarrow & \fun{last}(s).
\end{array}
\end{equation*}
This transformation is correct because the case where \(s\)~is empty
has already been matched by the first pattern. Notice also that we
indeed considered a definition as some data. (We should write more
accurately \emph{metadata} since definitions are not data processed by
the program, but by the programmer.) In passing, \fun{last/1} is in
tail form.

What if we had tried to find directly a definition in tail form? We
might have recalled that such definitions often need an accumulator
and we would have tried perhaps something along these lines:
\begin{equation*}
\begin{array}{r@{\;}c@{\;}l}
\fun{last}_0(s) & \rightarrow & \fun{last}_1(s,0).\\
\fun{last}_1(\el,y) & \rightarrow & y;\\
\fun{last}_1(\cons{x}{s},y) & \rightarrow & \fun{last}_1(s,x).
\end{array}
\end{equation*}
The first observation may be about the function name
\fun{last\(_1\)}. Why not write the following, in accordance with the
style up to now?
\begin{equation*}
\begin{array}{r@{\;}c@{\;}l}
\fun{last}_0(s) & \rightarrow & \fun{last}_0(s,0).\\
\fun{last}_0(\el,y) & \rightarrow & y;\\
\fun{last}_0(\cons{x}{s},y) & \rightarrow & \fun{last}_0(s,x).
\end{array}
\end{equation*}
There would be no confusion between \fun{last\(_0\)/1} and
\fun{last\(_0\)/2} because, each taking a different number of
arguments, they are logically considered different. The reason why we
recommend to distinguish the names and, in general, to use one name
for only one function, is that this discipline enables the compiler to
catch the error consisting in forgetting one argument. For instance,
the program
\begin{equation*}
\begin{array}{r@{\;}c@{\;}l}
\fun{last}_0(s) & \rightarrow & \fun{last}_0(s,0).\\
\fun{last}_0(\el,y) & \rightarrow & y;\\
\fun{last}_0(\cons{x}{s},y) & \rightarrow &
\underline{\fun{last}_0(s)}.
\end{array}
\end{equation*}
contains an error that goes unreported, while
\begin{equation*}
\begin{array}{r@{\;}c@{\;}l}
\fun{last}_0(s) & \rightarrow & \fun{last}_1(s,0).\\
\fun{last}_1(\el,y) & \rightarrow & y;\\
\fun{last}_1(\cons{x}{s},y) & \rightarrow &
\underline{\fun{last}_1(s)}.
\end{array}
\end{equation*}
raises an error. However, for didactic purposes in this book, we will
not always follow this recommendation of having unique function
names. The possibility to use the same name for different functions
which can be otherwise distinguished by the number of their arguments
is called \emph{overloading}. Overloading of functions in the
programming language \Cpp is permitted, but the rules used to
distinguish amongst the different functions sharing the same name is
different than in \Erlang, as it makes use of the number of parameters
but also their static types.

Computing the call \(\fun{last}_0([1,2,3])\) with the original
definition, we find that the three rules are covered until the correct
result is found, that is~\(3\). Because we recommended previously to
make some litmus test and the argument is a stack, we try the empty
stack and obtain the evaluation \(\fun{last}_0(\el) \rightarrow
\fun{last}_1(\el,0) \rightarrow 0\), which is unexpected, since this
test ought to fail (see how \fun{last/1} is not defined for the empty
stack). Can we fix this?

Let us simply change the left\hyp{}hand side of \fun{last\(_0\)/1} so
that only non\hyp{}empty stacks are matched. We find here a case where
more information on the structure of the input is needed and a
variable is too general a pattern. We need instead
\begin{equation*}
\begin{array}{r@{\;}c@{\;}l}
\fun{last}_0(\underline{\cons{x}{s}}) & \rightarrow &
  \fun{last}_0(\underline{\cons{x}{s}},0).\\
\fun{last}_0(\el,y) & \rightarrow & y;\\
\fun{last}_0(\cons{x}{s},y) & \rightarrow & \fun{last}_0(s,x).
\end{array}
\end{equation*}
This emendation seems to go against an improvement we made earlier,
when we replaced \(\cons{y}{t}\) by~\(s\), but it does not: here we
want to exclude some input, that is, we do not seek an equivalent
function, whilst before the purpose was to simplify and obtain an
equivalent function.

The definition of \fun{last\(_0\)/1} is correct and complete but a
careful review should raise some doubts about its actual
simplicity. For example, the initial value of the accumulator, given
in the unique right\hyp{}hand side of \fun{last\(_0\)/1} is~\(0\), but
this value is never used, because it is discarded immediately after in
the second right\hyp{}hand side of \fun{last\(_0\)/2}. Indeed, we
could write the equivalent definition:
\begin{equation*}
\begin{array}{r@{\;}c@{\;}l}
\fun{last}_1(\cons{x}{s}) & \rightarrow & \fun{last}_1(\cons{x}{s},\underline{7}).\\
\fun{last}_1(\el,y) & \rightarrow & y;\\
\fun{last}_1(\cons{x}{s},y) & \rightarrow & \fun{last}_1(s,x).
\end{array}
\end{equation*}
The initial value of the accumulator here does not even need to be an
integer, it could be of any type, like \([4,\el]\). This is the sign
that we should better give up this overly complicated definition,
which is the product of a method that does not consider programs as
data and is founded on the wrong assumption that definitions in tail
form often require an accumulator: in general, they do not.

Take for example the polymorphic identity: \(\fun{id}(x) \rightarrow
x\). It is trivially in tail form. In passing, being in tail form has
nothing to do, in general, with recursion, despite the widespread and
unfortunate locution ``tail\hyp{}recursive function''. A recursive
definition may only feature tail calls (at least one recursive,
directly or indirectly), but a definition with only tail calls may not
be recursive, like \fun{id/1}.
