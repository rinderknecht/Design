\section{Regular expressions}

In \Pascal, an identifier is a letter followed by zero or more letters
or digits, that is, and identifier is a member of the set defined by
\(L(L \cup D)^{*}\). The notation we introduced so far is comfortable
for mathematics but not for computers. Let us introduce another
notation, called \emph{regular expressions}, for describing the same
languages and define its meaning in terms of the mathematical
notation. With this notation, we might define \Pascal identifiers as
\begin{center}
\term{letter} \lparen\term{letter} \disj \term{digit}\rparen\kleene
\end{center}
where the vertical bar means `or', the parentheses group
subexpressions, the star means `zero or more instances of' the
previous expression and juxtaposition means concatenation. A regular
expression~\(r\) is built up out of simpler regular expressions using
a set of rules, as follows. Let~\(\Sigma\) be an alphabet and \(L(r)\)
the language denoted by~\(r\). Then
\begin{enumerate*}

   \item \(\epsilon\) is a regular expression that denotes
     \(\{\varepsilon\}\);\label{regexp_empty}

   \item if \(a \in \Sigma\), then \(a\)~is a regular expression that
     denotes~\(\{a\}\). This is ambiguous: \(a\)~can denote a
     language, a word or a letter --~it depends on the
     context; \label{regexp_sym}

   \item assume \(r\)~and~\(s\) denote the languages \(L(r)\)~and
     \(L(s)\); \(a\)~denotes a letter. Then \label{regexp_rec}
   \begin{enumerate*}

     \item \(r\) \disj \(s\) is a regular expression
     denoting \(L(r) \cup L(s)\);

     \item \(r s\) is a regular expression denoting \(L(r) L(s)\);

     \item \(r\)\kleene{} is a regular expression
     denoting \((L(r))^{*}\);

     \item \lparen\(r\)\rparen{} is a regular expression
     denoting \(L(r)\);

     \item \(\overline{a}\) is a regular expression denoting
       \(\Sigma\backslash \{a\}\).

   \end{enumerate*}

\end{enumerate*}
A language described by a regular expression is a \emph{regular
  language}. Rules~\ref{regexp_empty} and~\ref{regexp_sym} form the
base of the definition. Rule~\ref{regexp_rec} provides the inductive
step. Unnecessary parentheses can be avoided in regular expressions if
\begin{itemize*}

  \item the unary operator \kleene{} has the highest precedence and
  is left associative,

  \item concatenation has the second highest precedence and is left
  associative,

  \item \disj{} has the lowest precedence and is left associative.

\end{itemize*}
Under those conventions, \lparen\(a\)\rparen{} \disj
\lparen\lparen\(b\)\rparen\kleene\lparen\(c\)\rparen\rparen{} is
equivalent to \(a\) \disj \(b\)\kleene\(c\). Both expressions denote
the language containing either the string \(a\) or zero or more
\(b\)'s followed by one \(c\): \(\{a, c, bc, bbc, bbbc, \dots\}\). For
example,
\begin{itemize*}

  \item the regular expression \(a\) \disj \(b\) denotes the set
    \(\{a, b\}\);

  \item the regular expression \lparen\(a\) \disj
    \(b\)\rparen\lparen\(a\) \disj \(b\)\rparen{} denotes \(\{aa, ab,
    ba, bb\}\), the set of all strings of \(a\)'s and \(b\)'s of
    length two. Another regular expression for the set is \(aa\) \disj
    \(ab\) \disj \(ba\) \disj \(bb\);

  \item the regular expression \(a\)\kleene{} denotes the set of all
    strings of zero or more \(a\)'s, i.e. \(\{\varepsilon, a, aa, aaa,
    \dots\}\);

  \item the regular expression \lparen\(a\) \disj
    \(b\)\rparen\kleene{} denotes the set of all strings containing
    zero of more instances of an \(a\) or \(b\), that is the language
    of all words made of \(a\)'s and \(b\)'s. Another expression is
    \lparen\(a\)\kleene\(b\)\kleene\rparen\kleene.

\end{itemize*}
If two regular expressions \(r\)~and~\(s\) denote the same language,
we say that \(r\)~and~\(s\) are \emph{equivalent} and write \(r =
s\). In \fig~\vref{fig:laws},
\begin{figure}
\centering
\begin{tabular}{c|l}
\toprule
  \multicolumn{1}{c}{\textsc{Law}}
& \multicolumn{1}{c}{\textsc{Description}}\\
\midrule
  \(r\) \disj \(s\) = \(s\) \disj \(r\)
& \disj is commutative\\
\hline
  \(r\) \disj \lparen\(s\) \disj \(t\)\rparen{}
  = \lparen\(r\) \disj \(s\)\rparen{} \disj \(t\)
& \disj is associative\\
\hline
  \lparen\(rs\)\rparen \(t\) = \(r\)\lparen\(st\)\rparen
& concatenation is associative\\
\hline
  \(r\)\lparen\(s\) \disj \(t\)\rparen{} = \(rs\) \disj \(rt\)
& concatenation distributes over \disj\\
  \lparen\(s\) \disj \(t\)\rparen \(r\) = \(sr\) \disj \(tr\)
&\\
\hline
  \(\epsilon r = r\)
& \(\epsilon\) is the identity element\\
  \(r \epsilon = r\)
& for the concatenation\\
\hline
  \(r\)\kleene\kleene = \(r\)\kleene
& Kleene closure is idempotent\\
\hline
  \(r\)\kleene = \(r\)\plus \disj \(\epsilon\)
& Kleene closure and positive closure\\
  \(r\)\plus = \(r r\)\kleene
& are closely linked\\
\bottomrule
\end{tabular}
\caption{Algebraic laws on regular languages}
\label{fig:laws}
\end{figure}
we show useful algebraic laws on regular languages.

\subsection*{Regular definitions}

It is convenient to give names to regular expressions and define new
regular expressions using these names as if they were symbols. If
\(\Sigma\) is an alphabet, then a \emph{regular definition} is a
series of definitions of the form
\begin{align*}
    d_1 &\rightarrow r_1\\
    d_2 &\rightarrow r_2\\
    &\cdots\\
    d_n &\rightarrow r_n
\end{align*}
where each \(d_i\) is a distinct name and each \(r_i\) is a regular
expression over the alphabet \(\Sigma \cup \{d_1, d_2, \dots,
d_{i-1}\}\), \emph{i.e.,} the basic symbols and the previously defined
names. The restriction to \(d_j\) such \(j < i\) allows to construct a
regular expression over \(\Sigma\) only by repeatedly replacing all
the names in it. For instance, as we have stated, the set of \Pascal
identifiers can be defined by the regular definitions
\begin{align*}
\term{letter} & \rightarrow \text{\exc{A} \disj \exc{B} \disj
  \ldots \disj \exc{Z} \disj \exc{a} \disj \exc{b} \disj \ldots \disj
  \exc{z}} \\
\term{digit} & \rightarrow \text{\exc{0} \disj \exc{1} \disj \exc{2}
  \disj \exc{3} \disj \exc{4} \disj \exc{5} \disj \exc{6} \disj
  \exc{7} \disj \exc{8} \disj \exc{9}}\\
\term{id} & \rightarrow \text{\term{letter} \lparen\term{letter}
  \disj \term{digit}\rparen\kleene}
\intertext{Unsigned numbers in \Pascal are strings like
\texttt{5280}, \texttt{39.37}, \texttt{6.336E4}
or \texttt{1.894E-4}.}
\term{digit} & \rightarrow \text{ \exc{0} \disj \exc{1} \disj \exc{2}
  \disj \exc{3} \disj \exc{4} \disj \exc{5} \disj \exc{6} \disj
  \exc{7} \disj \exc{8} \disj \exc{9}}\\
\term{digits} & \rightarrow \text{\term{digit} \term{digit}\kleene}\\
\term{opt\_fraction} & \rightarrow \text{\exc{.} \term{digits}
  \disj} \, \epsilon\\
\term{opt\_exponent} & \rightarrow \text{\lparen\exc{E} \lparen
  \exc{+} \disj \exc{-} \disj} \, \epsilon \, \text{\rparen{}
  \term{digits}\rparen{} \disj} \, \epsilon\\
\term{num} & \rightarrow \text{\term{digits} \term{opt\_fraction}
  \term{opt\_exponent}}
\end{align*}
Certain constructs occur so frequently in regular expressions that it
is convenient to introduce notational shorthands for them:
\begin{itemize}

  \item \emph{Zero or one instance.} The unary operator `\opt{}' means
    `zero or one instance of.' Formally, by definition, if~\(r\) is a
    regular expression then \(r\)\opt = \(r\) \disj \(\epsilon\). In
    other words, \lparen\(r\)\rparen\opt{} denotes the language \(L(r)
    \cup \{\varepsilon\}\).
\begin{align*}
\term{digit} & \rightarrow \text{\exc{0} \disj \exc{1} \disj \exc{2}
  \disj \exc{3} \disj \exc{4} \disj \exc{5} \disj \exc{6} \disj
  \exc{7} \disj \exc{8} \disj \exc{9}}\\
\term{digits} & \rightarrow \text{\term{digit}\plus}\\
\term{opt\_fraction} & \rightarrow \text{\lparen\exc{.}
  \term{digits}\rparen\opt}\\
\term{opt\_exponent} & \rightarrow \text{\lparen\exc{E} \lparen
  \exc{+} \disj \exc{-}\rparen\opt{} \term{digits}\rparen\opt}\\
\term{num} & \rightarrow \text{\term{digits} \term{opt\_fraction}
  \term{opt\_exponent}}
\end{align*}

 \item It is also possible to write:
\begin{align*}
\term{digit} & \rightarrow \text{\exc{0} \disj \exc{1} \disj \exc{2}
  \disj \exc{3} \disj \exc{4} \disj \exc{5} \disj \exc{6} \disj
  \exc{7} \disj \exc{8} \disj \exc{9}}\\
\term{digits} & \rightarrow \text{\term{digit}\plus}\\
\term{fraction} & \rightarrow \text{\exc{.} \term{digits}}\\
\term{exponent} & \rightarrow \text{\exc{E} \lparen \exc{+} \disj
  \exc{-}\rparen\opt{} \term{digits}}\\
\term{num} & \rightarrow \text{\term{digits} \term{fraction}\opt{}
  \term{exponent}\opt}
\end{align*}

\end{itemize}
If we want to specify the characters `\texttt{?}', `\texttt{*}',
`\texttt{+}', `\texttt{|}', we write them with a preceding backslash,
\emph{e.g.,} `\verb+\?+', or between double-quotes, \emph{e.g.,}
\verb+"?"+. Then, of course, the character double-quote must have a
backslash: \verb+\"+. It is also sometimes useful to match against end
of lines and end of files: \verb+\n+ stands for the control character
`end of line' and \term{\$} is for `end of file'.

\subsection*{Non-regular languages}

Some languages cannot be described by any regular expression. For
example, the language of balanced parentheses cannot be recognised by
any regular expression: \lparen\rparen, \lparen\lparen\rparen\rparen,
\lparen\rparen\lparen\rparen,
\lparen\lparen\lparen\rparen\rparen\lparen\rparen\rparen{}
etc. Another example is the C programming language: it is not a
regular language because it contains embedded blocs between `\verb+{+'
and `\verb+}+'. Therefore, a lexer cannot recognise valid C programs:
we need a parser.

\paragraph{Exercises}

\input{regexp_question_01}
\input{regexp_answer_01}
\input{regexp_question_02}
\paragraph{Answer 2.}

When answering these questions, it is important to keep in mind that
the language of words made up on the alphabet~\(\Sigma\)
is~\(\Sigma^{*}\) and that there are, in general, several regular
expressions describing one language.

\begin{enumerate}

  \item The constraint on the words is that they must be of the shape
    \(a \ldots a\) where the dots stand for `any combination of \(a\)
    and \(b\)'. In other words, one answer is \(a \lparen a \;
    \disjM{} \, b\rparen\kleeneM \, a \, \disjM \, a\).

  \item This question is very simple since the language of all words
    is \(\lparen a \; \disjM{} \, b\rparen\kleeneM\), we have to
    remove \(\epsilon\), \emph{i.e.,} one simple answer is \(\lparen a \;
    \disjM{} \, b\rparen\plusM\).

  \item The question implies that the words we are looking for are of
    the form \(\ldots a \, \_ \, \_\) where the dots stand for `any
    sequence of \(a\) and \(b\)' and each `\_' stands for a regular
    expression denoting any letter. Any letter is described
    by \(\lparen a \, \disjM{} \, b\rparen\); therefore one possible
    answer is \(\lparen a \, \disjM{} \, b\rparen\kleeneM{}
    a \, \lparen a \, \disjM{} \, b\rparen \, \lparen a \, \disjM{} \,
    b\rparen\).

  \item The words we search contain, at any place, exactly three
    \(a\), so are of the form \(\ldots a \ldots a \ldots a \ldots\),
    where the dots stand for `any letter except \(a\)', \emph{i.e.,} `any
    number of \(b\)'. In other words: \(b\kleeneM a b\kleeneM a
    b\kleeneM a b\kleeneM\).

  \item Because the alphabet contains only two letters, the question
    is equivalent to: 'All words containing the substring
    ab', \emph{i.e.,} the words are of the form \(\ldots ab \ldots\)
    where the dots stand for `any sequence of \(a\) and \(b\)'. It is
    then easy to understand that a short answer is \(\lparen
    a \, \disjM{} \, b\rparen\kleeneM ab \lparen a \, \disjM{} \,
    b\rparen\kleeneM\).

  \item Because the alphabet is made only of two letters, the answer
    is easy: we put first all the \(a\) and then all the \(b\):
    \(a\kleeneM b\kleeneM\).

  \item Since the alphabet contains only two letters, the only way to
    not repeat a letter is to only have substrings \(ab\) or \(ba\) in
    the words we look for. In other words: \(abab\ldots ab\) or
    \(abab\ldots aba\) or \(baba\ldots ba\) or \(baba\ldots bab\). In
    short: \(\lparen ab \rparen\kleeneM a\opt \, \disjM{} \, \lparen
    ba \rparen\kleeneM b\opt\) or, even shorter: \(a\opt \lparen ba
    \rparen\kleeneM b\opt\).

\end{enumerate}

\input{regexp_question_03}
\input{regexp_answer_03}
