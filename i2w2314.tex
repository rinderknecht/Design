%%-*-latex-*-

\documentclass[11pt]{article}

\usepackage{pst-eps}
\usepackage{amsmath,amssymb}

\input{commands}

\begin{document}

\TeXtoEPS
\(
\boxed{
\begin{array}{@{}r@{\;}c@{\;}l@{}}
\fun{i2w}([2,3,1,4])
& \xrightarrow{\smash{\xi}} & \fun{i2w}([2,3,1,4],\el,\el)\\
& \xrightarrow{\smash{\upsilon}} & \fun{i2w}([3,1,4],\el,[2])\\
& \xrightarrow{\smash{\sigma}} & \fun{i2w}([3,1,4],[2],\el)\\
& \xrightarrow{\smash{\upsilon}} & \fun{i2w}([1,4],[2],[3])\\
& \xrightarrow{\smash{\tau}} & \fun{i2w}([1,4],\el,[2,3])\\
& \xrightarrow{\smash{\upsilon}} & \fun{i2w}([4],\el,[1,2,3])\\
& \xrightarrow{\smash{\sigma}} & \fun{i2w}([4],[1],[2,3])\\
& \xrightarrow{\smash{\sigma}} & \fun{i2w}([4],[2,1],[3])\\
& \xrightarrow{\smash{\sigma}} & \fun{i2w}([4],[3,2,1],\el)\\
& \xrightarrow{\smash{\upsilon}} & \fun{i2w}(\el,[3,2,1],[4])\\
& \xrightarrow{\smash{\rho}} & \fun{i2w}(\el,[2,1],[3,4])\\
& \xrightarrow{\smash{\rho}} & \fun{i2w}(\el,[1],[2,3,4])\\
& \xrightarrow{\smash{\rho}} & \fun{i2w}(\el,\el,[1,2,3,4])\\
& \xrightarrow{\smash{\pi}} & [1,2,3,4].
\end{array}}
\)
\endTeXtoEPS

\end{document}


