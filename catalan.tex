\chapter{Catalan Trees}
\label{chap:Catalan}

In part~\ref{part:linear}, we dealt with linear structures, like
stacks and queues, but, to really understand programs operating on
such structures, we needed the concept of tree. This is why we
introduced very early on abstract syntax trees\index{tree!abstract
  syntax $\sim$}, directed acyclic graphs (e.g.,
\fig~\vref{fig:cat_dag}), comparison trees (e.g.,
\fig~\vref{fig:cmp_tree}), binary trees (e.g.,
\fig~\vref{fig:comp_tree}), proof trees (e.g.,
\fig~\vref{fig:perm_proof}), evaluation trees (e.g.,
\fig~\vref{fig:2way_unbal}) and merge trees (e.g.,
\fig~\vref{fig:bot_up1}). Those trees were meta\hyp{}objects, or
concepts, used to understand the linear structure at hand.

In this chapter, we take a more abstract point of view and we consider
the general class of \emph{Catalan trees}, or general trees. We will
study them as mathematical objects with the aim to transfer our
results to the trees used as data structures to implement
algorithms. In particular, we will be interested in measuring,
counting them, and determining some average parameters relative to
their shape, the reason being that knowing what a random tree looks
like will tell us something about the cost of traversing it in
different ways.

% Wrapping figure better declared before a paragraph
%
\begin{wrapfigure}[10]{r}[0pt]{0pt}
% [10] vertical lines
% {r} mandatory right placement
% [0pt] of margin overhang
\centering
\includegraphics[bb=71 646 160 712]{catalan_tree}
\caption{Catalan tree of height~4}
\label{fig:catalan_tree}
\end{wrapfigure}
Catalan trees are a special kind of graph, that is, an object made of
\emph{nodes} (also called \emph{vertices}) connected by \emph{edges},
without orientation (only the connection matters). What makes Catalan
trees is the distinction of a node, called the \emph{root}, and the
absence of cycles, that is, closed paths made of nodes successively
connected. Catalan trees are often called \emph{ordered trees}, or
\emph{planted plane trees}, in graph theory, and \emph{unranked
  trees}, or \emph{\(n\)-ary trees} in programming theory. An example
is given in \fig~\ref{fig:catalan_tree}. Note how the root is the
topmost node and has four subtrees, whose root are called
\emph{children}, given in order. The nodes drawn as white disks
\((\circ)\) make up a maximal path starting from the root (the number
of nodes along it is maximal). The ending node has no children; there
are actually~\(8\) such nodes in total, called the \emph{leaves}. The
number of edges connecting white disks is the \emph{height} of the
Catalan tree (there may be several maximal paths of same length), so
the given example has height~\(4\).

Programmers implement Catalan trees as a data structure, \emph{e.g.},
using \textsf{XML}, in which case some information is stored in the
nodes and its retrieval may --~in the worst case~-- require the
reaching of a leaf. The maximum cost of a search is thus proportional
to the height of the tree and the determination of the average height
becomes relevant when performing a series of random searches
\citep{VitterFlajolet_1990}. For this kind of analysis, we need first
to find the number of Catalan trees with a given size. There are two
common measures for the size: either we quantify the trees by their
number of nodes or we count the edges. In fact, using one or the other
is a matter of convenience or style: there are \(n\)~edges if there
are \(n+1\) nodes, simply because each node, save the root, has one
parent. It is often the case that formulas about Catalan trees are a
bit simpler when using the number of edges, so this will be our
measure of size in this chapter.

\section{Enumeration}
\label{sec:Catalan_enumeration}

% Wrapping figure better declared before a paragraph
%
\begin{wrapfigure}[10]{r}[0pt]{0pt}
% [10] vertical lines
% {r} mandatory right placement
% [0pt] of margin overhang
\hspace*{0pt} % Somehow needed nowadays
\centering
\subfloat[Dyck path\label{fig:dyck}]{%
  \includegraphics[bb=71 634 158 712]{dyck}}
\quad
\subfloat[Catalan tree\label{fig:ctree}]{%
\includegraphics[bb=65 663 127 721]{ctree}
}
\caption{Bijection}
\label{fig:bijection}
\end{wrapfigure}
In most textbooks \cite[\S~5.1 \&~5.2]{SedgewickFlajolet_1996} is
shown how to find the number of Catalan trees with \(n\)~edges by
means of some quite powerful mathematical tools known as
\emph{generating functions}\index{enumerative combinatorics!generating
  function} \cite[chap.~7]{GrahamKnuthPatashnik_1994}. Instead, for
didactical purposes, we opt for a more intuitive technique in
enumerative combinatorics\index{enumerative combinatorics} which
consists in constructing a one\hyp{}to\hyp{}one correspondence between
two finite sets, so the cardinal of one set is the cardinal of the
other. In other words, we are going to relate bijectively, on the one
hand, Catalan trees, and, on the other hand, other combinatorial
objects which are relatively easy to count.

The objects most suitable for our purpose are \emph{monotonic lattice
  paths}\index{monotonic lattice path|see{Dyck path}} in a square grid
\citep{Mohanty_1979,Humphreys_2010}. These paths are made up of steps
oriented upwards (\(\uparrow\)), called \emph{rises}, and steps
rightwards (\(\rightarrow\)), called \emph{falls}, starting at the
bottom left corner \((0,0)\). \emph{Dyck paths}\index{Dyck path} of
length~\(2n\) are paths ending at \((n,n)\) which stay above the
diagonal, or touch it. We want to show that \emph{there exists a
  bijection between Dyck paths of length~\(2n\) and Catalan trees with
  \(n\)~edges.}

\begin{wrapfigure}[8]{r}[0pt]{0pt}
% [8] vertical lines
% {r} mandatory right placement
% [0pt] of margin overhang
\centering
\includegraphics[bb=71 661 128 720]{ctree_pre}
\caption{}
\label{fig:ctree_pre}
\end{wrapfigure}
To understand that bijection, we need first to present a particular
kind of \emph{traversal}\index{tree!traversal}, or
\emph{walk}\index{tree!walk|see{traversal}}, of Catalan trees. Let us
imagine that a tree is a map where nodes represent towns and edges
roads. A complete traversal of the tree consists then in starting our
trip at the root and, following edges, to visit all the nodes. (It is
allowed to visit several times the same nodes, since there are no
cycles.) Of course, there are many ways to achieve this tour and the
one we envisage here is called a \emph{preorder traversal}. At every
node, we take the leftmost unvisited edge and visit the subtree in
preorder; when back at the node, we repeat the choice with the
remaining unvisited children. For more clarity, we show in
\fig~\ref{fig:ctree_pre} the \emph{preorder
numbering}\index{tree!Catalan $\sim$!preorder numbering} of
\fig~\ref{fig:ctree}, where the order in which a node is visited first
is shown instead of a black disk \((\bullet)\).

\begin{wrapfigure}[14]{r}[0pt]{0pt}
% [13] vertical lines
% {r} mandatory right placement
% [0pt] of margin overhang
\centering
\includegraphics[bb=71 591 186 721]{reflection}
\caption{Reflection of a prefix w.r.t. \(y = x - 1\)}
\label{fig:reflection}
\end{wrapfigure}
The first part of the bijection is an injection from Catalan trees
with \(n\)~edges to Dyck paths of length~\(2n\). By traversing the
tree in preorder, we associate one rise to an edge on the way down,
and a fall to the same edge on the way up. Obviously, there are
\(2n\)~steps in the Dyck path. The surjection simply consists in
reversing the process by reading the Dyck path step by step,
rightwards, and build the corresponding tree. Now, we need to count
the number of Dyck paths of length~\(2n\), which we know now is also
the number of Catalan trees with \(n\)~edges.

The total number of monotonic paths of length~\(2n\) is the number of
choices of \(n\)~rises amongst \(2n\)~steps, that is,
\(\binom{2n}{n}\). We need now to subtract the number of paths that
start with a rise and cross the diagonal. Such a path is shown in
\fig~\ref{fig:reflection}, drawn as a bold continuous line. The first
point reached below the diagonal is used to plot a dotted line
parallel to the diagonal. All the steps from that point back to
\((0,0)\) are then changed into their counterpart: a rise by a fall
and vice-versa. The resulting segment is drawn as a dashed line. This
operation is called a \emph{reflection} \citep{Renault_2008}. The crux
of the matter is that we can reflect each monotonic path crossing the
diagonal into a distinct path from \((1,-1)\) to
\((n,n)\). Furthermore, all reflected paths can be reflected when they
reach the dotted line back into their original counterpart. In other
words, the reflection is bijective. (Another intuitive and visual
approach to the same result has been published by \cite{Callan_1995}.)
Consequently, there are as many monotonic paths from \((0,0)\) to
\((n,n)\) that cross the diagonal as there are monotonic paths from
\((1,-1)\) to \((n,n)\). The latter are readily enumerated:
\(\binom{2n}{n-1}\). As a conclusion, the number of Dyck paths of
length~\(2n\) is\label{eq:Ann}
\begin{align*}
C_n &= \binom{2n}{n} - \binom{2n}{n-1}
= \binom{2n}{n} - \frac{(2n)!}{(n-1)!(n+1)!}\\
&= \binom{2n}{n} - \frac{n}{n+1} \cdot \frac{(2n)!}{n!n!}
 = \binom{2n}{n} - \frac{n}{n+1} \binom{2n}{n} = \frac{1}{n+1}\binom{2n}{n}.
\end{align*}
The numbers \(C_n\) are called \emph{Catalan numbers}\index{Catalan
  numbers@$C_n$|see{Catalan numbers}}\index{Catalan
  numbers}. Using Stirling's
formula\index{Stirling's formula}, seen in
equation~\eqref{eq:Stirling} on page~\pageref{eq:Stirling}, we find
that the number of Catalan trees with \(n\)~edges is
\begin{equation}
  C_n = \frac{1}{n+1}\binom{2n}{n} \sim \frac{4^n}{n\sqrt{\pi n}}.
\label{eq:Cn}
\end{equation}

\section{Average path length}

The \emph{path length} of a Catalan tree is the sum of the lengths of
the paths from the root. We have seen this concept in the context of
binary trees, where it was declined in two variants, \emph{internal
  path length} (page~\pageref{insertion__internal_path_length}) and
\emph{external path length}
(page~\pageref{sorting__external_path_length}), depending on the end
node being internal or external. In the case of Catalan trees, the
pertinent distinction between nodes is to be a \emph{leaf} (that is, a
node without subtrees) or not, but some authors nevertheless speak of
external path length when referring to the distances to the leaves,
and of internal path length for the non\hyp{}leaf nodes, hence we must
bear in mind whether the context is the Catalan trees or the binary
trees.

In order to study the average path length of Catalan trees, and some
related parameters, we may follow \cite{DershowitzZaks_1981} by
finding first the average number of nodes of degree~\(d\) at
level~\(l\) in a Catalan tree with \(n\)~edges. The \emph{degree of a
  node} is the number of its children and its \emph{level} is its
distance to the root counted in edges and the root is at
level~\(0\).

% Wrapping figure better declared before a paragraph
%
%%\begin{wrapfigure}[10]{r}[0pt]{0pt}
\begin{figure}
% [10] vertical lines
% {r} mandatory right placement
% [0pt] of margin overhang
\centering
\subfloat[Dyck path\label{fig:dyck_deg}]{%
  \includegraphics[bb=71 634 158 712]{dyck_deg}}
\quad
\subfloat[Catalan tree\label{fig:ctree_deg}]{%
\includegraphics[bb=65 663 127 721]{ctree_deg}
}
\caption{Degree-based bijection}\label{fig:degree_based_bijection}
\end{figure}
%%\end{wrapfigure}
The first step of our method for finding the average path length
consists in finding an alternative bijection between Catalan trees and
Dyck paths. In \fig~\ref{fig:ctree}, we see a Catalan tree equivalent
to the Dyck path in \fig~\ref{fig:dyck}, built from the preorder
traversal of that tree. \Fig~\ref{fig:ctree_deg} shows the same tree,
where the contents of the nodes is their degree. The preorder
traversal (of the degrees) is \([3,0,0,2,1,0,0]\). Since the last
degree is always~\(0\) (a leaf), we remove it and settle for
\([3,0,0,2,1,0]\). Another equivalent Dyck path is obtained by mapping
the degrees of that list into as many occurrences of rises
(\(\uparrow\)) and one fall (\(\rightarrow\)), so, for instance,
\(3\)~is mapped to (\(\uparrow, \uparrow, \uparrow, \rightarrow\)) and
\(0\)~to (\(\rightarrow\)). In the end, \([3,0,0,2,1,0]\) is mapped
into \([\uparrow, \uparrow, \uparrow, \rightarrow, \rightarrow,
  \rightarrow, \uparrow, \uparrow, \rightarrow, \uparrow, \rightarrow,
  \rightarrow]\), which corresponds to the Dyck path in
\fig~\ref{fig:dyck_deg}. It is easy to convince ourself that we can
reconstruct the tree from the Dyck path, so we indeed have a
bijection.

The reason for this new bijection is that we need to find the average
number of Catalan trees whose root has a given degree. This number
will help us in finding the average path length, following an idea of
\cite{Ruskey_1983}. From the bijection, it is clear that the number of
trees whose root has degree~\(r=3\) is the number of Dyck paths made
of the segment from \((0,0)\) to \((0,r)\), followed by one fall (see
the dot at \((1,r)\) in \fig~\ref{fig:dyck_deg}), and then all
monotonic paths above the diagonal until the upper right corner
\((n,n)\). Therefore, we need to determine the number of such paths.

We have seen in section~\vref{sec:Catalan_height} the bijective
reflection of paths and the counting principle of inclusion and
exclusion. Let us add to our tools one more bijection which proves
often useful: the \emph{reversal}. It simply consists in reversing the
order of the steps making up a path. Consider for example
\fig~\ref{fig:path_reversal}.
\begin{figure}
\centering
\subfloat[Reversal of
\fig~\ref{fig:reflection} \label{fig:path_reversal}]{
\includegraphics[bb=71 606 188 721]{path_reversal}}
\qquad
\subfloat[Reversal and reflection of \fig~\ref{fig:dyck_deg} after \((1,3)\)\label{fig:path_deg}]{
\includegraphics[bb=71 634 162 721]{path_deg}}
\caption{Reversals and reflections}
\end{figure}
Of course, the composition of two bijections being a bijection, the
composition of a reversal and a reflection is bijective, hence the
monotonic paths above the diagonal from \((1,r)\) to \((n,n)\) are in
bijection with the monotonic paths above the diagonal from \((0,0)\)
to \((n-r,n-1)\). For example, \fig~\ref{fig:path_deg} shows the
reversal and reflection of the Dyck path of \fig~\ref{fig:dyck_deg}
after the point \((1,3)\), distinguished by the black disk
(\(\bullet\)).

Recalling that Catalan trees with \(n\)~edges are in bijection with
Dyck paths of length~\(2n\) (section~\vref{sec:Catalan_enumeration}),
we now know that the number of Catalan trees with \(n\)~edges and
whose root has degree~\(r\) is the number of monotonic paths above the
diagonal from the point \((0,0)\) to \((n-r,n-1)\). We can find this
number using the same technique we used for the total number~\(C_n\)
of Dyck paths. The principle of inclusion and exclusion says that we
should count the total number of paths with same extremities and
retract the number of paths that cross the diagonal. The former is
\(\binom{2n-r-1}{n-1}\), which enumerates the ways to interleave
\(n-1\)~rises (\(\uparrow\)) and \(n-r\) falls (\(\rightarrow\)). The
latter number is the same as the number of monotonic paths from
\((1,-1)\) to \((n-r,n-1)\), as shown by reflecting the paths up to
their first crossing, that is, \(\binom{2n-r-1}{n}\); in other words,
that is the number of interleavings of \(n\)~rises with \(n-r-1\)
falls. Finally, imitating the derivation of equation~\eqref{eq:Cn},
the number \(\mathcal{R}_n(r)\) of trees with \(n\)~edges and root of
degree~\(r\) is
\begin{equation*}
\mathcal{R}_n(r) = \binom{2n-r-1}{n-1} - \binom{2n-r-1}{n}
                 = \frac{r}{2n-r} \binom{2n-r}{n}.
\end{equation*}

Let \(\mathcal{N}_n(l,d)\) be the number of nodes in the set of all
Catalan trees with \(n\)~edges, which are at level~\(l\) and have
degree~\(d\). This number is the next step in determining the average
path length because \cite{Ruskey_1983} found a neat bijection to
relate it to \(\mathcal{R}_n(r)\) by the following equation:
\begin{equation*}
\mathcal{N}_n(l,d) = \mathcal{R}_{n+l}(2l+d).
\end{equation*}
In \fig~\ref{fig:level_deg}
\begin{figure}
\centering
\subfloat[\(n\)~edges, (\(\bullet\)) is of degree~\(d\) and at level~\(l\) \label{fig:level_deg}]{%
  \includegraphics[bb=71 594 150 721]{level_deg}}
\quad
\subfloat[\(n+l\) edges, root of degree~\(2l+d\)\label{fig:lifted_node}]{%
\includegraphics[bb=71 663 295 721]{lifted_node}
}
\caption{Bijection}
\label{fig:bij_root_level}
\end{figure}
is shown the general pattern of a Catalan tree with node (\(\bullet\))
of level~\(d\) and degree~\(d\). The double edges denote a set of
edges, so the \(\mathcal{L}_i\), \(\mathcal{R}_i\) and
\(\mathcal{B}_i\) actually represent forests. In
\fig~\ref{fig:lifted_node} we see a Catalan tree in bijection with the
former, from which it is made by lifting the node of interest
(\(\bullet\)) to become the root, the forests \(\mathcal{L}_i\) with
their respective parents are attached below it, then the
\(\mathcal{B}_i\), and, finally, the \(\mathcal{R}_i\) for which new
parents are needed (inside a dashed frame in the figure). Clearly, the
new root is of degree \(2l+d\) and there are \(n+l\)
edges. Importantly, the transformation can be inverted for any tree
(it is injective and surjective), so it is indeed a bijection. We
deduce
\begin{equation}
\mathcal{N}_n(l,d) = \frac{2l+d}{2n-d}\binom{2n-d}{n+l}
= \binom{2n-d-1}{n+l-1} - \binom{2n-d-1}{n+l},
\label{eq:Nn_l_d}
\end{equation}
where the last step follows from expressing the binomial coefficient
in terms of the factorial function. In particular, this entails that
the total number of nodes at level~\(l\) in all Catalan trees with
\(n\)~edges is
\begin{equation*}
\sum_{d=0}^{n}\mathcal{N}_n(l,d)
  = \sum_{d=0}^{n}\binom{2n-d-1}{n+l-1}
    - \sum_{d=0}^{n}\binom{2n-d-1}{n+l}.
\end{equation*}
Let us consider the first sum:
\begin{equation*}
\sum_{d=0}^{n}\binom{2n-d-1}{n+l-1}
  = \sum_{i=n-1}^{2n-1}\binom{i}{n+l-1}
  = \sum_{i=n+l-1}^{2n-1}\binom{i}{n+l-1}.
\end{equation*}
We can now make use of the identity \eqref{eq:binom_sum}
\vpageref{eq:binom_sum}, which is equivalent to
\(\sum_{i=j}^{k}\binom{i}{j} = \binom{k+1}{j+1}\), so
\(j=n+l-1\) and \(k=2n-1\) yields
\begin{equation*}
\sum_{d=0}^{n}\binom{2n-d-1}{n+l-1} = \binom{2n}{n+l}.
\end{equation*}
Furthermore, replacing \(l\)~by \(l+1\) gives
\(\sum_{d=0}^{n}\binom{2n-d-1}{n+l} = \binom{2n}{n+l+1}\), so the
total number of nodes at level~\(l\) in all Catalan trees with
\(n\)~edges is
\begin{equation}
\sum_{d=0}^{n}\mathcal{N}_n(l,d)
 = \binom{2n}{n+l} - \binom{2n}{n+l+1}
 = \frac{2l+1}{2n+1}\binom{2n+1}{n-l}.
\label{eq:N_n_l_d}
\end{equation}

Let \(\Expected{P_n}\) be the \emph{average path length} of a Catalan
tree with \(n\)~edges. We have
\begin{equation*}
  \Expected{P_n} := \frac{1}{C_{n}} \cdot \sum_{l=0}^{n}l\sum_{d=0}^{n}\mathcal{N}_n(l,d),
\end{equation*}
because there are \(C_n\)~trees and the double summation is the sum of
the path lengths of all the trees. If we average again by the number
of nodes, \emph{i.e.,} \(n+1\), we obtain the average level of a node
in a random Catalan tree and beware that some authors take this as the
definition of the average path length. Alternatively, if we pick
distinct Catalan trees with \(n\)~edges at random and then pick
random, distinct nodes in them, \(\Expected{P_n}/(n+1)\) is the limit
of the average cost of reaching the nodes in question from the
root. Using equations \eqref{eq:N_n_l_d} and~\eqref{eq:Cn}
\vpageref{eq:Cn}, we get
\begin{align*}
\Expected{P_n} \cdot C_n
 &= \sum_{l=0}^{n}l\left[\binom{2n}{n+l} - \binom{2n}{n+l+1}\right]\\
 &= \sum_{l=1}^{n}l \binom{2n}{n+l} - \sum_{l=0}^{n-1}l \binom{2n}{n+l+1}\\
 &= \sum_{l=1}^{n}l \binom{2n}{n+l} - \sum_{l=1}^{n}(l-1)\binom{2n}{n+l}\\
 &= \sum_{l=1}^{n}\binom{2n}{n+l}
  = \sum_{i=n+1}^{2n}\binom{2n}{i}.
\end{align*}
The remaining summation is easy to crack because it is the sum of one
half of an even row in Pascal's triangle. We see in
\fig~\vref{fig:pascal_triangle} that the first half equals the second
half, only the central element remaining (there is an odd number of
entries in an even row). This is readily proven as follows:
\(\sum_{j=0}^{n-1}\binom{2n}{j} = \sum_{j=0}^{n-1}\binom{2n}{2n-j} =
\sum_{i=n+1}^{2n}\binom{2n}{i}\). Therefore
\begin{equation*}
\sum_{i=0}^{2n}\binom{2n}{i} = 2 \cdot \!\! \sum_{i=n+1}^{2n}\binom{2n}{i}
+ \binom{2n}{n},
\end{equation*}
and we can continue as follows:
\begin{equation*}
\frac{\Expected{P_n}}{n+1}
  = \frac{1}{2}\binom{2n}{n}^{-1}\left[\sum_{i=0}^{2n}\binom{2n}{i} -
    \binom{2n}{n}\right]
  = \frac{1}{2}\left[\binom{2n}{n}^{-1}\sum_{i=0}^{2n}\binom{2n}{i} - 1\right].
\end{equation*}
The remaining sum is perhaps the most famous combinatorial identity
because it is a corollary of the venerable \emph{binomial theorem},
which states that, for all real numbers \(x\)~and~\(y\), and all
positive integers~\(n\), we have the following equality:
\begin{equation*}
(x+y)^n = \sum_{k=0}^{n}\binom{n}{k}x^{n-k}y^k.
\end{equation*}
The truth of this statement can be seen by the following
reckoning. Since, by definition, \((x+y)^n =
\underline{(x+y)(x+y)\ldots(x+y)}_{\text{\(n\) times}}\), each term in
the expansion of \((x+y)^{n}\) has the form \(x^{n-k}y^k\), for
some~\(k\) ranging from~\(0\) to~\(n\), included. The coefficient of
\(x^{n-k}y^{k}\) for a given~\(k\) is simply the number of ways to
choose~\(k\) variables~\(y\) from the \(n\)~factors of \((x+y)^{n}\),
the \(x\)~variables coming from the remaining \(n-k\) factors.

Setting \(x=y=1\) yields the identity \(2^n =
\sum_{k=0}^{n}\binom{n}{k}\), which finally unlocks our last step:
\begin{equation}
\Expected{P_n} = \frac{n+1}{2}\left[4^{n}\binom{2n}{n}^{-1} - 1\right].
\label{eq:EPn}
\end{equation}
Recalling \eqref{eq:Cn} \vpageref{eq:Cn}, we obtain the asymptotic
expansion:
\begin{equation}
\Expected{P_n} \sim \frac{1}{2}n\sqrt{\pi n}.
\label{eq:Pn_asymptotics}
\end{equation}
Note that this equivalence also holds if~\(n\) denotes a number of
nodes, instead of edges. The exact formula for the average path length
of Catalan trees with \(n\)~nodes is \(\Expected{P_{n-1}}\) because
there are then \(n-1\) edges.

For some applications, it may be useful to know the external and
internal path lengths, which are, respectively, the path lengths up to
the leaves and to inner nodes (not to be confused with the external
and internal path lengths of binary trees). Let \(\Expected{E_n}\)~be
the former and \(\Expected{I_n}\) the latter. We have
\begin{align*}
\Expected{E_n} \cdot C_n
  &:= \sum_{l=0}^{n} l \cdot \mathcal{N}_n(l,0)
   = \sum_{l=0}^{n}l\left[\binom{2n-1}{n+l-1} -
     \binom{2n-1}{n+l}\right]\\
  &= \sum_{l=0}^{n-1}(l+1)\binom{2n-1}{n+l} -
     \sum_{l=0}^{n-1}l \binom{2n-1}{n+l}
   = \sum_{l=0}^{n-1}\binom{2n-1}{n+l},\\
\Expected{E_n} \cdot C_n
  &= \sum_{i=n}^{2n-1}\binom{2n-1}{i}
   = \frac{1}{2}\sum_{j=0}^{2n-1}\binom{2n-1}{j} = 4^{n-1},
\end{align*}
where the penultimate step follows from the fact that an odd row in
Pascal's triangle contains an even number of coefficients and the two
halves have equal sums. We conclude:
\begin{equation}
\Expected{E_n} = (n+1) 4^{n-1}\binom{2n}{n}^{-1} \sim
\frac{1}{4}n\sqrt{\pi n}.
\label{eq:EEn}
\end{equation}
The derivation of \(\Expected{I_n}\) is easy because
\begin{equation}
\Expected{P_n} = \Expected{E_n} + \Expected{I_n}.
\label{eq:PPn_EEn_EIn}
\end{equation}
From~\eqref{eq:EPn} and~\eqref{eq:EEn}, we express \(\Expected{P_n}\)
in terms of \(\Expected{E_n}\):
\begin{equation*}
\Expected{P_n} = 2 \Expected{E_n} - \frac{n+1}{2},
\end{equation*}
then, replacing it in~\eqref{eq:PPn_EEn_EIn}, we finally draw
\begin{equation*}
\Expected{I_n} = \Expected{E_n} - \frac{n+1}{2},
\end{equation*}
and
\begin{equation}
\Expected{I_n}
  = (n+1)4^{n-1}\binom{2n}{n}^{-1} - \frac{n+1}{2} \sim
  \frac{1}{4}n\sqrt{\pi n}.
\label{eq:EIn}
\end{equation}
Moreover, formulas \eqref{eq:EPn}, \eqref{eq:EEn} and~\eqref{eq:EIn}
entail
\begin{equation*}
\Expected{I_n} \sim \Expected{E_n} \sim \frac{1}{2}\Expected{P_n}.
\end{equation*}

\section{Average number of leaves}

The degree\hyp{}based bijection we introduced in
\fig~\vref{fig:degree_based_bijection} implies that there are
\((n+1)/2\) leaves in average in a random Catalan tree with
\(n\)~edges. Indeed, a leaf is a corner in the ordinary lattice path,
and it is \emph{not} a corner in the degree\hyp{}based lattice path,
that is, an internal node (non\hyp{}leaf), therefore, leaves and
non\hyp{}leaves are equinumerous and, since their total number is
\(n+1\), the average number of leaves is \((n+1)/2\).

This fact has an interesting consequence for the programming of
algorithms on Catalan trees, in particular, the data
structure. Indeed, a first idea would be to use a data constructor
\fun{tree} with a pair argument, whose first component is the data
stored at the root and the second component is the list of the
subtrees, for instance \(\fun{tree}(0,[\fun{tree}(1,\el),
\fun{tree}(2,\el)])\) denotes a tree with a root~\(0\), and two
leaves, \(1\)~and~\(2\). Note how leaves are uniquely distinguished by
an empty list of subtrees, which makes function definitions by case
easy because only two cases are enough to cover all
configurations. The downside becomes apparent when implementing the
algorithms in a real functional language, because storing a list means
having an indirection to the heap, where the contents of the list is
actually laid out. Therefore, with the previous encoding of Catalan
trees, each leaf means one indirection and no contents, which may seem
wasteful. If we assume that the Catalan trees to be processed are
random, the question boils down to determining the average number of
leaves, which we happily know now to be half of the nodes. In other
words, \(50\%\) of the nodes are represented in memory with an
indirection without contents. An alternative encoding consists in
using lists only if they contain at least one subtree, that is, only
for inner nodes, and leaves would be represented by a constructor
without list. The previous example would be thus rewritten:
\(\fun{inner}(0,[\fun{leaf}(1),\fun{leaf}(2)])\). This is not quite
right, because we could inadvertently write \(\fun{inner}(0,\el)\),
which is incorrect. Instead, we need to define a data structure for
encoding non\hyp{}empty lists, like so: \(\fun{single}(x)\), for the
singleton list containing~\(x\), and \(\fun{many}(x,l)\), for lists
containing at least two items, the first being~\(x\) and the remaining
ones being found in the sublist~\(l\). Consequently, the previous
example is now correctly written as
\(\fun{inner}(0,\fun{many}(\fun{leaf}(1),\fun{single}(\fun{leaf}(2))))\).

For further reading, we recommend the papers by
\citet*{DershwowitzZaks_1980,DershowitzZaks_1981,DershowitzZaks_1990}.

\section{Average height}
\label{sec:Catalan_height}

As mentioned earlier, the \emph{height} of a tree is the number of
edges on a maximal path from the root to a leaf, that is, a node
without subtrees; for example, we can follow down and count the edges
connecting the nodes~\((\circ)\) in \fig~\ref{fig:catalan_tree}. A
tree reduced to a single leaf has height~\(0\).

We begin with the key observation that a Catalan tree with \(n\)~edges
and height~\(h\) is in bijection with a Dyck path of length~\(2n\) and
height~\(h\) (see \fig~\vref{fig:catalan_tree} and
\fig~\ref{fig:bijection}). This simple fact allows us to reckon on the
height of the Dyck paths and transfer the result back to Catalan
trees.

We have already seen the correspondence between lattice paths and
Catalan trees, in which a rise reaching the $l$th diagonal corresponds
to a node at level~$l$ in the tree, counting levels from root
level~$0$. A simple bijection between paths will show that for every
node on level~$l$ of a tree of height~$h$ and size~$n$, there is a
corresponding node on either level $h-l+1$ or $h-l$ in another tree of
same height and size. (This bijection was found by Nachum Dershowitz.)

Consider the Dyck path in~\Fig~\ref{fig:h},
\begin{figure}
\centering
\includegraphics{h}
\caption{A Dyck path of length~\(2n\) and height~\(h\)}
\label{fig:h}
\end{figure}
in bijection with a tree with~\(n=8\) edges and height~\(h=3\). Let us
find the last (rightmost) point on the path where it reaches its full
height (the dotted line of equation \(y = x + h\)), which we call the
\emph{apex} of the path (marked~$A$ in the figure). The immediately
following fall leads to~$B$ and it is drawn with a double line. Let us
rotate the segment from $(0,0)$ to~$A$, and the segment from~$B$
to~$(n,n)$ by 180$^\circ$. The invariant fall $(A,B)$ now connects the
rotated segments. This way, what was the apex becomes the origin and
vice\hyp{}versa, making this a height\hyp{}preserving bijection
between paths. See \fig~\ref{fig:h_prime}.
\begin{figure}
\centering
\includegraphics{h_prime}
\caption{Dyck path in bijection with \fig~\ref{fig:h}}
\label{fig:h_prime}
\end{figure}

The point is that every rise to level~\(l\) in \fig~\ref{fig:h},
representing a node on level~\(l\) in the corresponding Catalan tree,
ends up reaching level $h-l+1$ or $h-l$ in \fig~\ref{fig:h_prime},
depending on whether it was to the left (segment before~$A$) or right
(segment after~$B$) of the apex.  In the example in the figure, the
rise~$a$ reaches level~$1$, and its counterpart after the
transformation rises to level \(3-1+1=3\); the rise~$b$ reached
level~$2$ and still does so because \(3-2+1=2\); the rise~$c$ also
reached level~$2$, but because it was to the right of the apex, it
reaches now level \(3-2=1\). It follows from this bijection that
\emph{the average height is within one of twice the average level of a
  node, that is, the average path length}. Equation~\eqref{eq:EPn} is
equivalent to
\begin{equation*}
2\frac{\Expected{P_n}}{n+1} = 4^{n}\binom{2n}{n}^{-1} - 1.
\end{equation*}
If \(\Expected{H_n}\) is the average height of a Catalan tree with
\(n\)~edges, we then have, recalling~\eqref{eq:Pn_asymptotics},
\begin{equation*}
\Expected{H_n} \sim 2\frac{\Expected{P_n}}{n+1} \sim \sqrt{\pi n}.
\end{equation*}
