\section{Flattening}
\label{sec:flattening}
\index{stack!flattening|(}

Let us design a function~\fun{flat/1}\index{flat@\fun{flat/1}} such
that the call \(\fun{flat}(s)\), where \(s\)~is a stack, is rewritten
into a stack containing only the non\hyp{}stack items found in~\(s\),
in the same order. If \(s\)~contains no stack, then \(\fun{flat}(s)
\equiv s\). Let us review some
examples\index{stack!flattening!example} to grasp the concept:
\begin{equation*}
\fun{flat}(\el) \twoheadrightarrow \el; \quad
\fun{flat}([\el,[\el]]) \twoheadrightarrow \el;\quad
\fun{flat}([\el,[1,[2,\el],3],\el]) \twoheadrightarrow [1,2,3].
\end{equation*}
First, let us focus on designing the left\hyp{}hand sides of the
rules, in order to ensure that our definition is \emph{complete} (all
valid inputs are matched). A stack is either empty or not and, in the
latter case, the specific issue at hand appears clearly: we need to
distinguish the items which are stacks themselves from those which are
not. This is very simply achieved by ordering the patterns so that
\(\el\) and \(\cons{x}{s}\) \emph{as items} appear
first:\index{flat@\fun{flat/1}|(}
\begin{equation*}
\begin{array}{r@{\;}l@{\;}l}
\fun{flat}(\el)                   & \xrightarrow{\smash{\psi}}   & \fbcode{CCCCCCC}\,;\\
\fun{flat}(\cons{\el}{t})         & \xrightarrow{\smash{\omega}} & \fbcode{CCCCCCC}\,;\\
\fun{flat}(\cons{\cons{x}{s}}{t}) & \xrightarrow{\smash{\gamma}} & \fbcode{CCCCCCC}\,;\\
\fun{flat}(\cons{y}{t})           & \xrightarrow{\smash{\delta}} & \fbcode{CCCCCCC}\,.
\end{array}
\end{equation*}
We know that \(y\)~in the last line is not a stack, otherwise the
penultimate or antepenultimate pattern would have matched. Almost all
the right\hyp{}hand sides are easy to guess now:
\begin{equation*}
\begin{array}{r@{\;}l@{\;}l}
\fun{flat}(\el)                   & \xrightarrow{\smash{\psi}}   & \el;\\
\fun{flat}(\cons{\el}{t})         & \xrightarrow{\smash{\omega}} & \fun{flat}(t);\\
\fun{flat}(\cons{\cons{x}{s}}{t}) & \xrightarrow{\smash{\gamma}} & \fbcode{CCCCCCC}\,;\\
\fun{flat}(\cons{y}{t})           & \xrightarrow{\smash{\delta}} & \cons{y}{\fun{flat}(t)}.
\end{array}
\end{equation*}
\index{flat@\fun{flat/1}|)}The design of the remaining right\hyp{}hand
side can be guided by two slightly different principles. If we look
back at the definitions of~\fun{rev\(_0\)/1}
\index{rev0@\fun{rev\(_0\)/1}|(}
and~\fun{rev/1}\index{rev@\fun{rev/1}} in section~\ref{sec:reversal},
we see that the former was designed with the result in mind, as if the
arrows would reach a value, which is then decomposed in terms of the
variables of the corresponding left\hyp{}hand side:
\begin{equation*}
\begin{array}{r@{\;}l@{\;}l}
\fun{rev}_0(\el) & \rightarrow & \el;\\
\fun{rev}_0(\cons{x}{s}) & \rightarrow &
\fun{cat}(\fun{rev}_0(s),[x]).
\end{array}
\end{equation*}
\index{rev0@\fun{rev\(_0\)/1}|)}By contrast,
\fun{rev/1}\index{rev@\fun{rev/1}} relies on another function,
\fun{rcat/2}\index{rcat@\fun{rcat/2}}, to accumulate partial results,
as if each arrow covered a short distance, only contributing minimally
to the final value:
\begin{equation*}
\begin{array}{r@{\;}l@{\;}l}
\fun{rev}(s) & \smashedrightarrow{\epsilon} & \fun{rcat}(s,\el).\\
\fun{rcat}(\el,t) & \smashedrightarrow{\zeta} & t;\\
\fun{rcat}(\cons{x}{s},t) & \smashedrightarrow{\eta} &
\fun{rcat}(s,\cons{x}{t}).
\end{array}
\end{equation*}
The first approach might be called \emph{big\hyp{}step
  design}\index{design!big-step $\sim$}, and the other
\emph{small\hyp{}step design}\index{design!small-step $\sim$}. Another
vantage point is to see that the former uses the context of the
recursive call to build the value, whilst the latter relies
exclusively on an argument (the accumulator\index{functional
  language!accumulator}) and is in tail form\index{functional
  language!tail form}. For example, in section~\ref{sec:skipping}, we
may find that the definition of \fun{sfst/2}\index{sfst@\fun{sfst/2}}
follows a big\hyp{}step design, while
\fun{sfst\(_0\)/2}\index{sfst@\fun{sfst/2}} illustrates a
small\hyp{}step design.


\paragraph{Big-step design}
\index{design!big-step $\sim$}
\index{divide and conquer}
\label{big-step}

Abstractly, a big\hyp{}step design means that the right\hyp{}hand
sides are made up of recursive calls on substructures, for instance,
in the case of rule~\(\gamma\), the substructures of
\(\cons{\cons{x}{s}}{t}\) are~\(x\), \(s\), \(t\) and
\(\cons{x}{s}\). By thinking how to make up the value by means of
\(\fun{flat}(\cons{x}{s})\)\index{flat@\fun{flat/1}} and
\(\fun{flat}(t)\), we obtain a new version,
\fun{flat\(_0\)/1}\index{flat0@\fun{flat$_0$/1}}, in
\fig~\vref{fig:flat0}.
\begin{figure}[t]
\begin{equation*}
\boxed{%
\begin{array}{r@{\;}l@{\;}l}
\fun{flat}_0(\el)                   & \xrightarrow{\smash{\psi}}
                                    & \el;\\
\fun{flat}_0(\cons{\el}{t})         & \xrightarrow{\smash{\omega}}
                                    & \fun{flat}_0(t);\\
\fun{flat}_0(\cons{\cons{x}{s}}{t}) & \xrightarrow{\smash{\gamma}}
        & \fun{cat}(\fun{flat}_0(\cons{x}{s}),\fun{flat}_0(t));\\
\fun{flat}_0(\cons{y}{t})           & \xrightarrow{\smash{\delta}}
                                    & \cons{y}{\fun{flat}_0(t)}.
\end{array}}
\end{equation*}
\caption{Flattening a stack with \fun{flat\(_0\)/1}}
\label{fig:flat0}\index{stack!flattening!definition}
\end{figure}

Let us consider an example in \fig~\vref{fig:flat0_ex}, where the call
to be rewritten next is underlined in case of ambiguity.
\begin{figure}[!b]
\begin{equation*}
\boxed{%
\begin{array}{r@{\;}l@{\;}l}
\fun{flat}_0([\el,[[1],2],3])
& \xrightarrow{\smash{\omega}}
& \fun{flat}_0([[[1],2],3])\\
& \xrightarrow{\smash{\gamma}}
& \fun{cat}(\underline{\fun{flat}_0([[1],2])},\fun{flat}_0([3]))\\
& \xrightarrow{\smash{\gamma}}
& \fun{cat}(\fun{cat}(\underline{\fun{flat}_0([1])},\fun{flat}_0([2])),\fun{flat}_0([3]))\\
& \xrightarrow{\smash{\delta}}
& \fun{cat}(\fun{cat}(\cons{1}{\underline{\fun{flat}_0(\el)}},\fun{flat}_0([2])),\fun{flat}_0([3]))\\
& \xrightarrow{\smash{\psi}}
& \fun{cat}(\fun{cat}([1],\underline{\fun{flat}_0([2])}),\fun{flat}_0([3]))\\
& \xrightarrow{\smash{\delta}}
& \fun{cat}(\fun{cat}([1],\cons{2}{\underline{\fun{flat}_0(\el)}}),\fun{flat}_0([3]))\\
& \xrightarrow{\smash{\psi}}
& \fun{cat}(\fun{cat}([1],[2]),\underline{\fun{flat}_0([3])})\\
& \xrightarrow{\smash{\delta}}
& \fun{cat}(\fun{cat}([1],[2]),\cons{3}{\underline{\fun{flat}_0(\el)}})\\
& \xrightarrow{\smash{\psi}}
& \fun{cat}(\fun{cat}([1],[2]),[3])\\
& \twoheadrightarrow & [1,2,3].
\end{array}
}
\end{equation*}
\caption{\(\fun{flat}_0([\el,[[1],2],3]) \twoheadrightarrow [1,2,3]\)}
\label{fig:flat0_ex}\index{stack!flattening!example}
\end{figure}
Note that the call\hyp{}by\hyp{}value\index{functional
  language!call-by-value} strategy (section~\ref{sec:functional}) does
not specify the order of evaluation of the arguments of a function
call: in our example, we delayed the evaluation of
\(\fun{cat}([1],[2])\) \index{cat@\fun{cat/2}|(} after that of the
calls to \fun{flat\(_0\)/1}\index{flat0@\fun{flat$_0$/1}}. When
deriving complicated recurrences or traces, we may try instead
counting the number of times each rule is used in any
evaluation. Calling \fun{flat\(_0\)/1}\index{flat0@\fun{flat$_0$/1}}
yields
\begin{itemize*}

  \item using rule~\(\omega\) once for each empty stack originally in
    the input;

  \item using rule~\(\psi\) once when the end of the input is reached
    \emph{and} once for each empty stack \(t\) in rule~\(\gamma\);

  \item using rule~\(\delta\) once for each item which is not a stack;

  \item using rule~\(\gamma\) once for each non\hyp{}empty embedded
    stack;

  \item calling \fun{cat/2}\index{cat@\fun{cat/2}|)} once for each
    non\hyp{}empty embedded stack.

\end{itemize*}
We now know the parameters which the cost depends upon:
\begin{enumerate*}

  \item the length of\index{flat0@\fun{flat$_0$/1}}
    \(\fun{flat}_0(s)\), noted~\(n\);

  \item the number of non\hyp{}empty stacks embedded in the input,
    say~\(\Gamma\);

  \item the number of embedded empty stacks, denoted by~\(\Omega\).

\end{enumerate*}
The dependence upon the size of the output, \(n\), is an instance of
\emph{output\hyp{}dependent cost}\index{cost!output-dependent
  $\sim$}. We can reformulate the above analysis in the following
terms: rule~\(\psi\) is used \(1 + \Gamma\)~times, rule~\(\omega\) is
used \(\Omega\)~times, rule~\(\gamma\) is used \(\Gamma\)~times,
rule~\(\delta\) is used \(n\)~times. So the cost due to the rules of
\fun{flat\(_0\)/1}\index{flat0@\fun{flat$_0$/1}} alone is \(1 + n +
\Omega + 2\Gamma\). For instance, in the case of
\(\fun{flat}_0([\el,[[1],2],3])\), we correctly find \(1 + 3 + 1 + 2
\cdot 2 = 9 = \len{\omega\gamma^2(\delta\psi)^3}\).

As far as the costs of the calls to \fun{cat/2} are concerned, their
associativity\index{cat@\fun{cat/2}} was proved on
page~\pageref{proof:assoc_cat} and equation~\eqref{ineq:cat_assoc}
\vpageref{ineq:cat_assoc} suggests that there are configurations of
the input to \fun{flat\(_0\)/1}\index{flat0@\fun{flat$_0$/1}} that
lead to greater costs, when the parameters \(n\), \(\Omega\)
and~\(\Gamma\) are fixed. A similar pattern of calls to
\fun{cat/2}\index{cat@\fun{cat/2}} with a quadratic cost is generated
from the definition of
\fun{rev\(_0\)/1}\index{rev0@\fun{rev\(_0\)/1}}, as seen in
\fig~\vref{fig:rev0_321}, after the first application of
rule~\(\alpha\). The right\hyp{}hand side of rule~\(\gamma\) is
\(\fun{cat}(\fun{flat}_0(\cons{x}{s}),\fun{flat}_0(t))\)
\index{flat0@\fun{flat$_0$/1}} and it implies that the arguments of
\fun{cat/2}\index{cat@\fun{cat/2}} may be empty stacks.

Given \(\Gamma\) \fun{cat}\hyp{}nodes, \(n\)~non\hyp{}stack nodes
(\(x_1\), \dots, \(x_n\)), what are the abstract syntax trees with
minimum and maximum costs? We found that the minimum cost for
\fun{cat/2}\index{cat@\fun{cat/2}} is achieved when all the
\fun{cat}\hyp{}nodes make up the rightmost
\emph{branch}\index{tree!branch} of the abstract syntax tree. (A
branch is a series of nodes such that one is the parent of the next,
from the root to a leaf.)
\begin{itemize}

  \item If \(\Omega \geqslant \Gamma\), the minimum configuration is
  shown in \fig~\ref{fig:cat_min1}
\begin{figure}
\centering
\subfloat[If \(\Omega \geqslant \Gamma\), the minimum is \(\Gamma\).
\label{fig:cat_min1}]{\includegraphics[bb=66 660 191 722]{cat_min1}}
\quad
\subfloat[If \(\Omega < \Gamma\), the minimum is \(2\Gamma-\Omega\).
\label{fig:cat_min2}]{\includegraphics[bb=70 643 230 721]{cat_min2}}
\caption{Minimum costs for the concatenations in \fun{flat\(_0\)/1}}
\label{fig:cat_min}
\end{figure}
  (at least one empty stack must be placed in any non\hyp{}empty stack
  whose flattening results in an empty stack).

\item Otherwise, the abstract syntax tree of minimum cost is found in
  \fig~\ref{fig:cat_min2}, where all the available empty stacks
  (\(\Omega\)) are used for the bottommost \fun{cat}\hyp{}nodes. We
  draw that the minimum cost\index{stack!flattening!minimum cost}
  is\index{flat0@$\B{\fun{flat}_0}{n}$}
  \(\B{\fun{flat}_0}{n,\Omega,\Gamma} = 1 + n + \Omega + 3\Gamma +
  \min\{\Omega,\Gamma\}\).

\end{itemize}

The maximum cost\index{stack!flattening!maximum cost} in
\fig~\ref{fig:cat_max} occurs for the symmetrical tree of
\fig~\ref{fig:cat_min1}. We
have\index{flat0@$\W{\fun{flat}_0}{n,\Omega,\Gamma}$}
\(\W{\fun{flat}_0}{n,\Omega,\Gamma} = \Omega + (n+3)(\Gamma+1) - 2\).
\begin{figure}[t]
\centering
\subfloat[Maximum cost \((n+1)\Gamma\)\label{fig:cat_max}]{
\includegraphics[bb=71 660 188 721]{cat_max}}
\qquad
\subfloat[$\protect\fun{flat}(\cons{\cons{x}{s}}{t})$\label{fig:lift1}]%
{\includegraphics[bb=55 661 122 721]{flat_embed}}
\qquad
\subfloat[$\protect\fun{flat}(\cons{x,s}{t})$\label{fig:lift2}]%
{\includegraphics[bb=57 661 120 721]{flat_lift}}
\caption{Maximum cost and right rotations}
\end{figure}

\paragraph{Small-step design}
\index{design!small-step $\sim$}

An alternative method for flattening of a stack consists in
lifting~\(x\) in rule~\(\gamma\) one level up amongst the embedded
stacks, thus approaching little by little a flat stack. In terms of
the abstract syntax trees, this operation is a \emph{right
  rotation}\index{binary tree!rotation} of the tree associated with
the argument, as shown in \figs~\vrefrange{fig:lift1}{fig:lift2},
where the new function is called
\fun{flat/1}\index{flat@\fun{flat/1}|(} and defined in
\fig~\vref{fig:flat}.
% Wrapping figure better declared before a paragraph
%
%\vspace*{0pt} % Because starts at the top of the page...
%\begin{wrapfigure}[7]{r}[0pt]{0pt}
% [7] vertical lines
% {r} mandatory right placement
\begin{figure}
\centering
\includegraphics[bb=71 658 220 710]{flat}%[... 721]
\caption{Defining \fun{flat/1}}
\label{fig:flat}
\index{stack!flattening!definition}
\end{figure}
%\end{wrapfigure}
Let us run the previous example of
\fig~\vref{fig:flat0_ex} again in \fig~\vref{fig:flat_ex} with
\fun{flat/1}. The difference in cost with
\fun{flat\(_0\)/1}\index{flat0@\fun{flat$_0$/1}} resides in the number
of times rule~\(\gamma\) is used: once for each item in all the
embedded stacks. Hence the
cost\index{stack!flattening!cost}\label{cost:flat} is \(1 + n + \Omega
+ \Gamma + L\), where \(L\)~is the sum of the lengths of all the
embedded stacks.

\paragraph{Comparison}

Consider the following costs:
\begin{equation*}
\begin{array}{@{}r@{\;}c@{\;}r@{\;}c@{\;}r@{\;}c@{\;}l@{}}
\Call{\fun{flat}([[[[[1,2]]]]])}
  &=& 12 &<& 23 &=& \Call{\fun{flat\(_0\)}([[[[[1,2]]]]])};\\
\Call{\fun{flat}([\el,[[1],2],3])}
  &=& 10 &<& 14 &=& \Call{\fun{flat\(_0\)}([\el,[[1],2],3])};\\
\Call{\fun{flat}([\el,[[1,[2]]],3])}
  &=& 12 &<& 19 &=& \Call{\fun{flat\(_0\)}([\el,[[1,[2]]],3])};\\
\Call{\fun{flat}([[\el,\el,\el]])}
  &=&  8 &>&  7 &=& \Call{\fun{flat\(_0\)}([[\el,\el,\el]])}.
\end{array}
\end{equation*}
\begin{figure}
\begin{equation*}
\boxed{%
\begin{array}{r@{\;}l@{\;}l}
\fun{flat}([\el,[[1],2],3])
& \xrightarrow{\smash{\omega}} & \fun{flat}([[[1],2],3])\\
& \xrightarrow{\smash{\gamma}} & \fun{flat}([[1],[2],3])\\
& \xrightarrow{\smash{\gamma}} & \fun{flat}([1,\el,[2],3])\\
& \xrightarrow{\smash{\delta}} & \cons{1}{\fun{flat}([\el,[2],3])}\\
& \xrightarrow{\smash{\omega}} & \cons{1}{\fun{flat}([[2],3])}\\
& \xrightarrow{\smash{\gamma}} & \cons{1}{\fun{flat}([2,\el,3])}\\
& \xrightarrow{\smash{\delta}} & \cons{1,2}{\fun{flat}([\el,3])}\\
& \xrightarrow{\smash{\omega}} & \cons{1,2}{\fun{flat}([3])}\\
& \xrightarrow{\smash{\delta}} & \cons{1,2,3}{\fun{flat}(\el)}\\
& \xrightarrow{\smash{\psi}}   & [1,2,3].
\end{array}}
\end{equation*}
\caption{\(\fun{flat}([\el,[[1],2],3]) \twoheadrightarrow [1,2,3]\)}
\label{fig:flat_ex}
\index{stack!flattening!example}
\end{figure}
Simple algebra shows that\index{flat@$\C{\fun{flat}}{n,\Omega,\Gamma}$}
\index{flat0@$\B{\fun{flat}_0}{n,\Omega,\Gamma}$}
\begin{equation*}
\C{\fun{flat}}{n,\Omega,\Gamma}
\leqslant
\B{\fun{flat}_0}{n,\Omega,\Gamma}
\Leftrightarrow
\begin{cases}
  L \leqslant 3\Gamma - \Omega,
                         & \text{if \(\Omega \geqslant \Gamma\)};\\
  L \leqslant 2\Gamma,   & \text{otherwise.}
\end{cases}
\end{equation*}
\index{flat@\fun{flat/1}|)} This criterion is impractical to check and
it is inconclusive if the inequalities on the right\hyp{}hand side
fail. Things worsen if we settle for \(L \leqslant 2\Gamma \Rightarrow
\smash[t]{\C{\fun{flat}}{n,\Omega,\Gamma}} \leqslant
\smash[t]{\C{\fun{flat}_0}{n,\Omega,\Gamma}}\), because this condition
is too strong when \(\Omega\)~is large. Let us examine what happens at
the other extreme, when \(\Omega = 0\). Considering an example like
\([0,[1,[2]],3,[4,5,[6,7],8],9]\), it may occur to us that if there
are no empty stacks, the length of each stack is lower than or equal
to the number of non\hyp{}stack items in it and its embeddings (it is
equal if there is no further embedded stack). Therefore, summing up
all these inequalities, the following inequality ensues:
\begin{equation}
  \Omega = 0 \Rightarrow C - L \geqslant 0, \label{eq:CL}
\end{equation}
where \(C\)~is the cost of rewriting the calls
to~\fun{cat/2}\index{cat@\fun{cat/2}}
in~\fun{flat\(_0\)/1}\index{flat0@\fun{flat$_0$/1}}. Because
\(\C{\smash{\fun{flat}_0}}{n,\Omega,\Gamma} = (1 + n + \Omega +
\Gamma) + (C + \Gamma)\) and \(\C{\fun{flat}}{n,\Omega,\Gamma} = (1 +
n + \Omega + \Gamma) + L\),
\begin{equation*}
  \C{\smash{\fun{flat}_0}}{n,\Omega,\Gamma} -
  \C{\fun{flat}}{n,\Omega,\Gamma} =
  (C - L) + \Gamma.
\end{equation*}
From equation~\eqref{eq:CL}, we deduce that, in the absence of empty
stacks, we have the inequality
\(\C{\smash{\fun{flat}_0}}{n,\Omega,\Gamma} -
\C{\fun{flat}}{n,\Omega,\Gamma} \geqslant L\), that is,
\fun{flat/1}\index{flat@\fun{flat/1}} is faster.


\mypar{Termination}
\label{flattening:termination}
\index{stack!flattening!termination|(}

As we have seen with the simplified Ackermann's function
(section~\ref{par:ackermann}, \vpageref{par:ackermann}), termination
follows from finding a well\hyp{}founded order \((\succ)\) on the
recursive calls, which is entailed by the rewrite relation
\((\rightarrow)\), that is, \(x \rightarrow y \Rightarrow x \succ
y\). One well\hyp{}founded order for stacks is the \emph{immediate
  subterm order}, satisfying \(\cons{x}{s} \succ s\) and \(\cons{x}{s}
\succ x\). Since big\hyp{}step design uses recursive calls to subterms
(section~\ref{par:well-founded}, \vpageref{par:well-founded}), it
eases termination proofs based on such an ordering.

For instance, let us recall the definition of
\fun{flat\(_0\)/1}\index{flat0@\fun{flat$_0$/1}|(} in
\fig~\ref{fig:flat0} on page~\pageref{fig:flat0}. Since
\fun{cat/2}\index{cat@\fun{cat/2}} is independent of
\fun{flat\(_0\)/1}, we prove its termination in isolation by using the
proper subterm order on its first argument. Assuming now that
\fun{cat/2} terminates, let us prove the termination of
\fun{flat\(_0\)/1}. Because the recursive calls of \fun{flat\(_0\)/1}
contain only (stack) constructors, we can try to order their arguments
\citep{ArtsGiesl_1996}. Again, the same order works:
\begin{itemize*}

  \item \(\cons{y}{t} \succ t\) by rule~\(\delta\) and rule~\(\omega\)
    when \(y=\el\),

  \item \(\cons{\cons{x}{s}}{t} \succ t\) and

  \item \(\cons{\cons{x}{s}}{t} \succ \cons{x}{s}\) by
    rule~\(\gamma\).

\end{itemize*}
Termination ensues.\hfill\(\Box\)

Let us further recall the definition of
\fun{flat/1}\index{flat@\fun{flat/1}} in \fig~\ref{fig:flat} and
prove its termination. Here, the order we used for \fun{flat\(_0\)/1}
fails:
\begin{equation*}
  \cons{\cons{x}{s}}{t} \nsucc \cons{x,s}{t} =
  \cons{x}{\cons{s}{t}}.
\end{equation*}
We could try the more general \emph{proper subterm
  order}\index{induction!proper subterm order}, that is, the strict
inclusion of a term into another, but, despite \(\cons{x}{s} \succ
x\), we have \(t \nsucc \cons{s}{t}\). One way out is to define a
\emph{measure}\index{termination!measure} on the stacks
\citep{Giesl_1995a}.\index{flat0@\fun{flat$_0$/1}|)}

A measure \(\measure{\cdot}\) is a map from terms to a
well\hyp{}ordered set (\(A,\succ\)), which is \emph{monotone} with
respect to rewriting:
\begin{equation*}
  x \rightarrow y \Rightarrow \measure{x} \succ \measure{y}.
\end{equation*}
Actually, we will only consider \emph{dependency pairs}
\citep{ArtsGiesl_2000}\index{termination!dependency pair}, that is,
pairs of calls whose first component is the left\hyp{}hand side of a
rule and the second components are the calls in the righ\hyp{}hand
side of same rule. This is easier than working with \(x\)~and~\(y\) in
\(x \rightarrow y\), as only subterms of~\(y\) are considered. The
pairs are\index{flat@\fun{flat/1}}:
\begin{itemize*}

  \item \((\fun{flat}(\cons{\el}{t}),
    \fun{flat}(t))_\omega\),

  \item \((\fun{flat}(\cons{\cons{x}{s}}{t}),
    \fun{flat}(\cons{x,s}{t}))_\gamma\) and

  \item \((\fun{flat}(\cons{y}{t}), \fun{flat}(t))_\delta\), with \(y
    \not\in S\).
\end{itemize*}
We can actually drop the function names, as all the pairs involve
\fun{flat/1}\index{flat@\fun{flat/1}}. A common class of measures are
monotone embeddings into (\(\mathbb{N},>\)), so let us seek a measure
satisfying
\begin{itemize*}

  \item \(\measure{\cons{\cons{x}{s}}{t}} > \measure{\cons{x,s}{t}}\);

  \item \(\measure{\cons{y}{t}} > \measure{t}\), if \(y \not\in S\) or
    \(y=\el\).

\end{itemize*}
For instance, let us set the following \emph{polynomial
  measure}\index{termination!polynomial measure}:
\begin{itemize*}

  \item \(\measure{\cons{x}{s}} := 1 + 2\cdot\measure{x} +
    \measure{s}\);

  \item \(\measure{y} := 0\), if \(y \not\in S\) or \(y=\el\).

\end{itemize*}
We have, for each stack,
\begin{itemize*}

  \item \(\measure{\cons{\cons{x}{s}}{t}} = 3 + 4\cdot\measure{x} +
    2\cdot\measure{s} + \measure{t}\),

  \item \(\measure{\cons{x,s}{t}} = 2 + 2\cdot\measure{x} +
    2\cdot\measure{s} + \measure{t}\).

\end{itemize*}
Therefore \(\measure{\cons{\cons{x}{s}}{t}} = \measure{\cons{x,s}{t}}
+ 1 + 2\cdot\measure{x}\). Because \(\measure{x} \in \mathbb{N}\),
for all~\(x\), we have \(\measure{\cons{\cons{x}{s}}{t}} >
\measure{\cons{x,s}{t}}\). The second inequality yields faster:
\(\measure{\cons{y}{t}} = 1 + \measure{t} > \measure{t}\). This
entails the termination of
\fun{flat/1}\index{flat@\fun{flat/1}}.\index{stack!flattening!termination|)}\hfill\(\Box\)

\cite{Giesl_1997} tackled the termination of mutually recursive
functions. Functional programs, as special cases of
term\hyp{}rewriting systems, have been considered by
\cite{Giesl_1995b} and \cite{GieslWaltherBrauburger_1998}.

% Wrapping figure better declared before a paragraph
%
\begin{wrapfigure}[6]{r}[0pt]{0pt}
% [6] vertical lines
% {r} mandatory right placement
\centering
\includegraphics[bb=71 644 227 708]{flat1}% Originally [... 721]
\caption{Variant flattening}
\label{fig:flat1}
\index{stack!flattening!definition}
\end{wrapfigure}

\paragraph{Exercises}

\begin{enumerate*}

  \item Define \fun{gamma/1}, \fun{lambda/1} and \fun{omega/1}, which
    compute \(\Gamma\), \(L\) and~\(\Omega\), respectively.

  \item Compare the costs of \fun{flat/1}\index{flat@\fun{flat/1}} and
    \fun{flat\(_1\)/1} \index{flat1@\fun{flat$_1$/1}} defined in
    \fig~\ref{fig:flat1} (see~(\(\leadsto\))).

\end{enumerate*}
\index{stack!flattening|)}
